\documentclass[a4paper,12pt]{article}
\usepackage[a4paper, margin=1.27cm]{geometry}
\usepackage[indonesian]{babel}
\usepackage[utf8]{inputenc}
\usepackage{tcolorbox}
\usepackage{array}
\usepackage{multirow}
\usepackage{setspace}
\usepackage{amssymb}
\usepackage{graphicx}
\usepackage{adjustbox}
\usepackage{enumitem}
\usepackage{longtable}
\usepackage{xcolor}
\usepackage{amsmath}
\usepackage{fancyhdr}
\usepackage{titlesec}

% Define colors (black and white theme)
\definecolor{darkgray}{RGB}{64, 64, 64}
\definecolor{lightgray}{RGB}{240, 240, 240}
\definecolor{mediumgray}{RGB}{128, 128, 128}

% Custom tcolorbox styles
\tcbset{
    mainbox/.style={
        colback=white,
        colframe=black,
        boxrule=1pt,
        arc=3pt,
        left=8pt,
        right=8pt,
        top=8pt,
        bottom=8pt
    },
    sectionbox/.style={
        colback=white,
        colframe=black,
        boxrule=1.5pt,
        arc=2pt,
        left=6pt,
        right=6pt,
        top=6pt,
        bottom=6pt
    }
}

\begin{document}

\begin{center}
{\Large\textbf{MODUL AJAR: Lesung Batu Tradisional \& Prinsip Tuas dalam Kehidupan}}
\end{center}

\vspace{0.5cm}

\begin{tcolorbox}[mainbox]
\textbf{Nama Penyusun:} Irfan Ananda Ismail, S.Pd., M.Pd., Gr \\
\textbf{Institusi:} SMP \\
\textbf{Mata Pelajaran:} Ilmu Pengetahuan Alam (IPA) \\
\textbf{Tahun Ajaran:} 2025/2026 \\
\textbf{Semester:} Ganjil \\
\textbf{Jenjang Sekolah:} SMP \\
\textbf{Kelas/Fase:} VIII / D \\
\textbf{Alokasi waktu:} 3 x 40 Menit (1 Pertemuan)
\end{tcolorbox}

\section{DIMENSI PROFIL PELAJAR PANCASILA}
\textit{(Clue untuk Guru: Sebutkan dimensi ini secara eksplisit saat apersepsi agar siswa sadar tujuan non-akademis yang sedang mereka kembangkan).}

\begin{itemize}
\item \textbf{Berkebinekaan Global:} Mengenal dan menghargai budaya, khususnya menganalisis kearifan lokal Minangkabau dalam teknologi tradisional (lesung batu dan alat-alat tuas tradisional), lalu menghubungkannya dengan konsep sains universal tentang prinsip tuas dan keuntungan mekanis.
\item \textbf{Bernalar Kritis:} Ditempa secara intensif saat menganalisis informasi dari sumber sains dan budaya (Tahap S), menyintesiskan kedua perspektif (Tahap A), dan menyusun argumen berbasis bukti (Tahap N).
\item \textbf{Gotong Royong:} Kemampuan untuk bekerja sama secara kolaboratif dalam kelompok untuk merumuskan masalah (Tahap E), melakukan investigasi (Tahap S), dan membangun pemahaman bersama (Tahap A).
\item \textbf{Kreatif:} Menghasilkan argumen atau solusi orisinal yang terintegrasi pada tahap akhir pembelajaran (Tahap N).
\end{itemize}

\section{Sarana dan Prasarana}

\begin{itemize}
\item \textbf{Media:} LKPD 06 - Jurnal Investigasi Dual-Lensa, video singkat penggunaan lesung batu tradisional Minangkabau (durasi 3-4 menit), gambar/diagram berbagai jenis tuas, artikel tentang alat-alat tradisional berbasis tuas, model tuas sederhana atau penggaris dan fulkrum.
\item \textbf{Alat:} Papan tulis/whiteboard, spidol, Proyektor \& Speaker, Kertas Plano atau Karton (1 per kelompok), sticky notes warna-warni, penggaris, pensil sebagai fulkrum, beban kecil untuk demonstrasi tuas.
\item \textbf{Sumber Belajar:} Buku ajar IPA kelas VIII, Tautan video animasi prinsip tuas (misal: bit.ly/animasi-tuas-fisika), tautan artikel/video lesung tradisional (misal: bit.ly/lesung-minang-tradisional).
\end{itemize}

\section{Target Peserta didik}

\begin{itemize}
\item Peserta didik reguler kelas VIII (Fase D).
\end{itemize}

\section{Model Pembelajaran}

\begin{itemize}
\item Model Pembelajaran KESAN (Konektivitas Etnosains-Sains).
\end{itemize}

\section{Pemahaman Bermakna}
\textit{(Clue untuk Guru: Bacakan atau sampaikan narasi ini dengan intonasi yang menarik di akhir pembelajaran untuk mengikat semua pengalaman belajar siswa menjadi satu kesatuan yang bermakna).}

\begin{tcolorbox}[sectionbox]
"Ananda Semua, hari ini kita telah mengungkap keajaiban di balik lesung batu yang masih digunakan di dapur-dapur tradisional Minangkabau! Kita menemukan bahwa alat sederhana ini sebenarnya menerapkan prinsip tuas yang sangat cerdas - dengan posisi fulkrum yang tepat, nenek moyang kita bisa menumbuk bumbu dan beras dengan efisiensi maksimal. Setiap ayunan alu, setiap tumbukan yang menghasilkan bumbu halus, semuanya mengikuti prinsip-prinsip fisika yang memungkinkan manusia mengoptimalkan tenaga dan menghasilkan kualitas terbaik. Dengan ini, kita sadar bahwa teknologi tradisional adalah manifestasi pemahaman intuitif tentang hukum-hukum alam yang telah dipraktikkan turun-temurun."
\end{tcolorbox}

\section{PERTEMUAN KEDUA: Prinsip Tuas dalam Teknologi Tradisional Minangkabau}

\subsection{Capaian Pembelajaran (Fase D)}
Pada akhir Fase D, murid memiliki kemampuan menerapkan konsep gaya, gerak, dan energi dalam kehidupan sehari-hari.

\subsection{Tujuan Pembelajaran (TP) Pertemuan 2:}
\textit{(Clue untuk Guru: Tujuan ini adalah kompas Anda. Pastikan setiap tahapan KESAN yang Anda lalui berkontribusi pada pencapaian tujuan-tujuan ini).}

Melalui model pembelajaran KESAN, peserta didik mampu:
\begin{itemize}
\item Menghubungkan fenomena kearifan lokal Minangkabau (teknologi lesung batu dan alat-alat tuas tradisional) dengan konteks prinsip tuas dan keuntungan mekanis. (Sintaks K)
\item Merumuskan pertanyaan investigatif mengenai cara kerja tuas dan hubungannya dengan efisiensi kerja dalam alat-alat tradisional. (Sintaks E)
\item Mengumpulkan informasi mengenai prinsip tuas dari sumber ilmiah serta teknologi pengolahan makanan tradisional dari sumber kultural. (Sintaks S)
\item Menganalisis dan menyintesiskan hubungan sebab-akibat antara posisi fulkrum, lengan kuasa, dan lengan beban dalam optimalisasi kerja tradisional. (Sintaks A)
\item Menyusun sebuah penjelasan analitis yang logis mengenai bagaimana teknologi tradisional Minangkabau mengoptimalkan prinsip tuas untuk efisiensi maksimal. (Sintaks N)
\end{itemize}

\subsection{Pertanyaan Pemantik}
\textit{(Clue untuk Guru: Ajukan dua pertanyaan ini secara berurutan, berikan jeda agar siswa berpikir. Jangan langsung minta jawaban, biarkan pertanyaan ini menggantung untuk memicu rasa ingin tahu).}

\begin{itemize}
\item "Pernahkah kalian membantu ibu atau nenek menumbuk bumbu di lesung batu? Mengapa bentuk lesung dan alu dibuat seperti itu? Apakah ada rahasia di balik desain yang sudah berusia ratusan tahun ini?"
\item "Coba bayangkan kalian harus menumbuk cabai untuk sambal. Mana yang lebih efektif: menumbuk dengan alu pendek atau alu panjang? Dari sudut pandang fisika, apa yang membuat perbedaan ini terjadi?"
\end{itemize}

\section{Langkah-langkah Kegiatan Pembelajaran:}

\subsection{Kegiatan Pembuka (15 Menit)}
\begin{itemize}
\item Guru membuka pelajaran dengan salam, doa, dan memeriksa kehadiran.
\item \textbf{Asesmen Diagnostik Awal:} Guru membagikan lembar K-W-L.
    \begin{itemize}
    \item \textbf{Instruksi Guru:} "Ananda, setelah mengeksplorasi katrol dan pesawat sederhana, sekarang kita akan menyelidiki jenis pesawat sederhana lainnya: tuas! Di lembar K-W-L ini, tulis di kolom \textbf{K (Tahu)} apa yang sudah kalian ketahui tentang alat-alat yang menggunakan prinsip 'ungkit' ATAU tentang lesung dan alu. Di kolom \textbf{W (Ingin Tahu)}, tulis apa yang membuat kalian penasaran tentang cara kerja alat-alat tradisional ini." \textit{(Clue: Ini membantu Anda melihat pemahaman awal siswa tentang konsep 'mengungkit' dan efisiensi kerja).}
    \end{itemize}
\item \textbf{Apersepsi:}
    \begin{itemize}
    \item \textbf{Instruksi Guru:} "Kemarin kita telah mengungkap rahasia katrol dalam memudahkan pekerjaan. Hari ini, kita akan menyelidiki pesawat sederhana lainnya yang tidak kalah menakjubkan: tuas! Misi kita adalah memahami bagaimana lesung batu dan alat-alat tradisional lainnya menerapkan prinsip tuas untuk mengoptimalkan tenaga kita. Siap menjadi ahli fisika yang menghargai kecerdasan desain tradisional?"
    \end{itemize}
\end{itemize}

\subsection{Kegiatan Inti (90 Menit) - SINTAKS MODEL KESAN}

\subsubsection{Tahap 1: (K) Kaitkan Konteks Kultural (15 menit)}
\begin{itemize}
\item \textbf{Aktivitas Guru:}
    \begin{itemize}
    \item Menampilkan video singkat (3-4 menit) penggunaan lesung batu tradisional untuk menumbuk bumbu dan beras, menunjukkan teknik dan ritme yang efisien.
    \item Mengajukan Pertanyaan Pemantik yang sudah disiapkan di atas.
    \item \textit{(Clue: Fokuskan pada keingintahuan tentang mengapa desain lesung dan alu begitu efektif. Saat siswa menjawab, tuliskan semua ide mereka di papan tulis dengan judul "\textbf{RAHASIA DESAIN TRADISIONAL}". Validasi setiap kontribusi dan tunjukkan antusiasme terhadap pengamatan mereka).}
    \end{itemize}
\item \textbf{Aktivitas Siswa:} Mengamati video, mendengarkan pertanyaan, lalu secara sukarela berbagi pengalaman atau dugaan tentang keunggulan desain lesung dan alu. Menuliskan minimal satu pertanyaan atau pengamatan di sticky notes dan menempelkannya di '\textbf{Papan Penasaran}'.
\end{itemize}

\subsubsection{Tahap 2: (E) Eksplorasi Enigma (15 menit)}
\begin{itemize}
\item \textbf{Aktivitas Guru:} Membentuk siswa menjadi kelompok (3-4 orang).
    \begin{itemize}
    \item \textbf{Instruksi Guru:} "Keingintahuan kalian tentang desain alat-alat tradisional sangat menginspirasi! Sekarang, sebagai tim ahli teknologi tradisional, tugas kalian adalah mengubah keingintahuan ini menjadi misi investigasi yang sistematis. Dalam kelompok, diskusikan dan rumuskan minimal 3 pertanyaan kunci yang akan kita selidiki hari ini. Tuliskan dalam bentuk '\textbf{Misi Investigasi Tim [Nama Kelompok]}' di kertas plano."
    \item \textit{(Clue: Arahkan diskusi siswa agar pertanyaannya mencakup aspek 'bagaimana tuas bekerja' dan 'mengapa posisi fulkrum penting'. Jika kelompok kesulitan, berikan pancingan: "Kira-kira, apa yang terjadi dengan gaya yang kita berikan saat menggunakan alu? Dan bagaimana panjang alu mempengaruhi efektivitas menumbuk?").}
    \end{itemize}
\item \textbf{Aktivitas Siswa:} Berdiskusi dalam tim untuk merumuskan misi investigasi (daftar pertanyaan kunci) di kertas plano. \textit{(Contoh misi yang diharapkan: 1. Bagaimana prinsip tuas bekerja pada lesung dan alu? 2. Mengapa posisi pegangan alu mempengaruhi kekuatan tumbukan? 3. Apa hubungan antara panjang lengan kuasa dan lengan beban?)}
\end{itemize}

\subsubsection{Tahap 3: (S) Selidiki secara Sintetis (25 menit)}
\begin{itemize}
\item \textbf{Aktivitas Guru:} Membagikan "\textbf{LKPD 06 - Jurnal Investigasi Dual-Lensa}".
    \begin{itemize}
    \item \textbf{Instruksi Guru:} "Setiap tim akan melakukan investigasi dari dua lensa. Gunakan sumber yang disediakan untuk mencari jawabannya. Bagilah tugas dalam tim!"
    \item \textbf{Lensa Sains:} Buka link video bit.ly/animasi-tuas-fisika untuk memahami konsep tuas (F1 × L1 = F2 × L2), jenis-jenis tuas (kelas 1, 2, dan 3), dan keuntungan mekanis tuas.
    \item \textbf{Lensa Etnosains/Kultural:} Buka link artikel bit.ly/lesung-minang-tradisional untuk memahami sejarah, konstruksi, teknik penggunaan, dan variasi lesung dalam budaya Minangkabau.
    \item \textit{(Clue: Pastikan sumber belajar sudah disiapkan dan dapat diakses. Berkelilinglah untuk memastikan setiap kelompok mengeksplorasi kedua lensa dengan fokus pada prinsip kerja dan efisiensi).}
    \end{itemize}
\item \textbf{Aktivitas Siswa:} Dalam kelompok, siswa berbagi tugas mencari informasi dari sumber yang diberikan dan mencatat temuan kunci di dua kolom terpisah pada "\textbf{LKPD 06 - Jurnal Investigasi Dual-Lensa}".
\end{itemize}

\subsubsection{Tahap 4: (A) Asimilasi Analitis (20 menit)}
\begin{itemize}
\item \textbf{Aktivitas Guru:} Memfasilitasi diskusi untuk menjembatani kedua lensa.
    \begin{itemize}
    \item \textbf{Pertanyaan Pancingan Kunci untuk Guru:}
        \begin{itemize}
        \item "Dari Lensa Sains kita tahu prinsip tuas: F1 × L1 = F2 × L2. Dari Lensa Etnosains, kita tahu alu yang panjang lebih efektif untuk menumbuk. Nah, coba hubungkan! Bagaimana panjang alu (lengan kuasa) mempengaruhi gaya yang dihasilkan untuk menumbuk?"
        \item "Dari Lensa Sains, kita tahu ada 3 jenis tuas berdasarkan posisi fulkrum. Dari Lensa Budaya, kita tahu lesung memiliki desain yang unik. Termasuk jenis tuas yang mana sistem lesung-alu ini? Dan mengapa desain ini dipilih nenek moyang kita?"
        \end{itemize}
    \item \textit{(Clue: Bantu siswa melihat bahwa lesung-alu adalah tuas kelas 3 (fulkrum di ujung, beban di tengah, kuasa di ujung lain). Gunakan demonstrasi sederhana dengan penggaris dan fulkrum jika perlu).}
    \end{itemize}
\item \textbf{Aktivitas Siswa:} Berdiskusi intensif untuk menghubungkan temuan sains dan budaya. Menuliskan kesimpulan terpadu (sintesis) mereka di kertas plano.
\end{itemize}

\subsubsection{Tahap 5: (N) Nyatakan Pemahaman (15 menit)}
\begin{itemize}
\item \textbf{Aktivitas Guru:} Memberikan studi kasus individual atau per kelompok.
    \begin{itemize}
    \item \textbf{Instruksi Guru (tuliskan di papan tulis):}
    
    "\textbf{STUDI KASUS UNTUK DESAINER ALAT TRADISIONAL SAINS-BUDAYA:}
    
    Sebuah komunitas kuliner tradisional ingin mengembangkan alat penumbuk bumbu yang ergonomis dan efisien untuk produksi sambal dalam skala menengah. Mereka ingin mempertahankan kualitas rasa tradisional namun meningkatkan efisiensi kerja. Mereka meminta bantuanmu sebagai konsultan yang memahami baik fisika maupun teknologi tradisional.
    
    Tugasmu: Tuliskan sebuah rekomendasi desain singkat (3-5 kalimat) di buku latihanmu tentang bagaimana mengoptimalkan prinsip tuas dalam desain alat penumbuk modern yang tetap mempertahankan kearifan tradisional. Jelaskan mengapa pemahaman prinsip fisika penting dalam melestarikan dan mengembangkan teknologi tradisional."
    \end{itemize}
\item \textbf{Aktivitas Siswa:} Menyusun argumen tertulis untuk menjawab studi kasus yang diberikan, menggunakan bukti dari kedua lensa.
\end{itemize}

\subsection{Kegiatan Penutup (15 Menit)}
\begin{itemize}
\item \textbf{Presentasi \& Penguatan:} Guru meminta 2-3 siswa secara acak untuk membacakan rekomendasi desain mereka. Guru memberikan pujian dan penguatan positif, menekankan betapa hebatnya kemampuan mereka mengintegrasikan prinsip fisika dengan kearifan tradisional.
\item \textbf{Refleksi:}
    \begin{itemize}
    \item \textbf{Instruksi Guru:} "Sekarang, kembali ke lembar K-W-L kalian. Lengkapi kolom terakhir, \textbf{L (Learned)}, dengan hal-hal baru yang paling mengagumkan yang kalian pelajari hari ini tentang prinsip tuas."
    \item \textbf{Instruksi Guru:} "Angkat tangan, siapa yang setelah belajar hari ini jadi lebih menghargai kecerdasan desain dalam alat-alat tradisional yang kita gunakan sehari-hari?"
    \end{itemize}
\item \textbf{Tindak Lanjut:}
    \begin{itemize}
    \item \textbf{Instruksi Guru:} "Luar biasa, para desainer teknologi tradisional! Hari ini kita sudah mengungkap bagaimana prinsip tuas diterapkan dalam teknologi tradisional. Pertemuan berikutnya, kita akan mengeksplorasi dunia tumbuhan dan bagaimana struktur serta fungsinya mencerminkan kearifan alam yang bisa kita pelajari. Siapa tahu ada rahasia arsitektur alam yang bisa menginspirasi teknologi masa depan!"
    \item \textit{(Clue: Buat transisi yang menarik ke topik berikutnya sambil mempertahankan semangat eksplorasi sains dan budaya).}
    \end{itemize}
\item Guru menutup pelajaran dengan doa dan salam.
\end{itemize}

\section{Asesmen (Penilaian)}

\begin{itemize}
\item \textbf{Asesmen Diagnostik (Awal):} Analisis lembar K-W-L. (Untuk mengetahui pemahaman awal tentang prinsip ungkit dan alat tradisional).
\item \textbf{Asesmen Formatif (Proses):} Observasi keaktifan diskusi (gotong royong) dan penilaian kelengkapan "\textbf{LKPD 06 - Jurnal Investigasi Dual-Lensa}".
\item \textbf{Asesmen Sumatif (Akhir Siklus):} Penilaian jawaban studi kasus (Tahap N) menggunakan rubrik.
\end{itemize}

\subsection{Rubrik Penilaian Jawaban Studi Kasus (Tahap N)}

\begin{longtable}{|p{3cm}|p{3cm}|p{3cm}|p{3cm}|p{3cm}|}
\hline
\textbf{Kriteria Penilaian} & \textbf{Skor 4 (Sangat Baik)} & \textbf{Skor 3 (Baik)} & \textbf{Skor 2 (Cukup)} & \textbf{Skor 1 (Kurang)} \\
\hline
\textbf{Ketepatan Konsep Ilmiah} & Menggunakan konsep tuas, fulkrum, lengan kuasa, lengan beban, dan keuntungan mekanis dengan sangat tepat dan relevan untuk desain alat. & Menggunakan konsep tuas dengan tepat, namun kurang relevan dengan konteks desain. & Menggunakan konsep fisika namun ada beberapa kesalahan dalam penerapan prinsip tuas. & Tidak menggunakan konsep tuas atau salah total. \\
\hline
\textbf{Keterkaitan dengan Etnosains} & Mampu menghubungkan secara logis dan eksplisit antara prinsip tuas dengan teknologi tradisional secara mendalam dan akurat, serta menunjukkan apresiasi terhadap kearifan lokal. & Mampu menghubungkan prinsip fisika dengan teknologi tradisional, namun kurang mendalam dalam apresiasi kearifan lokal. & Hanya menyebutkan teknologi tradisional tanpa menghubungkan dengan prinsip fisika, atau sebaliknya. & Tidak ada keterkaitan antara sains dan teknologi tradisional yang ditunjukkan. \\
\hline
\textbf{Kelogisan \& Struktur Argumen} & Rekomendasi desain sangat logis, runtut, inovatif, dan mempertimbangkan aspek ergonomis serta efisiensi. & Rekomendasi desain logis dan runtut, namun kurang inovatif atau kurang mempertimbangkan aspek ergonomis. & Alur rekomendasi kurang runtut atau kurang praktis dalam implementasi. & Rekomendasi tidak logis dan tidak terstruktur. \\
\hline
\end{longtable}

\section{Daftar Pustaka Sumber Etnosains}

\begin{enumerate}
\item Dt. Rajo Penghulu. (1994). \textit{Teknologi Tradisional Minangkabau}. Padang: Pusat Dokumentasi dan Informasi Kebudayaan Minangkabau.
\item Navis, A.A. (1984). \textit{Alam Terkembang Jadi Guru: Adat dan Kebudayaan Minangkabau}. Jakarta: Grafiti Pers.
\item Kementerian Pendidikan dan Kebudayaan. (2018). \textit{Alat-alat Tradisional Pengolahan Pangan Nusantara}. Jakarta: Direktorat Warisan dan Diplomasi Budaya.
\item Syafwan, A. (2008). \textit{Kearifan Lokal dalam Teknologi Pertanian Minangkabau}. Padang: Universitas Andalas Press.
\item Yusriwal. (2010). \textit{Lesung: Teknologi Pengolahan Pangan Tradisional Minangkabau}. Padang: Balai Pelestarian Nilai Budaya Sumatera Barat.
\end{enumerate}

\end{document}