\documentclass[a4paper,12pt]{article}
\usepackage[a4paper, margin=1.27cm]{geometry}
\usepackage[indonesian]{babel}
\usepackage[utf8]{inputenc}
\usepackage{tcolorbox}
\usepackage{array}
\usepackage{multirow}
\usepackage{setspace}
\usepackage{amssymb}
\usepackage{graphicx}
\usepackage{adjustbox}
\usepackage{enumitem}
\usepackage{longtable}
\usepackage{xcolor}
\usepackage{amsmath}
\usepackage{fancyhdr}
\usepackage{titlesec}

% Define colors (black and white theme)
\definecolor{darkgray}{RGB}{64, 64, 64}
\definecolor{lightgray}{RGB}{240, 240, 240}
\definecolor{mediumgray}{RGB}{128, 128, 128}

% Custom tcolorbox styles
\tcbset{
    mainbox/.style={
        colback=white,
        colframe=black,
        boxrule=1pt,
        arc=3pt,
        left=8pt,
        right=8pt,
        top=8pt,
        bottom=8pt
    },
    sectionbox/.style={
        colback=white,
        colframe=black,
        boxrule=1.5pt,
        arc=2pt,
        left=6pt,
        right=6pt,
        top=6pt,
        bottom=6pt
    }
}

\begin{document}

\begin{center}
{\Large\textbf{MODUL AJAR: Rahasia Gerak Tari Piring \& Sistem Gerak Makhluk Hidup}}
\end{center}

\vspace{0.5cm}

\begin{tcolorbox}[mainbox]
\textbf{Nama Penyusun:} Irfan Ananda Ismail, S.Pd., M.Pd., Gr \\
\textbf{Institusi:} SMP \\
\textbf{Mata Pelajaran:} Ilmu Pengetahuan Alam (IPA) \\
\textbf{Tahun Ajaran:} 2025/2026 \\
\textbf{Semester:} Ganjil \\
\textbf{Jenjang Sekolah:} SMP \\
\textbf{Kelas/Fase:} VIII / D \\
\textbf{Alokasi waktu:} 3 x 40 Menit (1 Pertemuan)
\end{tcolorbox}

\section{DIMENSI PROFIL PELAJAR PANCASILA}
\textit{(Clue untuk Guru: Sebutkan dimensi ini secara eksplisit saat apersepsi agar siswa sadar tujuan non-akademis yang sedang mereka kembangkan).}

\begin{itemize}
\item \textbf{Berkebinekaan Global:} Mengenal dan menghargai budaya, khususnya menganalisis kearifan lokal Minangkabau dalam seni tari tradisional (Tari Piring), lalu menghubungkannya dengan konsep sains universal tentang sistem gerak makhluk hidup.
\item \textbf{Bernalar Kritis:} Ditempa secara intensif saat menganalisis informasi dari sumber sains dan budaya (Tahap S), menyintesiskan kedua perspektif (Tahap A), dan menyusun argumen berbasis bukti (Tahap N).
\item \textbf{Gotong Royong:} Kemampuan untuk bekerja sama secara kolaboratif dalam kelompok untuk merumuskan masalah (Tahap E), melakukan investigasi (Tahap S), dan membangun pemahaman bersama (Tahap A).
\item \textbf{Kreatif:} Menghasilkan argumen atau solusi orisinal yang terintegrasi pada tahap akhir pembelajaran (Tahap N).
\end{itemize}

\section{Sarana dan Prasarana}

\begin{itemize}
\item \textbf{Media:} LKPD 01 - Jurnal Investigasi Dual-Lensa, video singkat Tari Piring Minangkabau (durasi 2-3 menit), gambar/diagram sistem gerak manusia, artikel tentang filosofi gerakan dalam budaya Minang, model rangka manusia atau poster sistem gerak.
\item \textbf{Alat:} Papan tulis/whiteboard, spidol, Proyektor \& Speaker, Kertas Plano atau Karton (1 per kelompok), sticky notes warna-warni.
\item \textbf{Sumber Belajar:} Buku ajar IPA kelas VIII, Tautan video animasi sistem gerak (misal: bit.ly/animasi-gerak), tautan artikel/video filosofi Tari Piring (misal: bit.ly/tari-piring-minang).
\end{itemize}

\section{Target Peserta didik}

\begin{itemize}
\item Peserta didik reguler kelas VIII (Fase D).
\end{itemize}

\section{Model Pembelajaran}

\begin{itemize}
\item Model Pembelajaran KESAN (Konektivitas Etnosains-Sains).
\end{itemize}

\section{Pemahaman Bermakna}
\textit{(Clue untuk Guru: Bacakan atau sampaikan narasi ini dengan intonasi yang menarik di akhir pembelajaran untuk mengikat semua pengalaman belajar siswa menjadi satu kesatuan yang bermakna).}

\begin{tcolorbox}[sectionbox]
"Ananda Semua, hari ini kita telah membongkar rahasia di balik keindahan Tari Piring! Kita menemukan bahwa setiap gerakan anggun penari - dari langkah kaki yang lincah, ayunan tangan yang lentur, hingga keseimbangan tubuh saat memutar piring - semuanya adalah hasil kerja sama yang luar biasa antara tulang, otot, dan sendi dalam tubuh kita. Dengan ini, kita sadar bahwa sains tidak hanya ada di laboratorium, tapi juga ada dalam setiap gerakan budaya yang kita warisi dari leluhur Minangkabau."
\end{tcolorbox}

\section{PERTEMUAN PERTAMA: Sistem Gerak dan Kearifan Lokal Minangkabau}

\subsection{Capaian Pembelajaran (Fase D)}
Pada akhir Fase D, murid memiliki kemampuan menganalisis sistem organisasi kehidupan, fungsi, serta kelainan atau gangguan yang muncul pada sistem organ makhluk hidup.

\subsection{Tujuan Pembelajaran (TP) Pertemuan 1:}
\textit{(Clue untuk Guru: Tujuan ini adalah kompas Anda. Pastikan setiap tahapan KESAN yang Anda lalui berkontribusi pada pencapaian tujuan-tujuan ini).}

Melalui model pembelajaran KESAN, peserta didik mampu:
\begin{itemize}
\item Menghubungkan fenomena kearifan lokal Minangkabau (gerakan dalam Tari Piring) dengan konteks sistem gerak makhluk hidup. (Sintaks K)
\item Merumuskan pertanyaan investigatif mengenai cara kerja sistem gerak dan peran koordinasi tulang-otot-sendi dalam gerakan tari tradisional. (Sintaks E)
\item Mengumpulkan informasi mengenai struktur dan fungsi sistem gerak dari sumber ilmiah serta makna filosofis gerakan tari dari sumber kultural. (Sintaks S)
\item Menganalisis dan menyintesiskan hubungan sebab-akibat antara konsep ilmiah (kontraksi otot, persendian) dengan praktik budaya (teknik gerakan tari). (Sintaks A)
\item Menyusun sebuah penjelasan analitis yang logis mengenai bagaimana kearifan lokal dalam seni tari Minangkabau dapat dibenarkan secara saintifik melalui prinsip-prinsip sistem gerak. (Sintaks N)
\end{itemize}

\subsection{Pertanyaan Pemantik}
\textit{(Clue untuk Guru: Ajukan dua pertanyaan ini secara berurutan, berikan jeda agar siswa berpikir. Jangan langsung minta jawaban, biarkan pertanyaan ini menggantung untuk memicu rasa ingin tahu).}

\begin{itemize}
\item "Pernahkah kalian memperhatikan bagaimana penari Tari Piring bisa bergerak begitu lincah sambil menyeimbangkan piring di telapak tangan mereka? Apa yang membuat mereka tidak pernah menjatuhkan piring meski bergerak dengan cepat?"
\item "Dalam budaya Minang, setiap gerakan tari memiliki makna filosofis. Tapi dari sudut pandang sains, apa yang sebenarnya terjadi di dalam tubuh penari saat mereka melakukan gerakan-gerakan indah tersebut?"
\end{itemize}

\section{Langkah-langkah Kegiatan Pembelajaran:}

\subsection{Kegiatan Pembuka (15 Menit)}
\begin{itemize}
\item Guru membuka pelajaran dengan salam, doa, dan memeriksa kehadiran.
\item \textbf{Asesmen Diagnostik Awal:} Guru membagikan lembar K-W-L.
    \begin{itemize}
    \item \textbf{Instruksi Guru:} "Ananda, sebelum kita mulai berpetualang, tolong isi dua kolom pertama di lembar ini. Di kolom \textbf{K (Tahu)}, tulis apa saja yang sudah kalian ketahui tentang sistem gerak manusia ATAU gerakan dalam tari tradisional. Di kolom \textbf{W (Ingin Tahu)}, tulis apa yang membuat kalian penasaran tentang topik ini." \textit{(Clue: Ini membantu Anda memetakan pengetahuan awal dan minat siswa secara cepat).}
    \end{itemize}
\item \textbf{Apersepsi:}
    \begin{itemize}
    \item \textbf{Instruksi Guru:} "Hari ini kita akan menjadi detektif sains dan budaya. Kita akan menyelidiki bagaimana keindahan Tari Piring berhubungan erat dengan sistem gerak dalam tubuh kita. Dalam investigasi ini, kita akan melatih kemampuan \textbf{Bernalar Kritis} kita, menghargai budaya lewat \textbf{Berkebinekaan Global}, dan bekerja sama dalam semangat \textbf{Gotong Royong}. Siap?"
    \end{itemize}
\end{itemize}

\subsection{Kegiatan Inti (90 Menit) - SINTAKS MODEL KESAN}

\subsubsection{Tahap 1: (K) Kaitkan Konteks Kultural (15 menit)}
\begin{itemize}
\item \textbf{Aktivitas Guru:}
    \begin{itemize}
    \item Menampilkan video singkat (2-3 menit) pertunjukan Tari Piring yang menyorot keanggunan dan koordinasi gerakan penari.
    \item Mengajukan Pertanyaan Pemantik yang sudah disiapkan di atas.
    \item \textit{(Clue: Tujuan tahap ini adalah memvalidasi pengetahuan siswa dan memantik rasa heran, bukan mencari jawaban benar. Sediakan spidol dan papan tulis/whiteboard. Saat siswa menjawab, tuliskan semua ide mereka, bahkan yang keliru sekalipun, dengan judul "\textbf{IDE AWAL KITA}". Ini menunjukkan bahwa semua pemikiran dihargai).}
    \end{itemize}
\item \textbf{Aktivitas Siswa:} Mengamati video, mendengarkan pertanyaan, lalu secara sukarela berbagi pengalaman atau dugaan awal. Menuliskan minimal satu pertanyaan atau pengalaman di sticky notes dan menempelkannya di '\textbf{Papan Penasaran}'.
\end{itemize}

\subsubsection{Tahap 2: (E) Eksplorasi Enigma (15 menit)}
\begin{itemize}
\item \textbf{Aktivitas Guru:} Membentuk siswa menjadi kelompok (3-4 orang).
    \begin{itemize}
    \item \textbf{Instruksi Guru:} "Baik, rasa penasaran kalian luar biasa! Sekarang, tugas kita sebagai detektif adalah mengubah rasa penasaran ini menjadi misi yang jelas. Dalam kelompok, diskusikan dan rumuskan minimal 3 pertanyaan kunci yang akan kita selidiki hari ini. Tuliskan dalam bentuk '\textbf{Misi Penyelidikan Tim [Nama Kelompok]}' di kertas plano yang Bapak/Ibu berikan."
    \item \textit{(Clue: Arahkan diskusi siswa agar pertanyaannya mencakup aspek 'bagaimana tubuh bergerak' dan 'apa makna gerakan tari'. Jika kelompok kesulitan, berikan pancingan: "Kira-kira, apa dulu yang perlu kita tahu? Bagian tubuh yang bergerak atau makna di balik gerakannya?").}
    \end{itemize}
\item \textbf{Aktivitas Siswa:} Berdiskusi dalam tim untuk merumuskan misi penyelidikan (daftar pertanyaan kunci) di kertas plano. \textit{(Contoh misi yang diharapkan: 1. Apa saja bagian tubuh yang terlibat dalam gerakan tari? 2. Bagaimana otot dan tulang bekerja sama saat bergerak? 3. Apa makna filosofis gerakan Tari Piring menurut budaya Minang?)}
\end{itemize}

\subsubsection{Tahap 3: (S) Selidiki secara Sintetis (25 menit)}
\begin{itemize}
\item \textbf{Aktivitas Guru:} Membagikan "\textbf{LKPD 01 - Jurnal Investigasi Dual-Lensa}".
    \begin{itemize}
    \item \textbf{Instruksi Guru:} "Setiap tim akan melakukan investigasi dari dua lensa. Gunakan HP atau sumber yang disediakan untuk mencari jawabannya. Bagilah tugas dalam tim!"
    \item \textbf{Lensa Sains:} Buka link video bit.ly/animasi-gerak untuk memahami struktur dan fungsi tulang, otot, sendi. Baca juga diagram sistem gerak yang Bapak/Ibu bagikan.
    \item \textbf{Lensa Etnosains/Kultural:} Buka link artikel bit.ly/tari-piring-minang untuk tahu makna filosofis dan teknik gerakan dalam Tari Piring.
    \item \textit{(Clue: Pastikan sumber belajar sudah disiapkan dan link bisa diakses. Berkelilinglah untuk memastikan setiap kelompok membagi tugas dan tidak hanya fokus pada satu lensa saja).}
    \end{itemize}
\item \textbf{Aktivitas Siswa:} Dalam kelompok, siswa berbagi tugas mencari informasi dari sumber yang diberikan dan mencatat temuan kunci di dua kolom terpisah pada "\textbf{LKPD 01 - Jurnal Investigasi Dual-Lensa}".
\end{itemize}

\subsubsection{Tahap 4: (A) Asimilasi Analitis (20 menit)}
\begin{itemize}
\item \textbf{Aktivitas Guru:} Memfasilitasi diskusi untuk menjembatani kedua lensa.
    \begin{itemize}
    \item \textbf{Pertanyaan Pancingan Kunci untuk Guru:}
        \begin{itemize}
        \item "Oke, dari Lensa Sains kita tahu otot berkontraksi dan relaksasi untuk menggerakkan tulang. Dari Lensa Etnosains, kita tahu gerakan Tari Piring harus halus dan terkontrol. Nah, coba hubungkan! Bagaimana kontraksi otot yang tepat bisa menghasilkan gerakan tari yang indah?"
        \item "Dari Lensa Sains, kita tahu sendi memungkinkan tulang bergerak. Dari Lensa Budaya, setiap gerakan tari punya makna simbolis. Apa hubungannya fleksibilitas sendi dengan kemampuan penari mengekspresikan makna melalui gerakan?"
        \end{itemize}
    \item \textit{(Clue: Fokuskan untuk membuat siswa 'menemukan' hubungannya sendiri, bukan diberitahu. Gunakan kata "menurut kalian", "kira-kira kenapa", "ada yang punya ide?").}
    \end{itemize}
\item \textbf{Aktivitas Siswa:} Berdiskusi intensif untuk menghubungkan temuan sains dan budaya. Menuliskan kesimpulan terpadu (sintesis) mereka di kertas plano.
\end{itemize}

\subsubsection{Tahap 5: (N) Nyatakan Pemahaman (15 menit)}
\begin{itemize}
\item \textbf{Aktivitas Guru:} Memberikan studi kasus individual atau per kelompok.
    \begin{itemize}
    \item \textbf{Instruksi Guru (tuliskan di papan tulis):}
    
    "\textbf{STUDI KASUS UNTUK DETEKTIF SAINS-BUDAYA:}
    
    Adikmu yang berumur 12 tahun ingin belajar Tari Piring untuk lomba seni budaya di sekolah. Namun, setelah latihan beberapa hari, dia mengeluh tangannya sering kram dan sulit menjaga keseimbangan saat memutar piring. Ibumu menyarankan untuk berhenti, tapi adikmu tetap ingin belajar.
    
    Tugasmu: Tuliskan sebuah penjelasan singkat (3-5 kalimat) di buku latihanmu untuk membantu adikmu. Gunakan pengetahuan gabungan dari sains (misalnya, cara kerja otot, koordinasi gerak) dan kearifan lokal yang baru saja kamu pelajari untuk menjelaskan mengapa latihan yang tepat dan bertahap itu penting."
    \end{itemize}
\item \textbf{Aktivitas Siswa:} Menyusun argumen tertulis untuk menjawab studi kasus yang diberikan, menggunakan bukti dari kedua lensa.
\end{itemize}

\subsection{Kegiatan Penutup (15 Menit)}
\begin{itemize}
\item \textbf{Presentasi \& Penguatan:} Guru meminta 2-3 siswa secara acak untuk membacakan jawaban studi kasus mereka. Guru memberikan pujian dan penguatan positif, menekankan betapa hebatnya argumen yang memadukan sains dan budaya.
\item \textbf{Refleksi:}
    \begin{itemize}
    \item \textbf{Instruksi Guru:} "Sekarang, kembali ke lembar K-W-L kalian. Lengkapi kolom terakhir, \textbf{L (Learned)}, dengan hal-hal baru yang paling berkesan yang kalian pelajari hari ini."
    \item \textbf{Instruksi Guru:} "Angkat tangan, siapa yang setelah belajar hari ini jadi melihat tarian tradisional dengan cara yang berbeda?"
    \end{itemize}
\item \textbf{Tindak Lanjut:}
    \begin{itemize}
    \item \textbf{Instruksi Guru:} "Kerja bagus, para detektif! Hari ini kita sudah membongkar rahasia 'sistem gerak' dalam tubuh kita. Pertemuan berikutnya, kita akan menyelidiki bagaimana makhluk hidup lain bergerak dan apa yang bisa kita pelajari dari mereka. Siapa tahu ada kearifan lokal Minang yang terkait dengan gerak hewan? Kita akan lihat!"
    \item \textit{(Clue: Kaitkan secara eksplisit dengan pembelajaran hari ini untuk membangun alur narasi yang berkelanjutan. Ini membuat siswa merasa belajar IPA seperti mengikuti sebuah cerita bersambung).}
    \end{itemize}
\item Guru menutup pelajaran dengan doa dan salam.
\end{itemize}

\section{Asesmen (Penilaian)}

\begin{itemize}
\item \textbf{Asesmen Diagnostik (Awal):} Analisis lembar K-W-L. (Untuk mengetahui baseline siswa).
\item \textbf{Asesmen Formatif (Proses):} Observasi keaktifan diskusi (gotong royong) dan penilaian kelengkapan "\textbf{LKPD 01 - Jurnal Investigasi Dual-Lensa}".
\item \textbf{Asesmen Sumatif (Akhir Siklus):} Penilaian jawaban studi kasus (Tahap N) menggunakan rubrik.
\end{itemize}

\subsection{Rubrik Penilaian Jawaban Studi Kasus (Tahap N)}

\begin{longtable}{|p{3cm}|p{3cm}|p{3cm}|p{3cm}|p{3cm}|}
\hline
\textbf{Kriteria Penilaian} & \textbf{Skor 4 (Sangat Baik)} & \textbf{Skor 3 (Baik)} & \textbf{Skor 2 (Cukup)} & \textbf{Skor 1 (Kurang)} \\
\hline
\textbf{Ketepatan Konsep Ilmiah} & Menggunakan istilah ilmiah (otot, tulang, sendi, kontraksi, koordinasi) dengan sangat tepat dan relevan dengan kasus. & Menggunakan istilah ilmiah dengan tepat, namun kurang relevan. & Menggunakan istilah ilmiah namun ada beberapa kesalahan konsep. & Tidak menggunakan istilah ilmiah atau salah total. \\
\hline
\textbf{Keterkaitan dengan Etnosains} & Mampu menghubungkan secara logis dan eksplisit antara praktik budaya (teknik Tari Piring) dengan penjelasan ilmiahnya secara mendalam. & Mampu menghubungkan praktik budaya dengan penjelasan ilmiah, namun kurang mendalam. & Hanya menyebutkan praktik budaya tanpa menghubungkan dengan sains, atau sebaliknya. & Tidak ada keterkaitan antara sains dan budaya yang ditunjukkan. \\
\hline
\textbf{Kelogisan \& Struktur Argumen} & Penjelasan sangat logis, runtut, persuasif, dan mudah dipahami. & Penjelasan logis dan runtut, namun kurang persuasif. & Alur penjelasan kurang runtut atau sulit dipahami. & Penjelasan tidak logis dan tidak terstruktur. \\
\hline
\end{longtable}

\section{Daftar Pustaka Sumber Etnosains}

\begin{enumerate}
\item Amir, A. (2013). \textit{Tari Piring: Warisan Budaya Minangkabau}. Padang: Universitas Negeri Padang Press.
\item Navis, A.A. (1984). \textit{Alam Terkembang Jadi Guru: Adat dan Kebudayaan Minangkabau}. Jakarta: Grafiti Pers.
\item Kementerian Pendidikan dan Kebudayaan. (2017). \textit{Tari Piring Sumatera Barat}. Jakarta: Direktorat Warisan dan Diplomasi Budaya.
\item Hakimy, I. (1994). \textit{Rangkaian Mustika Adat Basandi Syarak di Minangkabau}. Bandung: Remaja Rosdakarya.
\end{enumerate}

\end{document}