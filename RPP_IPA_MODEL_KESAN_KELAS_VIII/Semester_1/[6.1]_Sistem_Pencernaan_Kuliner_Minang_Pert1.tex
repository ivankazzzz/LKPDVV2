\documentclass[a4paper,12pt]{article}
\usepackage[a4paper, margin=1.27cm]{geometry}
\usepackage[indonesian]{babel}
\usepackage[utf8]{inputenc}
\usepackage{tcolorbox}
\usepackage{array}
\usepackage{multirow}
\usepackage{setspace}
\usepackage{amssymb}
\usepackage{graphicx}
\usepackage{adjustbox}
\usepackage{enumitem}
\usepackage{longtable}
\usepackage{xcolor}
\usepackage{amsmath}
\usepackage{fancyhdr}
\usepackage{titlesec}

% Define colors (black and white theme)
\definecolor{darkgray}{RGB}{64, 64, 64}
\definecolor{lightgray}{RGB}{240, 240, 240}
\definecolor{mediumgray}{RGB}{128, 128, 128}

% Custom tcolorbox styles
\tcbset{
    mainbox/.style={
        colback=white,
        colframe=black,
        boxrule=1pt,
        arc=3pt,
        left=8pt,
        right=8pt,
        top=8pt,
        bottom=8pt
    },
    sectionbox/.style={
        colback=white,
        colframe=black,
        boxrule=1.5pt,
        arc=2pt,
        left=6pt,
        right=6pt,
        top=6pt,
        bottom=6pt
    }
}

\begin{document}

\begin{center}
{\Large\textbf{MODUL AJAR: Kearifan Kuliner Minang \& Sistem Pencernaan Manusia}}
\end{center}

\vspace{0.5cm}

\begin{tcolorbox}[mainbox]
\textbf{Nama Penyusun:} Irfan Ananda Ismail, S.Pd., M.Pd., Gr \\
\textbf{Institusi:} SMP \\
\textbf{Mata Pelajaran:} Ilmu Pengetahuan Alam (IPA) \\
\textbf{Tahun Ajaran:} 2025/2026 \\
\textbf{Semester:} Ganjil \\
\textbf{Jenjang Sekolah:} SMP \\
\textbf{Kelas/Fase:} VIII / D \\
\textbf{Alokasi waktu:} 3 x 40 Menit (1 Pertemuan)
\end{tcolorbox}

\section{DIMENSI PROFIL PELAJAR PANCASILA}
\textit{(Clue untuk Guru: Sebutkan dimensi ini secara eksplisit saat apersepsi agar siswa sadar tujuan non-akademis yang sedang mereka kembangkan).}

\begin{itemize}
\item \textbf{Berkebinekaan Global:} Mengenal dan menghargai budaya, khususnya menganalisis kearifan kuliner Minangkabau dalam proses pengolahan makanan dan menghubungkannya dengan proses pencernaan dalam tubuh manusia.
\item \textbf{Bernalar Kritis:} Ditempa secara intensif saat menganalisis informasi dari sumber sains dan budaya (Tahap S), menyintesiskan kedua perspektif (Tahap A), dan menyusun argumen berbasis bukti (Tahap N).
\item \textbf{Gotong Royong:} Kemampuan untuk bekerja sama secara kolaboratif dalam kelompok untuk merumuskan masalah (Tahap E), melakukan investigasi (Tahap S), dan membangun pemahaman bersama (Tahap A).
\item \textbf{Kreatif:} Menghasilkan argumen atau solusi orisinal yang terintegrasi pada tahap akhir pembelajaran (Tahap N).
\end{itemize}

\section{Sarana dan Prasarana}

\begin{itemize}
\item \textbf{Media:} LKPD 07 - Jurnal Investigasi Dual-Lensa, contoh makanan tradisional Minangkabau (rendang, gulai, asinan, dll), gambar/poster sistem pencernaan manusia, artikel tentang filosofi kuliner Minang, model torso sistem pencernaan.
\item \textbf{Alat:} Papan tulis/whiteboard, spidol, Proyektor \& Speaker, Kertas Plano atau Karton (1 per kelompok), sticky notes warna-warni, penggaris, kaca pembesar.
\item \textbf{Sumber Belajar:} Buku ajar IPA kelas VIII, Tautan video animasi sistem pencernaan (misal: bit.ly/animasi-sistem-pencernaan), tautan artikel/video kuliner tradisional Minang (misal: bit.ly/kuliner-tradisional-minang).
\end{itemize}

\section{Target Peserta didik}

\begin{itemize}
\item Peserta didik reguler kelas VIII (Fase D).
\end{itemize}

\section{Model Pembelajaran}

\begin{itemize}
\item Model Pembelajaran KESAN (Konektivitas Etnosains-Sains).
\end{itemize}

\section{Pemahaman Bermakna}
\textit{(Clue untuk Guru: Bacakan atau sampaikan narasi ini dengan intonasi yang menarik di akhir pembelajaran untuk mengikat semua pengalaman belajar siswa menjadi satu kesatuan yang bermakna).}

\begin{tcolorbox}[sectionbox]
"Ananda Semua, hari ini kita telah mengungkap rahasia di balik kearifan kuliner nenek moyang Minangkabau! Kita menemukan bahwa cara mereka mengolah makanan - dari memotong, menumbuk, merebus, hingga menambahkan rempah-rempah - ternyata memiliki dasar ilmiah yang kuat dalam membantu proses pencernaan. Setiap tahapan pengolahan makanan tradisional sebenarnya adalah 'pencernaan mekanis dan kimiawi' yang dilakukan di luar tubuh untuk memudahkan kerja sistem pencernaan kita. Dengan ini, kita sadar bahwa kearifan lokal bukan hanya soal rasa, tapi juga tentang kesehatan dan efisiensi biologis yang telah dipahami secara intuitif oleh nenek moyang kita."
\end{tcolorbox}

\section{PERTEMUAN PERTAMA: Sistem Pencernaan dan Kearifan Kuliner Minangkabau}

\subsection{Capaian Pembelajaran (Fase D)}
Pada akhir Fase D, murid memiliki kemampuan menganalisis sistem organisasi kehidupan, fungsi, serta kelainan atau gangguan yang muncul pada sistem organ makhluk hidup.

\subsection{Tujuan Pembelajaran (TP) Pertemuan 1:}
\textit{(Clue untuk Guru: Tujuan ini adalah kompas Anda. Pastikan setiap tahapan KESAN yang Anda lalui berkontribusi pada pencapaian tujuan-tujuan ini).}

Melalui model pembelajaran KESAN, peserta didik mampu:
\begin{itemize}
\item Menghubungkan fenomena kearifan lokal Minangkabau (teknik pengolahan makanan tradisional) dengan konteks sistem pencernaan manusia dalam sains modern. (Sintaks K)
\item Merumuskan pertanyaan investigatif mengenai proses pencernaan dan perbandingan teknik pengolahan makanan tradisional dengan proses pencernaan dalam tubuh. (Sintaks E)
\item Mengumpulkan informasi mengenai sistem pencernaan manusia dari sumber sains serta teknik pengolahan makanan dari sumber kultural. (Sintaks S)
\item Menganalisis dan menyintesiskan hubungan sebab-akibat antara teknik pengolahan makanan tradisional Minangkabau dengan proses pencernaan mekanis dan kimiawi dalam tubuh. (Sintaks A)
\item Menyusun sebuah penjelasan analitis yang logis mengenai bagaimana kearifan kuliner Minangkabau dapat menjelaskan dan mendukung efisiensi sistem pencernaan manusia. (Sintaks N)
\end{itemize}

\subsection{Pertanyaan Pemantik}
\textit{(Clue untuk Guru: Ajukan dua pertanyaan ini secara berurutan, berikan jeda agar siswa berpikir. Jangan langsung minta jawaban, biarkan pertanyaan ini menggantung untuk memicu rasa ingin tahu).}

\begin{itemize}
\item "Pernahkah kalian memperhatikan mengapa nenek atau ibu kalian selalu memotong daging kecil-kecil untuk rendang, menumbuk bumbu halus-halus, dan merebus dalam waktu lama? Padahal bisa saja langsung dimasak utuh. Apa tujuannya sebenarnya?"
\item "Dalam budaya Minang, ada filosofi 'Masak nan indak salah, makan nan indak rugi' - memasak yang tidak salah, makan yang tidak rugi. Dari sudut pandang biologi, bagaimana sebenarnya cara pengolahan makanan tradisional ini membantu tubuh kita mencerna makanan dengan lebih efisien?"
\end{itemize}

\section{Langkah-langkah Kegiatan Pembelajaran:}

\subsection{Kegiatan Pembuka (15 Menit)}
\begin{itemize}
\item Guru membuka pelajaran dengan salam, doa, dan memeriksa kehadiran.
\item \textbf{Asesmen Diagnostik Awal:} Guru membagikan lembar K-W-L.
    \begin{itemize}
    \item \textbf{Instruksi Guru:} "Ananda, sebelum kita mulai menjelajahi dunia pencernaan, tolong isi dua kolom pertama di lembar ini. Di kolom \textbf{K (Tahu)}, tulis apa saja yang sudah kalian ketahui tentang sistem pencernaan ATAU cara memasak tradisional Minang. Di kolom \textbf{W (Ingin Tahu)}, tulis apa yang membuat kalian penasaran tentang topik ini." \textit{(Clue: Ini membantu Anda memetakan pengetahuan awal dan minat siswa secara cepat).}
    \end{itemize}
\item \textbf{Apersepsi:}
    \begin{itemize}
    \item \textbf{Instruksi Guru:} "Hari ini kita akan menjadi ahli kuliner tradisional dan ahli biologi sekaligus. Kita akan menyelidiki bagaimana teknik memasak Minangkabau berhubungan dengan sistem pencernaan dalam tubuh kita. Dalam investigasi ini, kita akan melatih kemampuan \textbf{Bernalar Kritis} kita, menghargai budaya lewat \textbf{Berkebinekaan Global}, dan bekerja sama dalam semangat \textbf{Gotong Royong}. Siap menjadi peneliti kuliner-biologi?"
    \end{itemize}
\end{itemize}

\subsection{Kegiatan Inti (90 Menit) - SINTAKS MODEL KESAN}

\subsubsection{Tahap 1: (K) Kaitkan Konteks Kultural (15 menit)}
\begin{itemize}
\item \textbf{Aktivitas Guru:}
    \begin{itemize}
    \item Menampilkan contoh makanan tradisional Minangkabau (rendang, gulai, asinan) dan menunjukkan tahapan pengolahannya.
    \item Mengajukan Pertanyaan Pemantik yang sudah disiapkan di atas.
    \item \textit{(Clue: Tujuan tahap ini adalah memvalidasi pengetahuan siswa dan memantik rasa heran, bukan mencari jawaban benar. Sediakan spidol dan papan tulis/whiteboard. Saat siswa menjawab, tuliskan semua ide mereka, bahkan yang keliru sekalipun, dengan judul "\textbf{PENGETAHUAN AWAL KITA}". Ini menunjukkan bahwa semua pemikiran dihargai).}
    \end{itemize}
\item \textbf{Aktivitas Siswa:} Mengamati contoh makanan dan tahapan pengolahannya, mendengarkan pertanyaan, lalu secara sukarela berbagi pengalaman atau pengetahuan awal tentang cara memasak tradisional atau pencernaan. Menuliskan minimal satu pertanyaan atau pengalaman di sticky notes dan menempelkannya di '\textbf{Papan Penasaran}'.
\end{itemize}

\subsubsection{Tahap 2: (E) Eksplorasi Enigma (15 menit)}
\begin{itemize}
\item \textbf{Aktivitas Guru:} Membentuk siswa menjadi kelompok (3-4 orang).
    \begin{itemize}
    \item \textbf{Instruksi Guru:} "Baik, rasa penasaran kalian luar biasa! Sekarang, tugas kita sebagai peneliti adalah mengubah rasa penasaran ini menjadi misi yang jelas. Dalam kelompok, diskusikan dan rumuskan minimal 3 pertanyaan kunci yang akan kita selidiki hari ini. Tuliskan dalam bentuk '\textbf{Misi Penelitian Tim [Nama Kelompok]}' di kertas plano yang Bapak/Ibu berikan."
    \item \textit{(Clue: Arahkan diskusi siswa agar pertanyaannya mencakup aspek 'bagaimana proses pencernaan dalam tubuh' dan 'bagaimana teknik pengolahan makanan tradisional'. Jika kelompok kesulitan, berikan pancingan: "Kira-kira, apa dulu yang perlu kita tahu? Proses pencernaan dalam tubuh atau cara pengolahan makanan di luar tubuh?").}
    \end{itemize}
\item \textbf{Aktivitas Siswa:} Berdiskusi dalam tim untuk merumuskan misi penelitian (daftar pertanyaan kunci) di kertas plano. \textit{(Contoh misi yang diharapkan: 1. Bagaimana proses pencernaan makanan dalam tubuh manusia? 2. Apa saja teknik pengolahan makanan dalam kuliner Minangkabau? 3. Bagaimana hubungan antara pengolahan makanan dengan pencernaan?)}
\end{itemize}

\subsubsection{Tahap 3: (S) Selidiki secara Sintetis (25 menit)}
\begin{itemize}
\item \textbf{Aktivitas Guru:} Membagikan "\textbf{LKPD 07 - Jurnal Investigasi Dual-Lensa}".
    \begin{itemize}
    \item \textbf{Instruksi Guru:} "Setiap tim akan melakukan investigasi dari dua lensa. Gunakan HP atau sumber yang disediakan untuk mencari jawabannya. Bagilah tugas dalam tim!"
    \item \textbf{Lensa Sains:} Buka link video bit.ly/animasi-sistem-pencernaan untuk memahami proses pencernaan mekanis dan kimiawi dari mulut hingga usus besar. Pelajari juga fungsi enzim dan organ pencernaan.
    \item \textbf{Lensa Etnosains/Kultural:} Buka link artikel bit.ly/kuliner-tradisional-minang untuk memahami teknik pengolahan makanan Minangkabau seperti memotong, menumbuk, merebus, fermentasi, dan penggunaan rempah-rempah.
    \item \textit{(Clue: Pastikan sumber belajar sudah disiapkan dan link bisa diakses. Berkelilinglah untuk memastikan setiap kelompok membagi tugas dan tidak hanya fokus pada satu lensa saja).}
    \end{itemize}
\item \textbf{Aktivitas Siswa:} Dalam kelompok, siswa berbagi tugas mencari informasi dari sumber yang diberikan dan mencatat temuan kunci di dua kolom terpisah pada "\textbf{LKPD 07 - Jurnal Investigasi Dual-Lensa}".
\end{itemize}

\subsubsection{Tahap 4: (A) Asimilasi Analitis (20 menit)}
\begin{itemize}
\item \textbf{Aktivitas Guru:} Memfasilitasi diskusi untuk menjembatani kedua lensa.
    \begin{itemize}
    \item \textbf{Pertanyaan Pancingan Kunci untuk Guru:}
        \begin{itemize}
        \item "Oke, dari Lensa Sains kita tahu pencernaan dimulai dari mulut dengan mengunyah (mekanis) dan enzim (kimiawi). Dari Lensa Etnosains, kita tahu makanan Minang dipotong kecil dan ditumbuk halus. Nah, coba hubungkan! Apa kesamaannya?"
        \item "Dari Lensa Sains, kita tahu lambung menggunakan asam untuk mencerna protein. Dari Lensa Budaya, masakan Minang sering menggunakan asam jawa atau belimbing. Mengapa teknik ini membantu pencernaan?"
        \end{itemize}
    \item \textit{(Clue: Fokuskan untuk membuat siswa 'menemukan' hubungannya sendiri, bukan diberitahu. Gunakan kata "menurut kalian", "kira-kira kenapa", "ada yang punya ide?").}
    \end{itemize}
\item \textbf{Aktivitas Siswa:} Berdiskusi intensif untuk menghubungkan temuan sains dan budaya. Menuliskan kesimpulan terpadu (sintesis) mereka di kertas plano.
\end{itemize}

\subsubsection{Tahap 5: (N) Nyatakan Pemahaman (15 menit)}
\begin{itemize}
\item \textbf{Aktivitas Guru:} Memberikan studi kasus individual atau per kelompok.
    \begin{itemize}
    \item \textbf{Instruksi Guru (tuliskan di papan tulis):}
    
    "\textbf{STUDI KASUS UNTUK PENELITI SAINS-BUDAYA:}
    
    Seorang ahli gizi dari Jakarta datang ke Padang untuk mempelajari mengapa orang Minang jarang mengalami gangguan pencernaan meskipun makanannya pedas dan berlemak. Dia menemukan bahwa teknik memasak tradisional Minang memiliki keunikan: daging dipotong kecil, bumbu ditumbuk halus, dimasak lama dengan api kecil, dan selalu menggunakan rempah-rempah tertentu. Dia ingin memahami hubungan antara teknik memasak ini dengan kesehatan pencernaan.
    
    Tugasmu: Tuliskan sebuah penjelasan ilmiah singkat (4-6 kalimat) di buku latihanmu untuk membantu ahli gizi tersebut memahami bagaimana teknik memasak tradisional Minangkabau mendukung proses pencernaan yang sehat. Gunakan pengetahuan gabungan dari sains dan kearifan kuliner yang baru saja kamu pelajari."
    \end{itemize}
\item \textbf{Aktivitas Siswa:} Menyusun argumen tertulis untuk menjawab studi kasus yang diberikan, menggunakan bukti dari kedua lensa.
\end{itemize}

\subsection{Kegiatan Penutup (15 Menit)}
\begin{itemize}
\item \textbf{Presentasi \& Penguatan:} Guru meminta 2-3 siswa secara acak untuk membacakan penjelasan mereka. Guru memberikan pujian dan penguatan positif, menekankan betapa hebatnya kemampuan mereka mengintegrasikan sains dan budaya.
\item \textbf{Refleksi:}
    \begin{itemize}
    \item \textbf{Instruksi Guru:} "Sekarang, kembali ke lembar K-W-L kalian. Lengkapi kolom terakhir, \textbf{L (Learned)}, dengan hal-hal baru yang paling mengagumkan yang kalian pelajari hari ini."
    \item \textbf{Instruksi Guru:} "Angkat tangan, siapa yang setelah belajar hari ini jadi lebih menghargai kearifan nenek moyang kita dalam memasak?"
    \end{itemize}
\item \textbf{Tindak Lanjut:}
    \begin{itemize}
    \item \textbf{Instruksi Guru:} "Luar biasa, para peneliti! Hari ini kita sudah mengungkap bagaimana kearifan kuliner mendukung kesehatan pencernaan. Pertemuan berikutnya, kita akan mendalami gangguan sistem pencernaan dan cara pencegahannya. Siapa tahu ada kearifan pengobatan tradisional Minang yang bisa mengajarkan kita tentang menjaga kesehatan pencernaan? Kita akan lihat!"
    \item \textit{(Clue: Buat transisi yang menarik ke pertemuan berikutnya sambil mempertahankan semangat investigasi yang sudah terbangun).}
    \end{itemize}
\item Guru menutup pelajaran dengan doa dan salam.
\end{itemize}

\section{Asesmen (Penilaian)}

\begin{itemize}
\item \textbf{Asesmen Diagnostik (Awal):} Analisis lembar K-W-L. (Untuk mengetahui baseline siswa).
\item \textbf{Asesmen Formatif (Proses):} Observasi keaktifan diskusi (gotong royong) dan penilaian kelengkapan "\textbf{LKPD 07 - Jurnal Investigasi Dual-Lensa}".
\item \textbf{Asesmen Sumatif (Akhir Siklus):} Penilaian jawaban studi kasus (Tahap N) menggunakan rubrik.
\end{itemize}

\subsection{Rubrik Penilaian Jawaban Studi Kasus (Tahap N)}

\begin{longtable}{|p{3cm}|p{3cm}|p{3cm}|p{3cm}|p{3cm}|}
\hline
\textbf{Kriteria Penilaian} & \textbf{Skor 4 (Sangat Baik)} & \textbf{Skor 3 (Baik)} & \textbf{Skor 2 (Cukup)} & \textbf{Skor 1 (Kurang)} \\
\hline
\textbf{Ketepatan Konsep Ilmiah} & Menggunakan istilah ilmiah (pencernaan mekanis, kimiawi, enzim, organ pencernaan) dengan sangat tepat dan relevan dengan kasus. & Menggunakan istilah ilmiah dengan tepat, namun kurang relevan. & Menggunakan istilah ilmiah namun ada beberapa kesalahan konsep. & Tidak menggunakan istilah ilmiah atau salah total. \\
\hline
\textbf{Keterkaitan dengan Etnosains} & Mampu menghubungkan secara logis dan eksplisit antara teknik kuliner tradisional dengan proses pencernaan secara mendalam. & Mampu menghubungkan teknik kuliner dengan pencernaan, namun kurang mendalam. & Hanya menyebutkan teknik kuliner tanpa menghubungkan dengan pencernaan, atau sebaliknya. & Tidak ada keterkaitan antara sains dan budaya yang ditunjukkan. \\
\hline
\textbf{Kelogisan \& Struktur Argumen} & Penjelasan sangat logis, runtut, mudah dipahami, dan memberikan insight baru tentang kesehatan. & Penjelasan logis dan runtut, namun kurang memberikan insight kesehatan. & Alur penjelasan kurang runtut atau sulit dipahami. & Penjelasan tidak logis dan tidak terstruktur. \\
\hline
\end{longtable}

\section{Daftar Pustaka Sumber Etnosains}

\begin{enumerate}
\item Navis, A.A. (1984). \textit{Alam Terkembang Jadi Guru: Adat dan Kebudayaan Minangkabau}. Jakarta: Grafiti Pers.
\item Syukur, C. (2015). \textit{Kuliner Tradisional Minangkabau: Filosofi dan Kearifan Lokal}. Padang: Andalas University Press.
\item Kementerian Kesehatan RI. (2019). \textit{Pedoman Gizi Seimbang}. Jakarta: Direktorat Jenderal Kesehatan Masyarakat.
\item Dt. Rajo Penghulu, I. (1994). \textit{Pegangan Penghulu, Bundo Kanduang dan Pidato Alua Pasambahan Adat Minangkabau}. Bukittinggi: Pustaka Indonesia.
\end{enumerate}

\end{document}