\documentclass[a4paper,12pt]{article}
\usepackage[a4paper, margin=1.27cm]{geometry}
\usepackage[indonesian]{babel}
\usepackage[utf8]{inputenc}
\usepackage{tcolorbox}
\usepackage{array}
\usepackage{multirow}
\usepackage{setspace}
\usepackage{amssymb}
\usepackage{graphicx}
\usepackage{adjustbox}
\usepackage{enumitem}
\usepackage{longtable}
\usepackage{xcolor}
\usepackage{amsmath}
\usepackage{fancyhdr}
\usepackage{titlesec}

% Define colors (black and white theme)
\definecolor{darkgray}{RGB}{64, 64, 64}
\definecolor{lightgray}{RGB}{240, 240, 240}
\definecolor{mediumgray}{RGB}{128, 128, 128}

% Custom tcolorbox styles
\tcbset{
    mainbox/.style={
        colback=white,
        colframe=black,
        boxrule=1pt,
        arc=3pt,
        left=8pt,
        right=8pt,
        top=8pt,
        bottom=8pt
    },
    sectionbox/.style={
        colback=white,
        colframe=black,
        boxrule=1.5pt,
        arc=2pt,
        left=6pt,
        right=6pt,
        top=6pt,
        bottom=6pt
    }
}

\begin{document}

\begin{center}
{\Large\textbf{MODUL AJAR: Katrol Sumur Tradisional \& Prinsip Pesawat Sederhana}}
\end{center}

\vspace{0.5cm}

\begin{tcolorbox}[mainbox]
\textbf{Nama Penyusun:} Irfan Ananda Ismail, S.Pd., M.Pd., Gr \\
\textbf{Institusi:} SMP \\
\textbf{Mata Pelajaran:} Ilmu Pengetahuan Alam (IPA) \\
\textbf{Tahun Ajaran:} 2025/2026 \\
\textbf{Semester:} Ganjil \\
\textbf{Jenjang Sekolah:} SMP \\
\textbf{Kelas/Fase:} VIII / D \\
\textbf{Alokasi waktu:} 3 x 40 Menit (1 Pertemuan)
\end{tcolorbox}

\section{DIMENSI PROFIL PELAJAR PANCASILA}
\textit{(Clue untuk Guru: Sebutkan dimensi ini secara eksplisit saat apersepsi agar siswa sadar tujuan non-akademis yang sedang mereka kembangkan).}

\begin{itemize}
\item \textbf{Berkebinekaan Global:} Mengenal dan menghargai budaya, khususnya menganalisis kearifan lokal Minangkabau dalam teknologi tradisional (katrol sumur dan sistem pengangkatan air), lalu menghubungkannya dengan konsep sains universal tentang usaha dan pesawat sederhana.
\item \textbf{Bernalar Kritis:} Ditempa secara intensif saat menganalisis informasi dari sumber sains dan budaya (Tahap S), menyintesiskan kedua perspektif (Tahap A), dan menyusun argumen berbasis bukti (Tahap N).
\item \textbf{Gotong Royong:} Kemampuan untuk bekerja sama secara kolaboratif dalam kelompok untuk merumuskan masalah (Tahap E), melakukan investigasi (Tahap S), dan membangun pemahaman bersama (Tahap A).
\item \textbf{Kreatif:} Menghasilkan argumen atau solusi orisinal yang terintegrasi pada tahap akhir pembelajaran (Tahap N).
\end{itemize}

\section{Sarana dan Prasarana}

\begin{itemize}
\item \textbf{Media:} LKPD 05 - Jurnal Investigasi Dual-Lensa, video singkat katrol sumur tradisional Minangkabau (durasi 3-4 menit), gambar/diagram pesawat sederhana, artikel tentang sistem pengairan tradisional, model katrol sederhana atau tali dan roda.
\item \textbf{Alat:} Papan tulis/whiteboard, spidol, Proyektor \& Speaker, Kertas Plano atau Karton (1 per kelompok), sticky notes warna-warni, tali, beban kecil untuk demonstrasi katrol.
\item \textbf{Sumber Belajar:} Buku ajar IPA kelas VIII, Tautan video animasi pesawat sederhana (misal: bit.ly/animasi-pesawat-sederhana), tautan artikel/video katrol tradisional (misal: bit.ly/katrol-sumur-minang).
\end{itemize}

\section{Target Peserta didik}

\begin{itemize}
\item Peserta didik reguler kelas VIII (Fase D).
\end{itemize}

\section{Model Pembelajaran}

\begin{itemize}
\item Model Pembelajaran KESAN (Konektivitas Etnosains-Sains).
\end{itemize}

\section{Pemahaman Bermakna}
\textit{(Clue untuk Guru: Bacakan atau sampaikan narasi ini dengan intonasi yang menarik di akhir pembelajaran untuk mengikat semua pengalaman belajar siswa menjadi satu kesatuan yang bermakna).}

\begin{tcolorbox}[sectionbox]
"Ananda Semua, hari ini kita telah mengungkap keajaiban di balik katrol sumur tradisional yang masih bisa kita temukan di beberapa rumah gadang dan surau tua di Minangkabau! Kita menemukan bahwa sistem sederhana ini sebenarnya menerapkan prinsip pesawat sederhana yang sangat cerdas - dengan katrol, nenek moyang kita bisa mengangkat ember air yang berat dengan usaha yang jauh lebih ringan. Setiap putaran katrol, setiap tarikan tali, semuanya mengikuti prinsip-prinsip fisika yang memungkinkan manusia mengatasi keterbatasan kekuatan fisiknya. Dengan ini, kita sadar bahwa teknologi tradisional adalah solusi engineering yang telah teruji waktu."
\end{tcolorbox}

\section{PERTEMUAN PERTAMA: Usaha dan Pesawat Sederhana dalam Teknologi Tradisional}

\subsection{Capaian Pembelajaran (Fase D)}
Pada akhir Fase D, murid memiliki kemampuan menerapkan konsep gaya, gerak, dan energi dalam kehidupan sehari-hari.

\subsection{Tujuan Pembelajaran (TP) Pertemuan 1:}
\textit{(Clue untuk Guru: Tujuan ini adalah kompas Anda. Pastikan setiap tahapan KESAN yang Anda lalui berkontribusi pada pencapaian tujuan-tujuan ini).}

Melalui model pembelajaran KESAN, peserta didik mampu:
\begin{itemize}
\item Menghubungkan fenomena kearifan lokal Minangkabau (teknologi katrol sumur tradisional) dengan konteks usaha dan pesawat sederhana. (Sintaks K)
\item Merumuskan pertanyaan investigatif mengenai cara kerja katrol dan hubungannya dengan konsep usaha dan keuntungan mekanis. (Sintaks E)
\item Mengumpulkan informasi mengenai prinsip pesawat sederhana dari sumber ilmiah serta teknologi pengairan tradisional dari sumber kultural. (Sintaks S)
\item Menganalisis dan menyintesiskan hubungan sebab-akibat antara prinsip katrol dengan efisiensi kerja dalam teknologi tradisional. (Sintaks A)
\item Menyusun sebuah penjelasan analitis yang logis mengenai bagaimana teknologi tradisional Minangkabau mengoptimalkan usaha melalui pesawat sederhana. (Sintaks N)
\end{itemize}

\subsection{Pertanyaan Pemantik}
\textit{(Clue untuk Guru: Ajukan dua pertanyaan ini secara berurutan, berikan jeda agar siswa berpikir. Jangan langsung minta jawaban, biarkan pertanyaan ini menggantung untuk memicu rasa ingin tahu).}

\begin{itemize}
\item "Pernahkah kalian melihat sumur tua dengan katrol di rumah gadang atau surau? Mengapa nenek moyang kita repot-repot memasang katrol, padahal bisa saja langsung menarik ember dengan tali biasa?"
\item "Coba bayangkan kalian harus mengangkat ember berisi air seberat 10 kg dari sumur sedalam 5 meter. Mana yang lebih mudah: menarik langsung atau menggunakan katrol? Dari sudut pandang sains, apa yang membuat katrol bisa 'meringankan' pekerjaan kita?"
\end{itemize}

\section{Langkah-langkah Kegiatan Pembelajaran:}

\subsection{Kegiatan Pembuka (15 Menit)}
\begin{itemize}
\item Guru membuka pelajaran dengan salam, doa, dan memeriksa kehadiran.
\item \textbf{Asesmen Diagnostik Awal:} Guru membagikan lembar K-W-L.
    \begin{itemize}
    \item \textbf{Instruksi Guru:} "Ananda, kita telah menyelesaikan eksplorasi tentang gerak benda dan energi. Sekarang kita akan menjelajahi topik baru: usaha dan pesawat sederhana! Di lembar K-W-L ini, tulis di kolom \textbf{K (Tahu)} apa yang sudah kalian ketahui tentang alat-alat yang memudahkan pekerjaan ATAU tentang teknologi tradisional seperti katrol. Di kolom \textbf{W (Ingin Tahu)}, tulis apa yang membuat kalian penasaran tentang topik hari ini." \textit{(Clue: Ini membantu Anda melihat pemahaman awal siswa tentang konsep 'memudahkan pekerjaan').}
    \end{itemize}
\item \textbf{Apersepsi:}
    \begin{itemize}
    \item \textbf{Instruksi Guru:} "Setelah mengungkap rahasia gerak dan energi dalam teknologi tradisional, sekarang kita akan menyelidiki bagaimana nenek moyang kita menciptakan alat-alat yang bisa 'meringankan' pekerjaan berat. Hari ini, misi kita adalah memahami bagaimana katrol sumur tradisional menerapkan prinsip pesawat sederhana. Siap menjadi insinyur yang menghargai kecerdasan tradisional?"
    \end{itemize}
\end{itemize}

\subsection{Kegiatan Inti (90 Menit) - SINTAKS MODEL KESAN}

\subsubsection{Tahap 1: (K) Kaitkan Konteks Kultural (15 menit)}
\begin{itemize}
\item \textbf{Aktivitas Guru:}
    \begin{itemize}
    \item Menampilkan video singkat (3-4 menit) katrol sumur tradisional yang masih digunakan, menunjukkan proses pengambilan air yang efisien.
    \item Mengajukan Pertanyaan Pemantik yang sudah disiapkan di atas.
    \item \textit{(Clue: Fokuskan pada keingintahuan tentang mengapa teknologi sederhana ini begitu efektif. Saat siswa menjawab, tuliskan semua ide mereka di papan tulis dengan judul "\textbf{KEAJAIBAN ALAT SEDERHANA}". Validasi setiap kontribusi).}
    \end{itemize}
\item \textbf{Aktivitas Siswa:} Mengamati video, mendengarkan pertanyaan, lalu secara sukarela berbagi pengalaman atau dugaan tentang keunggulan katrol dibanding menarik langsung. Menuliskan minimal satu pertanyaan atau pengamatan di sticky notes dan menempelkannya di '\textbf{Papan Penasaran}'.
\end{itemize}

\subsubsection{Tahap 2: (E) Eksplorasi Enigma (15 menit)}
\begin{itemize}
\item \textbf{Aktivitas Guru:} Membentuk siswa menjadi kelompok (3-4 orang).
    \begin{itemize}
    \item \textbf{Instruksi Guru:} "Keingintahuan kalian tentang alat-alat sederhana yang powerful sangat menginspirasi! Sekarang, sebagai tim insinyur tradisional, tugas kalian adalah mengubah keingintahuan ini menjadi misi investigasi yang sistematis. Dalam kelompok, diskusikan dan rumuskan minimal 3 pertanyaan kunci yang akan kita selidiki hari ini. Tuliskan dalam bentuk '\textbf{Misi Investigasi Tim [Nama Kelompok]}' di kertas plano."
    \item \textit{(Clue: Arahkan diskusi siswa agar pertanyaannya mencakup aspek 'bagaimana katrol bekerja' dan 'mengapa bisa meringankan pekerjaan'. Jika kelompok kesulitan, berikan pancingan: "Kira-kira, apa yang terjadi dengan gaya yang kita berikan saat menggunakan katrol? Dan bagaimana hal itu mempengaruhi usaha yang kita keluarkan?").}
    \end{itemize}
\item \textbf{Aktivitas Siswa:} Berdiskusi dalam tim untuk merumuskan misi investigasi (daftar pertanyaan kunci) di kertas plano. \textit{(Contoh misi yang diharapkan: 1. Bagaimana katrol bisa meringankan pekerjaan? 2. Apa hubungan antara gaya, jarak, dan usaha pada katrol? 3. Mengapa teknologi sederhana ini begitu efektif?)}
\end{itemize}

\subsubsection{Tahap 3: (S) Selidiki secara Sintetis (25 menit)}
\begin{itemize}
\item \textbf{Aktivitas Guru:} Membagikan "\textbf{LKPD 05 - Jurnal Investigasi Dual-Lensa}".
    \begin{itemize}
    \item \textbf{Instruksi Guru:} "Setiap tim akan melakukan investigasi dari dua lensa. Gunakan sumber yang disediakan untuk mencari jawabannya. Bagilah tugas dalam tim!"
    \item \textbf{Lensa Sains:} Buka link video bit.ly/animasi-pesawat-sederhana untuk memahami konsep usaha (W = F × s), keuntungan mekanis, dan jenis-jenis pesawat sederhana termasuk katrol.
    \item \textbf{Lensa Etnosains/Kultural:} Buka link artikel bit.ly/katrol-sumur-minang untuk memahami sejarah, konstruksi, dan penggunaan katrol dalam sistem pengairan tradisional Minangkabau.
    \item \textit{(Clue: Pastikan sumber belajar sudah disiapkan dan dapat diakses. Berkelilinglah untuk memastikan setiap kelompok mengeksplorasi kedua lensa dengan fokus pada prinsip kerja).}
    \end{itemize}
\item \textbf{Aktivitas Siswa:} Dalam kelompok, siswa berbagi tugas mencari informasi dari sumber yang diberikan dan mencatat temuan kunci di dua kolom terpisah pada "\textbf{LKPD 05 - Jurnal Investigasi Dual-Lensa}".
\end{itemize}

\subsubsection{Tahap 4: (A) Asimilasi Analitis (20 menit)}
\begin{itemize}
\item \textbf{Aktivitas Guru:} Memfasilitasi diskusi untuk menjembatani kedua lensa.
    \begin{itemize}
    \item \textbf{Pertanyaan Pancingan Kunci untuk Guru:}
        \begin{itemize}
        \item "Dari Lensa Sains kita tahu usaha = gaya × jarak. Dari Lensa Etnosains, kita tahu katrol memungkinkan kita menarik ke bawah untuk mengangkat beban ke atas. Nah, coba hubungkan! Bagaimana katrol mengubah arah gaya dan mempengaruhi usaha yang kita keluarkan?"
        \item "Dari Lensa Sains, kita tahu keuntungan mekanis = beban/kuasa. Dari Lensa Budaya, kita tahu katrol membuat pekerjaan terasa lebih ringan. Apa hubungannya konsep keuntungan mekanis dengan pengalaman 'meringankan' pekerjaan dalam teknologi tradisional?"
        \end{itemize}
    \item \textit{(Clue: Bantu siswa melihat bahwa katrol tidak mengurangi usaha total, tapi mengubah cara kita mengeluarkan gaya. Gunakan demonstrasi sederhana dengan tali dan beban jika perlu).}
    \end{itemize}
\item \textbf{Aktivitas Siswa:} Berdiskusi intensif untuk menghubungkan temuan sains dan budaya. Menuliskan kesimpulan terpadu (sintesis) mereka di kertas plano.
\end{itemize}

\subsubsection{Tahap 5: (N) Nyatakan Pemahaman (15 menit)}
\begin{itemize}
\item \textbf{Aktivitas Guru:} Memberikan studi kasus individual atau per kelompok.
    \begin{itemize}
    \item \textbf{Instruksi Guru (tuliskan di papan tulis):}
    
    "\textbf{STUDI KASUS UNTUK INSINYUR TRADISIONAL SAINS-BUDAYA:}
    
    Sebuah komunitas petani organik ingin membangun sistem irigasi tradisional yang ramah lingkungan untuk mengangkat air dari sungai ke sawah mereka yang berada 3 meter lebih tinggi. Mereka meminta bantuanmu sebagai konsultan yang memahami baik fisika maupun teknologi tradisional.
    
    Tugasmu: Tuliskan sebuah rekomendasi singkat (3-5 kalimat) di buku latihanmu tentang bagaimana menggunakan prinsip katrol dan pesawat sederhana untuk memecahkan masalah mereka. Jelaskan mengapa solusi berbasis teknologi tradisional bisa lebih efisien dan berkelanjutan dibanding teknologi modern."
    \end{itemize}
\item \textbf{Aktivitas Siswa:} Menyusun argumen tertulis untuk menjawab studi kasus yang diberikan, menggunakan bukti dari kedua lensa.
\end{itemize}

\subsection{Kegiatan Penutup (15 Menit)}
\begin{itemize}
\item \textbf{Presentasi \& Penguatan:} Guru meminta 2-3 siswa secara acak untuk membacakan rekomendasi mereka. Guru memberikan pujian dan penguatan positif, menekankan betapa hebatnya kemampuan mereka mengaplikasikan prinsip fisika pada solusi berkelanjutan.
\item \textbf{Refleksi:}
    \begin{itemize}
    \item \textbf{Instruksi Guru:} "Sekarang, kembali ke lembar K-W-L kalian. Lengkapi kolom terakhir, \textbf{L (Learned)}, dengan hal-hal baru yang paling mengagumkan yang kalian pelajari hari ini."
    \item \textbf{Instruksi Guru:} "Angkat tangan, siapa yang setelah belajar hari ini jadi lebih menghargai kecerdasan nenek moyang kita dalam menciptakan solusi engineering yang berkelanjutan?"
    \end{itemize}
\item \textbf{Tindak Lanjut:}
    \begin{itemize}
    \item \textbf{Instruksi Guru:} "Bravo, para insinyur tradisional! Hari ini kita sudah mengungkap bagaimana prinsip pesawat sederhana diterapkan dalam teknologi tradisional. Pertemuan berikutnya, kita akan mengeksplorasi pesawat sederhana lainnya yang ada dalam kearifan lokal Minangkabau. Siapa tahu ada alat tradisional lain yang mengajarkan kita tentang tuas, bidang miring, atau roda? Kita akan lihat!"
    \item \textit{(Clue: Buat transisi yang menarik ke pertemuan berikutnya sambil mempertahankan semangat eksplorasi teknologi berkelanjutan).}
    \end{itemize}
\item Guru menutup pelajaran dengan doa dan salam.
\end{itemize}

\section{Asesmen (Penilaian)}

\begin{itemize}
\item \textbf{Asesmen Diagnostik (Awal):} Analisis lembar K-W-L. (Untuk mengetahui pemahaman awal tentang alat yang memudahkan pekerjaan).
\item \textbf{Asesmen Formatif (Proses):} Observasi keaktifan diskusi (gotong royong) dan penilaian kelengkapan "\textbf{LKPD 05 - Jurnal Investigasi Dual-Lensa}".
\item \textbf{Asesmen Sumatif (Akhir Siklus):} Penilaian jawaban studi kasus (Tahap N) menggunakan rubrik.
\end{itemize}

\subsection{Rubrik Penilaian Jawaban Studi Kasus (Tahap N)}

\begin{longtable}{|p{3cm}|p{3cm}|p{3cm}|p{3cm}|p{3cm}|}
\hline
\textbf{Kriteria Penilaian} & \textbf{Skor 4 (Sangat Baik)} & \textbf{Skor 3 (Baik)} & \textbf{Skor 2 (Cukup)} & \textbf{Skor 1 (Kurang)} \\
\hline
\textbf{Ketepatan Konsep Ilmiah} & Menggunakan konsep usaha, gaya, keuntungan mekanis, dan pesawat sederhana dengan sangat tepat dan relevan untuk solusi irigasi. & Menggunakan konsep pesawat sederhana dengan tepat, namun kurang relevan dengan konteks. & Menggunakan konsep fisika namun ada beberapa kesalahan dalam penerapan prinsip pesawat sederhana. & Tidak menggunakan konsep pesawat sederhana atau salah total. \\
\hline
\textbf{Keterkaitan dengan Etnosains} & Mampu menghubungkan secara logis dan eksplisit antara prinsip pesawat sederhana dengan teknologi tradisional secara mendalam dan akurat. & Mampu menghubungkan prinsip fisika dengan teknologi tradisional, namun kurang mendalam. & Hanya menyebutkan teknologi tradisional tanpa menghubungkan dengan prinsip fisika, atau sebaliknya. & Tidak ada keterkaitan antara sains dan teknologi tradisional yang ditunjukkan. \\
\hline
\textbf{Kelogisan \& Struktur Argumen} & Rekomendasi sangat logis, runtut, berkelanjutan, dan mudah diimplementasikan oleh petani. & Rekomendasi logis dan runtut, namun kurang berkelanjutan atau sulit diimplementasikan. & Alur rekomendasi kurang runtut atau kurang praktis. & Rekomendasi tidak logis dan tidak terstruktur. \\
\hline
\end{longtable}

\section{Daftar Pustaka Sumber Etnosains}

\begin{enumerate}
\item Dt. Rajo Penghulu. (1994). \textit{Teknologi Tradisional Minangkabau}. Padang: Pusat Dokumentasi dan Informasi Kebudayaan Minangkabau.
\item Navis, A.A. (1984). \textit{Alam Terkembang Jadi Guru: Adat dan Kebudayaan Minangkabau}. Jakarta: Grafiti Pers.
\item Kementerian Pendidikan dan Kebudayaan. (2018). \textit{Sistem Irigasi Tradisional Nusantara}. Jakarta: Direktorat Warisan dan Diplomasi Budaya.
\item Syafwan, A. (2008). \textit{Kearifan Lokal dalam Teknologi Pertanian Minangkabau}. Padang: Universitas Andalas Press.
\end{enumerate}

\end{document}