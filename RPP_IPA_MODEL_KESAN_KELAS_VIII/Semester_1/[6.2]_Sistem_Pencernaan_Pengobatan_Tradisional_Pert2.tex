\documentclass[a4paper,12pt]{article}
\usepackage[a4paper, margin=1.27cm]{geometry}
\usepackage[indonesian]{babel}
\usepackage[utf8]{inputenc}
\usepackage{tcolorbox}
\usepackage{array}
\usepackage{multirow}
\usepackage{setspace}
\usepackage{amssymb}
\usepackage{graphicx}
\usepackage{adjustbox}
\usepackage{enumitem}
\usepackage{longtable}
\usepackage{xcolor}
\usepackage{amsmath}
\usepackage{fancyhdr}
\usepackage{titlesec}

% Define colors (black and white theme)
\definecolor{darkgray}{RGB}{64, 64, 64}
\definecolor{lightgray}{RGB}{240, 240, 240}
\definecolor{mediumgray}{RGB}{128, 128, 128}

% Custom tcolorbox styles
\tcbset{
    mainbox/.style={
        colback=white,
        colframe=black,
        boxrule=1pt,
        arc=3pt,
        left=8pt,
        right=8pt,
        top=8pt,
        bottom=8pt
    },
    sectionbox/.style={
        colback=white,
        colframe=black,
        boxrule=1.5pt,
        arc=2pt,
        left=6pt,
        right=6pt,
        top=6pt,
        bottom=6pt
    }
}

\begin{document}

\begin{center}
{\Large\textbf{MODUL AJAR: Pengobatan Tradisional Minang \& Gangguan Sistem Pencernaan}}
\end{center}

\vspace{0.5cm}

\begin{tcolorbox}[mainbox]
\textbf{Nama Penyusun:} Irfan Ananda Ismail, S.Pd., M.Pd., Gr \\
\textbf{Institusi:} SMP \\
\textbf{Mata Pelajaran:} Ilmu Pengetahuan Alam (IPA) \\
\textbf{Tahun Ajaran:} 2025/2026 \\
\textbf{Semester:} Ganjil \\
\textbf{Jenjang Sekolah:} SMP \\
\textbf{Kelas/Fase:} VIII / D \\
\textbf{Alokasi waktu:} 3 x 40 Menit (1 Pertemuan)
\end{tcolorbox}

\section{DIMENSI PROFIL PELAJAR PANCASILA}
\textit{(Clue untuk Guru: Sebutkan dimensi ini secara eksplisit saat apersepsi agar siswa sadar tujuan non-akademis yang sedang mereka kembangkan).}

\begin{itemize}
\item \textbf{Berkebinekaan Global:} Mengenal dan menghargai budaya, khususnya menganalisis kearifan pengobatan tradisional Minangkabau untuk gangguan pencernaan dan menghubungkannya dengan pengetahuan medis modern.
\item \textbf{Bernalar Kritis:} Ditempa secara intensif saat menganalisis informasi dari sumber sains dan budaya (Tahap S), menyintesiskan kedua perspektif (Tahap A), dan menyusun argumen berbasis bukti (Tahap N).
\item \textbf{Gotong Royong:} Kemampuan untuk bekerja sama secara kolaboratif dalam kelompok untuk merumuskan masalah (Tahap E), melakukan investigasi (Tahap S), dan membangun pemahaman bersama (Tahap A).
\item \textbf{Kreatif:} Menghasilkan argumen atau solusi orisinal yang terintegrasi pada tahap akhir pembelajaran (Tahap N).
\end{itemize}

\section{Sarana dan Prasarana}

\begin{itemize}
\item \textbf{Media:} LKPD 08 - Jurnal Investigasi Dual-Lensa, koleksi tumbuhan obat tradisional untuk pencernaan (jahe, kunyit, daun mint, dll), gambar/poster gangguan sistem pencernaan, artikel tentang pengobatan tradisional Minang, studi kasus gangguan pencernaan.
\item \textbf{Alat:} Papan tulis/whiteboard, spidol, Proyektor \& Speaker, Kertas Plano atau Karton (1 per kelompok), sticky notes warna-warni, penggaris, kaca pembesar.
\item \textbf{Sumber Belajar:} Buku ajar IPA kelas VIII, Tautan video animasi gangguan pencernaan (misal: bit.ly/animasi-gangguan-pencernaan), tautan artikel/video pengobatan tradisional Minang (misal: bit.ly/obat-tradisional-pencernaan-minang).
\end{itemize}

\section{Target Peserta didik}

\begin{itemize}
\item Peserta didik reguler kelas VIII (Fase D).
\end{itemize}

\section{Model Pembelajaran}

\begin{itemize}
\item Model Pembelajaran KESAN (Konektivitas Etnosains-Sains).
\end{itemize}

\section{Pemahaman Bermakna}
\textit{(Clue untuk Guru: Bacakan atau sampaikan narasi ini dengan intonasi yang menarik di akhir pembelajaran untuk mengikat semua pengalaman belajar siswa menjadi satu kesatuan yang bermakna).}

\begin{tcolorbox}[sectionbox]
"Ananda Semua, hari ini kita telah membuka mata terhadap kearifan pengobatan tradisional nenek moyang Minangkabau! Kita menemukan bahwa ramuan-ramuan tradisional untuk mengatasi gangguan pencernaan - seperti air jahe untuk mual, kunyit untuk maag, dan daun mint untuk perut kembung - ternyata memiliki dasar ilmiah yang kuat. Setiap tumbuhan obat mengandung senyawa aktif yang bekerja secara spesifik mengatasi gangguan tertentu pada sistem pencernaan. Dengan ini, kita sadar bahwa pengobatan tradisional bukan hanya kepercayaan turun-temurun, tapi juga pengetahuan empiris yang telah teruji secara ilmiah dan dapat melengkapi pengobatan modern."
\end{tcolorbox}

\section{PERTEMUAN KEDUA: Gangguan Sistem Pencernaan dan Pengobatan Tradisional Minangkabau}

\subsection{Capaian Pembelajaran (Fase D)}
Pada akhir Fase D, murid memiliki kemampuan menganalisis sistem organisasi kehidupan, fungsi, serta kelainan atau gangguan yang muncul pada sistem organ makhluk hidup.

\subsection{Tujuan Pembelajaran (TP) Pertemuan 2:}
\textit{(Clue untuk Guru: Tujuan ini adalah kompas Anda. Pastikan setiap tahapan KESAN yang Anda lalui berkontribusi pada pencapaian tujuan-tujuan ini).}

Melalui model pembelajaran KESAN, peserta didik mampu:
\begin{itemize}
\item Menghubungkan fenomena kearifan lokal Minangkabau (pengobatan tradisional untuk gangguan pencernaan) dengan konteks gangguan sistem pencernaan dalam sains modern. (Sintaks K)
\item Merumuskan pertanyaan investigatif mengenai gangguan sistem pencernaan dan perbandingan pengobatan tradisional dengan pengobatan modern. (Sintaks E)
\item Mengumpulkan informasi mengenai gangguan sistem pencernaan dari sumber sains serta pengobatan tradisional dari sumber kultural. (Sintaks S)
\item Menganalisis dan menyintesiskan hubungan sebab-akibat antara senyawa aktif dalam tumbuhan obat tradisional dengan mekanisme penyembuhan gangguan pencernaan. (Sintaks A)
\item Menyusun sebuah penjelasan analitis yang logis mengenai bagaimana pengobatan tradisional Minangkabau dapat melengkapi pengobatan modern untuk gangguan sistem pencernaan. (Sintaks N)
\end{itemize}

\subsection{Pertanyaan Pemantik}
\textit{(Clue untuk Guru: Ajukan dua pertanyaan ini secara berurutan, berikan jeda agar siswa berpikir. Jangan langsung minta jawaban, biarkan pertanyaan ini menggantung untuk memicu rasa ingin tahu).}

\begin{itemize}
\item "Pernahkah kalian sakit perut lalu nenek atau orang tua kalian memberikan air jahe hangat, atau kunyit asam? Dan anehnya, setelah minum itu perut kalian jadi lebih baik. Bagaimana bisa tumbuhan sederhana ini menyembuhkan gangguan pencernaan?"
\item "Dalam budaya Minang ada pepatah 'Obat tumbuh di halaman, penyakit datang dari mulut' - obat tumbuh di halaman, penyakit datang dari mulut. Dari sudut pandang medis, bagaimana sebenarnya tumbuhan obat tradisional bekerja mengatasi gangguan sistem pencernaan kita?"
\end{itemize}

\section{Langkah-langkah Kegiatan Pembelajaran:}

\subsection{Kegiatan Pembuka (15 Menit)}
\begin{itemize}
\item Guru membuka pelajaran dengan salam, doa, dan memeriksa kehadiran.
\item \textbf{Asesmen Diagnostik Awal:} Guru membagikan lembar K-W-L.
    \begin{itemize}
    \item \textbf{Instruksi Guru:} "Ananda, sebelum kita mulai menjelajahi dunia gangguan pencernaan dan pengobatannya, tolong isi dua kolom pertama di lembar ini. Di kolom \textbf{K (Tahu)}, tulis apa saja yang sudah kalian ketahui tentang gangguan pencernaan ATAU obat tradisional. Di kolom \textbf{W (Ingin Tahu)}, tulis apa yang membuat kalian penasaran tentang topik ini." \textit{(Clue: Ini membantu Anda memetakan pengetahuan awal dan minat siswa secara cepat).}
    \end{itemize}
\item \textbf{Apersepsi:}
    \begin{itemize}
    \item \textbf{Instruksi Guru:} "Hari ini kita akan menjadi ahli pengobatan tradisional dan dokter modern sekaligus. Kita akan menyelidiki bagaimana pengobatan tradisional Minangkabau mengatasi gangguan pencernaan dan bagaimana hal ini dapat dijelaskan secara ilmiah. Dalam investigasi ini, kita akan melatih kemampuan \textbf{Bernalar Kritis} kita, menghargai budaya lewat \textbf{Berkebinekaan Global}, dan bekerja sama dalam semangat \textbf{Gotong Royong}. Siap menjadi peneliti medis-tradisional?"
    \end{itemize}
\end{itemize}

\subsection{Kegiatan Inti (90 Menit) - SINTAKS MODEL KESAN}

\subsubsection{Tahap 1: (K) Kaitkan Konteks Kultural (15 menit)}
\begin{itemize}
\item \textbf{Aktivitas Guru:}
    \begin{itemize}
    \item Menampilkan koleksi tumbuhan obat tradisional untuk pencernaan (jahe, kunyit, daun mint, dll) dan menceritakan penggunaannya dalam budaya Minang.
    \item Mengajukan Pertanyaan Pemantik yang sudah disiapkan di atas.
    \item \textit{(Clue: Tujuan tahap ini adalah memvalidasi pengetahuan siswa dan memantik rasa heran, bukan mencari jawaban benar. Sediakan spidol dan papan tulis/whiteboard. Saat siswa menjawab, tuliskan semua ide mereka, bahkan yang keliru sekalipun, dengan judul "\textbf{PENGETAHUAN AWAL KITA}". Ini menunjukkan bahwa semua pemikiran dihargai).}
    \end{itemize}
\item \textbf{Aktivitas Siswa:} Mengamati koleksi tumbuhan obat, mendengarkan pertanyaan, lalu secara sukarela berbagi pengalaman atau pengetahuan awal tentang pengobatan tradisional atau gangguan pencernaan. Menuliskan minimal satu pertanyaan atau pengalaman di sticky notes dan menempelkannya di '\textbf{Papan Penasaran}'.
\end{itemize}

\subsubsection{Tahap 2: (E) Eksplorasi Enigma (15 menit)}
\begin{itemize}
\item \textbf{Aktivitas Guru:} Membentuk siswa menjadi kelompok (3-4 orang).
    \begin{itemize}
    \item \textbf{Instruksi Guru:} "Baik, rasa penasaran kalian luar biasa! Sekarang, tugas kita sebagai peneliti adalah mengubah rasa penasaran ini menjadi misi yang jelas. Dalam kelompok, diskusikan dan rumuskan minimal 3 pertanyaan kunci yang akan kita selidiki hari ini. Tuliskan dalam bentuk '\textbf{Misi Penelitian Tim [Nama Kelompok]}' di kertas plano yang Bapak/Ibu berikan."
    \item \textit{(Clue: Arahkan diskusi siswa agar pertanyaannya mencakup aspek 'apa saja gangguan sistem pencernaan' dan 'bagaimana pengobatan tradisional mengatasi gangguan tersebut'. Jika kelompok kesulitan, berikan pancingan: "Kira-kira, apa dulu yang perlu kita tahu? Jenis-jenis gangguan pencernaan atau cara mengobatinya?").}
    \end{itemize}
\item \textbf{Aktivitas Siswa:} Berdiskusi dalam tim untuk merumuskan misi penelitian (daftar pertanyaan kunci) di kertas plano. \textit{(Contoh misi yang diharapkan: 1. Apa saja gangguan yang bisa terjadi pada sistem pencernaan? 2. Bagaimana pengobatan tradisional Minang mengatasi gangguan pencernaan? 3. Mengapa tumbuhan obat tradisional bisa menyembuhkan gangguan pencernaan?)}
\end{itemize}

\subsubsection{Tahap 3: (S) Selidiki secara Sintetis (25 menit)}
\begin{itemize}
\item \textbf{Aktivitas Guru:} Membagikan "\textbf{LKPD 08 - Jurnal Investigasi Dual-Lensa}".
    \begin{itemize}
    \item \textbf{Instruksi Guru:} "Setiap tim akan melakukan investigasi dari dua lensa. Gunakan HP atau sumber yang disediakan untuk mencari jawabannya. Bagilah tugas dalam tim!"
    \item \textbf{Lensa Sains:} Buka link video bit.ly/animasi-gangguan-pencernaan untuk memahami berbagai gangguan sistem pencernaan (gastritis, diare, konstipasi, dll) beserta penyebab dan gejalanya.
    \item \textbf{Lensa Etnosains/Kultural:} Buka link artikel bit.ly/obat-tradisional-pencernaan-minang untuk memahami penggunaan tumbuhan obat tradisional Minangkabau seperti jahe, kunyit, daun mint, dan lainnya untuk mengatasi gangguan pencernaan.
    \item \textit{(Clue: Pastikan sumber belajar sudah disiapkan dan link bisa diakses. Berkelilinglah untuk memastikan setiap kelompok membagi tugas dan tidak hanya fokus pada satu lensa saja).}
    \end{itemize}
\item \textbf{Aktivitas Siswa:} Dalam kelompok, siswa berbagi tugas mencari informasi dari sumber yang diberikan dan mencatat temuan kunci di dua kolom terpisah pada "\textbf{LKPD 08 - Jurnal Investigasi Dual-Lensa}".
\end{itemize}

\subsubsection{Tahap 4: (A) Asimilasi Analitis (20 menit)}
\begin{itemize}
\item \textbf{Aktivitas Guru:} Memfasilitasi diskusi untuk menjembatani kedua lensa.
    \begin{itemize}
    \item \textbf{Pertanyaan Pancingan Kunci untuk Guru:}
        \begin{itemize}
        \item "Oke, dari Lensa Sains kita tahu gastritis disebabkan oleh iritasi lambung dan diare karena infeksi usus. Dari Lensa Etnosains, kita tahu jahe untuk mual dan kunyit untuk maag. Nah, coba hubungkan! Mengapa tumbuhan ini efektif?"
        \item "Dari Lensa Sains, kita tahu gangguan pencernaan sering karena peradangan atau infeksi. Dari Lensa Budaya, tumbuhan obat Minang mengandung senyawa anti-inflamasi dan antibakteri. Bagaimana hubungannya?"
        \end{itemize}
    \item \textit{(Clue: Fokuskan untuk membuat siswa 'menemukan' hubungannya sendiri, bukan diberitahu. Gunakan kata "menurut kalian", "kira-kira kenapa", "ada yang punya ide?").}
    \end{itemize}
\item \textbf{Aktivitas Siswa:} Berdiskusi intensif untuk menghubungkan temuan sains dan budaya. Menuliskan kesimpulan terpadu (sintesis) mereka di kertas plano.
\end{itemize}

\subsubsection{Tahap 5: (N) Nyatakan Pemahaman (15 menit)}
\begin{itemize}
\item \textbf{Aktivitas Guru:} Memberikan studi kasus individual atau per kelompok.
    \begin{itemize}
    \item \textbf{Instruksi Guru (tuliskan di papan tulis):}
    
    "\textbf{STUDI KASUS UNTUK PENELITI SAINS-BUDAYA:}
    
    Seorang dokter muda yang baru bertugas di daerah terpencil Sumatera Barat menemukan bahwa masyarakat setempat jarang datang ke puskesmas untuk gangguan pencernaan ringan. Mereka lebih memilih menggunakan ramuan tradisional seperti air jahe untuk mual, kunyit asam untuk maag, dan daun mint untuk perut kembung. Dokter ini penasaran dan ingin memahami secara ilmiah mengapa pengobatan tradisional ini efektif, sehingga dia bisa mengintegrasikannya dengan pengobatan modern.
    
    Tugasmu: Tuliskan sebuah rekomendasi ilmiah singkat (4-6 kalimat) di buku latihanmu untuk membantu dokter tersebut memahami mekanisme kerja pengobatan tradisional dan bagaimana hal ini bisa melengkapi pengobatan modern. Gunakan pengetahuan gabungan dari sains dan kearifan tradisional yang baru saja kamu pelajari."
    \end{itemize}
\item \textbf{Aktivitas Siswa:} Menyusun argumen tertulis untuk menjawab studi kasus yang diberikan, menggunakan bukti dari kedua lensa.
\end{itemize}

\subsection{Kegiatan Penutup (15 Menit)}
\begin{itemize}
\item \textbf{Presentasi \& Penguatan:} Guru meminta 2-3 siswa secara acak untuk membacakan rekomendasi mereka. Guru memberikan pujian dan penguatan positif, menekankan betapa hebatnya kemampuan mereka mengintegrasikan sains dan budaya.
\item \textbf{Refleksi:}
    \begin{itemize}
    \item \textbf{Instruksi Guru:} "Sekarang, kembali ke lembar K-W-L kalian. Lengkapi kolom terakhir, \textbf{L (Learned)}, dengan hal-hal baru yang paling mengagumkan yang kalian pelajari hari ini."
    \item \textbf{Instruksi Guru:} "Angkat tangan, siapa yang setelah belajar hari ini jadi lebih menghargai kearifan pengobatan tradisional nenek moyang kita?"
    \end{itemize}
\item \textbf{Tindak Lanjut:}
    \begin{itemize}
    \item \textbf{Instruksi Guru:} "Luar biasa, para peneliti! Hari ini kita sudah mengungkap bagaimana pengobatan tradisional dapat melengkapi pengobatan modern. Pertemuan berikutnya, kita akan mempelajari zat aditif dan adiktif. Siapa tahu ada kearifan Minang tentang pengawetan makanan yang bisa mengajarkan kita tentang zat aditif alami? Kita akan lihat!"
    \item \textit{(Clue: Buat transisi yang menarik ke pertemuan berikutnya sambil mempertahankan semangat investigasi yang sudah terbangun).}
    \end{itemize}
\item Guru menutup pelajaran dengan doa dan salam.
\end{itemize}

\section{Asesmen (Penilaian)}

\begin{itemize}
\item \textbf{Asesmen Diagnostik (Awal):} Analisis lembar K-W-L. (Untuk mengetahui baseline siswa).
\item \textbf{Asesmen Formatif (Proses):} Observasi keaktifan diskusi (gotong royong) dan penilaian kelengkapan "\textbf{LKPD 08 - Jurnal Investigasi Dual-Lensa}".
\item \textbf{Asesmen Sumatif (Akhir Siklus):} Penilaian jawaban studi kasus (Tahap N) menggunakan rubrik.
\end{itemize}

\subsection{Rubrik Penilaian Jawaban Studi Kasus (Tahap N)}

\begin{longtable}{|p{3cm}|p{3cm}|p{3cm}|p{3cm}|p{3cm}|}
\hline
\textbf{Kriteria Penilaian} & \textbf{Skor 4 (Sangat Baik)} & \textbf{Skor 3 (Baik)} & \textbf{Skor 2 (Cukup)} & \textbf{Skor 1 (Kurang)} \\
\hline
\textbf{Ketepatan Konsep Ilmiah} & Menggunakan istilah ilmiah (gangguan pencernaan, senyawa aktif, anti-inflamasi, antibakteri) dengan sangat tepat dan relevan dengan kasus. & Menggunakan istilah ilmiah dengan tepat, namun kurang relevan. & Menggunakan istilah ilmiah namun ada beberapa kesalahan konsep. & Tidak menggunakan istilah ilmiah atau salah total. \\
\hline
\textbf{Keterkaitan dengan Etnosains} & Mampu menghubungkan secara logis dan eksplisit antara pengobatan tradisional dengan mekanisme penyembuhan secara mendalam. & Mampu menghubungkan pengobatan tradisional dengan sains, namun kurang mendalam. & Hanya menyebutkan pengobatan tradisional tanpa menghubungkan dengan sains, atau sebaliknya. & Tidak ada keterkaitan antara sains dan budaya yang ditunjukkan. \\
\hline
\textbf{Kelogisan \& Struktur Argumen} & Rekomendasi sangat logis, runtut, praktis untuk integrasi pengobatan, dan mudah dipahami. & Rekomendasi logis dan runtut, namun kurang praktis untuk integrasi. & Alur rekomendasi kurang runtut atau sulit dipahami. & Rekomendasi tidak logis dan tidak terstruktur. \\
\hline
\end{longtable}

\section{Daftar Pustaka Sumber Etnosains}

\begin{enumerate}
\item Zuhud, E.A.M. (2009). \textit{Potensi Hutan Tropika Indonesia sebagai Penyangga Bahan Obat Alam untuk Kesehatan Bangsa}. Jakarta: Fakultas Kehutanan IPB.
\item Syukur, C. (2011). \textit{Tumbuhan Obat Tradisional Sumatera Barat}. Padang: Andalas University Press.
\item Kementerian Kesehatan RI. (2018). \textit{Farmakope Herbal Indonesia Edisi II}. Jakarta: Direktorat Jenderal Kefarmasian dan Alat Kesehatan.
\item Navis, A.A. (1984). \textit{Alam Terkembang Jadi Guru: Adat dan Kebudayaan Minangkabau}. Jakarta: Grafiti Pers.
\end{enumerate}

\end{document}