\documentclass[a4paper,12pt]{article}
\usepackage[a4paper, margin=1.27cm]{geometry}
\usepackage[indonesian]{babel}
\usepackage[utf8]{inputenc}
\usepackage{tcolorbox}
\usepackage{array}
\usepackage{multirow}
\usepackage{setspace}
\usepackage{amssymb}
\usepackage{graphicx}
\usepackage{adjustbox}
\usepackage{enumitem}
\usepackage{longtable}
\usepackage{xcolor}
\usepackage{amsmath}
\usepackage{fancyhdr}
\usepackage{titlesec}

% Define colors (black and white theme)
\definecolor{darkgray}{RGB}{64, 64, 64}
\definecolor{lightgray}{RGB}{240, 240, 240}
\definecolor{mediumgray}{RGB}{128, 128, 128}

% Custom tcolorbox styles
\tcbset{
    mainbox/.style={
        colback=white,
        colframe=black,
        boxrule=1pt,
        arc=3pt,
        left=8pt,
        right=8pt,
        top=8pt,
        bottom=8pt
    },
    sectionbox/.style={
        colback=white,
        colframe=black,
        boxrule=1.5pt,
        arc=2pt,
        left=6pt,
        right=6pt,
        top=6pt,
        bottom=6pt
    }
}

\begin{document}

\begin{center}
{\Large\textbf{MODUL AJAR: Kearifan Pencegahan Minang \& Bahaya Zat Adiktif}}
\end{center}

\vspace{0.5cm}

\begin{tcolorbox}[mainbox]
\textbf{Nama Penyusun:} Irfan Ananda Ismail, S.Pd., M.Pd., Gr \\
\textbf{Institusi:} SMP \\
\textbf{Mata Pelajaran:} Ilmu Pengetahuan Alam (IPA) \\
\textbf{Tahun Ajaran:} 2025/2026 \\
\textbf{Semester:} Ganjil \\
\textbf{Jenjang Sekolah:} SMP \\
\textbf{Kelas/Fase:} VIII / D \\
\textbf{Alokasi waktu:} 3 x 40 Menit (1 Pertemuan)
\end{tcolorbox}

\section{DIMENSI PROFIL PELAJAR PANCASILA}
\textit{(Clue untuk Guru: Sebutkan dimensi ini secara eksplisit saat apersepsi agar siswa sadar tujuan non-akademis yang sedang mereka kembangkan).}

\begin{itemize}
\item \textbf{Berkebinekaan Global:} Mengenal dan menghargai budaya, khususnya menganalisis kearifan Minangkabau dalam pencegahan dan penanganan masalah sosial serta menghubungkannya dengan bahaya zat adiktif dalam sains modern.
\item \textbf{Bernalar Kritis:} Ditempa secara intensif saat menganalisis informasi dari sumber sains dan budaya (Tahap S), menyintesiskan kedua perspektif (Tahap A), dan menyusun argumen berbasis bukti (Tahap N).
\item \textbf{Gotong Royong:} Kemampuan untuk bekerja sama secara kolaboratif dalam kelompok untuk merumuskan masalah (Tahap E), melakukan investigasi (Tahap S), dan membangun pemahaman bersama (Tahap A).
\item \textbf{Kreatif:} Menghasilkan argumen atau solusi orisinal yang terintegrasi pada tahap akhir pembelajaran (Tahap N).
\end{itemize}

\section{Sarana dan Prasarana}

\begin{itemize}
\item \textbf{Media:} LKPD 10 - Jurnal Investigasi Dual-Lensa, poster/gambar dampak zat adiktif, artikel tentang filosofi pencegahan dalam budaya Minang, contoh kasus kecanduan, video testimoni korban narkoba.
\item \textbf{Alat:} Papan tulis/whiteboard, spidol, Proyektor \& Speaker, Kertas Plano atau Karton (1 per kelompok), sticky notes warna-warni, penggaris.
\item \textbf{Sumber Belajar:} Buku ajar IPA kelas VIII, Tautan video animasi dampak zat adiktif (misal: bit.ly/animasi-dampak-narkoba), tautan artikel/video kearifan pencegahan Minang (misal: bit.ly/kearifan-pencegahan-minang).
\end{itemize}

\section{Target Peserta didik}

\begin{itemize}
\item Peserta didik reguler kelas VIII (Fase D).
\end{itemize}

\section{Model Pembelajaran}

\begin{itemize}
\item Model Pembelajaran KESAN (Konektivitas Etnosains-Sains).
\end{itemize}

\section{Pemahaman Bermakna}
\textit{(Clue untuk Guru: Bacakan atau sampaikan narasi ini dengan intonasi yang menarik di akhir pembelajaran untuk mengikat semua pengalaman belajar siswa menjadi satu kesatuan yang bermakna).}

\begin{tcolorbox}[sectionbox]
"Ananda Semua, hari ini kita telah mengungkap kearifan mendalam di balik filosofi pencegahan dalam budaya Minangkabau! Kita menemukan bahwa prinsip-prinsip seperti 'Adat basandi syarak, syarak basandi kitabullah' dan sistem kontrol sosial nagari ternyata memiliki dasar ilmiah yang kuat dalam mencegah penyalahgunaan zat adiktif. Setiap mekanisme pencegahan tradisional - dari pendidikan karakter, pengawasan komunitas, hingga sanksi sosial - sebenarnya bekerja pada tingkat neurobiologis dan psikososial yang sama dengan program pencegahan modern. Dengan ini, kita sadar bahwa kearifan lokal bukan hanya warisan budaya, tapi juga strategi ilmiah yang telah terbukti efektif melindungi masyarakat dari bahaya zat adiktif."
\end{tcolorbox}

\section{PERTEMUAN KEDUA: Bahaya Zat Adiktif dan Kearifan Pencegahan Minangkabau}

\subsection{Capaian Pembelajaran (Fase D)}
Pada akhir Fase D, murid memiliki kemampuan menganalisis sistem organisasi kehidupan, fungsi, serta kelainan atau gangguan yang muncul pada sistem organ makhluk hidup.

\subsection{Tujuan Pembelajaran (TP) Pertemuan 2:}
\textit{(Clue untuk Guru: Tujuan ini adalah kompas Anda. Pastikan setiap tahapan KESAN yang Anda lalui berkontribusi pada pencapaian tujuan-tujuan ini).}

Melalui model pembelajaran KESAN, peserta didik mampu:
\begin{itemize}
\item Menghubungkan fenomena kearifan lokal Minangkabau (sistem pencegahan dan kontrol sosial) dengan konteks bahaya zat adiktif dalam sains modern. (Sintaks K)
\item Merumuskan pertanyaan investigatif mengenai dampak zat adiktif dan perbandingan sistem pencegahan tradisional dengan program pencegahan modern. (Sintaks E)
\item Mengumpulkan informasi mengenai dampak zat adiktif pada tubuh dari sumber sains serta sistem pencegahan tradisional dari sumber kultural. (Sintaks S)
\item Menganalisis dan menyintesiskan hubungan sebab-akibat antara mekanisme pencegahan dalam budaya Minangkabau dengan prinsip pencegahan kecanduan dalam sains modern. (Sintaks A)
\item Menyusun sebuah penjelasan analitis yang logis mengenai bagaimana kearifan pencegahan tradisional Minangkabau dapat memperkuat program pencegahan zat adiktif modern. (Sintaks N)
\end{itemize}

\subsection{Pertanyaan Pemantik}
\textit{(Clue untuk Guru: Ajukan dua pertanyaan ini secara berurutan, berikan jeda agar siswa berpikir. Jangan langsung minta jawaban, biarkan pertanyaan ini menggantung untuk memicu rasa ingin tahu).}

\begin{itemize}
\item "Pernahkah kalian memperhatikan mengapa di daerah yang masih kental adat Minangkabau, kasus penyalahgunaan narkoba relatif lebih rendah dibanding daerah lain? Apa yang membuat sistem adat begitu efektif melindungi masyarakatnya?"
\item "Dalam filosofi Minang ada ungkapan 'Bulek aie dek pembuluh, bulek kato dek mufakat' - bulat air karena pembuluh, bulat kata karena mufakat. Dari sudut pandang ilmu pencegahan kecanduan, bagaimana sebenarnya kekuatan komunitas dan norma sosial melindungi seseorang dari bahaya zat adiktif?"
\end{itemize}

\section{Langkah-langkah Kegiatan Pembelajaran:}

\subsection{Kegiatan Pembuka (15 Menit)}
\begin{itemize}
\item Guru membuka pelajaran dengan salam, doa, dan memeriksa kehadiran.
\item \textbf{Asesmen Diagnostik Awal:} Guru membagikan lembar K-W-L.
    \begin{itemize}
    \item \textbf{Instruksi Guru:} "Ananda, sebelum kita mulai menjelajahi dunia pencegahan zat adiktif, tolong isi dua kolom pertama di lembar ini. Di kolom \textbf{K (Tahu)}, tulis apa saja yang sudah kalian ketahui tentang bahaya narkoba ATAU sistem pencegahan dalam budaya Minang. Di kolom \textbf{W (Ingin Tahu)}, tulis apa yang membuat kalian penasaran tentang topik ini." \textit{(Clue: Ini membantu Anda memetakan pengetahuan awal dan minat siswa secara cepat).}
    \end{itemize}
\item \textbf{Apersepsi:}
    \begin{itemize}
    \item \textbf{Instruksi Guru:} "Hari ini kita akan menjadi ahli pencegahan tradisional dan modern sekaligus. Kita akan menyelidiki bagaimana sistem pencegahan dalam budaya Minangkabau dapat memperkuat program pencegahan zat adiktif modern. Dalam investigasi ini, kita akan melatih kemampuan \textbf{Bernalar Kritis} kita, menghargai budaya lewat \textbf{Berkebinekaan Global}, dan bekerja sama dalam semangat \textbf{Gotong Royong}. Siap menjadi peneliti pencegahan?"
    \end{itemize}
\end{itemize}

\subsection{Kegiatan Inti (90 Menit) - SINTAKS MODEL KESAN}

\subsubsection{Tahap 1: (K) Kaitkan Konteks Kultural (15 menit)}
\begin{itemize}
\item \textbf{Aktivitas Guru:}
    \begin{itemize}
    \item Menampilkan poster/gambar tentang sistem adat Minangkabau dan menceritakan bagaimana sistem ini melindungi masyarakat dari berbagai masalah sosial.
    \item Mengajukan Pertanyaan Pemantik yang sudah disiapkan di atas.
    \item \textit{(Clue: Tujuan tahap ini adalah memvalidasi pengetahuan siswa dan memantik rasa heran, bukan mencari jawaban benar. Sediakan spidol dan papan tulis/whiteboard. Saat siswa menjawab, tuliskan semua ide mereka, bahkan yang keliru sekalipun, dengan judul "\textbf{PENGETAHUAN AWAL KITA}". Ini menunjukkan bahwa semua pemikiran dihargai).}
    \end{itemize}
\item \textbf{Aktivitas Siswa:} Mengamati poster sistem adat, mendengarkan pertanyaan, lalu secara sukarela berbagi pengalaman atau pengetahuan awal tentang sistem pencegahan dalam budaya Minang atau bahaya narkoba. Menuliskan minimal satu pertanyaan atau pengalaman di sticky notes dan menempelkannya di '\textbf{Papan Penasaran}'.
\end{itemize}

\subsubsection{Tahap 2: (E) Eksplorasi Enigma (15 menit)}
\begin{itemize}
\item \textbf{Aktivitas Guru:} Membentuk siswa menjadi kelompok (3-4 orang).
    \begin{itemize}
    \item \textbf{Instruksi Guru:} "Baik, rasa penasaran kalian luar biasa! Sekarang, tugas kita sebagai peneliti adalah mengubah rasa penasaran ini menjadi misi yang jelas. Dalam kelompok, diskusikan dan rumuskan minimal 3 pertanyaan kunci yang akan kita selidiki hari ini. Tuliskan dalam bentuk '\textbf{Misi Penelitian Tim [Nama Kelompok]}' di kertas plano yang Bapak/Ibu berikan."
    \item \textit{(Clue: Arahkan diskusi siswa agar pertanyaannya mencakup aspek 'apa dampak zat adiktif pada tubuh' dan 'bagaimana sistem pencegahan tradisional bekerja'. Jika kelompok kesulitan, berikan pancingan: "Kira-kira, apa dulu yang perlu kita tahu? Dampak narkoba pada tubuh atau cara mencegahnya?").}
    \end{itemize}
\item \textbf{Aktivitas Siswa:} Berdiskusi dalam tim untuk merumuskan misi penelitian (daftar pertanyaan kunci) di kertas plano. \textit{(Contoh misi yang diharapkan: 1. Apa dampak zat adiktif pada sistem saraf dan organ tubuh? 2. Bagaimana sistem pencegahan dalam budaya Minangkabau bekerja? 3. Mengapa sistem pencegahan tradisional efektif melindungi masyarakat?)}
\end{itemize}

\subsubsection{Tahap 3: (S) Selidiki secara Sintetis (25 menit)}
\begin{itemize}
\item \textbf{Aktivitas Guru:} Membagikan "\textbf{LKPD 10 - Jurnal Investigasi Dual-Lensa}".
    \begin{itemize}
    \item \textbf{Instruksi Guru:} "Setiap tim akan melakukan investigasi dari dua lensa. Gunakan HP atau sumber yang disediakan untuk mencari jawabannya. Bagilah tugas dalam tim!"
    \item \textbf{Lensa Sains:} Buka link video bit.ly/animasi-dampak-narkoba untuk memahami dampak zat adiktif pada sistem saraf, jantung, paru-paru, dan organ tubuh lainnya, serta mekanisme kecanduan.
    \item \textbf{Lensa Etnosains/Kultural:} Buka link artikel bit.ly/kearifan-pencegahan-minang untuk memahami sistem pencegahan dalam budaya Minangkabau seperti pendidikan adat, pengawasan komunitas, sanksi sosial, dan peran lembaga adat.
    \item \textit{(Clue: Pastikan sumber belajar sudah disiapkan dan link bisa diakses. Berkelilinglah untuk memastikan setiap kelompok membagi tugas dan tidak hanya fokus pada satu lensa saja).}
    \end{itemize}
\item \textbf{Aktivitas Siswa:} Dalam kelompok, siswa berbagi tugas mencari informasi dari sumber yang diberikan dan mencatat temuan kunci di dua kolom terpisah pada "\textbf{LKPD 10 - Jurnal Investigasi Dual-Lensa}".
\end{itemize}

\subsubsection{Tahap 4: (A) Asimilasi Analitis (20 menit)}
\begin{itemize}
\item \textbf{Aktivitas Guru:} Memfasilitasi diskusi untuk menjembatani kedua lensa.
    \begin{itemize}
    \item \textbf{Pertanyaan Pancingan Kunci untuk Guru:}
        \begin{itemize}
        \item "Oke, dari Lensa Sains kita tahu zat adiktif merusak sistem saraf dan menyebabkan kecanduan. Dari Lensa Etnosains, kita tahu sistem adat Minang melindungi masyarakat melalui kontrol sosial. Nah, coba hubungkan! Mengapa kontrol sosial efektif mencegah kecanduan?"
        \item "Dari Lensa Sains, kita tahu pencegahan terbaik adalah tidak mencoba sama sekali. Dari Lensa Budaya, sistem adat mencegah dengan pendidikan nilai dan pengawasan komunitas. Apa kesamaan prinsipnya?"
        \end{itemize}
    \item \textit{(Clue: Fokuskan untuk membuat siswa 'menemukan' hubungannya sendiri, bukan diberitahu. Gunakan kata "menurut kalian", "kira-kira kenapa", "ada yang punya ide?").}
    \end{itemize}
\item \textbf{Aktivitas Siswa:} Berdiskusi intensif untuk menghubungkan temuan sains dan budaya. Menuliskan kesimpulan terpadu (sintesis) mereka di kertas plano.
\end{itemize}

\subsubsection{Tahap 5: (N) Nyatakan Pemahaman (15 menit)}
\begin{itemize}
\item \textbf{Aktivitas Guru:} Memberikan studi kasus individual atau per kelompok.
    \begin{itemize}
    \item \textbf{Instruksi Guru (tuliskan di papan tulis):}
    
    "\textbf{STUDI KASUS UNTUK PENELITI SAINS-BUDAYA:}
    
    Pemerintah daerah Sumatera Barat ingin mengembangkan program pencegahan narkoba yang lebih efektif untuk remaja. Mereka menyadari bahwa program konvensional sering kurang berhasil karena tidak mengakar pada budaya lokal. Tim ahli dari BNN (Badan Narkotika Nasional) ingin belajar dari sistem pencegahan tradisional Minangkabau yang terbukti efektif selama berabad-abad. Mereka membutuhkan rekomendasi bagaimana mengintegrasikan kearifan lokal dengan program pencegahan modern.
    
    Tugasmu: Tuliskan sebuah rekomendasi program singkat (4-6 kalimat) di buku latihanmu untuk membantu tim BNN mengembangkan program pencegahan narkoba yang mengintegrasikan kearifan budaya Minangkabau dengan pendekatan ilmiah modern. Gunakan pengetahuan gabungan dari sains dan kearifan tradisional yang baru saja kamu pelajari."
    \end{itemize}
\item \textbf{Aktivitas Siswa:} Menyusun argumen tertulis untuk menjawab studi kasus yang diberikan, menggunakan bukti dari kedua lensa.
\end{itemize}

\subsection{Kegiatan Penutup (15 Menit)}
\begin{itemize}
\item \textbf{Presentasi \& Penguatan:} Guru meminta 2-3 siswa secara acak untuk membacakan rekomendasi mereka. Guru memberikan pujian dan penguatan positif, menekankan betapa hebatnya kemampuan mereka mengintegrasikan sains dan budaya.
\item \textbf{Refleksi:}
    \begin{itemize}
    \item \textbf{Instruksi Guru:} "Sekarang, kembali ke lembar K-W-L kalian. Lengkapi kolom terakhir, \textbf{L (Learned)}, dengan hal-hal baru yang paling mengagumkan yang kalian pelajari hari ini."
    \item \textbf{Instruksi Guru:} "Angkat tangan, siapa yang setelah belajar hari ini jadi lebih bangga dengan kearifan budaya Minangkabau dalam melindungi masyarakat?"
    \end{itemize}
\item \textbf{Tindak Lanjut:}
    \begin{itemize}
    \item \textbf{Instruksi Guru:} "Luar biasa, para peneliti! Hari ini kita sudah mengungkap bagaimana kearifan tradisional dapat memperkuat program pencegahan modern. Pertemuan berikutnya, kita akan mempelajari sistem peredaran darah. Siapa tahu ada kearifan Minang tentang menjaga kesehatan jantung yang bisa mengajarkan kita tentang sistem kardiovaskular? Kita akan lihat!"
    \item \textit{(Clue: Buat transisi yang menarik ke pertemuan berikutnya sambil mempertahankan semangat investigasi yang sudah terbangun).}
    \end{itemize}
\item Guru menutup pelajaran dengan doa dan salam.
\end{itemize}

\section{Asesmen (Penilaian)}

\begin{itemize}
\item \textbf{Asesmen Diagnostik (Awal):} Analisis lembar K-W-L. (Untuk mengetahui baseline siswa).
\item \textbf{Asesmen Formatif (Proses):} Observasi keaktifan diskusi (gotong royong) dan penilaian kelengkapan "\textbf{LKPD 10 - Jurnal Investigasi Dual-Lensa}".
\item \textbf{Asesmen Sumatif (Akhir Siklus):} Penilaian jawaban studi kasus (Tahap N) menggunakan rubrik.
\end{itemize}

\subsection{Rubrik Penilaian Jawaban Studi Kasus (Tahap N)}

\begin{longtable}{|p{3cm}|p{3cm}|p{3cm}|p{3cm}|p{3cm}|}
\hline
\textbf{Kriteria Penilaian} & \textbf{Skor 4 (Sangat Baik)} & \textbf{Skor 3 (Baik)} & \textbf{Skor 2 (Cukup)} & \textbf{Skor 1 (Kurang)} \\
\hline
\textbf{Ketepatan Konsep Ilmiah} & Menggunakan istilah ilmiah (sistem saraf, kecanduan, neurotransmitter, pencegahan primer) dengan sangat tepat dan relevan dengan kasus. & Menggunakan istilah ilmiah dengan tepat, namun kurang relevan. & Menggunakan istilah ilmiah namun ada beberapa kesalahan konsep. & Tidak menggunakan istilah ilmiah atau salah total. \\
\hline
\textbf{Keterkaitan dengan Etnosains} & Mampu menghubungkan secara logis dan eksplisit antara sistem pencegahan tradisional dengan prinsip pencegahan modern secara mendalam. & Mampu menghubungkan sistem pencegahan tradisional dengan modern, namun kurang mendalam. & Hanya menyebutkan sistem pencegahan tanpa menghubungkan keduanya, atau sebaliknya. & Tidak ada keterkaitan antara sains dan budaya yang ditunjukkan. \\
\hline
\textbf{Kelogisan \& Struktur Argumen} & Rekomendasi sangat logis, runtut, praktis untuk implementasi, dan mudah dipahami. & Rekomendasi logis dan runtut, namun kurang praktis untuk implementasi. & Alur rekomendasi kurang runtut atau sulit dipahami. & Rekomendasi tidak logis dan tidak terstruktur. \\
\hline
\end{longtable}

\section{Daftar Pustaka Sumber Etnosains}

\begin{enumerate}
\item Navis, A.A. (1984). \textit{Alam Terkembang Jadi Guru: Adat dan Kebudayaan Minangkabau}. Jakarta: Grafiti Pers.
\item Kato, T. (2005). \textit{Adat Minangkabau dan Merantau dalam Perspektif Sejarah}. Jakarta: Balai Pustaka.
\item BNN RI. (2019). \textit{Panduan Pencegahan Penyalahgunaan Narkoba Berbasis Masyarakat}. Jakarta: BNN RI.
\item Dt. Rajo Penghulu, I. (1994). \textit{Pegangan Penghulu, Bundo Kanduang dan Pidato Alua Pasambahan Adat Minangkabau}. Bukittinggi: Pustaka Indonesia.
\end{enumerate}

\end{document}