\documentclass[a4paper,12pt]{article}
\usepackage[a4paper, margin=1.27cm]{geometry}
\usepackage[indonesian]{babel}
\usepackage[utf8]{inputenc}
\usepackage{tcolorbox}
\usepackage{array}
\usepackage{multirow}
\usepackage{setspace}
\usepackage{amssymb}
\usepackage{graphicx}
\usepackage{adjustbox}
\usepackage{enumitem}
\usepackage{longtable}
\usepackage{xcolor}
\usepackage{amsmath}
\usepackage{fancyhdr}
\usepackage{titlesec}

% Define colors (black and white theme)
\definecolor{darkgray}{RGB}{64, 64, 64}
\definecolor{lightgray}{RGB}{240, 240, 240}
\definecolor{mediumgray}{RGB}{128, 128, 128}

% Custom tcolorbox styles
\tcbset{
    mainbox/.style={
        colback=white,
        colframe=black,
        boxrule=1pt,
        arc=3pt,
        left=8pt,
        right=8pt,
        top=8pt,
        bottom=8pt
    },
    sectionbox/.style={
        colback=white,
        colframe=black,
        boxrule=1.5pt,
        arc=2pt,
        left=6pt,
        right=6pt,
        top=6pt,
        bottom=6pt
    }
}

\begin{document}

\begin{center}
{\Large\textbf{MODUL AJAR: Sawah Bertingkat Minangkabau \& Rahasia Fotosintesis}}
\end{center}

\vspace{0.5cm}

\begin{tcolorbox}[mainbox]
\textbf{Nama Penyusun:} Irfan Ananda Ismail, S.Pd., M.Pd., Gr \\
\textbf{Institusi:} SMP \\
\textbf{Mata Pelajaran:} Ilmu Pengetahuan Alam (IPA) \\
\textbf{Tahun Ajaran:} 2025/2026 \\
\textbf{Semester:} Ganjil \\
\textbf{Jenjang Sekolah:} SMP \\
\textbf{Kelas/Fase:} VIII / D \\
\textbf{Alokasi waktu:} 3 x 40 Menit (1 Pertemuan)
\end{tcolorbox}

\section{DIMENSI PROFIL PELAJAR PANCASILA}
\textit{(Clue untuk Guru: Sebutkan dimensi ini secara eksplisit saat apersepsi agar siswa sadar tujuan non-akademis yang sedang mereka kembangkan).}

\begin{itemize}
\item \textbf{Berkebinekaan Global:} Mengenal dan menghargai budaya, khususnya menganalisis kearifan lokal Minangkabau dalam sistem pertanian sawah bertingkat (terasering), lalu menghubungkannya dengan konsep sains universal tentang proses fotosintesis dan respirasi tumbuhan.
\item \textbf{Bernalar Kritis:} Ditempa secara intensif saat menganalisis informasi dari sumber sains dan budaya (Tahap S), menyintesiskan kedua perspektif (Tahap A), dan menyusun argumen berbasis bukti (Tahap N).
\item \textbf{Gotong Royong:} Kemampuan untuk bekerja sama secara kolaboratif dalam kelompok untuk merumuskan masalah (Tahap E), melakukan investigasi (Tahap S), dan membangun pemahaman bersama (Tahap A).
\item \textbf{Kreatif:} Menghasilkan argumen atau solusi orisinal yang terintegrasi pada tahap akhir pembelajaran (Tahap N).
\end{itemize}

\section{Sarana dan Prasarana}

\begin{itemize}
\item \textbf{Media:} LKPD 08 - Jurnal Investigasi Dual-Lensa, video singkat sistem sawah bertingkat tradisional Minangkabau (durasi 3-4 menit), gambar/diagram proses fotosintesis, artikel tentang sistem pertanian tradisional, tanaman hijau kecil (jika tersedia) untuk demonstrasi.
\item \textbf{Alat:} Papan tulis/whiteboard, spidol, Proyektor \& Speaker, Kertas Plano atau Karton (1 per kelompok), sticky notes warna-warni, senter/lampu untuk demonstrasi cahaya, gelas bening dan air untuk eksperimen sederhana.
\item \textbf{Sumber Belajar:} Buku ajar IPA kelas VIII, Tautan video animasi fotosintesis (misal: bit.ly/animasi-fotosintesis), tautan artikel/video sawah bertingkat Minangkabau (misal: bit.ly/sawah-terasering-minang).
\end{itemize}

\section{Target Peserta didik}

\begin{itemize}
\item Peserta didik reguler kelas VIII (Fase D).
\end{itemize}

\section{Model Pembelajaran}

\begin{itemize}
\item Model Pembelajaran KESAN (Konektivitas Etnosains-Sains).
\end{itemize}

\section{Pemahaman Bermakna}
\textit{(Clue untuk Guru: Bacakan atau sampaikan narasi ini dengan intonasi yang menarik di akhir pembelajaran untuk mengikat semua pengalaman belajar siswa menjadi satu kesatuan yang bermakna).}

\begin{tcolorbox}[sectionbox]
"Ananda Semua, hari ini kita telah mengungkap keajaiban di balik sawah bertingkat yang masih memukau mata di dataran tinggi Minangkabau! Kita menemukan bahwa sistem pertanian tradisional ini sebenarnya mengoptimalkan proses fotosintesis - pabrik makanan alami tumbuhan yang mengubah cahaya matahari, air, dan karbon dioksida menjadi makanan dan oksigen. Setiap teras sawah yang dirancang mengikuti kontur tanah, setiap pengaturan air yang cermat, semuanya memastikan tanaman padi mendapat cahaya matahari optimal untuk fotosintesis. Dengan ini, kita sadar bahwa nenek moyang kita telah memahami secara intuitif bagaimana mengoptimalkan 'pabrik oksigen' alami untuk keberlanjutan hidup dan lingkungan."
\end{tcolorbox}

\section{PERTEMUAN KEDUA: Proses Fotosintesis dalam Sistem Pertanian Tradisional}

\subsection{Capaian Pembelajaran (Fase D)}
Pada akhir Fase D, murid memiliki kemampuan menjelaskan keterkaitan struktur jaringan penyusun organ pada sistem organ dan mengaitkannya dengan fungsinya, melalui berbagai kegiatan seperti pengamatan mikroskopis, percobaan, dan penelusuran informasi dari berbagai sumber.

\subsection{Tujuan Pembelajaran (TP) Pertemuan 2:}
\textit{(Clue untuk Guru: Tujuan ini adalah kompas Anda. Pastikan setiap tahapan KESAN yang Anda lalui berkontribusi pada pencapaian tujuan-tujuan ini).}

Melalui model pembelajaran KESAN, peserta didik mampu:
\begin{itemize}
\item Menghubungkan fenomena kearifan lokal Minangkabau (sistem sawah bertingkat tradisional) dengan konteks proses fotosintesis dan respirasi tumbuhan. (Sintaks K)
\item Merumuskan pertanyaan investigatif mengenai hubungan antara kondisi lingkungan sawah dengan optimalisasi proses fotosintesis tanaman padi. (Sintaks E)
\item Mengumpulkan informasi mengenai proses fotosintesis dan respirasi dari sumber ilmiah serta sistem pertanian tradisional dari sumber kultural. (Sintaks S)
\item Menganalisis dan menyintesiskan hubungan sebab-akibat antara desain sawah bertingkat dengan optimalisasi fotosintesis untuk produktivitas pertanian. (Sintaks A)
\item Menyusun sebuah penjelasan analitis yang logis mengenai bagaimana sistem pertanian tradisional mendukung proses vital tumbuhan secara optimal. (Sintaks N)
\end{itemize}

\subsection{Pertanyaan Pemantik}
\textit{(Clue untuk Guru: Ajukan dua pertanyaan ini secara berurutan, berikan jeda agar siswa berpikir. Jangan langsung minta jawaban, biarkan pertanyaan ini menggantung untuk memicu rasa ingin tahu).}

\begin{itemize}
\item "Pernahkah kalian melihat sawah bertingkat yang indah di dataran tinggi Minangkabau? Mengapa petani tradisional repot-repot membuat teras-teras, padahal bisa saja menanam padi di lahan datar biasa?"
\item "Coba perhatikan tanaman padi di sawah: mengapa mereka tampak begitu hijau dan subur? Apa yang sebenarnya terjadi di dalam daun hijau itu yang membuat tanaman bisa tumbuh hanya dengan air, udara, dan sinar matahari?"
\end{itemize}

\section{Langkah-langkah Kegiatan Pembelajaran:}

\subsection{Kegiatan Pembuka (15 Menit)}
\begin{itemize}
\item Guru membuka pelajaran dengan salam, doa, dan memeriksa kehadiran.
\item \textbf{Asesmen Diagnostik Awal:} Guru membagikan lembar K-W-L.
    \begin{itemize}
    \item \textbf{Instruksi Guru:} "Ananda, setelah mengeksplorasi struktur organ tumbuhan dan apotek hidup, sekarang kita akan menyelidiki proses-proses yang terjadi dalam tumbuhan! Di lembar K-W-L ini, tulis di kolom \textbf{K (Tahu)} apa yang sudah kalian ketahui tentang bagaimana tumbuhan membuat makanan ATAU tentang sawah dan pertanian. Di kolom \textbf{W (Ingin Tahu)}, tulis apa yang membuat kalian penasaran tentang bagaimana tumbuhan bisa hidup hanya dengan air dan sinar matahari." \textit{(Clue: Ini membantu Anda melihat pemahaman awal siswa tentang fotosintesis dan pengalaman mereka dengan pertanian).}
    \end{itemize}
\item \textbf{Apersepsi:}
    \begin{itemize}
    \item \textbf{Instruksi Guru:} "Setelah mengungkap rahasia struktur tumbuhan dalam apotek hidup, sekarang kita akan menyelidiki 'pabrik makanan' alami yang ada di setiap daun hijau! Misi kita adalah memahami bagaimana proses fotosintesis dioptimalkan dalam sistem pertanian tradisional Minangkabau. Siap menjadi ahli fisiologi tumbuhan yang menghargai kearifan pertanian berkelanjutan?"
    \end{itemize}
\end{itemize}

\subsection{Kegiatan Inti (90 Menit) - SINTAKS MODEL KESAN}

\subsubsection{Tahap 1: (K) Kaitkan Konteks Kultural (15 menit)}
\begin{itemize}
\item \textbf{Aktivitas Guru:}
    \begin{itemize}
    \item Menampilkan video singkat (3-4 menit) sistem sawah bertingkat tradisional Minangkabau, menunjukkan keindahan dan kecerdasan desain yang mengikuti kontur alam.
    \item Mengajukan Pertanyaan Pemantik yang sudah disiapkan di atas.
    \item \textit{(Clue: Fokuskan pada keingintahuan tentang mengapa desain sawah bertingkat begitu efektif dan indah. Saat siswa menjawab, tuliskan semua ide mereka di papan tulis dengan judul "\textbf{KECERDASAN SAWAH BERTINGKAT}". Validasi setiap kontribusi dan tunjukkan antusiasme terhadap pengamatan mereka).}
    \end{itemize}
\item \textbf{Aktivitas Siswa:} Mengamati video, mendengarkan pertanyaan, lalu secara sukarela berbagi pengalaman atau pengetahuan tentang sawah atau pertanian yang pernah mereka lihat. Menuliskan minimal satu pertanyaan atau pengamatan di sticky notes dan menempelkannya di '\textbf{Papan Penasaran}'.
\end{itemize}

\subsubsection{Tahap 2: (E) Eksplorasi Enigma (15 menit)}
\begin{itemize}
\item \textbf{Aktivitas Guru:} Membentuk siswa menjadi kelompok (3-4 orang).
    \begin{itemize}
    \item \textbf{Instruksi Guru:} "Keingintahuan kalian tentang sistem pertanian tradisional sangat menginspirasi! Sekarang, sebagai tim peneliti agronomi tradisional, tugas kalian adalah mengubah keingintahuan ini menjadi misi investigasi yang sistematis. Dalam kelompok, diskusikan dan rumuskan minimal 3 pertanyaan kunci yang akan kita selidiki hari ini. Tuliskan dalam bentuk '\textbf{Misi Investigasi Tim [Nama Kelompok]}' di kertas plano."
    \item \textit{(Clue: Arahkan diskusi siswa agar pertanyaannya mencakup aspek 'bagaimana tumbuhan membuat makanan' dan 'mengapa sawah bertingkat efektif'. Jika kelompok kesulitan, berikan pancingan: "Kira-kira, apa yang dibutuhkan tumbuhan untuk membuat makanan? Dan bagaimana desain sawah membantu proses itu?").}
    \end{itemize}
\item \textbf{Aktivitas Siswa:} Berdiskusi dalam tim untuk merumuskan misi investigasi (daftar pertanyaan kunci) di kertas plano. \textit{(Contoh misi yang diharapkan: 1. Bagaimana proses fotosintesis terjadi? 2. Mengapa sawah bertingkat lebih efektif? 3. Apa hubungan antara cahaya matahari, air, dan produktivitas padi?)}
\end{itemize}

\subsubsection{Tahap 3: (S) Selidiki secara Sintetis (25 menit)}
\begin{itemize}
\item \textbf{Aktivitas Guru:} Membagikan "\textbf{LKPD 08 - Jurnal Investigasi Dual-Lensa}".
    \begin{itemize}
    \item \textbf{Instruksi Guru:} "Setiap tim akan melakukan investigasi dari dua lensa. Gunakan sumber yang disediakan untuk mencari jawabannya. Bagilah tugas dalam tim!"
    \item \textbf{Lensa Sains:} Buka link video bit.ly/animasi-fotosintesis untuk memahami proses fotosintesis (6CO₂ + 6H₂O + cahaya → C₆H₁₂O₆ + 6O₂), respirasi, dan faktor-faktor yang mempengaruhi fotosintesis.
    \item \textbf{Lensa Etnosains/Kultural:} Buka link artikel bit.ly/sawah-terasering-minang untuk memahami sejarah, konstruksi, sistem irigasi, dan keunggulan sawah bertingkat dalam budaya Minangkabau.
    \item \textit{(Clue: Pastikan sumber belajar sudah disiapkan dan dapat diakses. Berkelilinglah untuk memastikan setiap kelompok mengeksplorasi kedua lensa dengan fokus pada optimalisasi kondisi pertumbuhan).}
    \end{itemize}
\item \textbf{Aktivitas Siswa:} Dalam kelompok, siswa berbagi tugas mencari informasi dari sumber yang diberikan dan mencatat temuan kunci di dua kolom terpisah pada "\textbf{LKPD 08 - Jurnal Investigasi Dual-Lensa}".
\end{itemize}

\subsubsection{Tahap 4: (A) Asimilasi Analitis (20 menit)}
\begin{itemize}
\item \textbf{Aktivitas Guru:} Memfasilitasi diskusi untuk menjembatani kedua lensa.
    \begin{itemize}
    \item \textbf{Pertanyaan Pancingan Kunci untuk Guru:}
        \begin{itemize}
        \item "Dari Lensa Sains kita tahu fotosintesis membutuhkan cahaya, CO₂, dan air. Dari Lensa Etnosains, kita tahu sawah bertingkat mengoptimalkan paparan sinar matahari dan sistem air. Nah, coba hubungkan! Bagaimana desain sawah bertingkat mendukung proses fotosintesis?"
        \item "Dari Lensa Sains, kita tahu fotosintesis menghasilkan oksigen dan glukosa. Dari Lensa Budaya, kita tahu sawah bertingkat menghasilkan padi berkualitas tinggi. Mengapa sistem tradisional ini bisa mengoptimalkan 'pabrik makanan' alami tumbuhan?"
        \end{itemize}
    \item \textit{(Clue: Bantu siswa melihat bahwa sawah bertingkat memaksimalkan paparan cahaya matahari, mengatur distribusi air yang optimal, dan mencegah erosi yang bisa mengganggu pertumbuhan tanaman).}
    \end{itemize}
\item \textbf{Aktivitas Siswa:} Berdiskusi intensif untuk menghubungkan temuan sains dan budaya. Menuliskan kesimpulan terpadu (sintesis) mereka di kertas plano.
\end{itemize}

\subsubsection{Tahap 5: (N) Nyatakan Pemahaman (15 menit)}
\begin{itemize}
\item \textbf{Aktivitas Guru:} Memberikan studi kasus individual atau per kelompok.
    \begin{itemize}
    \item \textbf{Instruksi Guru (tuliskan di papan tulis):}
    
    "\textbf{STUDI KASUS UNTUK AHLI AGRONOMI SAINS-BUDAYA:}
    
    Sebuah komunitas petani muda ingin mengembangkan sistem pertanian berkelanjutan yang ramah lingkungan. Mereka tertarik dengan konsep sawah bertingkat tradisional namun ingin memahami dasar ilmiahnya untuk mengoptimalkan produktivitas. Mereka meminta bantuanmu sebagai konsultan yang memahami baik fisiologi tumbuhan maupun pertanian tradisional.
    
    Tugasmu: Tuliskan sebuah rekomendasi ilmiah singkat (3-5 kalimat) di buku latihanmu tentang bagaimana prinsip fotosintesis dapat dioptimalkan melalui desain sawah bertingkat untuk pertanian berkelanjutan. Jelaskan mengapa pendekatan tradisional ini relevan untuk pertanian modern yang ramah lingkungan."
    \end{itemize}
\item \textbf{Aktivitas Siswa:} Menyusun argumen tertulis untuk menjawab studi kasus yang diberikan, menggunakan bukti dari kedua lensa.
\end{itemize}

\subsection{Kegiatan Penutup (15 Menit)}
\begin{itemize}
\item \textbf{Presentasi \& Penguatan:} Guru meminta 2-3 siswa secara acak untuk membacakan rekomendasi ilmiah mereka. Guru memberikan pujian dan penguatan positif, menekankan betapa hebatnya kemampuan mereka mengintegrasikan pengetahuan fisiologi tumbuhan dengan kearifan pertanian tradisional.
\item \textbf{Refleksi:}
    \begin{itemize}
    \item \textbf{Instruksi Guru:} "Sekarang, kembali ke lembar K-W-L kalian. Lengkapi kolom terakhir, \textbf{L (Learned)}, dengan hal-hal baru yang paling mengagumkan yang kalian pelajari hari ini tentang fotosintesis dan pertanian tradisional."
    \item \textbf{Instruksi Guru:} "Angkat tangan, siapa yang setelah belajar hari ini jadi lebih menghargai kecerdasan nenek moyang kita dalam menciptakan sistem pertanian yang berkelanjutan dan ramah lingkungan?"
    \end{itemize}
\item \textbf{Tindak Lanjut:}
    \begin{itemize}
    \item \textbf{Instruksi Guru:} "Luar biasa, para ahli agronomi tradisional! Hari ini kita sudah mengungkap bagaimana proses fotosintesis dioptimalkan dalam sistem pertanian tradisional. Pertemuan berikutnya, kita akan mengeksplorasi sistem pencernaan manusia dan bagaimana makanan yang dihasilkan tumbuhan melalui fotosintesis diolah dalam tubuh kita. Siapa tahu kita bisa menemukan keterkaitan yang menakjubkan antara 'pabrik makanan' tumbuhan dan 'pabrik energi' tubuh kita!"
    \item \textit{(Clue: Buat transisi yang menarik ke topik berikutnya sambil mempertahankan semangat eksplorasi sains dan budaya).}
    \end{itemize}
\item Guru menutup pelajaran dengan doa dan salam.
\end{itemize}

\section{Asesmen (Penilaian)}

\begin{itemize}
\item \textbf{Asesmen Diagnostik (Awal):} Analisis lembar K-W-L. (Untuk mengetahui pemahaman awal tentang fotosintesis dan pertanian).
\item \textbf{Asesmen Formatif (Proses):} Observasi keaktifan diskusi (gotong royong) dan penilaian kelengkapan "\textbf{LKPD 08 - Jurnal Investigasi Dual-Lensa}".
\item \textbf{Asesmen Sumatif (Akhir Siklus):} Penilaian jawaban studi kasus (Tahap N) menggunakan rubrik.
\end{itemize}

\subsection{Rubrik Penilaian Jawaban Studi Kasus (Tahap N)}

\begin{longtable}{|p{3cm}|p{3cm}|p{3cm}|p{3cm}|p{3cm}|}
\hline
\textbf{Kriteria Penilaian} & \textbf{Skor 4 (Sangat Baik)} & \textbf{Skor 3 (Baik)} & \textbf{Skor 2 (Cukup)} & \textbf{Skor 1 (Kurang)} \\
\hline
\textbf{Ketepatan Konsep Ilmiah} & Menggunakan konsep fotosintesis, respirasi, dan faktor-faktor yang mempengaruhi dengan sangat tepat dan relevan untuk optimalisasi pertanian. & Menggunakan konsep fotosintesis dengan tepat, namun kurang relevan dengan konteks pertanian berkelanjutan. & Menggunakan konsep fisiologi tumbuhan namun ada beberapa kesalahan dalam menjelaskan proses fotosintesis. & Tidak menggunakan konsep fotosintesis atau salah total. \\
\hline
\textbf{Keterkaitan dengan Etnosains} & Mampu menghubungkan secara logis dan eksplisit antara proses fotosintesis dengan sistem pertanian tradisional secara mendalam dan akurat, menunjukkan apresiasi terhadap kearifan agronomi lokal. & Mampu menghubungkan konsep fisiologi tumbuhan dengan pertanian tradisional, namun kurang mendalam dalam apresiasi kearifan lokal. & Hanya menyebutkan pertanian tradisional tanpa menghubungkan dengan konsep fotosintesis, atau sebaliknya. & Tidak ada keterkaitan antara sains dan agronomi tradisional yang ditunjukkan. \\
\hline
\textbf{Kelogisan \& Struktur Argumen} & Rekomendasi sangat logis, runtut, berkelanjutan, dan menunjukkan pemahaman mendalam tentang relevansi tradisi untuk modernitas. & Rekomendasi logis dan runtut, namun kurang mendalam dalam aspek keberlanjutan. & Alur rekomendasi kurang runtut atau kurang praktis dalam implementasi modern. & Rekomendasi tidak logis dan tidak terstruktur. \\
\hline
\end{longtable}

\section{Daftar Pustaka Sumber Etnosains}

\begin{enumerate}
\item Dt. Rajo Penghulu. (1994). \textit{Teknologi Tradisional Minangkabau}. Padang: Pusat Dokumentasi dan Informasi Kebudayaan Minangkabau.
\item Navis, A.A. (1984). \textit{Alam Terkembang Jadi Guru: Adat dan Kebudayaan Minangkabau}. Jakarta: Grafiti Pers.
\item Kementerian Pendidikan dan Kebudayaan. (2018). \textit{Sistem Pertanian Tradisional Nusantara}. Jakarta: Direktorat Warisan dan Diplomasi Budaya.
\item Syafwan, A. (2008). \textit{Kearifan Lokal dalam Teknologi Pertanian Minangkabau}. Padang: Universitas Andalas Press.
\item Arsyad, S. (2010). \textit{Konservasi Tanah dan Air}. Bogor: IPB Press.
\item Suripin. (2004). \textit{Sistem Drainase Perkotaan yang Berkelanjutan}. Yogyakarta: Andi Offset.
\end{enumerate}

\end{document}