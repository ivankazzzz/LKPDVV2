\documentclass[a4paper,12pt]{article}
\usepackage[a4paper, margin=1.27cm]{geometry}
\usepackage[indonesian]{babel}
\usepackage[utf8]{inputenc}
\usepackage{tcolorbox}
\usepackage{array}
\usepackage{multirow}
\usepackage{setspace}
\usepackage{amssymb}
\usepackage{graphicx}
\usepackage{adjustbox}
\usepackage{enumitem}
\usepackage{longtable}
\usepackage{xcolor}
\usepackage{amsmath}
\usepackage{fancyhdr}
\usepackage{titlesec}

% Define colors (black and white theme)
\definecolor{darkgray}{RGB}{64, 64, 64}
\definecolor{lightgray}{RGB}{240, 240, 240}
\definecolor{mediumgray}{RGB}{128, 128, 128}

% Custom tcolorbox styles
\tcbset{
    mainbox/.style={
        colback=white,
        colframe=black,
        boxrule=1pt,
        arc=3pt,
        left=8pt,
        right=8pt,
        top=8pt,
        bottom=8pt
    },
    sectionbox/.style={
        colback=white,
        colframe=black,
        boxrule=1.5pt,
        arc=2pt,
        left=6pt,
        right=6pt,
        top=6pt,
        bottom=6pt
    }
}

\begin{document}

\begin{center}
{\Large\textbf{MODUL AJAR: Pengawetan Tradisional Minang \& Zat Aditif Makanan}}
\end{center}

\vspace{0.5cm}

\begin{tcolorbox}[mainbox]
\textbf{Nama Penyusun:} Irfan Ananda Ismail, S.Pd., M.Pd., Gr \\
\textbf{Institusi:} SMP \\
\textbf{Mata Pelajaran:} Ilmu Pengetahuan Alam (IPA) \\
\textbf{Tahun Ajaran:} 2025/2026 \\
\textbf{Semester:} Ganjil \\
\textbf{Jenjang Sekolah:} SMP \\
\textbf{Kelas/Fase:} VIII / D \\
\textbf{Alokasi waktu:} 3 x 40 Menit (1 Pertemuan)
\end{tcolorbox}

\section{DIMENSI PROFIL PELAJAR PANCASILA}
\textit{(Clue untuk Guru: Sebutkan dimensi ini secara eksplisit saat apersepsi agar siswa sadar tujuan non-akademis yang sedang mereka kembangkan).}

\begin{itemize}
\item \textbf{Berkebinekaan Global:} Mengenal dan menghargai budaya, khususnya menganalisis kearifan pengawetan makanan tradisional Minangkabau dan menghubungkannya dengan konsep zat aditif dalam sains modern.
\item \textbf{Bernalar Kritis:} Ditempa secara intensif saat menganalisis informasi dari sumber sains dan budaya (Tahap S), menyintesiskan kedua perspektif (Tahap A), dan menyusun argumen berbasis bukti (Tahap N).
\item \textbf{Gotong Royong:} Kemampuan untuk bekerja sama secara kolaboratif dalam kelompok untuk merumuskan masalah (Tahap E), melakukan investigasi (Tahap S), dan membangun pemahaman bersama (Tahap A).
\item \textbf{Kreatif:} Menghasilkan argumen atau solusi orisinal yang terintegrasi pada tahap akhir pembelajaran (Tahap N).
\end{itemize}

\section{Sarana dan Prasarana}

\begin{itemize}
\item \textbf{Media:} LKPD 09 - Jurnal Investigasi Dual-Lensa, contoh makanan awetan tradisional Minang (dendeng, ikan asin, asinan, kerupuk, dll), gambar/poster zat aditif makanan, artikel tentang teknik pengawetan tradisional, label makanan kemasan.
\item \textbf{Alat:} Papan tulis/whiteboard, spidol, Proyektor \& Speaker, Kertas Plano atau Karton (1 per kelompok), sticky notes warna-warni, penggaris, kaca pembesar, pH meter sederhana.
\item \textbf{Sumber Belajar:} Buku ajar IPA kelas VIII, Tautan video animasi zat aditif makanan (misal: bit.ly/animasi-zat-aditif), tautan artikel/video pengawetan tradisional Minang (misal: bit.ly/pengawetan-tradisional-minang).
\end{itemize}

\section{Target Peserta didik}

\begin{itemize}
\item Peserta didik reguler kelas VIII (Fase D).
\end{itemize}

\section{Model Pembelajaran}

\begin{itemize}
\item Model Pembelajaran KESAN (Konektivitas Etnosains-Sains).
\end{itemize}

\section{Pemahaman Bermakna}
\textit{(Clue untuk Guru: Bacakan atau sampaikan narasi ini dengan intonasi yang menarik di akhir pembelajaran untuk mengikat semua pengalaman belajar siswa menjadi satu kesatuan yang bermakna).}

\begin{tcolorbox}[sectionbox]
"Ananda Semua, hari ini kita telah mengungkap kearifan luar biasa di balik teknik pengawetan makanan nenek moyang Minangkabau! Kita menemukan bahwa cara mereka mengawetkan makanan - dengan garam untuk dendeng, asam untuk asinan, dan pengeringan untuk kerupuk - ternyata menggunakan prinsip-prinsip ilmiah yang sama dengan zat aditif modern. Setiap teknik pengawetan tradisional sebenarnya memanfaatkan zat aditif alami untuk menghambat pertumbuhan mikroorganisme perusak. Dengan ini, kita sadar bahwa nenek moyang kita telah memahami konsep kimia makanan jauh sebelum ilmu pengetahuan modern berkembang, dan kearifan ini dapat menjadi inspirasi untuk pengembangan zat aditif yang lebih aman dan alami."
\end{tcolorbox}

\section{PERTEMUAN PERTAMA: Zat Aditif Makanan dan Kearifan Pengawetan Tradisional Minangkabau}

\subsection{Capaian Pembelajaran (Fase D)}
Pada akhir Fase D, murid memiliki kemampuan menganalisis sistem organisasi kehidupan, fungsi, serta kelainan atau gangguan yang muncul pada sistem organ makhluk hidup.

\subsection{Tujuan Pembelajaran (TP) Pertemuan 1:}
\textit{(Clue untuk Guru: Tujuan ini adalah kompas Anda. Pastikan setiap tahapan KESAN yang Anda lalui berkontribusi pada pencapaian tujuan-tujuan ini).}

Melalui model pembelajaran KESAN, peserta didik mampu:
\begin{itemize}
\item Menghubungkan fenomena kearifan lokal Minangkabau (teknik pengawetan makanan tradisional) dengan konteks zat aditif makanan dalam sains modern. (Sintaks K)
\item Merumuskan pertanyaan investigatif mengenai zat aditif makanan dan perbandingan teknik pengawetan tradisional dengan pengawetan modern. (Sintaks E)
\item Mengumpulkan informasi mengenai jenis dan fungsi zat aditif dari sumber sains serta teknik pengawetan tradisional dari sumber kultural. (Sintaks S)
\item Menganalisis dan menyintesiskan hubungan sebab-akibat antara teknik pengawetan tradisional Minangkabau dengan prinsip kerja zat aditif modern. (Sintaks A)
\item Menyusun sebuah penjelasan analitis yang logis mengenai bagaimana kearifan pengawetan tradisional Minangkabau dapat menginspirasi pengembangan zat aditif yang lebih aman dan alami. (Sintaks N)
\end{itemize}

\subsection{Pertanyaan Pemantik}
\textit{(Clue untuk Guru: Ajukan dua pertanyaan ini secara berurutan, berikan jeda agar siswa berpikir. Jangan langsung minta jawaban, biarkan pertanyaan ini menggantung untuk memicu rasa ingin tahu).}

\begin{itemize}
\item "Pernahkah kalian memperhatikan mengapa dendeng bisa tahan berbulan-bulan tanpa kulkas, sementara daging segar hanya tahan beberapa hari? Atau mengapa asinan bisa awet lama padahal tidak pakai pengawet buatan? Apa rahasianya?"
\item "Dalam budaya Minang ada filosofi 'Nan basi dibuang, nan elok disimpan' - yang busuk dibuang, yang baik disimpan. Dari sudut pandang kimia, bagaimana sebenarnya nenek moyang kita bisa menyimpan makanan dalam jangka waktu lama tanpa teknologi modern?"
\end{itemize}

\section{Langkah-langkah Kegiatan Pembelajaran:}

\subsection{Kegiatan Pembuka (15 Menit)}
\begin{itemize}
\item Guru membuka pelajaran dengan salam, doa, dan memeriksa kehadiran.
\item \textbf{Asesmen Diagnostik Awal:} Guru membagikan lembar K-W-L.
    \begin{itemize}
    \item \textbf{Instruksi Guru:} "Ananda, sebelum kita mulai menjelajahi dunia zat aditif, tolong isi dua kolom pertama di lembar ini. Di kolom \textbf{K (Tahu)}, tulis apa saja yang sudah kalian ketahui tentang zat aditif makanan ATAU cara mengawetkan makanan tradisional. Di kolom \textbf{W (Ingin Tahu)}, tulis apa yang membuat kalian penasaran tentang topik ini." \textit{(Clue: Ini membantu Anda memetakan pengetahuan awal dan minat siswa secara cepat).}
    \end{itemize}
\item \textbf{Apersepsi:}
    \begin{itemize}
    \item \textbf{Instruksi Guru:} "Hari ini kita akan menjadi ahli teknologi pangan tradisional dan modern sekaligus. Kita akan menyelidiki bagaimana teknik pengawetan makanan Minangkabau berhubungan dengan zat aditif dalam makanan modern. Dalam investigasi ini, kita akan melatih kemampuan \textbf{Bernalar Kritis} kita, menghargai budaya lewat \textbf{Berkebinekaan Global}, dan bekerja sama dalam semangat \textbf{Gotong Royong}. Siap menjadi peneliti teknologi pangan?"
    \end{itemize}
\end{itemize}

\subsection{Kegiatan Inti (90 Menit) - SINTAKS MODEL KESAN}

\subsubsection{Tahap 1: (K) Kaitkan Konteks Kultural (15 menit)}
\begin{itemize}
\item \textbf{Aktivitas Guru:}
    \begin{itemize}
    \item Menampilkan contoh makanan awetan tradisional Minang (dendeng, ikan asin, asinan, kerupuk) dan menjelaskan proses pembuatannya secara singkat.
    \item Mengajukan Pertanyaan Pemantik yang sudah disiapkan di atas.
    \item \textit{(Clue: Tujuan tahap ini adalah memvalidasi pengetahuan siswa dan memantik rasa heran, bukan mencari jawaban benar. Sediakan spidol dan papan tulis/whiteboard. Saat siswa menjawab, tuliskan semua ide mereka, bahkan yang keliru sekalipun, dengan judul "\textbf{PENGETAHUAN AWAL KITA}". Ini menunjukkan bahwa semua pemikiran dihargai).}
    \end{itemize}
\item \textbf{Aktivitas Siswa:} Mengamati contoh makanan awetan, mendengarkan pertanyaan, lalu secara sukarela berbagi pengalaman atau pengetahuan awal tentang pengawetan makanan tradisional atau zat aditif. Menuliskan minimal satu pertanyaan atau pengalaman di sticky notes dan menempelkannya di '\textbf{Papan Penasaran}'.
\end{itemize}

\subsubsection{Tahap 2: (E) Eksplorasi Enigma (15 menit)}
\begin{itemize}
\item \textbf{Aktivitas Guru:} Membentuk siswa menjadi kelompok (3-4 orang).
    \begin{itemize}
    \item \textbf{Instruksi Guru:} "Baik, rasa penasaran kalian luar biasa! Sekarang, tugas kita sebagai peneliti adalah mengubah rasa penasaran ini menjadi misi yang jelas. Dalam kelompok, diskusikan dan rumuskan minimal 3 pertanyaan kunci yang akan kita selidiki hari ini. Tuliskan dalam bentuk '\textbf{Misi Penelitian Tim [Nama Kelompok]}' di kertas plano yang Bapak/Ibu berikan."
    \item \textit{(Clue: Arahkan diskusi siswa agar pertanyaannya mencakup aspek 'apa itu zat aditif dan fungsinya' dan 'bagaimana teknik pengawetan tradisional bekerja'. Jika kelompok kesulitan, berikan pancingan: "Kira-kira, apa dulu yang perlu kita tahu? Jenis-jenis zat aditif atau cara kerja pengawetan tradisional?").}
    \end{itemize}
\item \textbf{Aktivitas Siswa:} Berdiskusi dalam tim untuk merumuskan misi penelitian (daftar pertanyaan kunci) di kertas plano. \textit{(Contoh misi yang diharapkan: 1. Apa itu zat aditif dan apa fungsinya dalam makanan? 2. Bagaimana cara kerja teknik pengawetan tradisional Minang? 3. Apa hubungan antara pengawetan tradisional dengan zat aditif modern?)}
\end{itemize}

\subsubsection{Tahap 3: (S) Selidiki secara Sintetis (25 menit)}
\begin{itemize}
\item \textbf{Aktivitas Guru:} Membagikan "\textbf{LKPD 09 - Jurnal Investigasi Dual-Lensa}".
    \begin{itemize}
    \item \textbf{Instruksi Guru:} "Setiap tim akan melakukan investigasi dari dua lensa. Gunakan HP atau sumber yang disediakan untuk mencari jawabannya. Bagilah tugas dalam tim!"
    \item \textbf{Lensa Sains:} Buka link video bit.ly/animasi-zat-aditif untuk memahami jenis-jenis zat aditif (pengawet, pewarna, pemanis, penguat rasa) dan fungsinya dalam makanan modern.
    \item \textbf{Lensa Etnosains/Kultural:} Buka link artikel bit.ly/pengawetan-tradisional-minang untuk memahami teknik pengawetan makanan tradisional Minangkabau seperti penggaraman, pengeringan, pengasaman, dan fermentasi.
    \item \textit{(Clue: Pastikan sumber belajar sudah disiapkan dan link bisa diakses. Berkelilinglah untuk memastikan setiap kelompok membagi tugas dan tidak hanya fokus pada satu lensa saja).}
    \end{itemize}
\item \textbf{Aktivitas Siswa:} Dalam kelompok, siswa berbagi tugas mencari informasi dari sumber yang diberikan dan mencatat temuan kunci di dua kolom terpisah pada "\textbf{LKPD 09 - Jurnal Investigasi Dual-Lensa}".
\end{itemize}

\subsubsection{Tahap 4: (A) Asimilasi Analitis (20 menit)}
\begin{itemize}
\item \textbf{Aktivitas Guru:} Memfasilitasi diskusi untuk menjembatani kedua lensa.
    \begin{itemize}
    \item \textbf{Pertanyaan Pancingan Kunci untuk Guru:}
        \begin{itemize}
        \item "Oke, dari Lensa Sains kita tahu zat pengawet berfungsi menghambat pertumbuhan bakteri dan jamur. Dari Lensa Etnosains, kita tahu garam pada dendeng dan asam pada asinan mengawetkan makanan. Nah, coba hubungkan! Apa kesamaan prinsipnya?"
        \item "Dari Lensa Sains, kita tahu zat aditif mengubah kondisi kimia makanan. Dari Lensa Budaya, teknik tradisional juga mengubah pH, kadar air, dan lingkungan makanan. Mengapa hal ini efektif mengawetkan?"
        \end{itemize}
    \item \textit{(Clue: Fokuskan untuk membuat siswa 'menemukan' hubungannya sendiri, bukan diberitahu. Gunakan kata "menurut kalian", "kira-kira kenapa", "ada yang punya ide?").}
    \end{itemize}
\item \textbf{Aktivitas Siswa:} Berdiskusi intensif untuk menghubungkan temuan sains dan budaya. Menuliskan kesimpulan terpadu (sintesis) mereka di kertas plano.
\end{itemize}

\subsubsection{Tahap 5: (N) Nyatakan Pemahaman (15 menit)}
\begin{itemize}
\item \textbf{Aktivitas Guru:} Memberikan studi kasus individual atau per kelompok.
    \begin{itemize}
    \item \textbf{Instruksi Guru (tuliskan di papan tulis):}
    
    "\textbf{STUDI KASUS UNTUK PENELITI SAINS-BUDAYA:}
    
    Sebuah perusahaan makanan ingin mengembangkan produk makanan awetan yang lebih sehat dan alami. Mereka tertarik dengan teknik pengawetan tradisional Minangkabau yang terbukti efektif dan aman dikonsumsi dalam jangka panjang. Tim R\&D perusahaan ingin memahami prinsip ilmiah di balik teknik tradisional ini agar bisa diadaptasi untuk produksi skala industri tanpa kehilangan nilai keamanan dan kealamiannya.
    
    Tugasmu: Tuliskan sebuah proposal singkat (4-6 kalimat) di buku latihanmu untuk membantu tim R\&D tersebut memahami prinsip ilmiah teknik pengawetan tradisional Minangkabau dan bagaimana hal ini bisa diadaptasi menjadi zat aditif alami untuk industri makanan modern. Gunakan pengetahuan gabungan dari sains dan kearifan tradisional yang baru saja kamu pelajari."
    \end{itemize}
\item \textbf{Aktivitas Siswa:} Menyusun argumen tertulis untuk menjawab studi kasus yang diberikan, menggunakan bukti dari kedua lensa.
\end{itemize}

\subsection{Kegiatan Penutup (15 Menit)}
\begin{itemize}
\item \textbf{Presentasi \& Penguatan:} Guru meminta 2-3 siswa secara acak untuk membacakan proposal mereka. Guru memberikan pujian dan penguatan positif, menekankan betapa hebatnya kemampuan mereka mengintegrasikan sains dan budaya.
\item \textbf{Refleksi:}
    \begin{itemize}
    \item \textbf{Instruksi Guru:} "Sekarang, kembali ke lembar K-W-L kalian. Lengkapi kolom terakhir, \textbf{L (Learned)}, dengan hal-hal baru yang paling mengagumkan yang kalian pelajari hari ini."
    \item \textbf{Instruksi Guru:} "Angkat tangan, siapa yang setelah belajar hari ini jadi lebih menghargai kearifan nenek moyang kita dalam mengawetkan makanan?"
    \end{itemize}
\item \textbf{Tindak Lanjut:}
    \begin{itemize}
    \item \textbf{Instruksi Guru:} "Luar biasa, para peneliti! Hari ini kita sudah mengungkap bagaimana kearifan pengawetan tradisional dapat menginspirasi teknologi pangan modern. Pertemuan berikutnya, kita akan mempelajari zat adiktif dan bahayanya. Siapa tahu ada kearifan Minang tentang menghindari hal-hal berbahaya yang bisa mengajarkan kita tentang pencegahan kecanduan? Kita akan lihat!"
    \item \textit{(Clue: Buat transisi yang menarik ke pertemuan berikutnya sambil mempertahankan semangat investigasi yang sudah terbangun).}
    \end{itemize}
\item Guru menutup pelajaran dengan doa dan salam.
\end{itemize}

\section{Asesmen (Penilaian)}

\begin{itemize}
\item \textbf{Asesmen Diagnostik (Awal):} Analisis lembar K-W-L. (Untuk mengetahui baseline siswa).
\item \textbf{Asesmen Formatif (Proses):} Observasi keaktifan diskusi (gotong royong) dan penilaian kelengkapan "\textbf{LKPD 09 - Jurnal Investigasi Dual-Lensa}".
\item \textbf{Asesmen Sumatif (Akhir Siklus):} Penilaian jawaban studi kasus (Tahap N) menggunakan rubrik.
\end{itemize}

\subsection{Rubrik Penilaian Jawaban Studi Kasus (Tahap N)}

\begin{longtable}{|p{3cm}|p{3cm}|p{3cm}|p{3cm}|p{3cm}|}
\hline
\textbf{Kriteria Penilaian} & \textbf{Skor 4 (Sangat Baik)} & \textbf{Skor 3 (Baik)} & \textbf{Skor 2 (Cukup)} & \textbf{Skor 1 (Kurang)} \\
\hline
\textbf{Ketepatan Konsep Ilmiah} & Menggunakan istilah ilmiah (zat aditif, pengawet, pH, kadar air, mikroorganisme) dengan sangat tepat dan relevan dengan kasus. & Menggunakan istilah ilmiah dengan tepat, namun kurang relevan. & Menggunakan istilah ilmiah namun ada beberapa kesalahan konsep. & Tidak menggunakan istilah ilmiah atau salah total. \\
\hline
\textbf{Keterkaitan dengan Etnosains} & Mampu menghubungkan secara logis dan eksplisit antara teknik pengawetan tradisional dengan prinsip zat aditif secara mendalam. & Mampu menghubungkan teknik pengawetan tradisional dengan zat aditif, namun kurang mendalam. & Hanya menyebutkan teknik pengawetan tanpa menghubungkan dengan zat aditif, atau sebaliknya. & Tidak ada keterkaitan antara sains dan budaya yang ditunjukkan. \\
\hline
\textbf{Kelogisan \& Struktur Argumen} & Proposal sangat logis, runtut, inovatif untuk industri, dan mudah dipahami. & Proposal logis dan runtut, namun kurang inovatif untuk industri. & Alur proposal kurang runtut atau sulit dipahami. & Proposal tidak logis dan tidak terstruktur. \\
\hline
\end{longtable}

\section{Daftar Pustaka Sumber Etnosains}

\begin{enumerate}
\item Syukur, C. (2015). \textit{Kuliner Tradisional Minangkabau: Filosofi dan Kearifan Lokal}. Padang: Andalas University Press.
\item Navis, A.A. (1984). \textit{Alam Terkembang Jadi Guru: Adat dan Kebudayaan Minangkabau}. Jakarta: Grafiti Pers.
\item BPOM RI. (2019). \textit{Peraturan Badan Pengawas Obat dan Makanan tentang Bahan Tambahan Pangan}. Jakarta: BPOM RI.
\item Winarno, F.G. (2008). \textit{Kimia Pangan dan Gizi}. Jakarta: Gramedia Pustaka Utama.
\end{enumerate}

\end{document}