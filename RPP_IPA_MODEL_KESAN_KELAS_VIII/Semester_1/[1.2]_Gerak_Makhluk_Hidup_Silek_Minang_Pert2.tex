\documentclass[a4paper,12pt]{article}
\usepackage[a4paper, margin=1.27cm]{geometry}
\usepackage[indonesian]{babel}
\usepackage[utf8]{inputenc}
\usepackage{tcolorbox}
\usepackage{array}
\usepackage{multirow}
\usepackage{setspace}
\usepackage{amssymb}
\usepackage{graphicx}
\usepackage{adjustbox}
\usepackage{enumitem}
\usepackage{longtable}
\usepackage{xcolor}
\usepackage{amsmath}
\usepackage{fancyhdr}
\usepackage{titlesec}

% Define colors (black and white theme)
\definecolor{darkgray}{RGB}{64, 64, 64}
\definecolor{lightgray}{RGB}{240, 240, 240}
\definecolor{mediumgray}{RGB}{128, 128, 128}

% Custom tcolorbox styles
\tcbset{
    mainbox/.style={
        colback=white,
        colframe=black,
        boxrule=1pt,
        arc=3pt,
        left=8pt,
        right=8pt,
        top=8pt,
        bottom=8pt
    },
    sectionbox/.style={
        colback=white,
        colframe=black,
        boxrule=1.5pt,
        arc=2pt,
        left=6pt,
        right=6pt,
        top=6pt,
        bottom=6pt
    }
}

\begin{document}

\begin{center}
{\Large\textbf{MODUL AJAR: Kekuatan Silek Harimau \& Adaptasi Gerak Makhluk Hidup}}
\end{center}

\vspace{0.5cm}

\begin{tcolorbox}[mainbox]
\textbf{Nama Penyusun:} Irfan Ananda Ismail, S.Pd., M.Pd., Gr \\
\textbf{Institusi:} SMP \\
\textbf{Mata Pelajaran:} Ilmu Pengetahuan Alam (IPA) \\
\textbf{Tahun Ajaran:} 2025/2026 \\
\textbf{Semester:} Ganjil \\
\textbf{Jenjang Sekolah:} SMP \\
\textbf{Kelas/Fase:} VIII / D \\
\textbf{Alokasi waktu:} 3 x 40 Menit (1 Pertemuan)
\end{tcolorbox}

\section{DIMENSI PROFIL PELAJAR PANCASILA}
\textit{(Clue untuk Guru: Sebutkan dimensi ini secara eksplisit saat apersepsi agar siswa sadar tujuan non-akademis yang sedang mereka kembangkan).}

\begin{itemize}
\item \textbf{Berkebinekaan Global:} Mengenal dan menghargai budaya, khususnya menganalisis kearifan lokal Minangkabau dalam seni bela diri tradisional (Silek), lalu menghubungkannya dengan konsep sains universal tentang adaptasi gerak makhluk hidup.
\item \textbf{Bernalar Kritis:} Ditempa secara intensif saat menganalisis informasi dari sumber sains dan budaya (Tahap S), menyintesiskan kedua perspektif (Tahap A), dan menyusun argumen berbasis bukti (Tahap N).
\item \textbf{Gotong Royong:} Kemampuan untuk bekerja sama secara kolaboratif dalam kelompok untuk merumuskan masalah (Tahap E), melakukan investigasi (Tahap S), dan membangun pemahaman bersama (Tahap A).
\item \textbf{Kreatif:} Menghasilkan argumen atau solusi orisinal yang terintegrasi pada tahap akhir pembelajaran (Tahap N).
\end{itemize}

\section{Sarana dan Prasarana}

\begin{itemize}
\item \textbf{Media:} LKPD 02 - Jurnal Investigasi Dual-Lensa, video singkat demonstrasi Silek Harimau Minangkabau (durasi 3-4 menit), gambar/diagram perbandingan gerak hewan dan manusia, artikel tentang filosofi gerakan hewan dalam Silek Minang, poster/gambar berbagai jenis gerak makhluk hidup.
\item \textbf{Alat:} Papan tulis/whiteboard, spidol, Proyektor \& Speaker, Kertas Plano atau Karton (1 per kelompok), sticky notes warna-warni.
\item \textbf{Sumber Belajar:} Buku ajar IPA kelas VIII, Tautan video animasi gerak hewan (misal: bit.ly/animasi-gerak-hewan), tautan artikel/video filosofi Silek Harimau (misal: bit.ly/silek-harimau-minang).
\end{itemize}

\section{Target Peserta didik}

\begin{itemize}
\item Peserta didik reguler kelas VIII (Fase D).
\end{itemize}

\section{Model Pembelajaran}

\begin{itemize}
\item Model Pembelajaran KESAN (Konektivitas Etnosains-Sains).
\end{itemize}

\section{Pemahaman Bermakna}
\textit{(Clue untuk Guru: Bacakan atau sampaikan narasi ini dengan intonasi yang menarik di akhir pembelajaran untuk mengikat semua pengalaman belajar siswa menjadi satu kesatuan yang bermakna).}

\begin{tcolorbox}[sectionbox]
"Ananda Semua, hari ini kita telah mengungkap rahasia di balik kekuatan Silek Harimau! Kita menemukan bahwa setiap gerakan dalam seni bela diri tradisional Minangkabau ini - dari cakaran yang tajam, lompatan yang gesit, hingga sikap bertahan yang kokoh - semuanya terinspirasi dari cara hewan-hewan bergerak dan beradaptasi di alam. Dengan ini, kita sadar bahwa nenek moyang kita adalah ilmuwan sejati yang mengamati alam dan mengembangkan kearifan lokal berdasarkan prinsip-prinsip sains yang sama dengan yang kita pelajari hari ini."
\end{tcolorbox}

\section{PERTEMUAN KEDUA: Adaptasi Gerak dan Kearifan Lokal Minangkabau}

\subsection{Capaian Pembelajaran (Fase D)}
Pada akhir Fase D, murid memiliki kemampuan menganalisis sistem organisasi kehidupan, fungsi, serta kelainan atau gangguan yang muncul pada sistem organ makhluk hidup.

\subsection{Tujuan Pembelajaran (TP) Pertemuan 2:}
\textit{(Clue untuk Guru: Tujuan ini adalah kompas Anda. Pastikan setiap tahapan KESAN yang Anda lalui berkontribusi pada pencapaian tujuan-tujuan ini).}

Melalui model pembelajaran KESAN, peserta didik mampu:
\begin{itemize}
\item Menghubungkan fenomena kearifan lokal Minangkabau (gerakan dalam Silek Harimau) dengan konteks adaptasi gerak berbagai makhluk hidup. (Sintaks K)
\item Merumuskan pertanyaan investigatif mengenai cara berbagai makhluk hidup beradaptasi dalam bergerak dan bagaimana hal ini terinspirasi dalam seni bela diri tradisional. (Sintaks E)
\item Mengumpulkan informasi mengenai jenis-jenis gerak makhluk hidup dari sumber ilmiah serta makna filosofis gerakan hewan dalam Silek dari sumber kultural. (Sintaks S)
\item Menganalisis dan menyintesiskan hubungan sebab-akibat antara adaptasi gerak hewan dengan teknik gerakan dalam Silek Harimau. (Sintaks A)
\item Menyusun sebuah penjelasan analitis yang logis mengenai bagaimana kearifan lokal dalam seni bela diri Minangkabau mencerminkan pemahaman mendalam tentang adaptasi gerak makhluk hidup. (Sintaks N)
\end{itemize}

\subsection{Pertanyaan Pemantik}
\textit{(Clue untuk Guru: Ajukan dua pertanyaan ini secara berurutan, berikan jeda agar siswa berpikir. Jangan langsung minta jawaban, biarkan pertanyaan ini menggantung untuk memicu rasa ingin tahu).}

\begin{itemize}
\item "Pernahkah kalian memperhatikan bahwa gerakan dalam Silek Harimau sangat mirip dengan cara harimau sesungguhnya bergerak di alam liar? Mengapa nenek moyang kita memilih meniru gerakan hewan-hewan tertentu untuk seni bela diri?"
\item "Dari sudut pandang sains, apa yang membuat gerakan harimau, elang, atau ular begitu efektif sehingga manusia bisa belajar dan mengadaptasinya untuk kebutuhan bertahan hidup?"
\end{itemize}

\section{Langkah-langkah Kegiatan Pembelajaran:}

\subsection{Kegiatan Pembuka (15 Menit)}
\begin{itemize}
\item Guru membuka pelajaran dengan salam, doa, dan memeriksa kehadiran.
\item \textbf{Asesmen Diagnostik Awal:} Guru membagikan lembar K-W-L.
    \begin{itemize}
    \item \textbf{Instruksi Guru:} "Ananda, kemarin kita sudah menyelidiki sistem gerak dalam tubuh manusia melalui Tari Piring. Hari ini kita akan melanjutkan petualangan kita! Di lembar K-W-L ini, tulis di kolom \textbf{K (Tahu)} apa yang sudah kalian ketahui tentang cara hewan bergerak ATAU tentang seni bela diri tradisional. Di kolom \textbf{W (Ingin Tahu)}, tulis apa yang membuat kalian penasaran tentang topik hari ini." \textit{(Clue: Ini membantu Anda melihat koneksi yang siswa buat dengan pembelajaran sebelumnya).}
    \end{itemize}
\item \textbf{Apersepsi:}
    \begin{itemize}
    \item \textbf{Instruksi Guru:} "Kemarin kita sudah menjadi detektif yang membongkar rahasia Tari Piring dan sistem gerak manusia. Hari ini, misi kita berlanjut! Kita akan menyelidiki bagaimana makhluk hidup lain bergerak dan bagaimana kearifan lokal Minangkabau dalam Silek Harimau mengajarkan kita tentang adaptasi gerak. Siap melanjutkan petualangan sains dan budaya kita?"
    \end{itemize}
\end{itemize}

\subsection{Kegiatan Inti (90 Menit) - SINTAKS MODEL KESAN}

\subsubsection{Tahap 1: (K) Kaitkan Konteks Kultural (15 menit)}
\begin{itemize}
\item \textbf{Aktivitas Guru:}
    \begin{itemize}
    \item Menampilkan video singkat (3-4 menit) demonstrasi Silek Harimau yang menyorot gerakan-gerakan yang terinspirasi dari hewan.
    \item Mengajukan Pertanyaan Pemantik yang sudah disiapkan di atas.
    \item \textit{(Clue: Fokuskan pada kesamaan gerakan antara hewan dan manusia dalam Silek. Saat siswa menjawab, tuliskan semua ide mereka di papan tulis dengan judul "\textbf{PENGAMATAN AWAL KITA}". Validasi setiap kontribusi siswa).}
    \end{itemize}
\item \textbf{Aktivitas Siswa:} Mengamati video, mendengarkan pertanyaan, lalu secara sukarela berbagi pengamatan tentang kesamaan gerakan hewan dan manusia. Menuliskan minimal satu pertanyaan atau pengamatan di sticky notes dan menempelkannya di '\textbf{Papan Penasaran}'.
\end{itemize}

\subsubsection{Tahap 2: (E) Eksplorasi Enigma (15 menit)}
\begin{itemize}
\item \textbf{Aktivitas Guru:} Membentuk siswa menjadi kelompok (3-4 orang).
    \begin{itemize}
    \item \textbf{Instruksi Guru:} "Pengamatan kalian sangat tajam! Sekarang, sebagai tim peneliti, tugas kalian adalah mengubah pengamatan ini menjadi misi investigasi yang fokus. Dalam kelompok, diskusikan dan rumuskan minimal 3 pertanyaan kunci yang akan kita selidiki hari ini. Tuliskan dalam bentuk '\textbf{Misi Investigasi Tim [Nama Kelompok]}' di kertas plano."
    \item \textit{(Clue: Arahkan diskusi siswa agar pertanyaannya mencakup aspek 'bagaimana hewan bergerak' dan 'mengapa gerakan hewan diadaptasi dalam Silek'. Jika kelompok kesulitan, berikan pancingan: "Kira-kira, apa yang membuat gerakan hewan begitu efektif? Dan bagaimana manusia bisa belajar dari mereka?").}
    \end{itemize}
\item \textbf{Aktivitas Siswa:} Berdiskusi dalam tim untuk merumuskan misi investigasi (daftar pertanyaan kunci) di kertas plano. \textit{(Contoh misi yang diharapkan: 1. Bagaimana cara berbagai hewan bergerak dan beradaptasi? 2. Apa keunggulan gerakan hewan yang bisa ditiru manusia? 3. Bagaimana filosofi Silek Harimau mengadaptasi gerakan hewan?)}
\end{itemize}

\subsubsection{Tahap 3: (S) Selidiki secara Sintetis (25 menit)}
\begin{itemize}
\item \textbf{Aktivitas Guru:} Membagikan "\textbf{LKPD 02 - Jurnal Investigasi Dual-Lensa}".
    \begin{itemize}
    \item \textbf{Instruksi Guru:} "Setiap tim akan melakukan investigasi dari dua lensa. Gunakan sumber yang disediakan untuk mencari jawabannya. Bagilah tugas dalam tim!"
    \item \textbf{Lensa Sains:} Buka link video bit.ly/animasi-gerak-hewan untuk memahami berbagai jenis gerak makhluk hidup (gerak otot, gerak tropisme, gerak taksis, dll). Amati juga poster/gambar yang menunjukkan adaptasi gerak berbagai hewan.
    \item \textbf{Lensa Etnosains/Kultural:} Buka link artikel bit.ly/silek-harimau-minang untuk memahami filosofi dan teknik gerakan dalam Silek Harimau yang terinspirasi dari hewan.
    \item \textit{(Clue: Pastikan sumber belajar sudah disiapkan dan dapat diakses. Berkelilinglah untuk memastikan setiap kelompok mengeksplorasi kedua lensa secara seimbang).}
    \end{itemize}
\item \textbf{Aktivitas Siswa:} Dalam kelompok, siswa berbagi tugas mencari informasi dari sumber yang diberikan dan mencatat temuan kunci di dua kolom terpisah pada "\textbf{LKPD 02 - Jurnal Investigasi Dual-Lensa}".
\end{itemize}

\subsubsection{Tahap 4: (A) Asimilasi Analitis (20 menit)}
\begin{itemize}
\item \textbf{Aktivitas Guru:} Memfasilitasi diskusi untuk menjembatani kedua lensa.
    \begin{itemize}
    \item \textbf{Pertanyaan Pancingan Kunci untuk Guru:}
        \begin{itemize}
        \item "Dari Lensa Sains kita tahu hewan memiliki adaptasi gerak yang spesifik untuk bertahan hidup. Dari Lensa Etnosains, kita tahu Silek Harimau meniru gerakan hewan. Nah, coba hubungkan! Mengapa gerakan yang efektif untuk hewan juga efektif untuk manusia dalam bela diri?"
        \item "Dari Lensa Sains, kita tahu setiap hewan punya keunggulan gerak yang berbeda. Dari Lensa Budaya, Silek tidak hanya meniru satu hewan. Apa hubungannya keanekaragaman adaptasi gerak hewan dengan kekayaan teknik dalam Silek Minangkabau?"
        \end{itemize}
    \item \textit{(Clue: Bantu siswa melihat bahwa prinsip-prinsip sains yang sama (efisiensi energi, biomekanikal, adaptasi) berlaku baik untuk hewan maupun untuk seni bela diri manusia).}
    \end{itemize}
\item \textbf{Aktivitas Siswa:} Berdiskusi intensif untuk menghubungkan temuan sains dan budaya. Menuliskan kesimpulan terpadu (sintesis) mereka di kertas plano.
\end{itemize}

\subsubsection{Tahap 5: (N) Nyatakan Pemahaman (15 menit)}
\begin{itemize}
\item \textbf{Aktivitas Guru:} Memberikan studi kasus individual atau per kelompok.
    \begin{itemize}
    \item \textbf{Instruksi Guru (tuliskan di papan tulis):}
    
    "\textbf{STUDI KASUS UNTUK PENELITI SAINS-BUDAYA:}
    
    Sebuah komunitas pencinta alam ingin mengembangkan program latihan fisik yang terinspirasi dari gerakan hewan untuk meningkatkan kebugaran dan kemampuan bertahan hidup di alam. Mereka meminta bantuanmu sebagai ahli yang memahami baik sains maupun budaya lokal.
    
    Tugasmu: Tuliskan sebuah rekomendasi singkat (3-5 kalimat) di buku latihanmu. Pilih 2 jenis hewan dan jelaskan mengapa gerakan mereka efektif secara sains, lalu berikan contoh bagaimana gerakan tersebut sudah diadaptasi dalam Silek Minangkabau. Gunakan pengetahuan gabungan dari kedua lensa yang baru saja kamu pelajari."
    \end{itemize}
\item \textbf{Aktivitas Siswa:} Menyusun argumen tertulis untuk menjawab studi kasus yang diberikan, menggunakan bukti dari kedua lensa.
\end{itemize}

\subsection{Kegiatan Penutup (15 Menit)}
\begin{itemize}
\item \textbf{Presentasi \& Penguatan:} Guru meminta 2-3 siswa secara acak untuk membacakan rekomendasi mereka. Guru memberikan pujian dan penguatan positif, menekankan betapa hebatnya kemampuan mereka mengintegrasikan sains dan budaya.
\item \textbf{Refleksi:}
    \begin{itemize}
    \item \textbf{Instruksi Guru:} "Sekarang, kembali ke lembar K-W-L kalian. Lengkapi kolom terakhir, \textbf{L (Learned)}, dengan hal-hal baru yang paling mengejutkan yang kalian pelajari hari ini."
    \item \textbf{Instruksi Guru:} "Angkat tangan, siapa yang setelah belajar hari ini jadi lebih menghargai kearifan nenek moyang kita dalam mengamati alam?"
    \end{itemize}
\item \textbf{Tindak Lanjut:}
    \begin{itemize}
    \item \textbf{Instruksi Guru:} "Luar biasa, para peneliti! Dalam dua pertemuan ini kita sudah mengungkap rahasia gerak manusia dan hewan melalui kearifan lokal Minangkabau. Pertemuan berikutnya, kita akan beralih ke topik baru: bagaimana benda-benda di sekitar kita bergerak. Siapa tahu ada teknologi tradisional Minang yang bisa mengajarkan kita tentang gerak benda? Kita akan lihat!"
    \item \textit{(Clue: Buat transisi yang smooth ke topik berikutnya sambil tetap mempertahankan semangat investigasi yang sudah terbangun).}
    \end{itemize}
\item Guru menutup pelajaran dengan doa dan salam.
\end{itemize}

\section{Asesmen (Penilaian)}

\begin{itemize}
\item \textbf{Asesmen Diagnostik (Awal):} Analisis lembar K-W-L. (Untuk mengetahui koneksi dengan pembelajaran sebelumnya).
\item \textbf{Asesmen Formatif (Proses):} Observasi keaktifan diskusi (gotong royong) dan penilaian kelengkapan "\textbf{LKPD 02 - Jurnal Investigasi Dual-Lensa}".
\item \textbf{Asesmen Sumatif (Akhir Siklus):} Penilaian jawaban studi kasus (Tahap N) menggunakan rubrik.
\end{itemize}

\subsection{Rubrik Penilaian Jawaban Studi Kasus (Tahap N)}

\begin{longtable}{|p{3cm}|p{3cm}|p{3cm}|p{3cm}|p{3cm}|}
\hline
\textbf{Kriteria Penilaian} & \textbf{Skor 4 (Sangat Baik)} & \textbf{Skor 3 (Baik)} & \textbf{Skor 2 (Cukup)} & \textbf{Skor 1 (Kurang)} \\
\hline
\textbf{Ketepatan Konsep Ilmiah} & Menggunakan istilah ilmiah (adaptasi, biomekanikal, efisiensi energi, jenis gerak) dengan sangat tepat dan relevan dengan kasus. & Menggunakan istilah ilmiah dengan tepat, namun kurang relevan. & Menggunakan istilah ilmiah namun ada beberapa kesalahan konsep. & Tidak menggunakan istilah ilmiah atau salah total. \\
\hline
\textbf{Keterkaitan dengan Etnosains} & Mampu menghubungkan secara logis dan eksplisit antara adaptasi gerak hewan dengan teknik Silek Harimau secara mendalam dan akurat. & Mampu menghubungkan adaptasi gerak hewan dengan Silek, namun kurang mendalam. & Hanya menyebutkan Silek tanpa menghubungkan dengan adaptasi hewan, atau sebaliknya. & Tidak ada keterkaitan antara sains dan budaya yang ditunjukkan. \\
\hline
\textbf{Kelogisan \& Struktur Argumen} & Rekomendasi sangat logis, runtut, praktis, dan mudah dipahami dengan contoh yang tepat. & Rekomendasi logis dan runtut, namun kurang praktis atau contoh kurang tepat. & Alur rekomendasi kurang runtut atau sulit dipahami. & Rekomendasi tidak logis dan tidak terstruktur. \\
\hline
\end{longtable}

\section{Daftar Pustaka Sumber Etnosains}

\begin{enumerate}
\item Dt. Palimo Ameh. (2005). \textit{Silek Minangkabau: Seni Bela Diri dan Kearifan Budaya}. Padang: Pusat Pengkajian Islam dan Minangkabau.
\item Navis, A.A. (1984). \textit{Alam Terkembang Jadi Guru: Adat dan Kebudayaan Minangkabau}. Jakarta: Grafiti Pers.
\item Kementerian Pendidikan dan Kebudayaan. (2018). \textit{Silek: Warisan Budaya Takbenda Indonesia dari Sumatera Barat}. Jakarta: Direktorat Warisan dan Diplomasi Budaya.
\item Syahrul, R. (2010). \textit{Filosofi Gerakan dalam Silek Harimau Minangkabau}. Padang: Universitas Negeri Padang Press.
\end{enumerate}

\end{document}