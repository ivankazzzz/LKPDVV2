\documentclass[a4paper,12pt]{article}
\usepackage[a4paper, margin=1.27cm]{geometry}
\usepackage[indonesian]{babel}
\usepackage[utf8]{inputenc}
\usepackage{tcolorbox}
\usepackage{array}
\usepackage{multirow}
\usepackage{setspace}
\usepackage{amssymb}
\usepackage{graphicx}
\usepackage{adjustbox}
\usepackage{enumitem}
\usepackage{longtable}
\usepackage{xcolor}
\usepackage{amsmath}
\usepackage{fancyhdr}
\usepackage{titlesec}

% Define colors (black and white theme)
\definecolor{darkgray}{RGB}{64, 64, 64}
\definecolor{lightgray}{RGB}{240, 240, 240}
\definecolor{mediumgray}{RGB}{128, 128, 128}

% Custom tcolorbox styles
\tcbset{
    mainbox/.style={
        colback=white,
        colframe=black,
        boxrule=1pt,
        arc=3pt,
        left=8pt,
        right=8pt,
        top=8pt,
        bottom=8pt
    },
    sectionbox/.style={
        colback=white,
        colframe=black,
        boxrule=1.5pt,
        arc=2pt,
        left=6pt,
        right=6pt,
        top=6pt,
        bottom=6pt
    }
}

\begin{document}

\begin{center}
{\Large\textbf{MODUL AJAR: Kearifan Kesehatan Ginjal Minang \& Sistem Ekskresi}}
\end{center}

\vspace{0.5cm}

\begin{tcolorbox}[mainbox]
\textbf{Nama Penyusun:} Irfan Ananda Ismail, S.Pd., M.Pd., Gr \\
\textbf{Institusi:} SMP \\
\textbf{Mata Pelajaran:} Ilmu Pengetahuan Alam (IPA) \\
\textbf{Tahun Ajaran:} 2025/2026 \\
\textbf{Semester:} Ganjil \\
\textbf{Jenjang Sekolah:} SMP \\
\textbf{Kelas/Fase:} VIII / D \\
\textbf{Alokasi waktu:} 3 x 40 Menit (1 Pertemuan)
\end{tcolorbox}

\section{DIMENSI PROFIL PELAJAR PANCASILA}
\textit{(Clue untuk Guru: Sebutkan dimensi ini secara eksplisit saat apersepsi agar siswa sadar tujuan non-akademis yang sedang mereka kembangkan).}

\begin{itemize}
\item \textbf{Berkebinekaan Global:} Mengenal dan menghargai budaya, khususnya menganalisis kearifan tradisional Minangkabau dalam menjaga kesehatan ginjal dan sistem ekskresi serta menghubungkannya dengan prinsip fisiologi modern.
\item \textbf{Bernalar Kritis:} Ditempa secara intensif saat menganalisis informasi dari sumber sains dan budaya (Tahap S), menyintesiskan kedua perspektif (Tahap A), dan menyusun argumen berbasis bukti (Tahap N).
\item \textbf{Gotong Royong:} Kemampuan untuk bekerja sama secara kolaboratif dalam kelompok untuk merumuskan masalah (Tahap E), melakukan investigasi (Tahap S), dan membangun pemahaman bersama (Tahap A).
\item \textbf{Kreatif:} Menghasilkan argumen atau solusi orisinal yang terintegrasi pada tahap akhir pembelajaran (Tahap N).
\end{itemize}

\section{Sarana dan Prasarana}

\begin{itemize}
\item \textbf{Media:} LKPD 15 - Jurnal Investigasi Dual-Lensa, gambar/video tradisi minum air putih dan pola hidup sehat Minang, diagram sistem ekskresi, model ginjal sederhana.
\item \textbf{Alat:} Papan tulis/whiteboard, spidol, Proyektor \& Speaker, Kertas Plano atau Karton (1 per kelompok), sticky notes warna-warni, penggaris.
\item \textbf{Sumber Belajar:} Buku ajar IPA kelas VIII, Tautan video animasi sistem ekskresi (misal: bit.ly/animasi-sistem-ekskresi), tautan artikel kearifan kesehatan Minang (misal: bit.ly/kesehatan-tradisional-minang).
\end{itemize}

\section{Target Peserta didik}

\begin{itemize}
\item Peserta didik reguler kelas VIII (Fase D).
\end{itemize}

\section{Model Pembelajaran}

\begin{itemize}
\item Model Pembelajaran KESAN (Konektivitas Etnosains-Sains).
\end{itemize}

\section{Pemahaman Bermakna}
\textit{(Clue untuk Guru: Bacakan atau sampaikan narasi ini dengan intonasi yang menarik di akhir pembelajaran untuk mengikat semua pengalaman belajar siswa menjadi satu kesatuan yang bermakna).}

\begin{tcolorbox}[sectionbox]
"Ananda Semua, hari ini kita telah mengungkap kearifan mendalam di balik tradisi kesehatan Minangkabau dalam menjaga sistem ekskresi! Kita menemukan bahwa kebiasaan-kebiasaan tradisional seperti minum air putih yang cukup, mengonsumsi makanan berserat, menghindari makanan berlebihan, dan pola hidup aktif ternyata memiliki dasar ilmiah yang kuat dalam menjaga kesehatan ginjal dan sistem pembuangan. Setiap tradisi kesehatan - dari filosofi 'makan secukupnya', kebiasaan minum air kelapa muda, hingga aktivitas fisik tradisional - sebenarnya bekerja untuk mengoptimalkan fungsi filtrasi ginjal, melancarkan pembuangan limbah metabolisme, dan menjaga keseimbangan cairan tubuh. Dengan ini, kita sadar bahwa kearifan kesehatan tradisional bukan hanya warisan budaya, tapi juga strategi ilmiah yang telah terbukti menjaga kesehatan sistem ekskresi masyarakat Minangkabau selama berabad-abad."
\end{tcolorbox}

\section{PERTEMUAN PERTAMA: Sistem Ekskresi dan Kearifan Kesehatan Ginjal Minangkabau}

\subsection{Capaian Pembelajaran (Fase D)}
Pada akhir Fase D, murid memiliki kemampuan menganalisis sistem organisasi kehidupan, fungsi, serta kelainan atau gangguan yang muncul pada sistem organ makhluk hidup.

\subsection{Tujuan Pembelajaran (TP) Pertemuan 1:}
\textit{(Clue untuk Guru: Tujuan ini adalah kompas Anda. Pastikan setiap tahapan KESAN yang Anda lalui berkontribusi pada pencapaian tujuan-tujuan ini).}

Melalui model pembelajaran KESAN, peserta didik mampu:
\begin{itemize}
\item Menghubungkan fenomena kearifan lokal Minangkabau (tradisi kesehatan ginjal) dengan konteks sistem ekskresi dalam sains modern. (Sintaks K)
\item Merumuskan pertanyaan investigatif mengenai struktur dan fungsi sistem ekskresi serta hubungannya dengan tradisi kesehatan Minangkabau. (Sintaks E)
\item Mengumpulkan informasi mengenai komponen sistem ekskresi dari sumber sains serta tradisi kesehatan ginjal dari sumber kultural. (Sintaks S)
\item Menganalisis dan menyintesiskan hubungan sebab-akibat antara tradisi kesehatan Minangkabau dengan efisiensi fungsi sistem ekskresi. (Sintaks A)
\item Menyusun sebuah penjelasan analitis yang logis mengenai bagaimana kearifan kesehatan tradisional Minangkabau dapat menjadi model menjaga kesehatan sistem ekskresi. (Sintaks N)
\end{itemize}

\subsection{Pertanyaan Pemantik}
\textit{(Clue untuk Guru: Ajukan dua pertanyaan ini secara berurutan, berikan jeda agar siswa berpikir. Jangan langsung minta jawaban, biarkan pertanyaan ini menggantung untuk memicu rasa ingin tahu).}

\begin{itemize}
\item "Pernahkah kalian memperhatikan bahwa orang-orang tua di Minangkabau jarang mengalami masalah ginjal atau batu ginjal? Mereka memiliki kebiasaan minum air putih yang cukup dan pola makan yang teratur. Apa rahasia di balik tradisi kesehatan mereka?"
\item "Dalam filosofi Minang ada ungkapan 'Makan secukupnyo, minum secukupnyo' - makan secukupnya, minum secukupnya. Dari sudut pandang ilmu fisiologi, bagaimana sebenarnya keseimbangan asupan dan pembuangan dapat menjaga kesehatan sistem ekskresi?"
\end{itemize}

\section{Langkah-langkah Kegiatan Pembelajaran:}

\subsection{Kegiatan Pembuka (15 Menit)}
\begin{itemize}
\item Guru membuka pelajaran dengan salam, doa, dan memeriksa kehadiran.
\item \textbf{Asesmen Diagnostik Awal:} Guru membagikan lembar K-W-L.
    \begin{itemize}
    \item \textbf{Instruksi Guru:} "Ananda, sebelum kita mulai menjelajahi dunia sistem ekskresi, tolong isi dua kolom pertama di lembar ini. Di kolom \textbf{K (Tahu)}, tulis apa saja yang sudah kalian ketahui tentang ginjal dan pembuangan limbah tubuh ATAU tradisi kesehatan dalam keluarga. Di kolom \textbf{W (Ingin Tahu)}, tulis apa yang membuat kalian penasaran tentang topik ini." \textit{(Clue: Ini membantu Anda memetakan pengetahuan awal dan minat siswa secara cepat).}
    \end{itemize}
\item \textbf{Apersepsi:}
    \begin{itemize}
    \item \textbf{Instruksi Guru:} "Hari ini kita akan menjadi ahli nefrologi dan praktisi kesehatan tradisional sekaligus. Kita akan menyelidiki bagaimana kearifan kesehatan tradisional Minangkabau dapat menjaga kesehatan sistem ekskresi kita. Dalam investigasi ini, kita akan melatih kemampuan \textbf{Bernalar Kritis} kita, menghargai budaya lewat \textbf{Berkebinekaan Global}, dan bekerja sama dalam semangat \textbf{Gotong Royong}. Siap menjadi peneliti kesehatan tradisional?"
    \end{itemize}
\end{itemize}

\subsection{Kegiatan Inti (90 Menit) - SINTAKS MODEL KESAN}

\subsubsection{Tahap 1: (K) Kaitkan Konteks Kultural (15 menit)}
\begin{itemize}
\item \textbf{Aktivitas Guru:}
    \begin{itemize}
    \item Menampilkan gambar/video tradisi kesehatan Minangkabau seperti kebiasaan minum air putih, pola makan teratur, dan aktivitas fisik tradisional.
    \item Mengajukan Pertanyaan Pemantik yang sudah disiapkan di atas.
    \item \textit{(Clue: Tujuan tahap ini adalah memvalidasi pengetahuan siswa dan memantik rasa heran, bukan mencari jawaban benar. Sediakan spidol dan papan tulis/whiteboard. Saat siswa menjawab, tuliskan semua ide mereka, bahkan yang keliru sekalipun, dengan judul "\textbf{PENGETAHUAN AWAL KITA}". Ini menunjukkan bahwa semua pemikiran dihargai).}
    \end{itemize}
\item \textbf{Aktivitas Siswa:} Mengamati tradisi kesehatan, mendengarkan pertanyaan, lalu secara sukarela berbagi pengalaman atau pengetahuan awal tentang tradisi kesehatan keluarga atau sistem ekskresi. Menuliskan minimal satu pertanyaan atau pengalaman di sticky notes dan menempelkannya di '\textbf{Papan Penasaran}'.
\end{itemize}

\subsubsection{Tahap 2: (E) Eksplorasi Enigma (15 menit)}
\begin{itemize}
\item \textbf{Aktivitas Guru:} Membentuk siswa menjadi kelompok (3-4 orang).
    \begin{itemize}
    \item \textbf{Instruksi Guru:} "Baik, rasa penasaran kalian luar biasa! Sekarang, tugas kita sebagai peneliti adalah mengubah rasa penasaran ini menjadi misi yang jelas. Dalam kelompok, diskusikan dan rumuskan minimal 3 pertanyaan kunci yang akan kita selidiki hari ini. Tuliskan dalam bentuk '\textbf{Misi Penelitian Tim [Nama Kelompok]}' di kertas plano yang Bapak/Ibu berikan."
    \item \textit{(Clue: Arahkan diskusi siswa agar pertanyaannya mencakup aspek 'bagaimana sistem ekskresi bekerja' dan 'mengapa tradisi kesehatan Minang efektif'. Jika kelompok kesulitan, berikan pancingan: "Kira-kira, apa dulu yang perlu kita tahu? Bagaimana ginjal bekerja atau mengapa minum air putih baik untuk ginjal?").}
    \end{itemize}
\item \textbf{Aktivitas Siswa:} Berdiskusi dalam tim untuk merumuskan misi penelitian (daftar pertanyaan kunci) di kertas plano. \textit{(Contoh misi yang diharapkan: 1. Bagaimana struktur dan fungsi sistem ekskresi? 2. Apa tradisi kesehatan ginjal dalam budaya Minangkabau? 3. Mengapa tradisi kesehatan tradisional efektif untuk menjaga ginjal?)}
\end{itemize}

\subsubsection{Tahap 3: (S) Selidiki secara Sintetis (25 menit)}
\begin{itemize}
\item \textbf{Aktivitas Guru:} Membagikan "\textbf{LKPD 15 - Jurnal Investigasi Dual-Lensa}".
    \begin{itemize}
    \item \textbf{Instruksi Guru:} "Setiap tim akan melakukan investigasi dari dua lensa. Gunakan HP atau sumber yang disediakan untuk mencari jawabannya. Bagilah tugas dalam tim!"
    \item \textbf{Lensa Sains:} Buka link video bit.ly/animasi-sistem-ekskresi untuk memahami struktur sistem ekskresi (ginjal, ureter, kandung kemih, uretra), proses filtrasi, reabsorpsi, dan sekresi, serta fungsi pembuangan limbah metabolisme.
    \item \textbf{Lensa Etnosains/Kultural:} Buka link artikel bit.ly/kesehatan-tradisional-minang untuk memahami tradisi kesehatan Minangkabau seperti pola minum air putih, konsumsi makanan berserat, kebiasaan makan teratur, dan aktivitas fisik untuk menjaga kesehatan ginjal.
    \item \textit{(Clue: Pastikan sumber belajar sudah disiapkan dan link bisa diakses. Berkelilinglah untuk memastikan setiap kelompok membagi tugas dan tidak hanya fokus pada satu lensa saja).}
    \end{itemize}
\item \textbf{Aktivitas Siswa:} Dalam kelompok, siswa berbagi tugas mencari informasi dari sumber yang diberikan dan mencatat temuan kunci di dua kolom terpisah pada "\textbf{LKPD 15 - Jurnal Investigasi Dual-Lensa}".
\end{itemize}

\subsubsection{Tahap 4: (A) Asimilasi Analitis (20 menit)}
\begin{itemize}
\item \textbf{Aktivitas Guru:} Memfasilitasi diskusi untuk menjembatani kedua lensa.
    \begin{itemize}
    \item \textbf{Pertanyaan Pancingan Kunci untuk Guru:}
        \begin{itemize}
        \item "Oke, dari Lensa Sains kita tahu ginjal memfilter darah dan membuang limbah melalui urin. Dari Lensa Etnosains, kita tahu tradisi Minang menekankan minum air putih yang cukup. Nah, coba hubungkan! Mengapa air putih penting untuk ginjal?"
        \item "Dari Lensa Sains, kita tahu ginjal perlu bekerja optimal untuk menjaga keseimbangan cairan. Dari Lensa Budaya, filosofi 'makan secukupnya' mengatur asupan. Apa kesamaan prinsipnya?"
        \end{itemize}
    \item \textit{(Clue: Fokuskan untuk membuat siswa 'menemukan' hubungannya sendiri, bukan diberitahu. Gunakan kata "menurut kalian", "kira-kira kenapa", "ada yang punya ide?").}
    \end{itemize}
\item \textbf{Aktivitas Siswa:} Berdiskusi intensif untuk menghubungkan temuan sains dan budaya. Menuliskan kesimpulan terpadu (sintesis) mereka di kertas plano.
\end{itemize}

\subsubsection{Tahap 5: (N) Nyatakan Pemahaman (15 menit)}
\begin{itemize}
\item \textbf{Aktivitas Guru:} Memberikan studi kasus individual atau per kelompok.
    \begin{itemize}
    \item \textbf{Instruksi Guru (tuliskan di papan tulis):}
    
    "\textbf{STUDI KASUS UNTUK PENELITI SAINS-BUDAYA:}
    
    Sebuah pusat kesehatan masyarakat di Padang ingin mengembangkan program pencegahan penyakit ginjal yang mengintegrasikan kearifan kesehatan tradisional Minangkabau dengan edukasi kesehatan modern. Mereka percaya bahwa program ini akan lebih mudah diterima masyarakat lokal dan sekaligus melestarikan budaya. Tim dokter membutuhkan penjelasan ilmiah mengapa tradisi kesehatan Minang efektif untuk menjaga kesehatan ginjal.
    
    Tugasmu: Tuliskan sebuah penjelasan ilmiah singkat (4-6 kalimat) di buku latihanmu untuk membantu tim dokter memahami mengapa kearifan kesehatan tradisional Minangkabau dapat menjadi model pencegahan penyakit ginjal yang efektif. Gunakan pengetahuan gabungan dari sains dan kearifan tradisional yang baru saja kamu pelajari."
    \end{itemize}
\item \textbf{Aktivitas Siswa:} Menyusun argumen tertulis untuk menjawab studi kasus yang diberikan, menggunakan bukti dari kedua lensa.
\end{itemize}

\subsection{Kegiatan Penutup (15 Menit)}
\begin{itemize}
\item \textbf{Presentasi \& Penguatan:} Guru meminta 2-3 siswa secara acak untuk membacakan penjelasan mereka. Guru memberikan pujian dan penguatan positif, menekankan betapa hebatnya kemampuan mereka mengintegrasikan sains dan budaya.
\item \textbf{Refleksi:}
    \begin{itemize}
    \item \textbf{Instruksi Guru:} "Sekarang, kembali ke lembar K-W-L kalian. Lengkapi kolom terakhir, \textbf{L (Learned)}, dengan hal-hal baru yang paling mengagumkan yang kalian pelajari hari ini."
    \item \textbf{Instruksi Guru:} "Angkat tangan, siapa yang setelah belajar hari ini ingin menerapkan tradisi kesehatan Minang untuk menjaga ginjal?"
    \end{itemize}
\item \textbf{Tindak Lanjut:}
    \begin{itemize}
    \item \textbf{Instruksi Guru:} "Luar biasa, para peneliti! Hari ini kita sudah mengungkap bagaimana kearifan kesehatan tradisional dapat menjaga sistem ekskresi. Pertemuan berikutnya, kita akan mempelajari gangguan sistem ekskresi dan bagaimana pengobatan tradisional Minang dapat membantu pencegahannya. Kita akan lihat!"
    \item \textit{(Clue: Buat transisi yang menarik ke pertemuan berikutnya sambil mempertahankan semangat investigasi yang sudah terbangun).}
    \end{itemize}
\item Guru menutup pelajaran dengan doa dan salam.
\end{itemize}

\section{Asesmen (Penilaian)}

\begin{itemize}
\item \textbf{Asesmen Diagnostik (Awal):} Analisis lembar K-W-L. (Untuk mengetahui baseline siswa).
\item \textbf{Asesmen Formatif (Proses):} Observasi keaktifan diskusi (gotong royong) dan penilaian kelengkapan "\textbf{LKPD 15 - Jurnal Investigasi Dual-Lensa}".
\item \textbf{Asesmen Sumatif (Akhir Siklus):} Penilaian jawaban studi kasus (Tahap N) menggunakan rubrik.
\end{itemize}

\subsection{Rubrik Penilaian Jawaban Studi Kasus (Tahap N)}

\begin{longtable}{|p{3cm}|p{3cm}|p{3cm}|p{3cm}|p{3cm}|}
\hline
\textbf{Kriteria Penilaian} & \textbf{Skor 4 (Sangat Baik)} & \textbf{Skor 3 (Baik)} & \textbf{Skor 2 (Cukup)} & \textbf{Skor 1 (Kurang)} \\
\hline
\textbf{Ketepatan Konsep Ilmiah} & Menggunakan istilah ilmiah (filtrasi, reabsorpsi, sekresi, nefron) dengan sangat tepat dan relevan dengan kasus. & Menggunakan istilah ilmiah dengan tepat, namun kurang relevan. & Menggunakan istilah ilmiah namun ada beberapa kesalahan konsep. & Tidak menggunakan istilah ilmiah atau salah total. \\
\hline
\textbf{Keterkaitan dengan Etnosains} & Mampu menghubungkan secara logis dan eksplisit antara tradisi kesehatan Minang dengan fungsi ginjal secara mendalam. & Mampu menghubungkan tradisi kesehatan dengan kesehatan ginjal, namun kurang mendalam. & Hanya menyebutkan tradisi kesehatan tanpa menghubungkan dengan sains, atau sebaliknya. & Tidak ada keterkaitan antara sains dan budaya yang ditunjukkan. \\
\hline
\textbf{Kelogisan \& Struktur Argumen} & Penjelasan sangat logis, runtut, mudah dipahami, dan didukung bukti yang kuat. & Penjelasan logis dan runtut, namun kurang didukung bukti. & Alur penjelasan kurang runtut atau sulit dipahami. & Penjelasan tidak logis dan tidak terstruktur. \\
\hline
\end{longtable}

\section{Daftar Pustaka Sumber Etnosains}

\begin{enumerate}
\item Navis, A.A. (1984). \textit{Alam Terkembang Jadi Guru: Adat dan Kebudayaan Minangkabau}. Jakarta: Grafiti Pers.
\item Kato, T. (2005). \textit{Adat Minangkabau dan Merantau dalam Perspektif Sejarah}. Jakarta: Balai Pustaka.
\item Dt. Rajo Penghulu, I. (1994). \textit{Pegangan Penghulu, Bundo Kanduang dan Pidato Alua Pasambahan Adat Minangkabau}. Bukittinggi: Pustaka Indonesia.
\item Zuhud, E.A.M. (2009). \textit{Potensi Hutan Tropika Indonesia sebagai Penyangga Bahan Obat Alam untuk Kesehatan Bangsa}. Jakarta: IPB Press.
\end{enumerate}

\end{document}