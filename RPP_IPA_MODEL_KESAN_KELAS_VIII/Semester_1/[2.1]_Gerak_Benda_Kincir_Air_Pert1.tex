\documentclass[a4paper,12pt]{article}
\usepackage[a4paper, margin=1.27cm]{geometry}
\usepackage[indonesian]{babel}
\usepackage[utf8]{inputenc}
\usepackage{tcolorbox}
\usepackage{array}
\usepackage{multirow}
\usepackage{setspace}
\usepackage{amssymb}
\usepackage{graphicx}
\usepackage{adjustbox}
\usepackage{enumitem}
\usepackage{longtable}
\usepackage{xcolor}
\usepackage{amsmath}
\usepackage{fancyhdr}
\usepackage{titlesec}

% Define colors (black and white theme)
\definecolor{darkgray}{RGB}{64, 64, 64}
\definecolor{lightgray}{RGB}{240, 240, 240}
\definecolor{mediumgray}{RGB}{128, 128, 128}

% Custom tcolorbox styles
\tcbset{
    mainbox/.style={
        colback=white,
        colframe=black,
        boxrule=1pt,
        arc=3pt,
        left=8pt,
        right=8pt,
        top=8pt,
        bottom=8pt
    },
    sectionbox/.style={
        colback=white,
        colframe=black,
        boxrule=1.5pt,
        arc=2pt,
        left=6pt,
        right=6pt,
        top=6pt,
        bottom=6pt
    }
}

\begin{document}

\begin{center}
{\Large\textbf{MODUL AJAR: Misteri Kincir Air Tradisional \& Hukum Gerak Newton}}
\end{center}

\vspace{0.5cm}

\begin{tcolorbox}[mainbox]
\textbf{Nama Penyusun:} Irfan Ananda Ismail, S.Pd., M.Pd., Gr \\
\textbf{Institusi:} SMP \\
\textbf{Mata Pelajaran:} Ilmu Pengetahuan Alam (IPA) \\
\textbf{Tahun Ajaran:} 2025/2026 \\
\textbf{Semester:} Ganjil \\
\textbf{Jenjang Sekolah:} SMP \\
\textbf{Kelas/Fase:} VIII / D \\
\textbf{Alokasi waktu:} 3 x 40 Menit (1 Pertemuan)
\end{tcolorbox}

\section{DIMENSI PROFIL PELAJAR PANCASILA}
\textit{(Clue untuk Guru: Sebutkan dimensi ini secara eksplisit saat apersepsi agar siswa sadar tujuan non-akademis yang sedang mereka kembangkan).}

\begin{itemize}
\item \textbf{Berkebinekaan Global:} Mengenal dan menghargai budaya, khususnya menganalisis kearifan lokal Minangkabau dalam teknologi tradisional (kincir air untuk penggilingan padi), lalu menghubungkannya dengan konsep sains universal tentang hukum gerak Newton.
\item \textbf{Bernalar Kritis:} Ditempa secara intensif saat menganalisis informasi dari sumber sains dan budaya (Tahap S), menyintesiskan kedua perspektif (Tahap A), dan menyusun argumen berbasis bukti (Tahap N).
\item \textbf{Gotong Royong:} Kemampuan untuk bekerja sama secara kolaboratif dalam kelompok untuk merumuskan masalah (Tahap E), melakukan investigasi (Tahap S), dan membangun pemahaman bersama (Tahap A).
\item \textbf{Kreatif:} Menghasilkan argumen atau solusi orisinal yang terintegrasi pada tahap akhir pembelajaran (Tahap N).
\end{itemize}

\section{Sarana dan Prasarana}

\begin{itemize}
\item \textbf{Media:} LKPD 03 - Jurnal Investigasi Dual-Lensa, video singkat kincir air tradisional Minangkabau (durasi 3-4 menit), gambar/diagram hukum Newton, artikel tentang teknologi penggilingan padi tradisional, model sederhana kincir air atau roda berputar.
\item \textbf{Alat:} Papan tulis/whiteboard, spidol, Proyektor \& Speaker, Kertas Plano atau Karton (1 per kelompok), sticky notes warna-warni, penggaris, bola kecil untuk demonstrasi.
\item \textbf{Sumber Belajar:} Buku ajar IPA kelas VIII, Tautan video animasi hukum Newton (misal: bit.ly/animasi-newton), tautan artikel/video kincir air tradisional (misal: bit.ly/kincir-air-minang).
\end{itemize}

\section{Target Peserta didik}

\begin{itemize}
\item Peserta didik reguler kelas VIII (Fase D).
\end{itemize}

\section{Model Pembelajaran}

\begin{itemize}
\item Model Pembelajaran KESAN (Konektivitas Etnosains-Sains).
\end{itemize}

\section{Pemahaman Bermakna}
\textit{(Clue untuk Guru: Bacakan atau sampaikan narasi ini dengan intonasi yang menarik di akhir pembelajaran untuk mengikat semua pengalaman belajar siswa menjadi satu kesatuan yang bermakna).}

\begin{tcolorbox}[sectionbox]
"Ananda Semua, hari ini kita telah membongkar misteri di balik kincir air tradisional yang masih digunakan di beberapa nagari di Minangkabau! Kita menemukan bahwa teknologi sederhana yang diwariskan nenek moyang kita ini sebenarnya menerapkan hukum-hukum fisika yang sama dengan yang ditemukan Newton ratusan tahun kemudian. Setiap putaran kincir, setiap tumbukan air dengan bilah kayu, semuanya mengikuti prinsip-prinsip sains yang universal. Dengan ini, kita sadar bahwa kearifan lokal bukan hanya tradisi, tapi juga sains yang telah teruji waktu."
\end{tcolorbox}

\section{PERTEMUAN PERTAMA: Hukum Gerak Newton dan Teknologi Tradisional Minangkabau}

\subsection{Capaian Pembelajaran (Fase D)}
Pada akhir Fase D, murid memiliki kemampuan menerapkan konsep gaya, gerak, dan energi dalam kehidupan sehari-hari.

\subsection{Tujuan Pembelajaran (TP) Pertemuan 1:}
\textit{(Clue untuk Guru: Tujuan ini adalah kompas Anda. Pastikan setiap tahapan KESAN yang Anda lalui berkontribusi pada pencapaian tujuan-tujuan ini).}

Melalui model pembelajaran KESAN, peserta didik mampu:
\begin{itemize}
\item Menghubungkan fenomena kearifan lokal Minangkabau (teknologi kincir air tradisional) dengan konteks hukum gerak Newton. (Sintaks K)
\item Merumuskan pertanyaan investigatif mengenai cara kerja kincir air dan hubungannya dengan hukum-hukum gerak. (Sintaks E)
\item Mengumpulkan informasi mengenai hukum Newton dari sumber ilmiah serta prinsip kerja teknologi tradisional dari sumber kultural. (Sintaks S)
\item Menganalisis dan menyintesiskan hubungan sebab-akibat antara hukum Newton dengan cara kerja kincir air tradisional. (Sintaks A)
\item Menyusun sebuah penjelasan analitis yang logis mengenai bagaimana teknologi tradisional Minangkabau menerapkan prinsip-prinsip fisika modern. (Sintaks N)
\end{itemize}

\subsection{Pertanyaan Pemantik}
\textit{(Clue untuk Guru: Ajukan dua pertanyaan ini secara berurutan, berikan jeda agar siswa berpikir. Jangan langsung minta jawaban, biarkan pertanyaan ini menggantung untuk memicu rasa ingin tahu).}

\begin{itemize}
\item "Pernahkah kalian melihat kincir air tradisional yang masih digunakan untuk menggiling padi di beberapa nagari di Minangkabau? Mengapa air yang mengalir bisa membuat kincir berputar dan menggiling padi dengan kekuatan yang luar biasa?"
\item "Newton merumuskan hukum gerak pada abad ke-17, tapi nenek moyang kita sudah menggunakan kincir air jauh sebelum itu. Apakah mungkin teknologi tradisional kita sudah menerapkan hukum-hukum fisika yang sama tanpa kita sadari?"
\end{itemize}

\section{Langkah-langkah Kegiatan Pembelajaran:}

\subsection{Kegiatan Pembuka (15 Menit)}
\begin{itemize}
\item Guru membuka pelajaran dengan salam, doa, dan memeriksa kehadiran.
\item \textbf{Asesmen Diagnostik Awal:} Guru membagikan lembar K-W-L.
    \begin{itemize}
    \item \textbf{Instruksi Guru:} "Ananda, kita telah menyelesaikan petualangan tentang gerak makhluk hidup. Sekarang kita beralih ke gerak benda! Di lembar K-W-L ini, tulis di kolom \textbf{K (Tahu)} apa yang sudah kalian ketahui tentang gerak benda ATAU tentang teknologi tradisional seperti kincir air. Di kolom \textbf{W (Ingin Tahu)}, tulis apa yang membuat kalian penasaran tentang topik hari ini." \textit{(Clue: Ini membantu Anda melihat transisi pemahaman siswa dari gerak makhluk hidup ke gerak benda).}
    \end{itemize}
\item \textbf{Apersepsi:}
    \begin{itemize}
    \item \textbf{Instruksi Guru:} "Setelah menyelidiki rahasia gerak makhluk hidup melalui Tari Piring dan Silek Harimau, sekarang kita akan mengeksplorasi dunia gerak benda! Hari ini, misi kita adalah mengungkap bagaimana teknologi tradisional Minangkabau seperti kincir air sebenarnya menerapkan hukum-hukum fisika yang sama dengan yang dipelajari ilmuwan modern. Siap menjadi detektif teknologi tradisional?"
    \end{itemize}
\end{itemize}

\subsection{Kegiatan Inti (90 Menit) - SINTAKS MODEL KESAN}

\subsubsection{Tahap 1: (K) Kaitkan Konteks Kultural (15 menit)}
\begin{itemize}
\item \textbf{Aktivitas Guru:}
    \begin{itemize}
    \item Menampilkan video singkat (3-4 menit) kincir air tradisional yang masih beroperasi di Minangkabau, menunjukkan proses penggilingan padi.
    \item Mengajukan Pertanyaan Pemantik yang sudah disiapkan di atas.
    \item \textit{(Clue: Fokuskan pada keajaiban teknologi sederhana yang bisa menghasilkan kekuatan besar. Saat siswa menjawab, tuliskan semua ide mereka di papan tulis dengan judul "\textbf{KEAJAIBAN TEKNOLOGI TRADISIONAL}". Hargai setiap kontribusi).}
    \end{itemize}
\item \textbf{Aktivitas Siswa:} Mengamati video, mendengarkan pertanyaan, lalu secara sukarela berbagi pengalaman atau dugaan tentang cara kerja kincir air. Menuliskan minimal satu pertanyaan atau pengamatan di sticky notes dan menempelkannya di '\textbf{Papan Penasaran}'.
\end{itemize}

\subsubsection{Tahap 2: (E) Eksplorasi Enigma (15 menit)}
\begin{itemize}
\item \textbf{Aktivitas Guru:} Membentuk siswa menjadi kelompok (3-4 orang).
    \begin{itemize}
    \item \textbf{Instruksi Guru:} "Keingintahuan kalian tentang teknologi tradisional sangat menginspirasi! Sekarang, sebagai tim insinyur muda, tugas kalian adalah mengubah keingintahuan ini menjadi misi investigasi yang sistematis. Dalam kelompok, diskusikan dan rumuskan minimal 3 pertanyaan kunci yang akan kita selidiki hari ini. Tuliskan dalam bentuk '\textbf{Misi Investigasi Tim [Nama Kelompok]}' di kertas plano."
    \item \textit{(Clue: Arahkan diskusi siswa agar pertanyaannya mencakup aspek 'bagaimana kincir air bekerja' dan 'hukum fisika apa yang terlibat'. Jika kelompok kesulitan, berikan pancingan: "Kira-kira, gaya apa yang membuat kincir berputar? Dan mengapa putarannya bisa menghasilkan kekuatan untuk menggiling?").}
    \end{itemize}
\item \textbf{Aktivitas Siswa:} Berdiskusi dalam tim untuk merumuskan misi investigasi (daftar pertanyaan kunci) di kertas plano. \textit{(Contoh misi yang diharapkan: 1. Bagaimana air bisa membuat kincir berputar? 2. Hukum fisika apa yang bekerja pada kincir air? 3. Mengapa teknologi sederhana ini begitu efektif?)}
\end{itemize}

\subsubsection{Tahap 3: (S) Selidiki secara Sintetis (25 menit)}
\begin{itemize}
\item \textbf{Aktivitas Guru:} Membagikan "\textbf{LKPD 03 - Jurnal Investigasi Dual-Lensa}".
    \begin{itemize}
    \item \textbf{Instruksi Guru:} "Setiap tim akan melakukan investigasi dari dua lensa. Gunakan sumber yang disediakan untuk mencari jawabannya. Bagilah tugas dalam tim!"
    \item \textbf{Lensa Sains:} Buka link video bit.ly/animasi-newton untuk memahami hukum I, II, dan III Newton. Pelajari juga diagram yang menunjukkan konsep gaya, massa, dan percepatan.
    \item \textbf{Lensa Etnosains/Kultural:} Buka link artikel bit.ly/kincir-air-minang untuk memahami sejarah, konstruksi, dan prinsip kerja kincir air tradisional Minangkabau.
    \item \textit{(Clue: Pastikan sumber belajar sudah disiapkan dan dapat diakses. Berkelilinglah untuk memastikan setiap kelompok mengeksplorasi kedua lensa dengan fokus yang seimbang).}
    \end{itemize}
\item \textbf{Aktivitas Siswa:} Dalam kelompok, siswa berbagi tugas mencari informasi dari sumber yang diberikan dan mencatat temuan kunci di dua kolom terpisah pada "\textbf{LKPD 03 - Jurnal Investigasi Dual-Lensa}".
\end{itemize}

\subsubsection{Tahap 4: (A) Asimilasi Analitis (20 menit)}
\begin{itemize}
\item \textbf{Aktivitas Guru:} Memfasilitasi diskusi untuk menjembatani kedua lensa.
    \begin{itemize}
    \item \textbf{Pertanyaan Pancingan Kunci untuk Guru:}
        \begin{itemize}
        \item "Dari Lensa Sains kita tahu hukum I Newton: benda diam akan tetap diam kecuali ada gaya. Dari Lensa Etnosains, kita tahu air mengalir menumbuk bilah kincir. Nah, coba hubungkan! Bagaimana aliran air memberikan gaya untuk menggerakkan kincir yang awalnya diam?"
        \item "Dari Lensa Sains, kita tahu hukum III Newton: aksi-reaksi. Dari Lensa Budaya, kita tahu kincir air bisa menggiling padi dengan kekuatan besar. Apa hubungannya gaya aksi-reaksi dengan kemampuan kincir menghasilkan kekuatan untuk menggiling?"
        \end{itemize}
    \item \textit{(Clue: Bantu siswa melihat bahwa setiap hukum Newton dapat ditemukan dalam cara kerja kincir air. Gunakan demonstrasi sederhana dengan bola dan penggaris jika perlu).}
    \end{itemize}
\item \textbf{Aktivitas Siswa:} Berdiskusi intensif untuk menghubungkan temuan sains dan budaya. Menuliskan kesimpulan terpadu (sintesis) mereka di kertas plano.
\end{itemize}

\subsubsection{Tahap 5: (N) Nyatakan Pemahaman (15 menit)}
\begin{itemize}
\item \textbf{Aktivitas Guru:} Memberikan studi kasus individual atau per kelompok.
    \begin{itemize}
    \item \textbf{Instruksi Guru (tuliskan di papan tulis):}
    
    "\textbf{STUDI KASUS UNTUK INSINYUR MUDA SAINS-BUDAYA:}
    
    Sebuah desa wisata di Minangkabau ingin membangun kincir air tradisional sebagai atraksi sekaligus untuk menghasilkan listrik sederhana. Mereka meminta bantuanmu sebagai konsultan yang memahami baik sains maupun teknologi tradisional.
    
    Tugasmu: Tuliskan sebuah penjelasan singkat (3-5 kalimat) di buku latihanmu untuk menjelaskan kepada masyarakat desa mengapa kincir air tradisional bisa bekerja secara ilmiah. Gunakan hukum Newton yang relevan dan jelaskan bagaimana prinsip-prinsip ini sudah diterapkan nenek moyang kita dalam teknologi tradisional."
    \end{itemize}
\item \textbf{Aktivitas Siswa:} Menyusun argumen tertulis untuk menjawab studi kasus yang diberikan, menggunakan bukti dari kedua lensa.
\end{itemize}

\subsection{Kegiatan Penutup (15 Menit)}
\begin{itemize}
\item \textbf{Presentasi \& Penguatan:} Guru meminta 2-3 siswa secara acak untuk membacakan penjelasan mereka. Guru memberikan pujian dan penguatan positif, menekankan betapa hebatnya kemampuan mereka menghubungkan fisika modern dengan teknologi tradisional.
\item \textbf{Refleksi:}
    \begin{itemize}
    \item \textbf{Instruksi Guru:} "Sekarang, kembali ke lembar K-W-L kalian. Lengkapi kolom terakhir, \textbf{L (Learned)}, dengan hal-hal baru yang paling memukau yang kalian pelajari hari ini."
    \item \textbf{Instruksi Guru:} "Angkat tangan, siapa yang setelah belajar hari ini jadi lebih bangga dengan kecerdasan nenek moyang kita dalam menciptakan teknologi?"
    \end{itemize}
\item \textbf{Tindak Lanjut:}
    \begin{itemize}
    \item \textbf{Instruksi Guru:} "Fantastis, para insinyur muda! Hari ini kita sudah mengungkap bahwa hukum Newton sudah diterapkan dalam teknologi tradisional kita. Pertemuan berikutnya, kita akan mendalami lebih lanjut tentang gerak benda dan energi. Siapa tahu ada teknologi tradisional Minang lainnya yang bisa mengajarkan kita tentang konsep energi? Kita akan lihat!"
    \item \textit{(Clue: Buat transisi yang menarik ke pertemuan berikutnya sambil mempertahankan semangat eksplorasi teknologi tradisional).}
    \end{itemize}
\item Guru menutup pelajaran dengan doa dan salam.
\end{itemize}

\section{Asesmen (Penilaian)}

\begin{itemize}
\item \textbf{Asesmen Diagnostik (Awal):} Analisis lembar K-W-L. (Untuk mengetahui transisi pemahaman dari gerak makhluk hidup ke gerak benda).
\item \textbf{Asesmen Formatif (Proses):} Observasi keaktifan diskusi (gotong royong) dan penilaian kelengkapan "\textbf{LKPD 03 - Jurnal Investigasi Dual-Lensa}".
\item \textbf{Asesmen Sumatif (Akhir Siklus):} Penilaian jawaban studi kasus (Tahap N) menggunakan rubrik.
\end{itemize}

\subsection{Rubrik Penilaian Jawaban Studi Kasus (Tahap N)}

\begin{longtable}{|p{3cm}|p{3cm}|p{3cm}|p{3cm}|p{3cm}|}
\hline
\textbf{Kriteria Penilaian} & \textbf{Skor 4 (Sangat Baik)} & \textbf{Skor 3 (Baik)} & \textbf{Skor 2 (Cukup)} & \textbf{Skor 1 (Kurang)} \\
\hline
\textbf{Ketepatan Konsep Ilmiah} & Menggunakan hukum Newton (I, II, III) dengan sangat tepat dan relevan untuk menjelaskan cara kerja kincir air. & Menggunakan hukum Newton dengan tepat, namun kurang relevan dengan konteks. & Menggunakan konsep fisika namun ada beberapa kesalahan dalam penerapan hukum Newton. & Tidak menggunakan hukum Newton atau salah total. \\
\hline
\textbf{Keterkaitan dengan Etnosains} & Mampu menghubungkan secara logis dan eksplisit antara hukum Newton dengan teknologi kincir air tradisional secara mendalam dan akurat. & Mampu menghubungkan hukum Newton dengan teknologi tradisional, namun kurang mendalam. & Hanya menyebutkan teknologi tradisional tanpa menghubungkan dengan hukum Newton, atau sebaliknya. & Tidak ada keterkaitan antara sains dan teknologi tradisional yang ditunjukkan. \\
\hline
\textbf{Kelogisan \& Struktur Argumen} & Penjelasan sangat logis, runtut, mudah dipahami masyarakat awam, dan persuasif. & Penjelasan logis dan runtut, namun kurang mudah dipahami atau kurang persuasif. & Alur penjelasan kurang runtut atau sulit dipahami. & Penjelasan tidak logis dan tidak terstruktur. \\
\hline
\end{longtable}

\section{Daftar Pustaka Sumber Etnosains}

\begin{enumerate}
\item Dt. Rajo Penghulu. (1994). \textit{Teknologi Tradisional Minangkabau}. Padang: Pusat Dokumentasi dan Informasi Kebudayaan Minangkabau.
\item Navis, A.A. (1984). \textit{Alam Terkembang Jadi Guru: Adat dan Kebudayaan Minangkabau}. Jakarta: Grafiti Pers.
\item Kementerian Pendidikan dan Kebudayaan. (2016). \textit{Kincir Air: Teknologi Ramah Lingkungan Nusantara}. Jakarta: Direktorat Warisan dan Diplomasi Budaya.
\item Syafwan, A. (2008). \textit{Kearifan Lokal dalam Teknologi Pertanian Minangkabau}. Padang: Universitas Andalas Press.
\end{enumerate}

\end{document}