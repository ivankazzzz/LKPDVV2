\documentclass[a4paper,12pt]{article}
\usepackage[a4paper, margin=1.27cm]{geometry}
\usepackage[indonesian]{babel}
\usepackage[utf8]{inputenc}
\usepackage{tcolorbox}
\usepackage{array}
\usepackage{multirow}
\usepackage{setspace}
\usepackage{amssymb}
\usepackage{graphicx}
\usepackage{adjustbox}
\usepackage{enumitem}
\usepackage{longtable}
\usepackage{xcolor}
\usepackage{amsmath}
\usepackage{fancyhdr}
\usepackage{titlesec}

% Define colors (black and white theme)
\definecolor{darkgray}{RGB}{64, 64, 64}
\definecolor{lightgray}{RGB}{240, 240, 240}
\definecolor{mediumgray}{RGB}{128, 128, 128}

% Custom tcolorbox styles
\tcbset{
    mainbox/.style={
        colback=white,
        colframe=black,
        boxrule=1pt,
        arc=3pt,
        left=8pt,
        right=8pt,
        top=8pt,
        bottom=8pt
    },
    sectionbox/.style={
        colback=white,
        colframe=black,
        boxrule=1.5pt,
        arc=2pt,
        left=6pt,
        right=6pt,
        top=6pt,
        bottom=6pt
    }
}

\begin{document}

\begin{center}
{\Large\textbf{MODUL AJAR: Teknik Pernapasan Tradisional Minang \& Sistem Respirasi}}
\end{center}

\vspace{0.5cm}

\begin{tcolorbox}[mainbox]
\textbf{Nama Penyusun:} Irfan Ananda Ismail, S.Pd., M.Pd., Gr \\
\textbf{Institusi:} SMP \\
\textbf{Mata Pelajaran:} Ilmu Pengetahuan Alam (IPA) \\
\textbf{Tahun Ajaran:} 2025/2026 \\
\textbf{Semester:} Ganjil \\
\textbf{Jenjang Sekolah:} SMP \\
\textbf{Kelas/Fase:} VIII / D \\
\textbf{Alokasi waktu:} 3 x 40 Menit (1 Pertemuan)
\end{tcolorbox}

\section{DIMENSI PROFIL PELAJAR PANCASILA}
\textit{(Clue untuk Guru: Sebutkan dimensi ini secara eksplisit saat apersepsi agar siswa sadar tujuan non-akademis yang sedang mereka kembangkan).}

\begin{itemize}
\item \textbf{Berkebinekaan Global:} Mengenal dan menghargai budaya, khususnya menganalisis teknik pernapasan dalam aktivitas tradisional Minangkabau dan menghubungkannya dengan sistem respirasi dalam sains modern.
\item \textbf{Bernalar Kritis:} Ditempa secara intensif saat menganalisis informasi dari sumber sains dan budaya (Tahap S), menyintesiskan kedua perspektif (Tahap A), dan menyusun argumen berbasis bukti (Tahap N).
\item \textbf{Gotong Royong:} Kemampuan untuk bekerja sama secara kolaboratif dalam kelompok untuk merumuskan masalah (Tahap E), melakukan investigasi (Tahap S), dan membangun pemahaman bersama (Tahap A).
\item \textbf{Kreatif:} Menghasilkan argumen atau solusi orisinal yang terintegrasi pada tahap akhir pembelajaran (Tahap N).
\end{itemize}

\section{Sarana dan Prasarana}

\begin{itemize}
\item \textbf{Media:} LKPD 13 - Jurnal Investigasi Dual-Lensa, video/gambar teknik pernapasan dalam Silek Minang dan Randai, diagram sistem pernapasan, balon untuk simulasi paru-paru.
\item \textbf{Alat:} Papan tulis/whiteboard, spidol, Proyektor \& Speaker, Kertas Plano atau Karton (1 per kelompok), sticky notes warna-warni, penggaris, stopwatch.
\item \textbf{Sumber Belajar:} Buku ajar IPA kelas VIII, Tautan video animasi sistem pernapasan (misal: bit.ly/animasi-sistem-pernapasan), tautan video teknik pernapasan tradisional (misal: bit.ly/pernapasan-tradisional-minang).
\end{itemize}

\section{Target Peserta didik}

\begin{itemize}
\item Peserta didik reguler kelas VIII (Fase D).
\end{itemize}

\section{Model Pembelajaran}

\begin{itemize}
\item Model Pembelajaran KESAN (Konektivitas Etnosains-Sains).
\end{itemize}

\section{Pemahaman Bermakna}
\textit{(Clue untuk Guru: Bacakan atau sampaikan narasi ini dengan intonasi yang menarik di akhir pembelajaran untuk mengikat semua pengalaman belajar siswa menjadi satu kesatuan yang bermakna).}

\begin{tcolorbox}[sectionbox]
"Ananda Semua, hari ini kita telah mengungkap kearifan mendalam di balik teknik pernapasan dalam aktivitas tradisional Minangkabau! Kita menemukan bahwa teknik pernapasan dalam Silek Minang, Randai, dan aktivitas spiritual tradisional ternyata memiliki dasar ilmiah yang kuat dalam mengoptimalkan fungsi sistem respirasi. Setiap teknik pernapasan tradisional - dari pernapasan diafragma yang dalam, kontrol ritme napas yang teratur, hingga koordinasi pernapasan dengan gerakan - sebenarnya bekerja untuk meningkatkan kapasitas paru-paru, efisiensi pertukaran gas, dan kesehatan sistem pernapasan secara keseluruhan. Dengan ini, kita sadar bahwa teknik pernapasan tradisional bukan hanya warisan spiritual, tapi juga strategi ilmiah yang telah terbukti menjaga kesehatan sistem respirasi masyarakat Minangkabau selama berabad-abad."
\end{tcolorbox}

\section{PERTEMUAN PERTAMA: Sistem Pernapasan dan Teknik Pernapasan Tradisional Minangkabau}

\subsection{Capaian Pembelajaran (Fase D)}
Pada akhir Fase D, murid memiliki kemampuan menganalisis sistem organisasi kehidupan, fungsi, serta kelainan atau gangguan yang muncul pada sistem organ makhluk hidup.

\subsection{Tujuan Pembelajaran (TP) Pertemuan 1:}
\textit{(Clue untuk Guru: Tujuan ini adalah kompas Anda. Pastikan setiap tahapan KESAN yang Anda lalui berkontribusi pada pencapaian tujuan-tujuan ini).}

Melalui model pembelajaran KESAN, peserta didik mampu:
\begin{itemize}
\item Menghubungkan fenomena kearifan lokal Minangkabau (teknik pernapasan tradisional) dengan konteks sistem pernapasan dalam sains modern. (Sintaks K)
\item Merumuskan pertanyaan investigatif mengenai struktur dan fungsi sistem pernapasan serta hubungannya dengan teknik pernapasan tradisional. (Sintaks E)
\item Mengumpulkan informasi mengenai komponen sistem pernapasan dari sumber sains serta teknik pernapasan tradisional dari sumber kultural. (Sintaks S)
\item Menganalisis dan menyintesiskan hubungan sebab-akibat antara teknik pernapasan tradisional Minangkabau dengan efisiensi sistem respirasi. (Sintaks A)
\item Menyusun sebuah penjelasan analitis yang logis mengenai bagaimana teknik pernapasan tradisional Minangkabau dapat menjadi model latihan respirasi yang efektif. (Sintaks N)
\end{itemize}

\subsection{Pertanyaan Pemantik}
\textit{(Clue untuk Guru: Ajukan dua pertanyaan ini secara berurutan, berikan jeda agar siswa berpikir. Jangan langsung minta jawaban, biarkan pertanyaan ini menggantung untuk memicu rasa ingin tahu).}

\begin{itemize}
\item "Pernahkah kalian memperhatikan bahwa pesilat Minang atau penari Randai memiliki stamina pernapasan yang luar biasa? Mereka bisa bergerak intens dalam waktu lama tanpa terengah-engah. Apa rahasia di balik teknik pernapasan mereka?"
\item "Dalam filosofi Minang ada ungkapan 'Napeh panjang, hiduik panjang' - napas panjang, hidup panjang. Dari sudut pandang ilmu respirasi, bagaimana sebenarnya teknik pernapasan yang benar dapat memperpanjang dan meningkatkan kualitas hidup?"
\end{itemize}

\section{Langkah-langkah Kegiatan Pembelajaran:}

\subsection{Kegiatan Pembuka (15 Menit)}
\begin{itemize}
\item Guru membuka pelajaran dengan salam, doa, dan memeriksa kehadiran.
\item \textbf{Asesmen Diagnostik Awal:} Guru membagikan lembar K-W-L.
    \begin{itemize}
    \item \textbf{Instruksi Guru:} "Ananda, sebelum kita mulai menjelajahi dunia sistem pernapasan, tolong isi dua kolom pertama di lembar ini. Di kolom \textbf{K (Tahu)}, tulis apa saja yang sudah kalian ketahui tentang paru-paru dan pernapasan ATAU teknik pernapasan dalam aktivitas tradisional. Di kolom \textbf{W (Ingin Tahu)}, tulis apa yang membuat kalian penasaran tentang topik ini." \textit{(Clue: Ini membantu Anda memetakan pengetahuan awal dan minat siswa secara cepat).}
    \end{itemize}
\item \textbf{Apersepsi:}
    \begin{itemize}
    \item \textbf{Instruksi Guru:} "Hari ini kita akan menjadi ahli respirasi dan praktisi pernapasan tradisional sekaligus. Kita akan menyelidiki bagaimana teknik pernapasan dalam aktivitas tradisional Minangkabau dapat mengoptimalkan fungsi sistem pernapasan kita. Dalam investigasi ini, kita akan melatih kemampuan \textbf{Bernalar Kritis} kita, menghargai budaya lewat \textbf{Berkebinekaan Global}, dan bekerja sama dalam semangat \textbf{Gotong Royong}. Siap menjadi peneliti pernapasan tradisional?"
    \end{itemize}
\end{itemize}

\subsection{Kegiatan Inti (90 Menit) - SINTAKS MODEL KESAN}

\subsubsection{Tahap 1: (K) Kaitkan Konteks Kultural (15 menit)}
\begin{itemize}
\item \textbf{Aktivitas Guru:}
    \begin{itemize}
    \item Menampilkan video/gambar teknik pernapasan dalam Silek Minang dan Randai, serta mengajak siswa mengamati pola pernapasan yang digunakan.
    \item Mengajukan Pertanyaan Pemantik yang sudah disiapkan di atas.
    \item \textit{(Clue: Tujuan tahap ini adalah memvalidasi pengetahuan siswa dan memantik rasa heran, bukan mencari jawaban benar. Sediakan spidol dan papan tulis/whiteboard. Saat siswa menjawab, tuliskan semua ide mereka, bahkan yang keliru sekalipun, dengan judul "\textbf{PENGETAHUAN AWAL KITA}". Ini menunjukkan bahwa semua pemikiran dihargai).}
    \end{itemize}
\item \textbf{Aktivitas Siswa:} Mengamati video teknik pernapasan, mendengarkan pertanyaan, lalu secara sukarela berbagi pengalaman atau pengetahuan awal tentang teknik pernapasan tradisional atau sistem pernapasan. Menuliskan minimal satu pertanyaan atau pengalaman di sticky notes dan menempelkannya di '\textbf{Papan Penasaran}'.
\end{itemize}

\subsubsection{Tahap 2: (E) Eksplorasi Enigma (15 menit)}
\begin{itemize}
\item \textbf{Aktivitas Guru:} Membentuk siswa menjadi kelompok (3-4 orang).
    \begin{itemize}
    \item \textbf{Instruksi Guru:} "Baik, rasa penasaran kalian luar biasa! Sekarang, tugas kita sebagai peneliti adalah mengubah rasa penasaran ini menjadi misi yang jelas. Dalam kelompok, diskusikan dan rumuskan minimal 3 pertanyaan kunci yang akan kita selidiki hari ini. Tuliskan dalam bentuk '\textbf{Misi Penelitian Tim [Nama Kelompok]}' di kertas plano yang Bapak/Ibu berikan."
    \item \textit{(Clue: Arahkan diskusi siswa agar pertanyaannya mencakup aspek 'bagaimana sistem pernapasan bekerja' dan 'mengapa teknik pernapasan tradisional efektif'. Jika kelompok kesulitan, berikan pancingan: "Kira-kira, apa dulu yang perlu kita tahu? Bagaimana paru-paru bekerja atau mengapa pernapasan dalam baik untuk kesehatan?").}
    \end{itemize}
\item \textbf{Aktivitas Siswa:} Berdiskusi dalam tim untuk merumuskan misi penelitian (daftar pertanyaan kunci) di kertas plano. \textit{(Contoh misi yang diharapkan: 1. Bagaimana struktur dan fungsi sistem pernapasan? 2. Apa teknik pernapasan yang digunakan dalam aktivitas tradisional Minang? 3. Mengapa teknik pernapasan tradisional efektif untuk kesehatan?)}
\end{itemize}

\subsubsection{Tahap 3: (S) Selidiki secara Sintetis (25 menit)}
\begin{itemize}
\item \textbf{Aktivitas Guru:} Membagikan "\textbf{LKPD 13 - Jurnal Investigasi Dual-Lensa}".
    \begin{itemize}
    \item \textbf{Instruksi Guru:} "Setiap tim akan melakukan investigasi dari dua lensa. Gunakan HP atau sumber yang disediakan untuk mencari jawabannya. Bagilah tugas dalam tim!"
    \item \textbf{Lensa Sains:} Buka link video bit.ly/animasi-sistem-pernapasan untuk memahami struktur sistem pernapasan (hidung, trakea, bronkus, paru-paru, alveolus), proses inspirasi dan ekspirasi, serta pertukaran gas oksigen dan karbon dioksida.
    \item \textbf{Lensa Etnosains/Kultural:} Buka link video bit.ly/pernapasan-tradisional-minang untuk memahami teknik pernapasan dalam Silek Minang, Randai, dan praktik spiritual tradisional serta manfaatnya menurut kearifan lokal.
    \item \textit{(Clue: Pastikan sumber belajar sudah disiapkan dan link bisa diakses. Berkelilinglah untuk memastikan setiap kelompok membagi tugas dan tidak hanya fokus pada satu lensa saja).}
    \end{itemize}
\item \textbf{Aktivitas Siswa:} Dalam kelompok, siswa berbagi tugas mencari informasi dari sumber yang diberikan dan mencatat temuan kunci di dua kolom terpisah pada "\textbf{LKPD 13 - Jurnal Investigasi Dual-Lensa}".
\end{itemize}

\subsubsection{Tahap 4: (A) Asimilasi Analitis (20 menit)}
\begin{itemize}
\item \textbf{Aktivitas Guru:} Memfasilitasi diskusi untuk menjembatani kedua lensa.
    \begin{itemize}
    \item \textbf{Pertanyaan Pancingan Kunci untuk Guru:}
        \begin{itemize}
        \item "Oke, dari Lensa Sains kita tahu pertukaran gas terjadi di alveolus dan pernapasan diafragma lebih efisien. Dari Lensa Etnosains, kita tahu teknik pernapasan dalam Silek menggunakan pernapasan dalam dan teratur. Nah, coba hubungkan! Mengapa pernapasan dalam lebih baik?"
        \item "Dari Lensa Sains, kita tahu oksigen dibutuhkan untuk metabolisme sel. Dari Lensa Budaya, teknik pernapasan tradisional meningkatkan stamina dan konsentrasi. Apa kesamaan prinsipnya?"
        \end{itemize}
    \item \textit{(Clue: Fokuskan untuk membuat siswa 'menemukan' hubungannya sendiri, bukan diberitahu. Gunakan kata "menurut kalian", "kira-kira kenapa", "ada yang punya ide?").}
    \end{itemize}
\item \textbf{Aktivitas Siswa:} Berdiskusi intensif untuk menghubungkan temuan sains dan budaya. Menuliskan kesimpulan terpadu (sintesis) mereka di kertas plano.
\end{itemize}

\subsubsection{Tahap 5: (N) Nyatakan Pemahaman (15 menit)}
\begin{itemize}
\item \textbf{Aktivitas Guru:} Memberikan studi kasus individual atau per kelompok.
    \begin{itemize}
    \item \textbf{Instruksi Guru (tuliskan di papan tulis):}
    
    "\textbf{STUDI KASUS UNTUK PENELITI SAINS-BUDAYA:}
    
    Sebuah pusat terapi pernapasan di Padang ingin mengembangkan program latihan pernapasan yang mengintegrasikan teknik pernapasan tradisional Minangkabau dengan terapi respirasi modern. Mereka percaya bahwa program ini akan lebih mudah diterima masyarakat lokal dan sekaligus melestarikan budaya. Tim terapis membutuhkan penjelasan ilmiah mengapa teknik pernapasan tradisional Minang efektif untuk kesehatan sistem respirasi.
    
    Tugasmu: Tuliskan sebuah penjelasan ilmiah singkat (4-6 kalimat) di buku latihanmu untuk membantu tim terapis memahami mengapa teknik pernapasan tradisional Minangkabau dapat menjadi terapi respirasi yang efektif. Gunakan pengetahuan gabungan dari sains dan kearifan tradisional yang baru saja kamu pelajari."
    \end{itemize}
\item \textbf{Aktivitas Siswa:} Menyusun argumen tertulis untuk menjawab studi kasus yang diberikan, menggunakan bukti dari kedua lensa.
\end{itemize}

\subsection{Kegiatan Penutup (15 Menit)}
\begin{itemize}
\item \textbf{Presentasi \& Penguatan:} Guru meminta 2-3 siswa secara acak untuk membacakan penjelasan mereka. Guru memberikan pujian dan penguatan positif, menekankan betapa hebatnya kemampuan mereka mengintegrasikan sains dan budaya.
\item \textbf{Refleksi:}
    \begin{itemize}
    \item \textbf{Instruksi Guru:} "Sekarang, kembali ke lembar K-W-L kalian. Lengkapi kolom terakhir, \textbf{L (Learned)}, dengan hal-hal baru yang paling mengagumkan yang kalian pelajari hari ini."
    \item \textbf{Instruksi Guru:} "Angkat tangan, siapa yang setelah belajar hari ini ingin mencoba teknik pernapasan tradisional Minang untuk meningkatkan kesehatan?"
    \end{itemize}
\item \textbf{Tindak Lanjut:}
    \begin{itemize}
    \item \textbf{Instruksi Guru:} "Luar biasa, para peneliti! Hari ini kita sudah mengungkap bagaimana teknik pernapasan tradisional dapat mengoptimalkan sistem respirasi. Pertemuan berikutnya, kita akan mempelajari gangguan sistem pernapasan dan bagaimana pengobatan tradisional Minang dapat membantu pencegahannya. Kita akan lihat!"
    \item \textit{(Clue: Buat transisi yang menarik ke pertemuan berikutnya sambil mempertahankan semangat investigasi yang sudah terbangun).}
    \end{itemize}
\item Guru menutup pelajaran dengan doa dan salam.
\end{itemize}

\section{Asesmen (Penilaian)}

\begin{itemize}
\item \textbf{Asesmen Diagnostik (Awal):} Analisis lembar K-W-L. (Untuk mengetahui baseline siswa).
\item \textbf{Asesmen Formatif (Proses):} Observasi keaktifan diskusi (gotong royong) dan penilaian kelengkapan "\textbf{LKPD 13 - Jurnal Investigasi Dual-Lensa}".
\item \textbf{Asesmen Sumatif (Akhir Siklus):} Penilaian jawaban studi kasus (Tahap N) menggunakan rubrik.
\end{itemize}

\subsection{Rubrik Penilaian Jawaban Studi Kasus (Tahap N)}

\begin{longtable}{|p{3cm}|p{3cm}|p{3cm}|p{3cm}|p{3cm}|}
\hline
\textbf{Kriteria Penilaian} & \textbf{Skor 4 (Sangat Baik)} & \textbf{Skor 3 (Baik)} & \textbf{Skor 2 (Cukup)} & \textbf{Skor 1 (Kurang)} \\
\hline
\textbf{Ketepatan Konsep Ilmiah} & Menggunakan istilah ilmiah (alveolus, diafragma, pertukaran gas, respirasi) dengan sangat tepat dan relevan dengan kasus. & Menggunakan istilah ilmiah dengan tepat, namun kurang relevan. & Menggunakan istilah ilmiah namun ada beberapa kesalahan konsep. & Tidak menggunakan istilah ilmiah atau salah total. \\
\hline
\textbf{Keterkaitan dengan Etnosains} & Mampu menghubungkan secara logis dan eksplisit antara teknik pernapasan tradisional dengan efisiensi respirasi secara mendalam. & Mampu menghubungkan teknik pernapasan tradisional dengan kesehatan, namun kurang mendalam. & Hanya menyebutkan teknik pernapasan tanpa menghubungkan dengan sains, atau sebaliknya. & Tidak ada keterkaitan antara sains dan budaya yang ditunjukkan. \\
\hline
\textbf{Kelogisan \& Struktur Argumen} & Penjelasan sangat logis, runtut, mudah dipahami, dan didukung bukti yang kuat. & Penjelasan logis dan runtut, namun kurang didukung bukti. & Alur penjelasan kurang runtut atau sulit dipahami. & Penjelasan tidak logis dan tidak terstruktur. \\
\hline
\end{longtable}

\section{Daftar Pustaka Sumber Etnosains}

\begin{enumerate}
\item Navis, A.A. (1984). \textit{Alam Terkembang Jadi Guru: Adat dan Kebudayaan Minangkabau}. Jakarta: Grafiti Pers.
\item Kato, T. (2005). \textit{Adat Minangkabau dan Merantau dalam Perspektif Sejarah}. Jakarta: Balai Pustaka.
\item Dt. Rajo Penghulu, I. (1994). \textit{Pegangan Penghulu, Bundo Kanduang dan Pidato Alua Pasambahan Adat Minangkabau}. Bukittinggi: Pustaka Indonesia.
\item Syafwan, A. (2018). \textit{Silek Minangkabau: Filosofi dan Teknik Bela Diri Tradisional}. Padang: Andalas University Press.
\end{enumerate}

\end{document}