\documentclass[a4paper,12pt]{article}
\usepackage[a4paper, margin=1.27cm]{geometry}
\usepackage[indonesian]{babel}
\usepackage[utf8]{inputenc}
\usepackage{tcolorbox}
\usepackage{array}
\usepackage{multirow}
\usepackage{setspace}
\usepackage{amssymb}
\usepackage{graphicx}
\usepackage{adjustbox}
\usepackage{enumitem}
\usepackage{longtable}
\usepackage{xcolor}
\usepackage{amsmath}
\usepackage{fancyhdr}
\usepackage{titlesec}

% Define colors (black and white theme)
\definecolor{darkgray}{RGB}{64, 64, 64}
\definecolor{lightgray}{RGB}{240, 240, 240}
\definecolor{mediumgray}{RGB}{128, 128, 128}

% Custom tcolorbox styles
\tcbset{
    mainbox/.style={
        colback=white,
        colframe=black,
        boxrule=1pt,
        arc=3pt,
        left=8pt,
        right=8pt,
        top=8pt,
        bottom=8pt
    },
    sectionbox/.style={
        colback=white,
        colframe=black,
        boxrule=1.5pt,
        arc=2pt,
        left=6pt,
        right=6pt,
        top=6pt,
        bottom=6pt
    }
}

\begin{document}

\begin{center}
{\Large\textbf{MODUL AJAR: Apotek Hidup Minangkabau \& Rahasia Struktur Tumbuhan}}
\end{center}

\vspace{0.5cm}

\begin{tcolorbox}[mainbox]
\textbf{Nama Penyusun:} Irfan Ananda Ismail, S.Pd., M.Pd., Gr \\
\textbf{Institusi:} SMP \\
\textbf{Mata Pelajaran:} Ilmu Pengetahuan Alam (IPA) \\
\textbf{Tahun Ajaran:} 2025/2026 \\
\textbf{Semester:} Ganjil \\
\textbf{Jenjang Sekolah:} SMP \\
\textbf{Kelas/Fase:} VIII / D \\
\textbf{Alokasi waktu:} 3 x 40 Menit (1 Pertemuan)
\end{tcolorbox}

\section{DIMENSI PROFIL PELAJAR PANCASILA}
\textit{(Clue untuk Guru: Sebutkan dimensi ini secara eksplisit saat apersepsi agar siswa sadar tujuan non-akademis yang sedang mereka kembangkan).}

\begin{itemize}
\item \textbf{Berkebinekaan Global:} Mengenal dan menghargai budaya, khususnya menganalisis kearifan lokal Minangkabau dalam pemanfaatan tanaman obat tradisional (apotek hidup), lalu menghubungkannya dengan konsep sains universal tentang struktur dan fungsi organ tumbuhan.
\item \textbf{Bernalar Kritis:} Ditempa secara intensif saat menganalisis informasi dari sumber sains dan budaya (Tahap S), menyintesiskan kedua perspektif (Tahap A), dan menyusun argumen berbasis bukti (Tahap N).
\item \textbf{Gotong Royong:} Kemampuan untuk bekerja sama secara kolaboratif dalam kelompok untuk merumuskan masalah (Tahap E), melakukan investigasi (Tahap S), dan membangun pemahaman bersama (Tahap A).
\item \textbf{Kreatif:} Menghasilkan argumen atau solusi orisinal yang terintegrasi pada tahap akhir pembelajaran (Tahap N).
\end{itemize}

\section{Sarana dan Prasarana}

\begin{itemize}
\item \textbf{Media:} LKPD 07 - Jurnal Investigasi Dual-Lensa, video singkat apotek hidup tradisional Minangkabau (durasi 3-4 menit), gambar/diagram struktur organ tumbuhan, artikel tentang tanaman obat tradisional, sampel tanaman obat (jika tersedia) atau gambar berkualitas tinggi.
\item \textbf{Alat:} Papan tulis/whiteboard, spidol, Proyektor \& Speaker, Kertas Plano atau Karton (1 per kelompok), sticky notes warna-warni, lup/kaca pembesar (jika ada), penggaris untuk mengukur.
\item \textbf{Sumber Belajar:} Buku ajar IPA kelas VIII, Tautan video animasi struktur tumbuhan (misal: bit.ly/animasi-struktur-tumbuhan), tautan artikel/video tanaman obat Minangkabau (misal: bit.ly/apotek-hidup-minang).
\end{itemize}

\section{Target Peserta didik}

\begin{itemize}
\item Peserta didik reguler kelas VIII (Fase D).
\end{itemize}

\section{Model Pembelajaran}

\begin{itemize}
\item Model Pembelajaran KESAN (Konektivitas Etnosains-Sains).
\end{itemize}

\section{Pemahaman Bermakna}
\textit{(Clue untuk Guru: Bacakan atau sampaikan narasi ini dengan intonasi yang menarik di akhir pembelajaran untuk mengikat semua pengalaman belajar siswa menjadi satu kesatuan yang bermakna).}

\begin{tcolorbox}[sectionbox]
"Ananda Semua, hari ini kita telah mengungkap keajaiban di balik apotek hidup yang masih terpelihara di halaman-halaman rumah tradisional Minangkabau! Kita menemukan bahwa setiap bagian tumbuhan - dari akar hingga daun - memiliki struktur dan fungsi yang sangat spesifik, dan nenek moyang kita telah memahami hal ini secara intuitif untuk pengobatan. Setiap helai daun sirih yang dikunyah, setiap rebusan kunyit yang diminum, semuanya berdasarkan pemahaman mendalam tentang bagaimana struktur organ tumbuhan menghasilkan senyawa aktif yang bermanfaat bagi kesehatan. Dengan ini, kita sadar bahwa pengetahuan tradisional adalah hasil observasi sistematis terhadap alam yang telah dipraktikkan turun-temurun."
\end{tcolorbox}

\section{PERTEMUAN PERTAMA: Struktur dan Fungsi Organ Tumbuhan dalam Kearifan Tradisional}

\subsection{Capaian Pembelajaran (Fase D)}
Pada akhir Fase D, murid memiliki kemampuan menjelaskan keterkaitan struktur jaringan penyusun organ pada sistem organ dan mengaitkannya dengan fungsinya, melalui berbagai kegiatan seperti pengamatan mikroskopis, percobaan, dan penelusuran informasi dari berbagai sumber.

\subsection{Tujuan Pembelajaran (TP) Pertemuan 1:}
\textit{(Clue untuk Guru: Tujuan ini adalah kompas Anda. Pastikan setiap tahapan KESAN yang Anda lalui berkontribusi pada pencapaian tujuan-tujuan ini).}

Melalui model pembelajaran KESAN, peserta didik mampu:
\begin{itemize}
\item Menghubungkan fenomena kearifan lokal Minangkabau (pemanfaatan tanaman obat tradisional) dengan konteks struktur dan fungsi organ tumbuhan. (Sintaks K)
\item Merumuskan pertanyaan investigatif mengenai hubungan antara struktur organ tumbuhan dengan khasiat obat yang dihasilkan. (Sintaks E)
\item Mengumpulkan informasi mengenai struktur dan fungsi organ tumbuhan dari sumber ilmiah serta pengetahuan tanaman obat tradisional dari sumber kultural. (Sintaks S)
\item Menganalisis dan menyintesiskan hubungan sebab-akibat antara struktur organ tumbuhan dengan produksi senyawa aktif dalam pengobatan tradisional. (Sintaks A)
\item Menyusun sebuah penjelasan analitis yang logis mengenai bagaimana pengetahuan struktur tumbuhan mendukung pemanfaatan tanaman obat secara optimal. (Sintaks N)
\end{itemize}

\subsection{Pertanyaan Pemantik}
\textit{(Clue untuk Guru: Ajukan dua pertanyaan ini secara berurutan, berikan jeda agar siswa berpikir. Jangan langsung minta jawaban, biarkan pertanyaan ini menggantung untuk memicu rasa ingin tahu).}

\begin{itemize}
\item "Pernahkah kalian melihat nenek atau ibu menggunakan daun sirih untuk obat sariawan, atau kunyit untuk sakit perut? Mengapa bagian tumbuhan yang berbeda digunakan untuk penyakit yang berbeda pula?"
\item "Coba perhatikan tanaman di sekitar kita: mengapa obat untuk batuk biasanya dari akar (seperti akar alang-alang), sedangkan obat untuk luka luar dari daun (seperti daun sirih)? Apa hubungannya dengan struktur dan fungsi organ tumbuhan tersebut?"
\end{itemize}

\section{Langkah-langkah Kegiatan Pembelajaran:}

\subsection{Kegiatan Pembuka (15 Menit)}
\begin{itemize}
\item Guru membuka pelajaran dengan salam, doa, dan memeriksa kehadiran.
\item \textbf{Asesmen Diagnostik Awal:} Guru membagikan lembar K-W-L.
    \begin{itemize}
    \item \textbf{Instruksi Guru:} "Ananda, setelah mengeksplorasi pesawat sederhana dalam teknologi tradisional, sekarang kita akan menyelidiki keajaiban alam: tumbuhan! Di lembar K-W-L ini, tulis di kolom \textbf{K (Tahu)} apa yang sudah kalian ketahui tentang bagian-bagian tumbuhan ATAU tentang tanaman obat yang pernah kalian lihat atau gunakan. Di kolom \textbf{W (Ingin Tahu)}, tulis apa yang membuat kalian penasaran tentang mengapa tumbuhan bisa menjadi obat." \textit{(Clue: Ini membantu Anda melihat pemahaman awal siswa tentang organ tumbuhan dan pengalaman mereka dengan tanaman obat).}
    \end{itemize}
\item \textbf{Apersepsi:}
    \begin{itemize}
    \item \textbf{Instruksi Guru:} "Setelah mengungkap rahasia teknologi tradisional, sekarang kita akan menyelidiki apotek hidup yang ada di sekitar kita! Misi kita adalah memahami bagaimana struktur dan fungsi organ tumbuhan memungkinkan nenek moyang kita menciptakan sistem pengobatan tradisional yang efektif. Siap menjadi ahli botani yang menghargai kearifan pengobatan tradisional?"
    \end{itemize}
\end{itemize}

\subsection{Kegiatan Inti (90 Menit) - SINTAKS MODEL KESAN}

\subsubsection{Tahap 1: (K) Kaitkan Konteks Kultural (15 menit)}
\begin{itemize}
\item \textbf{Aktivitas Guru:}
    \begin{itemize}
    \item Menampilkan video singkat (3-4 menit) apotek hidup tradisional Minangkabau, menunjukkan berbagai tanaman obat dan cara penggunaannya oleh masyarakat.
    \item Mengajukan Pertanyaan Pemantik yang sudah disiapkan di atas.
    \item \textit{(Clue: Fokuskan pada keingintahuan tentang mengapa bagian tumbuhan yang berbeda memiliki khasiat yang berbeda. Saat siswa menjawab, tuliskan semua ide mereka di papan tulis dengan judul "\textbf{APOTEK HIDUP DI SEKITAR KITA}". Validasi setiap kontribusi dan tunjukkan antusiasme terhadap pengalaman mereka).}
    \end{itemize}
\item \textbf{Aktivitas Siswa:} Mengamati video, mendengarkan pertanyaan, lalu secara sukarela berbagi pengalaman atau pengetahuan tentang tanaman obat yang pernah mereka lihat atau gunakan. Menuliskan minimal satu pertanyaan atau pengamatan di sticky notes dan menempelkannya di '\textbf{Papan Penasaran}'.
\end{itemize}

\subsubsection{Tahap 2: (E) Eksplorasi Enigma (15 menit)}
\begin{itemize}
\item \textbf{Aktivitas Guru:} Membentuk siswa menjadi kelompok (3-4 orang).
    \begin{itemize}
    \item \textbf{Instruksi Guru:} "Keingintahuan kalian tentang apotek hidup sangat menginspirasi! Sekarang, sebagai tim peneliti etnobotani, tugas kalian adalah mengubah keingintahuan ini menjadi misi investigasi yang sistematis. Dalam kelompok, diskusikan dan rumuskan minimal 3 pertanyaan kunci yang akan kita selidiki hari ini. Tuliskan dalam bentuk '\textbf{Misi Investigasi Tim [Nama Kelompok]}' di kertas plano."
    \item \textit{(Clue: Arahkan diskusi siswa agar pertanyaannya mencakup aspek 'bagaimana struktur organ tumbuhan' dan 'mengapa bagian tertentu berkhasiat obat'. Jika kelompok kesulitan, berikan pancingan: "Kira-kira, apa yang membuat akar, batang, daun, dan bunga memiliki fungsi yang berbeda? Dan bagaimana hal itu berhubungan dengan khasiat obatnya?").}
    \end{itemize}
\item \textbf{Aktivitas Siswa:} Berdiskusi dalam tim untuk merumuskan misi investigasi (daftar pertanyaan kunci) di kertas plano. \textit{(Contoh misi yang diharapkan: 1. Bagaimana struktur akar, batang, daun, dan bunga berbeda? 2. Mengapa bagian tumbuhan tertentu berkhasiat obat? 3. Apa hubungan antara fungsi organ dengan senyawa yang dihasilkan?)}
\end{itemize}

\subsubsection{Tahap 3: (S) Selidiki secara Sintetis (25 menit)}
\begin{itemize}
\item \textbf{Aktivitas Guru:} Membagikan "\textbf{LKPD 07 - Jurnal Investigasi Dual-Lensa}".
    \begin{itemize}
    \item \textbf{Instruksi Guru:} "Setiap tim akan melakukan investigasi dari dua lensa. Gunakan sumber yang disediakan untuk mencari jawabannya. Bagilah tugas dalam tim!"
    \item \textbf{Lensa Sains:} Buka link video bit.ly/animasi-struktur-tumbuhan untuk memahami struktur dan fungsi organ tumbuhan (akar, batang, daun, bunga), jaringan penyusun, dan proses fisiologi tumbuhan.
    \item \textbf{Lensa Etnosains/Kultural:} Buka link artikel bit.ly/apotek-hidup-minang untuk memahami jenis-jenis tanaman obat tradisional Minangkabau, bagian yang digunakan, cara pengolahan, dan khasiatnya.
    \item \textit{(Clue: Pastikan sumber belajar sudah disiapkan dan dapat diakses. Berkelilinglah untuk memastikan setiap kelompok mengeksplorasi kedua lensa dengan fokus pada hubungan struktur-fungsi).}
    \end{itemize}
\item \textbf{Aktivitas Siswa:} Dalam kelompok, siswa berbagi tugas mencari informasi dari sumber yang diberikan dan mencatat temuan kunci di dua kolom terpisah pada "\textbf{LKPD 07 - Jurnal Investigasi Dual-Lensa}".
\end{itemize}

\subsubsection{Tahap 4: (A) Asimilasi Analitis (20 menit)}
\begin{itemize}
\item \textbf{Aktivitas Guru:} Memfasilitasi diskusi untuk menjembatani kedua lensa.
    \begin{itemize}
    \item \textbf{Pertanyaan Pancingan Kunci untuk Guru:}
        \begin{itemize}
        \item "Dari Lensa Sains kita tahu akar berfungsi menyerap air dan mineral, daun untuk fotosintesis. Dari Lensa Etnosains, kita tahu akar kunyit untuk sakit perut, daun sirih untuk antiseptik. Nah, coba hubungkan! Bagaimana fungsi organ mempengaruhi jenis senyawa yang dihasilkan?"
        \item "Dari Lensa Sains, kita tahu tumbuhan menghasilkan metabolit sekunder untuk pertahanan. Dari Lensa Budaya, kita tahu senyawa ini yang berkhasiat obat. Mengapa nenek moyang kita bisa memilih bagian tumbuhan yang tepat untuk penyakit tertentu?"
        \end{itemize}
    \item \textit{(Clue: Bantu siswa melihat bahwa organ yang berbeda menghasilkan senyawa yang berbeda sesuai fungsinya. Akar menyimpan cadangan makanan dan senyawa aktif, daun menghasilkan senyawa untuk fotosintesis dan pertahanan).}
    \end{itemize}
\item \textbf{Aktivitas Siswa:} Berdiskusi intensif untuk menghubungkan temuan sains dan budaya. Menuliskan kesimpulan terpadu (sintesis) mereka di kertas plano.
\end{itemize}

\subsubsection{Tahap 5: (N) Nyatakan Pemahaman (15 menit)}
\begin{itemize}
\item \textbf{Aktivitas Guru:} Memberikan studi kasus individual atau per kelompok.
    \begin{itemize}
    \item \textbf{Instruksi Guru (tuliskan di papan tulis):}
    
    "\textbf{STUDI KASUS UNTUK PENELITI ETNOBOTANI SAINS-BUDAYA:}
    
    Sebuah komunitas peneliti ingin mengembangkan obat herbal modern dari tanaman tradisional Minangkabau. Mereka menemukan bahwa masyarakat tradisional menggunakan akar jahe untuk masuk angin, daun sirih untuk luka, dan bunga rosella untuk hipertensi. Mereka meminta bantuanmu sebagai konsultan yang memahami baik botani maupun etnobotani.
    
    Tugasmu: Tuliskan sebuah penjelasan ilmiah singkat (3-5 kalimat) di buku latihanmu tentang mengapa bagian tumbuhan yang berbeda digunakan untuk penyakit yang berbeda, berdasarkan pemahaman struktur dan fungsi organ tumbuhan. Jelaskan bagaimana pengetahuan sains modern dapat memvalidasi dan mengembangkan kearifan tradisional."
    \end{itemize}
\item \textbf{Aktivitas Siswa:} Menyusun argumen tertulis untuk menjawab studi kasus yang diberikan, menggunakan bukti dari kedua lensa.
\end{itemize}

\subsection{Kegiatan Penutup (15 Menit)}
\begin{itemize}
\item \textbf{Presentasi \& Penguatan:} Guru meminta 2-3 siswa secara acak untuk membacakan penjelasan ilmiah mereka. Guru memberikan pujian dan penguatan positif, menekankan betapa hebatnya kemampuan mereka mengintegrasikan pengetahuan botani dengan kearifan tradisional.
\item \textbf{Refleksi:}
    \begin{itemize}
    \item \textbf{Instruksi Guru:} "Sekarang, kembali ke lembar K-W-L kalian. Lengkapi kolom terakhir, \textbf{L (Learned)}, dengan hal-hal baru yang paling mengagumkan yang kalian pelajari hari ini tentang struktur dan fungsi tumbuhan."
    \item \textbf{Instruksi Guru:} "Angkat tangan, siapa yang setelah belajar hari ini jadi lebih menghargai kearifan nenek moyang kita dalam memahami dan memanfaatkan tumbuhan untuk kesehatan?"
    \end{itemize}
\item \textbf{Tindak Lanjut:}
    \begin{itemize}
    \item \textbf{Instruksi Guru:} "Luar biasa, para peneliti etnobotani! Hari ini kita sudah mengungkap bagaimana struktur organ tumbuhan mendukung kearifan pengobatan tradisional. Pertemuan berikutnya, kita akan mengeksplorasi lebih dalam tentang proses-proses yang terjadi dalam tumbuhan dan bagaimana hal itu mempengaruhi kualitas tanaman obat. Siapa tahu kita bisa menemukan rahasia lain dari apotek hidup di sekitar kita!"
    \item \textit{(Clue: Buat transisi yang menarik ke pertemuan berikutnya sambil mempertahankan semangat eksplorasi etnobotani).}
    \end{itemize}
\item Guru menutup pelajaran dengan doa dan salam.
\end{itemize}

\section{Asesmen (Penilaian)}

\begin{itemize}
\item \textbf{Asesmen Diagnostik (Awal):} Analisis lembar K-W-L. (Untuk mengetahui pemahaman awal tentang organ tumbuhan dan tanaman obat).
\item \textbf{Asesmen Formatif (Proses):} Observasi keaktifan diskusi (gotong royong) dan penilaian kelengkapan "\textbf{LKPD 07 - Jurnal Investigasi Dual-Lensa}".
\item \textbf{Asesmen Sumatif (Akhir Siklus):} Penilaian jawaban studi kasus (Tahap N) menggunakan rubrik.
\end{itemize}

\subsection{Rubrik Penilaian Jawaban Studi Kasus (Tahap N)}

\begin{longtable}{|p{3cm}|p{3cm}|p{3cm}|p{3cm}|p{3cm}|}
\hline
\textbf{Kriteria Penilaian} & \textbf{Skor 4 (Sangat Baik)} & \textbf{Skor 3 (Baik)} & \textbf{Skor 2 (Cukup)} & \textbf{Skor 1 (Kurang)} \\
\hline
\textbf{Ketepatan Konsep Ilmiah} & Menggunakan konsep struktur dan fungsi organ tumbuhan, metabolit sekunder, dan proses fisiologi dengan sangat tepat dan relevan untuk menjelaskan khasiat obat. & Menggunakan konsep struktur dan fungsi organ tumbuhan dengan tepat, namun kurang relevan dengan konteks etnobotani. & Menggunakan konsep botani namun ada beberapa kesalahan dalam menjelaskan hubungan struktur-fungsi. & Tidak menggunakan konsep struktur-fungsi tumbuhan atau salah total. \\
\hline
\textbf{Keterkaitan dengan Etnosains} & Mampu menghubungkan secara logis dan eksplisit antara struktur organ tumbuhan dengan penggunaan tradisional secara mendalam dan akurat, menunjukkan apresiasi terhadap kearifan etnobotani. & Mampu menghubungkan konsep botani dengan penggunaan tradisional, namun kurang mendalam dalam apresiasi kearifan lokal. & Hanya menyebutkan penggunaan tradisional tanpa menghubungkan dengan konsep botani, atau sebaliknya. & Tidak ada keterkaitan antara sains dan etnobotani yang ditunjukkan. \\
\hline
\textbf{Kelogisan \& Struktur Argumen} & Penjelasan sangat logis, runtut, ilmiah, dan menunjukkan pemahaman mendalam tentang validasi sains terhadap kearifan tradisional. & Penjelasan logis dan runtut, namun kurang mendalam dalam aspek validasi ilmiah. & Alur penjelasan kurang runtut atau kurang ilmiah dalam pendekatan. & Penjelasan tidak logis dan tidak terstruktur. \\
\hline
\end{longtable}

\section{Daftar Pustaka Sumber Etnosains}

\begin{enumerate}
\item Dt. Rajo Penghulu. (1994). \textit{Teknologi Tradisional Minangkabau}. Padang: Pusat Dokumentasi dan Informasi Kebudayaan Minangkabau.
\item Navis, A.A. (1984). \textit{Alam Terkembang Jadi Guru: Adat dan Kebudayaan Minangkabau}. Jakarta: Grafiti Pers.
\item Kementerian Pendidikan dan Kebudayaan. (2018). \textit{Tanaman Obat Tradisional Nusantara}. Jakarta: Direktorat Warisan dan Diplomasi Budaya.
\item Syafwan, A. (2008). \textit{Kearifan Lokal dalam Teknologi Pertanian Minangkabau}. Padang: Universitas Andalas Press.
\item Zuhud, E.A.M. (2009). \textit{Potensi Hutan Tropika Indonesia sebagai Penyangga Bahan Obat Alam untuk Kesehatan Bangsa}. Jakarta: Fakultas Kehutanan IPB.
\item Hariana, A. (2013). \textit{Tumbuhan Obat dan Khasiatnya Seri 1}. Jakarta: Penebar Swadaya.
\end{enumerate}

\end{document}