\documentclass[a4paper,12pt]{article}
\usepackage[a4paper, margin=1.27cm]{geometry}
\usepackage[indonesian]{babel}
\usepackage[utf8]{inputenc}
\usepackage{tcolorbox}
\usepackage{array}
\usepackage{multirow}
\usepackage{setspace}
\usepackage{amssymb}
\usepackage{graphicx}
\usepackage{adjustbox}
\usepackage{enumitem}
\usepackage{longtable}
\usepackage{xcolor}
\usepackage{amsmath}
\usepackage{fancyhdr}
\usepackage{titlesec}

% Define colors (black and white theme)
\definecolor{darkgray}{RGB}{64, 64, 64}
\definecolor{lightgray}{RGB}{240, 240, 240}
\definecolor{mediumgray}{RGB}{128, 128, 128}

% Custom tcolorbox styles
\tcbset{
    mainbox/.style={
        colback=white,
        colframe=black,
        boxrule=1pt,
        arc=3pt,
        left=8pt,
        right=8pt,
        top=8pt,
        bottom=8pt
    },
    sectionbox/.style={
        colback=white,
        colframe=black,
        boxrule=1.5pt,
        arc=2pt,
        left=6pt,
        right=6pt,
        top=6pt,
        bottom=6pt
    }
}

\begin{document}

\begin{center}
{\Large\textbf{MODUL AJAR: Apotek Hidup Minangkabau \& Sistem Klasifikasi Makhluk Hidup}}
\end{center}

\vspace{0.5cm}

\begin{tcolorbox}[mainbox]
\textbf{Nama Penyusun:} Irfan Ananda Ismail, S.Pd., M.Pd., Gr \\
\textbf{Institusi:} SMP \\
\textbf{Mata Pelajaran:} Ilmu Pengetahuan Alam (IPA) \\
\textbf{Tahun Ajaran:} 2025/2026 \\
\textbf{Semester:} Ganjil \\
\textbf{Jenjang Sekolah:} SMP \\
\textbf{Kelas/Fase:} VIII / D \\
\textbf{Alokasi waktu:} 3 x 40 Menit (1 Pertemuan)
\end{tcolorbox}

\section{DIMENSI PROFIL PELAJAR PANCASILA}
\textit{(Clue untuk Guru: Sebutkan dimensi ini secara eksplisit saat apersepsi agar siswa sadar tujuan non-akademis yang sedang mereka kembangkan).}

\begin{itemize}
\item \textbf{Berkebinekaan Global:} Mengenal dan menghargai budaya, khususnya menganalisis kearifan lokal Minangkabau dalam sistem klasifikasi tumbuhan obat tradisional, lalu menghubungkannya dengan sistem klasifikasi ilmiah modern.
\item \textbf{Bernalar Kritis:} Ditempa secara intensif saat menganalisis informasi dari sumber sains dan budaya (Tahap S), menyintesiskan kedua perspektif (Tahap A), dan menyusun argumen berbasis bukti (Tahap N).
\item \textbf{Gotong Royong:} Kemampuan untuk bekerja sama secara kolaboratif dalam kelompok untuk merumuskan masalah (Tahap E), melakukan investigasi (Tahap S), dan membangun pemahaman bersama (Tahap A).
\item \textbf{Kreatif:} Menghasilkan argumen atau solusi orisinal yang terintegrasi pada tahap akhir pembelajaran (Tahap N).
\end{itemize}

\section{Sarana dan Prasarana}

\begin{itemize}
\item \textbf{Media:} LKPD 05 - Jurnal Investigasi Dual-Lensa, koleksi tumbuhan obat tradisional Minangkabau (segar atau kering), gambar/poster sistem klasifikasi makhluk hidup, artikel tentang pengobatan tradisional Minang, kunci determinasi sederhana.
\item \textbf{Alat:} Papan tulis/whiteboard, spidol, Proyektor \& Speaker, Kertas Plano atau Karton (1 per kelompok), sticky notes warna-warni, lup/kaca pembesar, penggaris.
\item \textbf{Sumber Belajar:} Buku ajar IPA kelas VIII, Tautan video animasi klasifikasi makhluk hidup (misal: bit.ly/animasi-klasifikasi), tautan artikel/video pengobatan tradisional Minang (misal: bit.ly/obat-tradisional-minang).
\end{itemize}

\section{Target Peserta didik}

\begin{itemize}
\item Peserta didik reguler kelas VIII (Fase D).
\end{itemize}

\section{Model Pembelajaran}

\begin{itemize}
\item Model Pembelajaran KESAN (Konektivitas Etnosains-Sains).
\end{itemize}

\section{Pemahaman Bermakna}
\textit{(Clue untuk Guru: Bacakan atau sampaikan narasi ini dengan intonasi yang menarik di akhir pembelajaran untuk mengikat semua pengalaman belajar siswa menjadi satu kesatuan yang bermakna).}

\begin{tcolorbox}[sectionbox]
"Ananda Semua, hari ini kita telah membuka tabir rahasia di balik kearifan nenek moyang Minangkabau dalam mengelompokkan tumbuhan obat! Kita menemukan bahwa sistem klasifikasi tradisional yang diwariskan turun-temurun ternyata memiliki dasar ilmiah yang kuat. Setiap pengelompokan berdasarkan khasiat, bentuk daun, atau cara penggunaan sebenarnya mencerminkan kesamaan karakteristik biologis yang dipelajari dalam sains modern. Dengan ini, kita sadar bahwa ilmu pengetahuan tidak hanya lahir dari laboratorium, tapi juga dari kebijaksanaan lokal yang telah teruji ratusan tahun."
\end{tcolorbox}

\section{PERTEMUAN PERTAMA: Klasifikasi Makhluk Hidup dan Kearifan Lokal Minangkabau}

\subsection{Capaian Pembelajaran (Fase D)}
Pada akhir Fase D, murid memiliki kemampuan menganalisis sistem organisasi kehidupan, fungsi, serta kelainan atau gangguan yang muncul pada sistem organ makhluk hidup.

\subsection{Tujuan Pembelajaran (TP) Pertemuan 1:}
\textit{(Clue untuk Guru: Tujuan ini adalah kompas Anda. Pastikan setiap tahapan KESAN yang Anda lalui berkontribusi pada pencapaian tujuan-tujuan ini).}

Melalui model pembelajaran KESAN, peserta didik mampu:
\begin{itemize}
\item Menghubungkan fenomena kearifan lokal Minangkabau (sistem klasifikasi tumbuhan obat tradisional) dengan konteks klasifikasi makhluk hidup dalam sains modern. (Sintaks K)
\item Merumuskan pertanyaan investigatif mengenai dasar pengelompokan makhluk hidup dan perbandingan sistem klasifikasi tradisional dengan ilmiah. (Sintaks E)
\item Mengumpulkan informasi mengenai sistem klasifikasi ilmiah dari sumber sains serta sistem pengelompokan tumbuhan obat dari sumber kultural. (Sintaks S)
\item Menganalisis dan menyintesiskan hubungan sebab-akibat antara karakteristik morfologi tumbuhan dengan pengelompokan berdasarkan khasiat dalam pengobatan tradisional. (Sintaks A)
\item Menyusun sebuah penjelasan analitis yang logis mengenai bagaimana kearifan lokal dalam klasifikasi tumbuhan obat Minangkabau dapat dibenarkan secara saintifik melalui prinsip-prinsip taksonomi modern. (Sintaks N)
\end{itemize}

\subsection{Pertanyaan Pemantik}
\textit{(Clue untuk Guru: Ajukan dua pertanyaan ini secara berurutan, berikan jeda agar siswa berpikir. Jangan langsung minta jawaban, biarkan pertanyaan ini menggantung untuk memicu rasa ingin tahu).}

\begin{itemize}
\item "Pernahkah kalian memperhatikan bagaimana nenek atau orang tua kalian bisa dengan mudah membedakan tumbuhan obat? Mereka tahu mana yang untuk sakit perut, mana yang untuk demam, padahal tidak pernah belajar biologi. Bagaimana mereka bisa begitu yakin?"
\item "Dalam budaya Minang, ada istilah 'alam takambang jadi guru' - alam terbentang menjadi guru. Dari sudut pandang sains, apa yang sebenarnya dipelajari nenek moyang kita dari alam sehingga mereka bisa mengelompokkan tumbuhan dengan begitu tepat?"
\end{itemize}

\section{Langkah-langkah Kegiatan Pembelajaran:}

\subsection{Kegiatan Pembuka (15 Menit)}
\begin{itemize}
\item Guru membuka pelajaran dengan salam, doa, dan memeriksa kehadiran.
\item \textbf{Asesmen Diagnostik Awal:} Guru membagikan lembar K-W-L.
    \begin{itemize}
    \item \textbf{Instruksi Guru:} "Ananda, sebelum kita mulai menjelajahi dunia klasifikasi, tolong isi dua kolom pertama di lembar ini. Di kolom \textbf{K (Tahu)}, tulis apa saja yang sudah kalian ketahui tentang pengelompokan makhluk hidup ATAU tumbuhan obat tradisional. Di kolom \textbf{W (Ingin Tahu)}, tulis apa yang membuat kalian penasaran tentang topik ini." \textit{(Clue: Ini membantu Anda memetakan pengetahuan awal dan minat siswa secara cepat).}
    \end{itemize}
\item \textbf{Apersepsi:}
    \begin{itemize}
    \item \textbf{Instruksi Guru:} "Hari ini kita akan menjadi ahli botani tradisional dan modern sekaligus. Kita akan menyelidiki bagaimana sistem klasifikasi tumbuhan obat Minangkabau berhubungan dengan ilmu taksonomi modern. Dalam investigasi ini, kita akan melatih kemampuan \textbf{Bernalar Kritis} kita, menghargai budaya lewat \textbf{Berkebinekaan Global}, dan bekerja sama dalam semangat \textbf{Gotong Royong}. Siap menjadi peneliti?"
    \end{itemize}
\end{itemize}

\subsection{Kegiatan Inti (90 Menit) - SINTAKS MODEL KESAN}

\subsubsection{Tahap 1: (K) Kaitkan Konteks Kultural (15 menit)}
\begin{itemize}
\item \textbf{Aktivitas Guru:}
    \begin{itemize}
    \item Menampilkan koleksi tumbuhan obat tradisional Minangkabau (segar atau kering) seperti kunyit, jahe, lengkuas, daun sirih, dll.
    \item Mengajukan Pertanyaan Pemantik yang sudah disiapkan di atas.
    \item \textit{(Clue: Tujuan tahap ini adalah memvalidasi pengetahuan siswa dan memantik rasa heran, bukan mencari jawaban benar. Sediakan spidol dan papan tulis/whiteboard. Saat siswa menjawab, tuliskan semua ide mereka, bahkan yang keliru sekalipun, dengan judul "\textbf{PENGETAHUAN AWAL KITA}". Ini menunjukkan bahwa semua pemikiran dihargai).}
    \end{itemize}
\item \textbf{Aktivitas Siswa:} Mengamati koleksi tumbuhan, mendengarkan pertanyaan, lalu secara sukarela berbagi pengalaman atau pengetahuan awal tentang tumbuhan obat. Menuliskan minimal satu pertanyaan atau pengalaman di sticky notes dan menempelkannya di '\textbf{Papan Penasaran}'.
\end{itemize}

\subsubsection{Tahap 2: (E) Eksplorasi Enigma (15 menit)}
\begin{itemize}
\item \textbf{Aktivitas Guru:} Membentuk siswa menjadi kelompok (3-4 orang).
    \begin{itemize}
    \item \textbf{Instruksi Guru:} "Baik, rasa penasaran kalian luar biasa! Sekarang, tugas kita sebagai peneliti adalah mengubah rasa penasaran ini menjadi misi yang jelas. Dalam kelompok, diskusikan dan rumuskan minimal 3 pertanyaan kunci yang akan kita selidiki hari ini. Tuliskan dalam bentuk '\textbf{Misi Penelitian Tim [Nama Kelompok]}' di kertas plano yang Bapak/Ibu berikan."
    \item \textit{(Clue: Arahkan diskusi siswa agar pertanyaannya mencakup aspek 'bagaimana cara mengelompokkan makhluk hidup' dan 'apa dasar pengelompokan tumbuhan obat tradisional'. Jika kelompok kesulitan, berikan pancingan: "Kira-kira, apa dulu yang perlu kita tahu? Cara ilmuwan mengelompokkan atau cara nenek moyang mengelompokkan?").}
    \end{itemize}
\item \textbf{Aktivitas Siswa:} Berdiskusi dalam tim untuk merumuskan misi penelitian (daftar pertanyaan kunci) di kertas plano. \textit{(Contoh misi yang diharapkan: 1. Bagaimana cara ilmuwan mengelompokkan makhluk hidup? 2. Apa dasar nenek moyang mengelompokkan tumbuhan obat? 3. Apakah ada kesamaan antara kedua sistem klasifikasi tersebut?)}
\end{itemize}

\subsubsection{Tahap 3: (S) Selidiki secara Sintetis (25 menit)}
\begin{itemize}
\item \textbf{Aktivitas Guru:} Membagikan "\textbf{LKPD 05 - Jurnal Investigasi Dual-Lensa}".
    \begin{itemize}
    \item \textbf{Instruksi Guru:} "Setiap tim akan melakukan investigasi dari dua lensa. Gunakan HP atau sumber yang disediakan untuk mencari jawabannya. Bagilah tugas dalam tim!"
    \item \textbf{Lensa Sains:} Buka link video bit.ly/animasi-klasifikasi untuk memahami sistem klasifikasi ilmiah (kingdom, filum, kelas, ordo, famili, genus, spesies). Pelajari juga kunci determinasi yang disediakan.
    \item \textbf{Lensa Etnosains/Kultural:} Buka link artikel bit.ly/obat-tradisional-minang untuk tahu sistem pengelompokan tumbuhan obat dalam budaya Minangkabau berdasarkan khasiat, bentuk, dan cara penggunaan.
    \item \textit{(Clue: Pastikan sumber belajar sudah disiapkan dan link bisa diakses. Berkelilinglah untuk memastikan setiap kelompok membagi tugas dan tidak hanya fokus pada satu lensa saja).}
    \end{itemize}
\item \textbf{Aktivitas Siswa:} Dalam kelompok, siswa berbagi tugas mencari informasi dari sumber yang diberikan dan mencatat temuan kunci di dua kolom terpisah pada "\textbf{LKPD 05 - Jurnal Investigasi Dual-Lensa}".
\end{itemize}

\subsubsection{Tahap 4: (A) Asimilasi Analitis (20 menit)}
\begin{itemize}
\item \textbf{Aktivitas Guru:} Memfasilitasi diskusi untuk menjembatani kedua lensa.
    \begin{itemize}
    \item \textbf{Pertanyaan Pancingan Kunci untuk Guru:}
        \begin{itemize}
        \item "Oke, dari Lensa Sains kita tahu makhluk hidup dikelompokkan berdasarkan kesamaan karakteristik morfologi dan genetik. Dari Lensa Etnosains, kita tahu tumbuhan obat dikelompokkan berdasarkan khasiat dan bentuk. Nah, coba hubungkan! Apakah ada kaitan antara bentuk tumbuhan dengan khasiatnya?"
        \item "Dari Lensa Sains, kita tahu tumbuhan satu famili memiliki karakteristik serupa. Dari Lensa Budaya, tumbuhan dengan bentuk daun serupa sering punya khasiat serupa. Apa hubungannya?"
        \end{itemize}
    \item \textit{(Clue: Fokuskan untuk membuat siswa 'menemukan' hubungannya sendiri, bukan diberitahu. Gunakan kata "menurut kalian", "kira-kira kenapa", "ada yang punya ide?").}
    \end{itemize}
\item \textbf{Aktivitas Siswa:} Berdiskusi intensif untuk menghubungkan temuan sains dan budaya. Menuliskan kesimpulan terpadu (sintesis) mereka di kertas plano.
\end{itemize}

\subsubsection{Tahap 5: (N) Nyatakan Pemahaman (15 menit)}
\begin{itemize}
\item \textbf{Aktivitas Guru:} Memberikan studi kasus individual atau per kelompok.
    \begin{itemize}
    \item \textbf{Instruksi Guru (tuliskan di papan tulis):}
    
    "\textbf{STUDI KASUS UNTUK PENELITI SAINS-BUDAYA:}
    
    Seorang peneliti dari universitas ingin mendokumentasikan kearifan lokal Minangkabau tentang tumbuhan obat untuk dijadikan database ilmiah. Dia menemukan bahwa masyarakat setempat mengelompokkan tumbuhan obat menjadi: 1) Obat 'panas' (untuk penyakit 'dingin' seperti masuk angin), 2) Obat 'dingin' (untuk penyakit 'panas' seperti demam), 3) Obat 'netral' (untuk berbagai penyakit). Namun, dia bingung bagaimana menghubungkan sistem ini dengan klasifikasi ilmiah.
    
    Tugasmu: Tuliskan sebuah rekomendasi singkat (3-5 kalimat) di buku latihanmu untuk membantu peneliti tersebut. Gunakan pengetahuan gabungan dari sains (sistem klasifikasi ilmiah) dan kearifan lokal yang baru saja kamu pelajari untuk menjelaskan bagaimana kedua sistem klasifikasi ini bisa saling melengkapi."
    \end{itemize}
\item \textbf{Aktivitas Siswa:} Menyusun argumen tertulis untuk menjawab studi kasus yang diberikan, menggunakan bukti dari kedua lensa.
\end{itemize}

\subsection{Kegiatan Penutup (15 Menit)}
\begin{itemize}
\item \textbf{Presentasi \& Penguatan:} Guru meminta 2-3 siswa secara acak untuk membacakan rekomendasi mereka. Guru memberikan pujian dan penguatan positif, menekankan betapa hebatnya kemampuan mereka mengintegrasikan sains dan budaya.
\item \textbf{Refleksi:}
    \begin{itemize}
    \item \textbf{Instruksi Guru:} "Sekarang, kembali ke lembar K-W-L kalian. Lengkapi kolom terakhir, \textbf{L (Learned)}, dengan hal-hal baru yang paling mengagumkan yang kalian pelajari hari ini."
    \item \textbf{Instruksi Guru:} "Angkat tangan, siapa yang setelah belajar hari ini jadi lebih menghargai kearifan nenek moyang kita dalam memahami alam?"
    \end{itemize}
\item \textbf{Tindak Lanjut:}
    \begin{itemize}
    \item \textbf{Instruksi Guru:} "Luar biasa, para peneliti! Hari ini kita sudah mengungkap bagaimana sistem klasifikasi tradisional dan modern saling melengkapi. Pertemuan berikutnya, kita akan mendalami lebih jauh tentang sistem organisasi kehidupan. Siapa tahu ada kearifan lokal Minang lainnya yang bisa mengajarkan kita tentang tingkatan organisasi kehidupan? Kita akan lihat!"
    \item \textit{(Clue: Buat transisi yang menarik ke pertemuan berikutnya sambil mempertahankan semangat investigasi yang sudah terbangun).}
    \end{itemize}
\item Guru menutup pelajaran dengan doa dan salam.
\end{itemize}

\section{Asesmen (Penilaian)}

\begin{itemize}
\item \textbf{Asesmen Diagnostik (Awal):} Analisis lembar K-W-L. (Untuk mengetahui baseline siswa).
\item \textbf{Asesmen Formatif (Proses):} Observasi keaktifan diskusi (gotong royong) dan penilaian kelengkapan "\textbf{LKPD 05 - Jurnal Investigasi Dual-Lensa}".
\item \textbf{Asesmen Sumatif (Akhir Siklus):} Penilaian jawaban studi kasus (Tahap N) menggunakan rubrik.
\end{itemize}

\subsection{Rubrik Penilaian Jawaban Studi Kasus (Tahap N)}

\begin{longtable}{|p{3cm}|p{3cm}|p{3cm}|p{3cm}|p{3cm}|}
\hline
\textbf{Kriteria Penilaian} & \textbf{Skor 4 (Sangat Baik)} & \textbf{Skor 3 (Baik)} & \textbf{Skor 2 (Cukup)} & \textbf{Skor 1 (Kurang)} \\
\hline
\textbf{Ketepatan Konsep Ilmiah} & Menggunakan istilah ilmiah (taksonomi, klasifikasi, karakteristik morfologi, famili) dengan sangat tepat dan relevan dengan kasus. & Menggunakan istilah ilmiah dengan tepat, namun kurang relevan. & Menggunakan istilah ilmiah namun ada beberapa kesalahan konsep. & Tidak menggunakan istilah ilmiah atau salah total. \\
\hline
\textbf{Keterkaitan dengan Etnosains} & Mampu menghubungkan secara logis dan eksplisit antara sistem klasifikasi tradisional dengan klasifikasi ilmiah secara mendalam. & Mampu menghubungkan sistem klasifikasi tradisional dengan ilmiah, namun kurang mendalam. & Hanya menyebutkan sistem klasifikasi tanpa menghubungkan keduanya, atau sebaliknya. & Tidak ada keterkaitan antara sains dan budaya yang ditunjukkan. \\
\hline
\textbf{Kelogisan \& Struktur Argumen} & Rekomendasi sangat logis, runtut, praktis, dan mudah dipahami. & Rekomendasi logis dan runtut, namun kurang praktis. & Alur rekomendasi kurang runtut atau sulit dipahami. & Rekomendasi tidak logis dan tidak terstruktur. \\
\hline
\end{longtable}

\section{Daftar Pustaka Sumber Etnosains}

\begin{enumerate}
\item Zuhud, E.A.M. (2009). \textit{Potensi Hutan Tropika Indonesia sebagai Penyangga Bahan Obat Alam untuk Kesehatan Bangsa}. Jakarta: Fakultas Kehutanan IPB.
\item Navis, A.A. (1984). \textit{Alam Terkembang Jadi Guru: Adat dan Kebudayaan Minangkabau}. Jakarta: Grafiti Pers.
\item Kementerian Kesehatan RI. (2018). \textit{Farmakope Herbal Indonesia Edisi II}. Jakarta: Direktorat Jenderal Kefarmasian dan Alat Kesehatan.
\item Syukur, C. (2011). \textit{Tumbuhan Obat Tradisional Sumatera Barat}. Padang: Andalas University Press.
\end{enumerate}

\end{document}