\documentclass[a4paper,12pt]{article}
\usepackage[a4paper, margin=1.27cm]{geometry}
\usepackage[indonesian]{babel}
\usepackage[utf8]{inputenc}
\usepackage{tcolorbox}
\usepackage{array}
\usepackage{multirow}
\usepackage{setspace}
\usepackage{amssymb}
\usepackage{graphicx}
\usepackage{adjustbox}
\usepackage{enumitem}
\usepackage{longtable}
\usepackage{xcolor}
\usepackage{amsmath}
\usepackage{fancyhdr}
\usepackage{titlesec}

% Define colors (black and white theme)
\definecolor{darkgray}{RGB}{64, 64, 64}
\definecolor{lightgray}{RGB}{240, 240, 240}
\definecolor{mediumgray}{RGB}{128, 128, 128}

% Custom tcolorbox styles
\tcbset{
    mainbox/.style={
        colback=white,
        colframe=black,
        boxrule=1pt,
        arc=3pt,
        left=8pt,
        right=8pt,
        top=8pt,
        bottom=8pt
    },
    sectionbox/.style={
        colback=white,
        colframe=black,
        boxrule=1.5pt,
        arc=2pt,
        left=6pt,
        right=6pt,
        top=6pt,
        bottom=6pt
    }
}

\begin{document}

\begin{center}
{\Large\textbf{MODUL AJAR: Pengobatan Tradisional Minang \& Gangguan Peredaran Darah}}
\end{center}

\vspace{0.5cm}

\begin{tcolorbox}[mainbox]
\textbf{Nama Penyusun:} Irfan Ananda Ismail, S.Pd., M.Pd., Gr \\
\textbf{Institusi:} SMP \\
\textbf{Mata Pelajaran:} Ilmu Pengetahuan Alam (IPA) \\
\textbf{Tahun Ajaran:} 2025/2026 \\
\textbf{Semester:} Ganjil \\
\textbf{Jenjang Sekolah:} SMP \\
\textbf{Kelas/Fase:} VIII / D \\
\textbf{Alokasi waktu:} 3 x 40 Menit (1 Pertemuan)
\end{tcolorbox}

\section{DIMENSI PROFIL PELAJAR PANCASILA}
\textit{(Clue untuk Guru: Sebutkan dimensi ini secara eksplisit saat apersepsi agar siswa sadar tujuan non-akademis yang sedang mereka kembangkan).}

\begin{itemize}
\item \textbf{Berkebinekaan Global:} Mengenal dan menghargai budaya, khususnya menganalisis pengobatan tradisional Minangkabau untuk gangguan peredaran darah dan menghubungkannya dengan prinsip medis modern.
\item \textbf{Bernalar Kritis:} Ditempa secara intensif saat menganalisis informasi dari sumber sains dan budaya (Tahap S), menyintesiskan kedua perspektif (Tahap A), dan menyusun argumen berbasis bukti (Tahap N).
\item \textbf{Gotong Royong:} Kemampuan untuk bekerja sama secara kolaboratif dalam kelompok untuk merumuskan masalah (Tahap E), melakukan investigasi (Tahap S), dan membangun pemahaman bersama (Tahap A).
\item \textbf{Kreatif:} Menghasilkan argumen atau solusi orisinal yang terintegrasi pada tahap akhir pembelajaran (Tahap N).
\end{itemize}

\section{Sarana dan Prasarana}

\begin{itemize}
\item \textbf{Media:} LKPD 12 - Jurnal Investigasi Dual-Lensa, gambar/contoh tanaman obat tradisional Minang, diagram gangguan sistem peredaran darah, artikel tentang hipertensi dan anemia.
\item \textbf{Alat:} Papan tulis/whiteboard, spidol, Proyektor \& Speaker, Kertas Plano atau Karton (1 per kelompok), sticky notes warna-warni, penggaris.
\item \textbf{Sumber Belajar:} Buku ajar IPA kelas VIII, Tautan video gangguan sistem peredaran darah (misal: bit.ly/gangguan-peredaran-darah), tautan artikel pengobatan tradisional Minang (misal: bit.ly/obat-tradisional-minang).
\end{itemize}

\section{Target Peserta didik}

\begin{itemize}
\item Peserta didik reguler kelas VIII (Fase D).
\end{itemize}

\section{Model Pembelajaran}

\begin{itemize}
\item Model Pembelajaran KESAN (Konektivitas Etnosains-Sains).
\end{itemize}

\section{Pemahaman Bermakna}
\textit{(Clue untuk Guru: Bacakan atau sampaikan narasi ini dengan intonasi yang menarik di akhir pembelajaran untuk mengikat semua pengalaman belajar siswa menjadi satu kesatuan yang bermakna).}

\begin{tcolorbox}[sectionbox]
"Ananda Semua, hari ini kita telah mengungkap kearifan mendalam di balik pengobatan tradisional Minangkabau untuk gangguan sistem peredaran darah! Kita menemukan bahwa tanaman-tanaman obat seperti daun katuk, kunyit, jahe merah, dan daun sirsak ternyata memiliki kandungan senyawa aktif yang secara ilmiah terbukti bermanfaat untuk kesehatan kardiovaskular. Setiap ramuan tradisional - dari yang membantu menurunkan tekanan darah, meningkatkan hemoglobin, hingga melancarkan peredaran darah - sebenarnya bekerja pada tingkat molekuler yang sama dengan obat-obatan modern. Dengan ini, kita sadar bahwa pengobatan tradisional bukan hanya warisan nenek moyang, tapi juga sumber pengetahuan ilmiah yang dapat memperkaya dunia medis modern dalam mengatasi gangguan sistem peredaran darah."
\end{tcolorbox}

\section{PERTEMUAN KEDUA: Gangguan Sistem Peredaran Darah dan Pengobatan Tradisional Minangkabau}

\subsection{Capaian Pembelajaran (Fase D)}
Pada akhir Fase D, murid memiliki kemampuan menganalisis sistem organisasi kehidupan, fungsi, serta kelainan atau gangguan yang muncul pada sistem organ makhluk hidup.

\subsection{Tujuan Pembelajaran (TP) Pertemuan 2:}
\textit{(Clue untuk Guru: Tujuan ini adalah kompas Anda. Pastikan setiap tahapan KESAN yang Anda lalui berkontribusi pada pencapaian tujuan-tujuan ini).}

Melalui model pembelajaran KESAN, peserta didik mampu:
\begin{itemize}
\item Menghubungkan fenomena kearifan lokal Minangkabau (pengobatan tradisional) dengan konteks gangguan sistem peredaran darah dalam sains modern. (Sintaks K)
\item Merumuskan pertanyaan investigatif mengenai gangguan sistem peredaran darah dan pengobatan tradisional yang digunakan masyarakat Minangkabau. (Sintaks E)
\item Mengumpulkan informasi mengenai jenis-jenis gangguan peredaran darah dari sumber sains serta pengobatan tradisional dari sumber kultural. (Sintaks S)
\item Menganalisis dan menyintesiskan hubungan sebab-akibat antara kandungan senyawa aktif dalam tanaman obat tradisional dengan mekanisme penyembuhan gangguan peredaran darah. (Sintaks A)
\item Menyusun sebuah penjelasan analitis yang logis mengenai bagaimana pengobatan tradisional Minangkabau dapat mendukung pengobatan modern untuk gangguan sistem peredaran darah. (Sintaks N)
\end{itemize}

\subsection{Pertanyaan Pemantik}
\textit{(Clue untuk Guru: Ajukan dua pertanyaan ini secara berurutan, berikan jeda agar siswa berpikir. Jangan langsung minta jawaban, biarkan pertanyaan ini menggantung untuk memicu rasa ingin tahu).}

\begin{itemize}
\item "Pernahkah kalian mendengar nenek atau orang tua menggunakan daun katuk untuk menambah darah, atau kunyit untuk melancarkan peredaran darah? Mengapa tanaman-tanaman ini dipercaya efektif untuk mengatasi masalah peredaran darah?"
\item "Dalam filosofi Minang ada ungkapan 'Alam takambang jadi guru' - alam terkembang jadi guru. Dari sudut pandang ilmu kedokteran, bagaimana sebenarnya alam mengajarkan kita tentang pengobatan gangguan sistem peredaran darah melalui tanaman-tanaman obat?"
\end{itemize}

\section{Langkah-langkah Kegiatan Pembelajaran:}

\subsection{Kegiatan Pembuka (15 Menit)}
\begin{itemize}
\item Guru membuka pelajaran dengan salam, doa, dan memeriksa kehadiran.
\item \textbf{Asesmen Diagnostik Awal:} Guru membagikan lembar K-W-L.
    \begin{itemize}
    \item \textbf{Instruksi Guru:} "Ananda, sebelum kita mulai menjelajahi dunia gangguan peredaran darah dan pengobatannya, tolong isi dua kolom pertama di lembar ini. Di kolom \textbf{K (Tahu)}, tulis apa saja yang sudah kalian ketahui tentang penyakit jantung atau darah tinggi ATAU tanaman obat tradisional. Di kolom \textbf{W (Ingin Tahu)}, tulis apa yang membuat kalian penasaran tentang topik ini." \textit{(Clue: Ini membantu Anda memetakan pengetahuan awal dan minat siswa secara cepat).}
    \end{itemize}
\item \textbf{Apersepsi:}
    \begin{itemize}
    \item \textbf{Instruksi Guru:} "Hari ini kita akan menjadi ahli farmakologi tradisional dan dokter modern sekaligus. Kita akan menyelidiki bagaimana pengobatan tradisional Minangkabau dapat membantu mengatasi gangguan sistem peredaran darah. Dalam investigasi ini, kita akan melatih kemampuan \textbf{Bernalar Kritis} kita, menghargai budaya lewat \textbf{Berkebinekaan Global}, dan bekerja sama dalam semangat \textbf{Gotong Royong}. Siap menjadi peneliti pengobatan tradisional?"
    \end{itemize}
\end{itemize}

\subsection{Kegiatan Inti (90 Menit) - SINTAKS MODEL KESAN}

\subsubsection{Tahap 1: (K) Kaitkan Konteks Kultural (15 menit)}
\begin{itemize}
\item \textbf{Aktivitas Guru:}
    \begin{itemize}
    \item Menampilkan gambar/contoh tanaman obat tradisional Minangkabau dan menceritakan penggunaannya dalam pengobatan tradisional.
    \item Mengajukan Pertanyaan Pemantik yang sudah disiapkan di atas.
    \item \textit{(Clue: Tujuan tahap ini adalah memvalidasi pengetahuan siswa dan memantik rasa heran, bukan mencari jawaban benar. Sediakan spidol dan papan tulis/whiteboard. Saat siswa menjawab, tuliskan semua ide mereka, bahkan yang keliru sekalipun, dengan judul "\textbf{PENGETAHUAN AWAL KITA}". Ini menunjukkan bahwa semua pemikiran dihargai).}
    \end{itemize}
\item \textbf{Aktivitas Siswa:} Mengamati tanaman obat, mendengarkan pertanyaan, lalu secara sukarela berbagi pengalaman atau pengetahuan awal tentang pengobatan tradisional atau gangguan peredaran darah. Menuliskan minimal satu pertanyaan atau pengalaman di sticky notes dan menempelkannya di '\textbf{Papan Penasaran}'.
\end{itemize}

\subsubsection{Tahap 2: (E) Eksplorasi Enigma (15 menit)}
\begin{itemize}
\item \textbf{Aktivitas Guru:} Membentuk siswa menjadi kelompok (3-4 orang).
    \begin{itemize}
    \item \textbf{Instruksi Guru:} "Baik, rasa penasaran kalian luar biasa! Sekarang, tugas kita sebagai peneliti adalah mengubah rasa penasaran ini menjadi misi yang jelas. Dalam kelompok, diskusikan dan rumuskan minimal 3 pertanyaan kunci yang akan kita selidiki hari ini. Tuliskan dalam bentuk '\textbf{Misi Penelitian Tim [Nama Kelompok]}' di kertas plano yang Bapak/Ibu berikan."
    \item \textit{(Clue: Arahkan diskusi siswa agar pertanyaannya mencakup aspek 'apa saja gangguan sistem peredaran darah' dan 'bagaimana tanaman obat dapat membantu pengobatan'. Jika kelompok kesulitan, berikan pancingan: "Kira-kira, apa dulu yang perlu kita tahu? Jenis-jenis penyakit jantung atau cara kerja obat tradisional?").}
    \end{itemize}
\item \textbf{Aktivitas Siswa:} Berdiskusi dalam tim untuk merumuskan misi penelitian (daftar pertanyaan kunci) di kertas plano. \textit{(Contoh misi yang diharapkan: 1. Apa saja gangguan yang dapat terjadi pada sistem peredaran darah? 2. Tanaman obat apa saja yang digunakan dalam pengobatan tradisional Minang? 3. Bagaimana cara kerja tanaman obat dalam mengatasi gangguan peredaran darah?)}
\end{itemize}

\subsubsection{Tahap 3: (S) Selidiki secara Sintetis (25 menit)}
\begin{itemize}
\item \textbf{Aktivitas Guru:} Membagikan "\textbf{LKPD 12 - Jurnal Investigasi Dual-Lensa}".
    \begin{itemize}
    \item \textbf{Instruksi Guru:} "Setiap tim akan melakukan investigasi dari dua lensa. Gunakan HP atau sumber yang disediakan untuk mencari jawabannya. Bagilah tugas dalam tim!"
    \item \textbf{Lensa Sains:} Buka link video bit.ly/gangguan-peredaran-darah untuk memahami jenis-jenis gangguan sistem peredaran darah seperti hipertensi, anemia, aterosklerosis, dan penyakit jantung koroner beserta penyebab dan gejalanya.
    \item \textbf{Lensa Etnosains/Kultural:} Buka link artikel bit.ly/obat-tradisional-minang untuk memahami tanaman obat tradisional Minangkabau seperti daun katuk, kunyit, jahe merah, daun sirsak, dan penggunaannya untuk mengatasi gangguan peredaran darah.
    \item \textit{(Clue: Pastikan sumber belajar sudah disiapkan dan link bisa diakses. Berkelilinglah untuk memastikan setiap kelompok membagi tugas dan tidak hanya fokus pada satu lensa saja).}
    \end{itemize}
\item \textbf{Aktivitas Siswa:} Dalam kelompok, siswa berbagi tugas mencari informasi dari sumber yang diberikan dan mencatat temuan kunci di dua kolom terpisah pada "\textbf{LKPD 12 - Jurnal Investigasi Dual-Lensa}".
\end{itemize}

\subsubsection{Tahap 4: (A) Asimilasi Analitis (20 menit)}
\begin{itemize}
\item \textbf{Aktivitas Guru:} Memfasilitasi diskusi untuk menjembatani kedua lensa.
    \begin{itemize}
    \item \textbf{Pertanyaan Pancingan Kunci untuk Guru:}
        \begin{itemize}
        \item "Oke, dari Lensa Sains kita tahu hipertensi disebabkan oleh penyempitan pembuluh darah. Dari Lensa Etnosains, kita tahu daun sirsak dipercaya dapat menurunkan tekanan darah. Nah, coba hubungkan! Mengapa daun sirsak bisa membantu mengatasi hipertensi?"
        \item "Dari Lensa Sains, kita tahu anemia terjadi karena kurangnya hemoglobin. Dari Lensa Budaya, daun katuk digunakan untuk 'menambah darah'. Apa kesamaan prinsipnya?"
        \end{itemize}
    \item \textit{(Clue: Fokuskan untuk membuat siswa 'menemukan' hubungannya sendiri, bukan diberitahu. Gunakan kata "menurut kalian", "kira-kira kenapa", "ada yang punya ide?").}
    \end{itemize}
\item \textbf{Aktivitas Siswa:} Berdiskusi intensif untuk menghubungkan temuan sains dan budaya. Menuliskan kesimpulan terpadu (sintesis) mereka di kertas plano.
\end{itemize}

\subsubsection{Tahap 5: (N) Nyatakan Pemahaman (15 menit)}
\begin{itemize}
\item \textbf{Aktivitas Guru:} Memberikan studi kasus individual atau per kelompok.
    \begin{itemize}
    \item \textbf{Instruksi Guru (tuliskan di papan tulis):}
    
    "\textbf{STUDI KASUS UNTUK PENELITI SAINS-BUDAYA:}
    
    Sebuah rumah sakit di Padang ingin mengembangkan program pengobatan komplementer yang mengintegrasikan pengobatan modern dengan pengobatan tradisional Minangkabau untuk pasien dengan gangguan sistem peredaran darah. Tim dokter membutuhkan penjelasan ilmiah tentang bagaimana tanaman obat tradisional dapat mendukung pengobatan medis konvensional. Mereka ingin memastikan bahwa integrasi ini aman dan efektif.
    
    Tugasmu: Tuliskan sebuah rekomendasi ilmiah singkat (4-6 kalimat) di buku latihanmu untuk membantu tim dokter memahami bagaimana pengobatan tradisional Minangkabau dapat mendukung pengobatan modern untuk gangguan sistem peredaran darah. Gunakan pengetahuan gabungan dari sains dan kearifan tradisional yang baru saja kamu pelajari."
    \end{itemize}
\item \textbf{Aktivitas Siswa:} Menyusun argumen tertulis untuk menjawab studi kasus yang diberikan, menggunakan bukti dari kedua lensa.
\end{itemize}

\subsection{Kegiatan Penutup (15 Menit)}
\begin{itemize}
\item \textbf{Presentasi \& Penguatan:} Guru meminta 2-3 siswa secara acak untuk membacakan rekomendasi mereka. Guru memberikan pujian dan penguatan positif, menekankan betapa hebatnya kemampuan mereka mengintegrasikan sains dan budaya.
\item \textbf{Refleksi:}
    \begin{itemize}
    \item \textbf{Instruksi Guru:} "Sekarang, kembali ke lembar K-W-L kalian. Lengkapi kolom terakhir, \textbf{L (Learned)}, dengan hal-hal baru yang paling mengagumkan yang kalian pelajari hari ini."
    \item \textbf{Instruksi Guru:} "Angkat tangan, siapa yang setelah belajar hari ini jadi lebih menghargai kearifan pengobatan tradisional Minangkabau?"
    \end{itemize}
\item \textbf{Tindak Lanjut:}
    \begin{itemize}
    \item \textbf{Instruksi Guru:} "Luar biasa, para peneliti! Hari ini kita sudah mengungkap bagaimana pengobatan tradisional dapat mendukung pengobatan modern. Pertemuan berikutnya, kita akan mempelajari sistem pernapasan. Siapa tahu ada kearifan Minang tentang menjaga kesehatan paru-paru yang bisa mengajarkan kita tentang sistem respirasi? Kita akan lihat!"
    \item \textit{(Clue: Buat transisi yang menarik ke pertemuan berikutnya sambil mempertahankan semangat investigasi yang sudah terbangun).}
    \end{itemize}
\item Guru menutup pelajaran dengan doa dan salam.
\end{itemize}

\section{Asesmen (Penilaian)}

\begin{itemize}
\item \textbf{Asesmen Diagnostik (Awal):} Analisis lembar K-W-L. (Untuk mengetahui baseline siswa).
\item \textbf{Asesmen Formatif (Proses):} Observasi keaktifan diskusi (gotong royong) dan penilaian kelengkapan "\textbf{LKPD 12 - Jurnal Investigasi Dual-Lensa}".
\item \textbf{Asesmen Sumatif (Akhir Siklus):} Penilaian jawaban studi kasus (Tahap N) menggunakan rubrik.
\end{itemize}

\subsection{Rubrik Penilaian Jawaban Studi Kasus (Tahap N)}

\begin{longtable}{|p{3cm}|p{3cm}|p{3cm}|p{3cm}|p{3cm}|}
\hline
\textbf{Kriteria Penilaian} & \textbf{Skor 4 (Sangat Baik)} & \textbf{Skor 3 (Baik)} & \textbf{Skor 2 (Cukup)} & \textbf{Skor 1 (Kurang)} \\
\hline
\textbf{Ketepatan Konsep Ilmiah} & Menggunakan istilah ilmiah (hipertensi, anemia, senyawa aktif, farmakologi) dengan sangat tepat dan relevan dengan kasus. & Menggunakan istilah ilmiah dengan tepat, namun kurang relevan. & Menggunakan istilah ilmiah namun ada beberapa kesalahan konsep. & Tidak menggunakan istilah ilmiah atau salah total. \\
\hline
\textbf{Keterkaitan dengan Etnosains} & Mampu menghubungkan secara logis dan eksplisit antara kandungan tanaman obat dengan mekanisme penyembuhan secara mendalam. & Mampu menghubungkan tanaman obat dengan pengobatan, namun kurang mendalam. & Hanya menyebutkan tanaman obat tanpa menghubungkan dengan sains, atau sebaliknya. & Tidak ada keterkaitan antara sains dan budaya yang ditunjukkan. \\
\hline
\textbf{Kelogisan \& Struktur Argumen} & Rekomendasi sangat logis, runtut, praktis untuk implementasi medis, dan mudah dipahami. & Rekomendasi logis dan runtut, namun kurang praktis untuk implementasi. & Alur rekomendasi kurang runtut atau sulit dipahami. & Rekomendasi tidak logis dan tidak terstruktur. \\
\hline
\end{longtable}

\section{Daftar Pustaka Sumber Etnosains}

\begin{enumerate}
\item Navis, A.A. (1984). \textit{Alam Terkembang Jadi Guru: Adat dan Kebudayaan Minangkabau}. Jakarta: Grafiti Pers.
\item Kato, T. (2005). \textit{Adat Minangkabau dan Merantau dalam Perspektif Sejarah}. Jakarta: Balai Pustaka.
\item Dt. Rajo Penghulu, I. (1994). \textit{Pegangan Penghulu, Bundo Kanduang dan Pidato Alua Pasambahan Adat Minangkabau}. Bukittinggi: Pustaka Indonesia.
\item Zuhud, E.A.M. (2009). \textit{Potensi Hutan Tropika Indonesia sebagai Penyangga Bahan Obat Alam untuk Kesehatan Bangsa}. Jakarta: IPB Press.
\end{enumerate}

\end{document}