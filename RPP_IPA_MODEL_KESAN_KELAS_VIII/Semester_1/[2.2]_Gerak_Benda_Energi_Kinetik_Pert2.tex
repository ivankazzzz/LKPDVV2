\documentclass[a4paper,12pt]{article}
\usepackage[a4paper, margin=1.27cm]{geometry}
\usepackage[indonesian]{babel}
\usepackage[utf8]{inputenc}
\usepackage{tcolorbox}
\usepackage{array}
\usepackage{multirow}
\usepackage{setspace}
\usepackage{amssymb}
\usepackage{graphicx}
\usepackage{adjustbox}
\usepackage{enumitem}
\usepackage{longtable}
\usepackage{xcolor}
\usepackage{amsmath}
\usepackage{fancyhdr}
\usepackage{titlesec}

% Define colors (black and white theme)
\definecolor{darkgray}{RGB}{64, 64, 64}
\definecolor{lightgray}{RGB}{240, 240, 240}
\definecolor{mediumgray}{RGB}{128, 128, 128}

% Custom tcolorbox styles
\tcbset{
    mainbox/.style={
        colback=white,
        colframe=black,
        boxrule=1pt,
        arc=3pt,
        left=8pt,
        right=8pt,
        top=8pt,
        bottom=8pt
    },
    sectionbox/.style={
        colback=white,
        colframe=black,
        boxrule=1.5pt,
        arc=2pt,
        left=6pt,
        right=6pt,
        top=6pt,
        bottom=6pt
    }
}

\begin{document}

\begin{center}
{\Large\textbf{MODUL AJAR: Energi Lesung Batu Tradisional \& Konsep Energi Kinetik}}
\end{center}

\vspace{0.5cm}

\begin{tcolorbox}[mainbox]
\textbf{Nama Penyusun:} Irfan Ananda Ismail, S.Pd., M.Pd., Gr \\
\textbf{Institusi:} SMP \\
\textbf{Mata Pelajaran:} Ilmu Pengetahuan Alam (IPA) \\
\textbf{Tahun Ajaran:} 2025/2026 \\
\textbf{Semester:} Ganjil \\
\textbf{Jenjang Sekolah:} SMP \\
\textbf{Kelas/Fase:} VIII / D \\
\textbf{Alokasi waktu:} 3 x 40 Menit (1 Pertemuan)
\end{tcolorbox}

\section{DIMENSI PROFIL PELAJAR PANCASILA}
\textit{(Clue untuk Guru: Sebutkan dimensi ini secara eksplisit saat apersepsi agar siswa sadar tujuan non-akademis yang sedang mereka kembangkan).}

\begin{itemize}
\item \textbf{Berkebinekaan Global:} Mengenal dan menghargai budaya, khususnya menganalisis kearifan lokal Minangkabau dalam teknologi tradisional (lesung batu untuk menumbuk padi), lalu menghubungkannya dengan konsep sains universal tentang energi kinetik dan momentum.
\item \textbf{Bernalar Kritis:} Ditempa secara intensif saat menganalisis informasi dari sumber sains dan budaya (Tahap S), menyintesiskan kedua perspektif (Tahap A), dan menyusun argumen berbasis bukti (Tahap N).
\item \textbf{Gotong Royong:} Kemampuan untuk bekerja sama secara kolaboratif dalam kelompok untuk merumuskan masalah (Tahap E), melakukan investigasi (Tahap S), dan membangun pemahaman bersama (Tahap A).
\item \textbf{Kreatif:} Menghasilkan argumen atau solusi orisinal yang terintegrasi pada tahap akhir pembelajaran (Tahap N).
\end{itemize}

\section{Sarana dan Prasarana}

\begin{itemize}
\item \textbf{Media:} LKPD 04 - Jurnal Investigasi Dual-Lensa, video singkat penggunaan lesung batu tradisional Minangkabau (durasi 3-4 menit), gambar/diagram energi kinetik dan momentum, artikel tentang teknologi pengolahan padi tradisional, model sederhana lesung atau alat tumbuk.
\item \textbf{Alat:} Papan tulis/whiteboard, spidol, Proyektor \& Speaker, Kertas Plano atau Karton (1 per kelompok), sticky notes warna-warni, bola dengan massa berbeda untuk demonstrasi.
\item \textbf{Sumber Belajar:} Buku ajar IPA kelas VIII, Tautan video animasi energi kinetik (misal: bit.ly/animasi-energi-kinetik), tautan artikel/video lesung batu tradisional (misal: bit.ly/lesung-batu-minang).
\end{itemize}

\section{Target Peserta didik}

\begin{itemize}
\item Peserta didik reguler kelas VIII (Fase D).
\end{itemize}

\section{Model Pembelajaran}

\begin{itemize}
\item Model Pembelajaran KESAN (Konektivitas Etnosains-Sains).
\end{itemize}

\section{Pemahaman Bermakna}
\textit{(Clue untuk Guru: Bacakan atau sampaikan narasi ini dengan intonasi yang menarik di akhir pembelajaran untuk mengikat semua pengalaman belajar siswa menjadi satu kesatuan yang bermakna).}

\begin{tcolorbox}[sectionbox]
"Ananda Semua, hari ini kita telah mengungkap rahasia di balik kekuatan lesung batu yang masih digunakan di rumah-rumah tradisional Minangkabau! Kita menemukan bahwa setiap pukulan alu ke dalam lesung bukan sekadar gerakan sederhana, tapi penerapan konsep energi kinetik dan momentum yang sangat efisien. Nenek moyang kita telah memahami bahwa massa alu yang berat dan kecepatan pukulan yang tepat akan menghasilkan energi yang cukup untuk menumbuk padi menjadi beras. Dengan ini, kita sadar bahwa teknologi tradisional adalah sains yang telah teruji ribuan tahun."
\end{tcolorbox}

\section{PERTEMUAN KEDUA: Energi Kinetik dan Momentum dalam Teknologi Tradisional}

\subsection{Capaian Pembelajaran (Fase D)}
Pada akhir Fase D, murid memiliki kemampuan menerapkan konsep gaya, gerak, dan energi dalam kehidupan sehari-hari.

\subsection{Tujuan Pembelajaran (TP) Pertemuan 2:}
\textit{(Clue untuk Guru: Tujuan ini adalah kompas Anda. Pastikan setiap tahapan KESAN yang Anda lalui berkontribusi pada pencapaian tujuan-tujuan ini).}

Melalui model pembelajaran KESAN, peserta didik mampu:
\begin{itemize}
\item Menghubungkan fenomena kearifan lokal Minangkabau (teknologi lesung batu tradisional) dengan konteks energi kinetik dan momentum. (Sintaks K)
\item Merumuskan pertanyaan investigatif mengenai cara kerja lesung batu dan hubungannya dengan konsep energi dan momentum. (Sintaks E)
\item Mengumpulkan informasi mengenai energi kinetik dan momentum dari sumber ilmiah serta prinsip kerja teknologi tradisional dari sumber kultural. (Sintaks S)
\item Menganalisis dan menyintesiskan hubungan sebab-akibat antara konsep energi kinetik dengan efektivitas lesung batu tradisional. (Sintaks A)
\item Menyusun sebuah penjelasan analitis yang logis mengenai bagaimana teknologi tradisional Minangkabau mengoptimalkan transfer energi untuk efisiensi kerja. (Sintaks N)
\end{itemize}

\subsection{Pertanyaan Pemantik}
\textit{(Clue untuk Guru: Ajukan dua pertanyaan ini secara berurutan, berikan jeda agar siswa berpikir. Jangan langsung minta jawaban, biarkan pertanyaan ini menggantung untuk memicu rasa ingin tahu).}

\begin{itemize}
\item "Pernahkah kalian memperhatikan mengapa alu lesung dibuat dari kayu yang berat dan panjang? Mengapa tidak dibuat ringan saja agar lebih mudah diangkat? Apa rahasia di balik desain tradisional ini?"
\item "Ketika nenek atau ibu kalian menumbuk padi di lesung, mereka selalu mengangkat alu tinggi-tinggi sebelum menumbukkannya. Dari sudut pandang sains, apa yang sebenarnya terjadi dengan energi saat alu diangkat dan dijatuhkan?"
\end{itemize}

\section{Langkah-langkah Kegiatan Pembelajaran:}

\subsection{Kegiatan Pembuka (15 Menit)}
\begin{itemize}
\item Guru membuka pelajaran dengan salam, doa, dan memeriksa kehadiran.
\item \textbf{Asesmen Diagnostik Awal:} Guru membagikan lembar K-W-L.
    \begin{itemize}
    \item \textbf{Instruksi Guru:} "Ananda, kemarin kita sudah mengungkap rahasia kincir air dan hukum Newton. Hari ini kita akan melanjutkan eksplorasi gerak benda dengan fokus pada energi! Di lembar K-W-L ini, tulis di kolom \textbf{K (Tahu)} apa yang sudah kalian ketahui tentang energi ATAU tentang alat-alat tradisional seperti lesung batu. Di kolom \textbf{W (Ingin Tahu)}, tulis apa yang membuat kalian penasaran tentang topik hari ini." \textit{(Clue: Ini membantu Anda melihat koneksi yang siswa buat dengan pembelajaran sebelumnya tentang gerak).}
    \end{itemize}
\item \textbf{Apersepsi:}
    \begin{itemize}
    \item \textbf{Instruksi Guru:} "Kemarin kita sudah menjadi insinyur muda yang mengungkap hukum Newton dalam kincir air. Hari ini, misi kita berlanjut! Kita akan menyelidiki bagaimana energi bekerja dalam teknologi tradisional Minangkabau, khususnya lesung batu. Siap menjadi ahli energi yang menghargai kearifan lokal?"
    \end{itemize}
\end{itemize}

\subsection{Kegiatan Inti (90 Menit) - SINTAKS MODEL KESAN}

\subsubsection{Tahap 1: (K) Kaitkan Konteks Kultural (15 menit)}
\begin{itemize}
\item \textbf{Aktivitas Guru:}
    \begin{itemize}
    \item Menampilkan video singkat (3-4 menit) penggunaan lesung batu tradisional, menunjukkan teknik menumbuk padi yang efisien.
    \item Mengajukan Pertanyaan Pemantik yang sudah disiapkan di atas.
    \item \textit{(Clue: Fokuskan pada keunikan desain lesung dan alu yang telah dioptimalkan selama berabad-abad. Saat siswa menjawab, tuliskan semua ide mereka di papan tulis dengan judul "\textbf{KEARIFAN DESAIN TRADISIONAL}". Hargai setiap observasi).}
    \end{itemize}
\item \textbf{Aktivitas Siswa:} Mengamati video, mendengarkan pertanyaan, lalu secara sukarela berbagi pengalaman atau dugaan tentang mengapa lesung dan alu didesain seperti itu. Menuliskan minimal satu pertanyaan atau pengamatan di sticky notes dan menempelkannya di '\textbf{Papan Penasaran}'.
\end{itemize}

\subsubsection{Tahap 2: (E) Eksplorasi Enigma (15 menit)}
\begin{itemize}
\item \textbf{Aktivitas Guru:} Membentuk siswa menjadi kelompok (3-4 orang).
    \begin{itemize}
    \item \textbf{Instruksi Guru:} "Observasi kalian tentang desain tradisional sangat menarik! Sekarang, sebagai tim ahli energi, tugas kalian adalah mengubah observasi ini menjadi misi investigasi yang fokus. Dalam kelompok, diskusikan dan rumuskan minimal 3 pertanyaan kunci yang akan kita selidiki hari ini. Tuliskan dalam bentuk '\textbf{Misi Investigasi Tim [Nama Kelompok]}' di kertas plano."
    \item \textit{(Clue: Arahkan diskusi siswa agar pertanyaannya mencakup aspek 'bagaimana energi bekerja dalam lesung' dan 'mengapa desain tradisional begitu efektif'. Jika kelompok kesulitan, berikan pancingan: "Kira-kira, dari mana energi untuk menumbuk berasal? Dan bagaimana energi itu ditransfer ke padi?").}
    \end{itemize}
\item \textbf{Aktivitas Siswa:} Berdiskusi dalam tim untuk merumuskan misi investigasi (daftar pertanyaan kunci) di kertas plano. \textit{(Contoh misi yang diharapkan: 1. Dari mana energi untuk menumbuk berasal? 2. Bagaimana massa dan kecepatan alu mempengaruhi hasil tumbukan? 3. Mengapa desain lesung dan alu begitu efektif?)}
\end{itemize}

\subsubsection{Tahap 3: (S) Selidiki secara Sintetis (25 menit)}
\begin{itemize}
\item \textbf{Aktivitas Guru:} Membagikan "\textbf{LKPD 04 - Jurnal Investigasi Dual-Lensa}".
    \begin{itemize}
    \item \textbf{Instruksi Guru:} "Setiap tim akan melakukan investigasi dari dua lensa. Gunakan sumber yang disediakan untuk mencari jawabannya. Bagilah tugas dalam tim!"
    \item \textbf{Lensa Sains:} Buka link video bit.ly/animasi-energi-kinetik untuk memahami konsep energi kinetik, energi potensial, dan momentum. Pelajari juga rumus-rumus dasar dan contoh penerapannya.
    \item \textbf{Lensa Etnosains/Kultural:} Buka link artikel bit.ly/lesung-batu-minang untuk memahami sejarah, konstruksi, dan teknik penggunaan lesung batu tradisional Minangkabau.
    \item \textit{(Clue: Pastikan sumber belajar sudah disiapkan dan dapat diakses. Berkelilinglah untuk memastikan setiap kelompok mengeksplorasi kedua lensa dengan kedalaman yang memadai).}
    \end{itemize}
\item \textbf{Aktivitas Siswa:} Dalam kelompok, siswa berbagi tugas mencari informasi dari sumber yang diberikan dan mencatat temuan kunci di dua kolom terpisah pada "\textbf{LKPD 04 - Jurnal Investigasi Dual-Lensa}".
\end{itemize}

\subsubsection{Tahap 4: (A) Asimilasi Analitis (20 menit)}
\begin{itemize}
\item \textbf{Aktivitas Guru:} Memfasilitasi diskusi untuk menjembatani kedua lensa.
    \begin{itemize}
    \item \textbf{Pertanyaan Pancingan Kunci untuk Guru:}
        \begin{itemize}
        \item "Dari Lensa Sains kita tahu energi kinetik = ½mv². Dari Lensa Etnosains, kita tahu alu dibuat berat dan diangkat tinggi. Nah, coba hubungkan! Bagaimana massa (m) dan kecepatan (v) alu mempengaruhi energi yang dihasilkan untuk menumbuk?"
        \item "Dari Lensa Sains, kita tahu energi potensial berubah menjadi energi kinetik. Dari Lensa Budaya, kita tahu teknik mengangkat alu tinggi-tinggi. Apa hubungannya ketinggian alu dengan efektivitas menumbuk padi?"
        \end{itemize}
    \item \textit{(Clue: Bantu siswa melihat bahwa setiap aspek desain lesung (massa alu, ketinggian angkatan, bentuk lesung) dioptimalkan untuk transfer energi yang maksimal. Gunakan demonstrasi dengan bola bermassa berbeda jika perlu).}
    \end{itemize}
\item \textbf{Aktivitas Siswa:} Berdiskusi intensif untuk menghubungkan temuan sains dan budaya. Menuliskan kesimpulan terpadu (sintesis) mereka di kertas plano.
\end{itemize}

\subsubsection{Tahap 5: (N) Nyatakan Pemahaman (15 menit)}
\begin{itemize}
\item \textbf{Aktivitas Guru:} Memberikan studi kasus individual atau per kelompok.
    \begin{itemize}
    \item \textbf{Instruksi Guru (tuliskan di papan tulis):}
    
    "\textbf{STUDI KASUS UNTUK AHLI ENERGI SAINS-BUDAYA:}
    
    Seorang pengrajin ingin membuat lesung batu modern yang lebih efisien untuk dijual sebagai produk wisata budaya. Dia meminta bantuanmu sebagai konsultan yang memahami baik fisika maupun teknologi tradisional untuk memberikan rekomendasi desain.
    
    Tugasmu: Tuliskan sebuah rekomendasi singkat (3-5 kalimat) di buku latihanmu tentang bagaimana mengoptimalkan desain lesung dan alu berdasarkan prinsip energi kinetik dan momentum. Jelaskan mengapa aspek-aspek tertentu dari desain tradisional sudah optimal secara sains dan tidak perlu diubah."
    \end{itemize}
\item \textbf{Aktivitas Siswa:} Menyusun argumen tertulis untuk menjawab studi kasus yang diberikan, menggunakan bukti dari kedua lensa.
\end{itemize}

\subsection{Kegiatan Penutup (15 Menit)}
\begin{itemize}
\item \textbf{Presentasi \& Penguatan:} Guru meminta 2-3 siswa secara acak untuk membacakan rekomendasi mereka. Guru memberikan pujian dan penguatan positif, menekankan betapa hebatnya kemampuan mereka mengaplikasikan konsep energi pada teknologi tradisional.
\item \textbf{Refleksi:}
    \begin{itemize}
    \item \textbf{Instruksi Guru:} "Sekarang, kembali ke lembar K-W-L kalian. Lengkapi kolom terakhir, \textbf{L (Learned)}, dengan hal-hal baru yang paling menakjubkan yang kalian pelajari hari ini."
    \item \textbf{Instruksi Guru:} "Angkat tangan, siapa yang setelah belajar hari ini jadi lebih menghargai kecerdasan nenek moyang kita dalam merancang teknologi yang efisien?"
    \end{itemize}
\item \textbf{Tindak Lanjut:}
    \begin{itemize}
    \item \textbf{Instruksi Guru:} "Luar biasa, para ahli energi! Dalam dua pertemuan tentang gerak benda, kita sudah mengungkap hukum Newton dan konsep energi melalui teknologi tradisional Minangkabau. Pertemuan berikutnya, kita akan beralih ke topik baru: usaha dan pesawat sederhana. Siapa tahu ada alat tradisional Minang yang mengajarkan kita tentang prinsip-prinsip mekanika? Kita akan lihat!"
    \item \textit{(Clue: Buat transisi yang menarik ke topik berikutnya sambil mempertahankan apresiasi terhadap kearifan lokal).}
    \end{itemize}
\item Guru menutup pelajaran dengan doa dan salam.
\end{itemize}

\section{Asesmen (Penilaian)}

\begin{itemize}
\item \textbf{Asesmen Diagnostik (Awal):} Analisis lembar K-W-L. (Untuk mengetahui koneksi dengan pembelajaran sebelumnya tentang gerak).
\item \textbf{Asesmen Formatif (Proses):} Observasi keaktifan diskusi (gotong royong) dan penilaian kelengkapan "\textbf{LKPD 04 - Jurnal Investigasi Dual-Lensa}".
\item \textbf{Asesmen Sumatif (Akhir Siklus):} Penilaian jawaban studi kasus (Tahap N) menggunakan rubrik.
\end{itemize}

\subsection{Rubrik Penilaian Jawaban Studi Kasus (Tahap N)}

\begin{longtable}{|p{3cm}|p{3cm}|p{3cm}|p{3cm}|p{3cm}|}
\hline
\textbf{Kriteria Penilaian} & \textbf{Skor 4 (Sangat Baik)} & \textbf{Skor 3 (Baik)} & \textbf{Skor 2 (Cukup)} & \textbf{Skor 1 (Kurang)} \\
\hline
\textbf{Ketepatan Konsep Ilmiah} & Menggunakan konsep energi kinetik, energi potensial, dan momentum dengan sangat tepat dan relevan untuk optimasi desain. & Menggunakan konsep energi dengan tepat, namun kurang relevan dengan konteks desain. & Menggunakan konsep fisika namun ada beberapa kesalahan dalam penerapan konsep energi. & Tidak menggunakan konsep energi atau salah total. \\
\hline
\textbf{Keterkaitan dengan Etnosains} & Mampu menghubungkan secara logis dan eksplisit antara prinsip energi dengan desain lesung tradisional secara mendalam dan akurat. & Mampu menghubungkan prinsip energi dengan teknologi tradisional, namun kurang mendalam. & Hanya menyebutkan teknologi tradisional tanpa menghubungkan dengan konsep energi, atau sebaliknya. & Tidak ada keterkaitan antara sains dan teknologi tradisional yang ditunjukkan. \\
\hline
\textbf{Kelogisan \& Struktur Argumen} & Rekomendasi sangat logis, runtut, praktis untuk implementasi, dan mudah dipahami. & Rekomendasi logis dan runtut, namun kurang praktis atau sulit diimplementasikan. & Alur rekomendasi kurang runtut atau sulit dipahami. & Rekomendasi tidak logis dan tidak terstruktur. \\
\hline
\end{longtable}

\section{Daftar Pustaka Sumber Etnosains}

\begin{enumerate}
\item Dt. Rajo Penghulu. (1994). \textit{Teknologi Tradisional Minangkabau}. Padang: Pusat Dokumentasi dan Informasi Kebudayaan Minangkabau.
\item Navis, A.A. (1984). \textit{Alam Terkembang Jadi Guru: Adat dan Kebudayaan Minangkabau}. Jakarta: Grafiti Pers.
\item Kementerian Pendidikan dan Kebudayaan. (2017). \textit{Lesung Batu: Warisan Teknologi Pengolahan Pangan Nusantara}. Jakarta: Direktorat Warisan dan Diplomasi Budaya.
\item Syafwan, A. (2008). \textit{Kearifan Lokal dalam Teknologi Pertanian Minangkabau}. Padang: Universitas Andalas Press.
\end{enumerate}

\end{document}