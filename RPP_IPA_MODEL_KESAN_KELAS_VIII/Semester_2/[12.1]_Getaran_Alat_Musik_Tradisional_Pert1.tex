\documentclass[12pt,a4paper]{article}
\usepackage[margin=1.5cm]{geometry}
\usepackage[utf8]{inputenc}
\usepackage{amsmath}
\usepackage{amsfonts}
\usepackage{amssymb}
\usepackage{graphicx}
\usepackage{xcolor}
\usepackage{tcolorbox}
\usepackage{enumitem}
\usepackage{multicol}
\usepackage{fancyhdr}
\usepackage{titlesec}
\usepackage{array}
\usepackage{longtable}
\usepackage{booktabs}

% Setup header and footer
\pagestyle{fancy}
\fancyhf{}
\fancyhead[L]{\textbf{RPP IPA Kelas VIII - Model KESAN}}
\fancyhead[R]{\textbf{Semester Genap 2025/2026}}
\fancyfoot[C]{\thepage}

% Define colors
\definecolor{primarycolor}{RGB}{0,0,0}
\definecolor{secondarycolor}{RGB}{255,255,255}

% Custom tcolorbox styles
\tcbset{
    mystyle/.style={
        colback=white,
        colframe=black,
        boxrule=1pt,
        arc=2pt,
        left=3pt,
        right=3pt,
        top=3pt,
        bottom=3pt
    }
}

\begin{document}

\begin{center}
\textbf{\Large MODUL AJAR: Harmoni Saluang \& Misteri Getaran dalam Musik Tradisional}
\end{center}

\vspace{0.5cm}

\begin{tcolorbox}[mystyle]
\textbf{Nama Penyusun:} Irfan Ananda \\
\textbf{Institusi:} SMP \\
\textbf{Mata Pelajaran:} Ilmu Pengetahuan Alam (IPA) \\
\textbf{Tahun Ajaran:} 2025/2026 \\
\textbf{Semester:} Genap \\
\textbf{Jenjang Sekolah:} SMP \\
\textbf{Kelas/Fase:} VIII / D \\
\textbf{Alokasi waktu:} 3 x 40 Menit (1 Pertemuan)
\end{tcolorbox}

\section{DIMENSI PROFIL PELAJAR PANCASILA}
\textit{(Clue untuk Guru: Sebutkan dimensi ini secara eksplisit saat apersepsi agar siswa sadar tujuan non-akademis yang sedang mereka kembangkan).}

\begin{itemize}
\item \textbf{Berkebinekaan Global:} Mengenal dan menghargai budaya, khususnya menganalisis kearifan lokal pembuatan alat musik tradisional Minangkabau seperti saluang, talempong, dan gandang dalam menerapkan prinsip-prinsip getaran dan resonansi, lalu menghubungkannya dengan konsep fisika universal.
\item \textbf{Bernalar Kritis:} Ditempa secara intensif saat menganalisis informasi dari sumber sains dan budaya (Tahap S), menyintesiskan kedua perspektif (Tahap A), dan menyusun argumen berbasis bukti (Tahap N).
\item \textbf{Gotong Royong:} Kemampuan untuk bekerja sama secara kolaboratif dalam kelompok untuk merumuskan masalah (Tahap E), melakukan investigasi (Tahap S), dan membangun pemahaman bersama (Tahap A).
\item \textbf{Kreatif:} Menghasilkan argumen atau solusi orisinal yang terintegrasi pada tahap akhir pembelajaran (Tahap N).
\end{itemize}

\section{Sarana dan Prasarana}

\begin{itemize}
\item \textbf{Media:} LKPD 14 - Jurnal Investigasi Dual-Lensa, video singkat pertunjukan musik tradisional Minangkabau dengan saluang dan talempong (durasi 2-3 menit), gambar/diagram struktur alat musik tradisional, artikel/infografis prinsip getaran dan resonansi, penggaris elastis dan garpu tala untuk demonstrasi.
\item \textbf{Alat:} Papan tulis/whiteboard, spidol, Proyektor \& Speaker, Kertas Plano atau Karton (1 per kelompok), sticky notes warna-warni, penggaris, garpu tala, tali/karet gelang, botol kosong berbagai ukuran.
\item \textbf{Sumber Belajar:} Buku ajar IPA kelas VIII, Tautan video animasi konsep getaran (misal: bit.ly/animasi-getaran), tautan artikel/video musik tradisional Minangkabau (misal: bit.ly/musik-minang).
\end{itemize}

\section{Target Peserta didik}

\begin{itemize}
\item Peserta didik reguler kelas VIII (Fase D).
\end{itemize}

\section{Model Pembelajaran}

\begin{itemize}
\item Model Pembelajaran KESAN (Konektivitas Etnosains-Sains).
\end{itemize}

\section{Pemahaman Bermakna}
\textit{(Clue untuk Guru: Bacakan atau sampaikan narasi ini dengan intonasi yang menarik di akhir pembelajaran untuk mengikat semua pengalaman belajar siswa menjadi satu kesatuan yang bermakna).}

\begin{tcolorbox}[mystyle]
"Ananda Semua, hari ini kita telah mengungkap rahasia di balik keindahan musik tradisional Minangkabau! Kita menemukan bahwa para pengrajin saluang, talempong, dan gandang telah menguasai prinsip-prinsip getaran, frekuensi, dan resonansi dengan sangat mahir. Mereka memahami bagaimana panjang dan diameter bamboo saluang mempengaruhi nada yang dihasilkan, bagaimana ketebalan logam talempong menentukan pitch, dan bagaimana resonansi dalam rongga gandang memperkuat suara. Semua ini diciptakan berdasarkan kepekaan pendengaran dan pemahaman intuitif tentang akustik. Dengan ini, kita sadar bahwa setiap nada yang indah dalam musik tradisional adalah hasil penerapan hukum-hukum fisika yang menakjubkan!"
\end{tcolorbox}

\section{PERTEMUAN 1: Getaran dalam Alat Musik Tradisional Minangkabau}

\subsection{Capaian Pembelajaran (Fase D)}
Menganalisis getaran, gelombang, bunyi, dan cahaya dalam konteks penerapan teknologi sehari-hari.

\subsection{Tujuan Pembelajaran Berbasis Sintaks KESAN}
\textit{(Clue untuk Guru: Setiap TP ini adalah checkpoint yang harus Anda pastikan tercapai di setiap tahap)}

Melalui Model Pembelajaran KESAN, peserta didik mampu:

\begin{enumerate}
\item \textbf{(Tahap K)} Mengidentifikasi dan mengapresiasi kearifan pembuatan alat musik tradisional Minangkabau dalam menerapkan prinsip-prinsip getaran, frekuensi, dan resonansi.
\item \textbf{(Tahap E)} Merumuskan pertanyaan investigatif tentang hubungan antara desain alat musik tradisional dan konsep getaran dalam fisika.
\item \textbf{(Tahap S)} Menganalisis secara sistematis prinsip getaran dari perspektif sains modern dan kearifan pembuatan alat musik tradisional Minangkabau.
\item \textbf{(Tahap A)} Menyintesiskan pemahaman tentang konsep getaran dengan mengintegrasikan pengetahuan fisika dan kearifan musikal tradisional.
\item \textbf{(Tahap N)} Menerapkan pemahaman terintegrasi tentang getaran untuk menganalisis dan merancang alat musik sederhana yang menggabungkan prinsip sains dan kearifan tradisional.
\end{enumerate}

\subsection{Pertanyaan Pemantik Berbasis Model KESAN}
\textit{(Clue untuk Guru: Ajukan secara berurutan dengan jeda reflektif. Biarkan ketegangan kognitif terbangun sebelum masuk ke tahap investigasi)}

\begin{itemize}
\item "Mengapa saluang yang terbuat dari bambu sederhana bisa menghasilkan nada yang indah dan bervariasi? Apa rahasia di balik suara merdu talempong yang dibuat dari logam?"
\item "Bagaimana para pengrajin tradisional bisa menentukan ukuran dan bentuk alat musik untuk menghasilkan nada yang tepat tanpa menggunakan alat pengukur frekuensi modern?"
\end{itemize}

\section{LANGKAH-LANGKAH KEGIATAN PEMBELAJARAN BERBASIS SINTAKS KESAN}

\subsection{Kegiatan Pembuka (15 Menit)}

\begin{enumerate}
\item \textbf{Orientasi} (3 menit): Salam, doa, presensi, dan pengkondisian kelas
\item \textbf{Asesmen Diagnostik} (7 menit): K-W-L Chart untuk mengidentifikasi prior knowledge tentang getaran dan musik tradisional
   \begin{itemize}
   \item \textbf{Instruksi Guru}: "Anak-anak, mari kita isi kolom K (Know) dengan apa yang kalian ketahui tentang bunyi dan alat musik tradisional. Lalu isi kolom W (Want to know) dengan apa yang ingin kalian ketahui tentang bagaimana alat musik menghasilkan suara."
   \end{itemize}
\item \textbf{Apersepsi \& Motivasi} (5 menit): Pengenalan dimensi PPP dan misi pembelajaran
\end{enumerate}

\subsection{Kegiatan Inti (90 Menit) - IMPLEMENTASI SINTAKS MODEL KESAN}

\subsubsection{Tahap 1: (K) Kaitkan Konteks Kultural (15 menit)}

\textbf{Aktivitas Guru:}
\begin{itemize}
\item Menampilkan video singkat pertunjukan musik tradisional Minangkabau dengan saluang, talempong, dan gandang
\item Mengajukan pertanyaan pemantik tentang keunikan suara alat musik tradisional
\item Memfasilitasi sharing pengalaman siswa tentang musik tradisional yang pernah mereka dengar atau alat musik yang pernah mereka mainkan
\item \textit{(Clue: Mainkan audio berbagai alat musik tradisional dan minta siswa menebak cara suara itu dihasilkan. Ciptakan ketertarikan pada kecanggihan teknologi akustik tradisional)}
\end{itemize}

\textbf{Aktivitas Siswa:}
\begin{itemize}
\item Mengamati video dan mendengarkan musik tradisional yang disajikan
\item Berbagi pengalaman pribadi tentang musik tradisional atau alat musik yang pernah dimainkan
\item Menuliskan pertanyaan awal di sticky notes untuk "Papan Penasaran"
\end{itemize}

\subsubsection{Tahap 2: (E) Eksplorasi Enigma (15 menit)}

\textbf{Aktivitas Guru:}
\begin{itemize}
\item Membentuk kelompok heterogen (3-4 siswa)
\item \textbf{Instruksi Guru}: "Sekarang, dalam kelompok kalian, diskusikan dan rumuskan 'Misi Penyelidikan Tim' kalian. Apa yang ingin kalian selidiki tentang hubungan antara cara kerja alat musik tradisional dan konsep getaran? Tuliskan minimal 3 pertanyaan kunci yang akan menjadi fokus investigasi kelompok kalian."
\item \textit{(Clue: Pastikan pertanyaan yang dirumuskan menghubungkan aspek fisik alat musik (bahan, ukuran, bentuk) dengan sifat suara yang dihasilkan. Arahkan pada hubungan antara struktur dan fungsi)}
\end{itemize}

\textbf{Aktivitas Siswa:}
\begin{itemize}
\item Berdiskusi dalam kelompok untuk merumuskan misi investigasi
\item Menuliskan minimal 3 pertanyaan kunci di kertas plano
\item Mempresentasikan rumusan masalah secara singkat
\end{itemize}

\subsubsection{Tahap 3: (S) Selidiki secara Sintetis (25 menit)}

\textbf{Aktivitas Guru:}
\begin{itemize}
\item Membagikan "LKPD 14 - Jurnal Investigasi Dual-Lensa"
\item \textbf{Instruksi Guru}: "Sekarang saatnya investigasi mendalam! Gunakan LKPD ini untuk mencatat temuan kalian dari dua perspektif. Di kolom 'Lensa Sains', catat semua informasi tentang konsep getaran, frekuensi, amplitudo, dan resonansi dari sumber-sumber ilmiah. Di kolom 'Lensa Etnosains', catat semua informasi tentang kearifan pembuatan alat musik tradisional Minangkabau dari sumber-sumber budaya."
\item \textit{(Clue: Demonstrasikan getaran dengan penggaris yang digetarkan di ujung meja, garpu tala, dan tali yang dipetik. Pastikan siswa mengamati hubungan antara frekuensi getaran dan tinggi rendah nada)}
\end{itemize}

\textbf{Aktivitas Siswa:}
\begin{itemize}
\item Melakukan investigasi dengan pembagian tugas dalam tim
\item Mencatat temuan dari perspektif sains dan etnosains secara terpisah pada LKPD
\item Mengumpulkan data tentang prinsip getaran dan kearifan pembuatan alat musik tradisional
\end{itemize}

\subsubsection{Tahap 4: (A) Asimilasi Analitis (20 menit)}

\textbf{Aktivitas Guru:}
\begin{itemize}
\item \textbf{Pertanyaan Pancingan Kunci}: "Bagaimana prinsip-prinsip getaran dan resonansi yang kalian pelajari dapat menjelaskan mengapa alat musik tradisional menghasilkan suara yang berbeda-beda? Apa kesamaan prinsip antara alat musik modern dan tradisional dalam menghasilkan bunyi?"
\item \textit{(Clue: Dorong siswa untuk mengidentifikasi bahwa baik alat musik tradisional maupun modern menggunakan prinsip fisika yang sama. Bantu mereka melihat bahwa perbedaan suara disebabkan oleh perbedaan frekuensi getaran)}
\end{itemize}

\textbf{Aktivitas Siswa:}
\begin{itemize}
\item Berdiskusi intensif untuk menghubungkan temuan dari kedua lensa
\item Mengidentifikasi kesamaan prinsip fisika dalam alat musik tradisional dan modern
\item Menuliskan kesimpulan terpadu di kertas plano
\end{itemize}

\subsubsection{Tahap 5: (N) Nyatakan Pemahaman (15 menit)}

\textbf{Aktivitas Guru:}
\begin{itemize}
\item Memberikan studi kasus: "Seorang siswa ingin membuat alat musik sederhana untuk pertunjukan sekolah dengan mengadaptasi prinsip saluang tradisional. Dia memiliki botol plastik berbagai ukuran, sedotan, dan tali karet. Berdasarkan pemahaman kalian tentang getaran dan kearifan alat musik tradisional, rancanglah alat musik sederhana yang bisa menghasilkan nada berbeda-beda."
\item \textbf{Instruksi Guru}: "Tuliskan jawaban individual kalian dalam 3-5 kalimat. Gunakan bukti dari kedua lensa yang telah kalian investigasi untuk mendukung rancangan kalian."
\end{itemize}

\textbf{Aktivitas Siswa:}
\begin{itemize}
\item Menyusun rancangan alat musik sederhana berdasarkan studi kasus
\item Mengintegrasikan pemahaman dari perspektif sains dan etnosains
\item Memberikan justifikasi yang didukung bukti dari investigasi
\end{itemize}

\subsection{Kegiatan Penutup (15 Menit)}

\begin{enumerate}
\item \textbf{Presentasi \& Penguatan} (7 menit): Beberapa siswa mempresentasikan rancangan mereka, guru memberikan penguatan konsep
\item \textbf{Refleksi} (5 menit): Melengkapi kolom L (Learned) pada K-W-L Chart, refleksi pembelajaran
\item \textbf{Tindak Lanjut \& Penutup} (3 menit): Preview pembelajaran berikutnya tentang gelombang bunyi dengan narasi berkelanjutan tentang perambatan suara musik
\end{enumerate}

\section{ASESMEN BERBASIS MODEL KESAN}

\subsection{Asesmen Diagnostik (Tahap K)}
\begin{itemize}
\item \textbf{Instrumen}: K-W-L Chart
\item \textbf{Tujuan}: Mengidentifikasi pengetahuan awal siswa tentang getaran dan musik tradisional
\end{itemize}

\subsection{Asesmen Formatif (Tahap E-S-A)}
\begin{itemize}
\item \textbf{Tahap E}: Penilaian kualitas "Misi Penyelidikan" kelompok
\item \textbf{Tahap S}: Kelengkapan dan kualitas "Jurnal Investigasi Dual-Lensa"
\item \textbf{Tahap A}: Observasi diskusi dan kemampuan sintesis kelompok
\end{itemize}

\subsection{Asesmen Sumatif (Tahap N)}
\begin{itemize}
\item \textbf{Instrumen}: Rancangan alat musik sederhana dengan rubrik 4 level
\item \textbf{Aspek}: Ketepatan konsep getaran, keterkaitan etnosains, kreativitas rancangan
\end{itemize}

\section{RUBRIK PENILAIAN STUDI KASUS}

\begin{longtable}{|p{3cm}|p{3cm}|p{3cm}|p{3cm}|p{3cm}|}
\hline
\textbf{Kriteria Penilaian} & \textbf{Skor 4 (Sangat Baik)} & \textbf{Skor 3 (Baik)} & \textbf{Skor 2 (Cukup)} & \textbf{Skor 1 (Kurang)} \\
\hline
\textbf{Ketepatan Konsep Getaran} & Menjelaskan konsep getaran, frekuensi, dan resonansi dengan tepat dan mengaitkannya secara akurat dengan rancangan alat musik & Menjelaskan konsep getaran dengan cukup tepat dan mengaitkannya dengan rancangan meski ada beberapa kekeliruan minor & Menjelaskan konsep getaran secara umum namun kaitan dengan rancangan kurang jelas atau mengandung kesalahan konsep & Penjelasan konsep getaran tidak tepat atau tidak dikaitkan dengan rancangan alat musik \\
\hline
\textbf{Keterkaitan dengan Etnosains} & Mengidentifikasi dengan jelas kearifan pembuatan alat musik tradisional dan mengintegrasikannya secara relevan dalam rancangan & Mengidentifikasi kearifan musik tradisional dan cukup berhasil mengintegrasikannya dalam rancangan meski kurang detail & Menyebutkan aspek tradisional namun integrasinya dengan rancangan masih superfisial & Tidak mengaitkan atau salah mengaitkan kearifan tradisional dengan rancangan \\
\hline
\textbf{Kreativitas \& Kelayakan Rancangan} & Rancangan sangat kreatif, inovatif, dan realistis untuk dibuat dengan bahan yang tersedia & Rancangan cukup kreatif dan layak dengan pertimbangan teknis yang memadai & Rancangan standar dengan kreativitas terbatas namun masih bisa direalisasikan & Rancangan tidak kreatif, tidak realistis, atau tidak bisa dibuat dengan bahan yang ada \\
\hline
\end{longtable}

\vspace{1cm}

\section{DAFTAR PUSTAKA ETNOSAINS}

\begin{enumerate}
\item Navis, A.A. (1984). \textit{Alam Terkembang Jadi Guru: Adat dan Kebudayaan Minangkabau}. Jakarta: Grafiti Pers.
\item Kartomi, Margaret J. (2012). \textit{Musical Journeys in Sumatra}. Urbana: University of Illinois Press.
\item Schaareman, Danker. (1993). \textit{Tabuik: A Shia Ritual Transplanted from Iraq to West Sumatra}. Jakarta: INIS.
\item Kausar, Akmal Sjafril. (2011). \textit{Saluang: Warisan Budaya Minangkabau}. Padang: Universitas Negeri Padang Press.
\item Dokumentasi Etnomusicology Sumatra Barat. (2019). Balai Pelestarian Nilai Budaya Sumatra Barat.
\end{enumerate}

\end{document}