\documentclass[12pt,a4paper]{article}
\usepackage[margin=1.5cm]{geometry}
\usepackage[utf8]{inputenc}
\usepackage{amsmath}
\usepackage{amsfonts}
\usepackage{amssymb}
\usepackage{graphicx}
\usepackage{xcolor}
\usepackage{tcolorbox}
\usepackage{enumitem}
\usepackage{multicol}
\usepackage{fancyhdr}
\usepackage{titlesec}
\usepackage{array}
\usepackage{longtable}
\usepackage{booktabs}

% Setup header and footer
\pagestyle{fancy}
\fancyhf{}
\fancyhead[L]{\textbf{RPP IPA Kelas VIII - Model KESAN}}
\fancyhead[R]{\textbf{Semester Genap 2025/2026}}
\fancyfoot[C]{\thepage}

% Define colors
\definecolor{primarycolor}{RGB}{0,0,0}
\definecolor{secondarycolor}{RGB}{255,255,255}

% Custom tcolorbox styles
\tcbset{
    mystyle/.style={
        colback=white,
        colframe=black,
        boxrule=1pt,
        arc=2pt,
        left=3pt,
        right=3pt,
        top=3pt,
        bottom=3pt
    }
}

\begin{document}

\begin{center}
\textbf{\Large MODUL AJAR: Gaung Randai \& Rahasia Gelombang Bunyi}
\end{center}

\vspace{0.5cm}

\begin{tcolorbox}[mystyle]
\textbf{Nama Penyusun:} Irfan Ananda \\
\textbf{Institusi:} SMP \\
\textbf{Mata Pelajaran:} Ilmu Pengetahuan Alam (IPA) \\
\textbf{Tahun Ajaran:} 2025/2026 \\
\textbf{Semester:} Genap \\
\textbf{Jenjang Sekolah:} SMP \\
\textbf{Kelas/Fase:} VIII / D \\
\textbf{Alokasi waktu:} 3 x 40 Menit (1 Pertemuan)
\end{tcolorbox}

\section{DIMENSI PROFIL PELAJAR PANCASILA}
\textit{(Clue untuk Guru: Sebutkan dimensi ini secara eksplisit saat apersepsi agar siswa sadar tujuan non-akademis yang sedang mereka kembangkan).}

\begin{itemize}
\item \textbf{Berkebinekaan Global:} Mengenal dan menghargai budaya, khususnya menganalisis kearifan pertunjukan Randai Minangkabau dalam memanfaatkan akustik ruang terbuka/lingkaran untuk penyebaran suara (gelombang bunyi), lalu menghubungkannya dengan konsep fisika universal.
\item \textbf{Bernalar Kritis:} Ditempa secara intensif saat menganalisis informasi dari sumber sains dan budaya (Tahap S), menyintesiskan kedua perspektif (Tahap A), dan menyusun argumen berbasis bukti (Tahap N).
\item \textbf{Gotong Royong:} Kemampuan untuk bekerja sama secara kolaboratif dalam kelompok untuk merumuskan masalah (Tahap E), melakukan investigasi (Tahap S), dan membangun pemahaman bersama (Tahap A).
\item \textbf{Kreatif:} Menghasilkan argumen atau solusi orisinal yang terintegrasi pada tahap akhir pembelajaran (Tahap N).
\end{itemize}

\section{Sarana dan Prasarana}

\begin{itemize}
\item \textbf{Media:} LKPD 15 - Jurnal Investigasi Dual-Lensa, video singkat pertunjukan Randai di lapangan/halaman rumah gadang (durasi 2-3 menit), gambar/diagram perambatan gelombang bunyi, artikel/infografis sifat-sifat gelombang (amplitudo, frekuensi, panjang gelombang, cepat rambat), aplikasi generator bunyi/speaker kecil.
\item \textbf{Alat:} Papan tulis/whiteboard, spidol, Proyektor \& Speaker, Kertas Plano atau Karton (1 per kelompok), sticky notes warna-warni, garpu tala, pipa PVC pendek, meteran/penggaris panjang, smartphone dengan aplikasi decibel meter (bila ada).
\item \textbf{Sumber Belajar:} Buku ajar IPA kelas VIII, Tautan video animasi gelombang bunyi (misal: bit.ly/animasi-bunyi), tautan artikel/video Randai Minangkabau (misal: bit.ly/randai-minang).
\end{itemize}

\section{Target Peserta didik}

\begin{itemize}
\item Peserta didik reguler kelas VIII (Fase D).
\end{itemize}

\section{Model Pembelajaran}

\begin{itemize}
\item Model Pembelajaran KESAN (Konektivitas Etnosains-Sains).
\end{itemize}

\section{Pemahaman Bermakna}
\textit{(Clue untuk Guru: Bacakan atau sampaikan narasi ini dengan intonasi yang menarik di akhir pembelajaran untuk mengikat semua pengalaman belajar siswa menjadi satu kesatuan yang bermakna).}

\begin{tcolorbox}[mystyle]
"Ananda Semua, hari ini kita belajar bahwa keindahan Randai tidak hanya terletak pada gerak dan cerita, tetapi juga pada sains bunyi yang membuat setiap dialog dan dendang terdengar jelas bagi penonton. Lingkaran Randai, posisi pemain, tempo tepukan tangan, hingga pantulan suara dari dinding rumah gadang—semuanya adalah desain akustik yang memanfaatkan sifat-sifat gelombang bunyi. Dengan memahami sains di balik budaya, kita semakin bangga pada kearifan \'urang awak' yang peka terhadap alam dan cerdas dalam berkesenian."
\end{tcolorbox}

\section{PERTEMUAN 2: Sifat Gelombang Bunyi dalam Randai Minangkabau}

\subsection{Capaian Pembelajaran (Fase D)}
Menganalisis getaran, gelombang, dan bunyi dalam konteks teknologi dan budaya, termasuk sifat-sifat gelombang dan perambatan bunyi pada berbagai media.

\subsection{Tujuan Pembelajaran Berbasis Sintaks KESAN}
\textit{(Clue untuk Guru: Setiap TP ini adalah checkpoint yang harus Anda pastikan tercapai di setiap tahap)}

Melalui Model Pembelajaran KESAN, peserta didik mampu:

\begin{enumerate}
\item \textbf{(Tahap K)} Mengidentifikasi dan mengapresiasi praktik akustik dalam pertunjukan Randai (formasi lingkaran, tempo tepuk, proyeksi suara) yang terkait dengan sifat gelombang bunyi.
\item \textbf{(Tahap E)} Merumuskan pertanyaan investigatif tentang hubungan antara tata panggung Randai dan konsep gelombang bunyi (amplitudo, frekuensi, panjang gelombang, cepat rambat, refleksi, absorpsi).
\item \textbf{(Tahap S)} Menganalisis secara sistematis sifat gelombang bunyi dari perspektif sains modern dan kearifan penataan pertunjukan Randai.
\item \textbf{(Tahap A)} Menyintesiskan pemahaman tentang gelombang bunyi dengan mengintegrasikan konsep fisika dan kearifan akustik pertunjukan tradisional.
\item \textbf{(Tahap N)} Menerapkan pemahaman untuk merekomendasikan desain tata panggung Randai yang akustiknya optimal pada lingkungan sekolah.
\end{enumerate}

\subsection{Pertanyaan Pemantik Berbasis Model KESAN}
\textit{(Clue untuk Guru: Ajukan secara berurutan dengan jeda reflektif. Biarkan ketegangan kognitif terbangun sebelum masuk ke tahap investigasi)}

\begin{itemize}
\item "Mengapa dialog dan dendang dalam Randai tetap terdengar jelas meskipun dilakukan di ruang terbuka tanpa mikrofon?"
\item "Bagaimana formasi lingkaran dan posisi pemain mempengaruhi kuat lemahnya suara yang didengar penonton? Apakah ada kaitan dengan sifat gelombang bunyi?"
\end{itemize}

\section{LANGKAH-LANGKAH KEGIATAN PEMBELAJARAN BERBASIS SINTAKS KESAN}

\subsection{Kegiatan Pembuka (15 Menit)}

\begin{enumerate}
\item \textbf{Orientasi} (3 menit): Salam, doa, presensi, dan pengkondisian kelas
\item \textbf{Asesmen Diagnostik} (7 menit): K-W-L Chart untuk mengidentifikasi prior knowledge tentang gelombang bunyi dan Randai
   \begin{itemize}
   \item \textbf{Instruksi Guru}: "Anak-anak, mari kita isi kolom K (Know) dengan apa yang kalian ketahui tentang bunyi dan Randai. Lalu isi kolom W (Want to know) dengan apa yang ingin kalian ketahui tentang penyebaran suara pada pertunjukan Randai."
   \end{itemize}
\item \textbf{Apersepsi \& Motivasi} (5 menit): Pengenalan dimensi PPP dan misi pembelajaran
\end{enumerate}

\subsection{Kegiatan Inti (90 Menit) - IMPLEMENTASI SINTAKS MODEL KESAN}

\subsubsection{Tahap 1: (K) Kaitkan Konteks Kultural (15 menit)}

\textbf{Aktivitas Guru:}
\begin{itemize}
\item Menampilkan video singkat pertunjukan Randai di ruang terbuka/halaman rumah gadang
\item Mengajukan pertanyaan pemantik tentang kejelasan suara dan gaung pada pertunjukan tradisional
\item Memfasilitasi sharing pengalaman siswa tentang menonton pertunjukan Randai atau teater rakyat lain
\item \textit{(Clue: Ajak siswa menirukan formasi lingkaran kecil di kelas untuk merasakan perubahan intensitas bunyi saat posisi berubah)}
\end{itemize}

\textbf{Aktivitas Siswa:}
\begin{itemize}
\item Mengamati video dan mendiskusikan fenomena akustik yang terlihat
\item Berbagi pengalaman pribadi tentang pertunjukan Randai atau teater rakyat
\item Menuliskan pertanyaan awal di sticky notes untuk "Papan Penasaran"
\end{itemize}

\subsubsection{Tahap 2: (E) Eksplorasi Enigma (15 menit)}

\textbf{Aktivitas Guru:}
\begin{itemize}
\item Membentuk kelompok heterogen (3-4 siswa)
\item \textbf{Instruksi Guru}: "Sekarang, dalam kelompok kalian, rumuskan 'Misi Penyelidikan Tim' tentang bagaimana sifat-sifat gelombang bunyi mempengaruhi kejelasan suara pada Randai. Tuliskan minimal 3 pertanyaan kunci yang menjadi fokus investigasi."
\item \textit{(Clue: Arahkan pertanyaan ke topik amplitudo vs keras-lembut bunyi, frekuensi vs tinggi-rendah nada, refleksi vs gaung/echo, absorpsi oleh bahan)}
\end{itemize}

\textbf{Aktivitas Siswa:}
\begin{itemize}
\item Berdiskusi dalam kelompok untuk merumuskan misi investigasi
\item Menuliskan minimal 3 pertanyaan kunci di kertas plano
\item Mempresentasikan rumusan masalah secara singkat
\end{itemize}

\subsubsection{Tahap 3: (S) Selidiki secara Sintetis (25 menit)}

\textbf{Aktivitas Guru:}
\begin{itemize}
\item Membagikan "LKPD 15 - Jurnal Investigasi Dual-Lensa"
\item \textbf{Instruksi Guru}: "Gunakan LKPD ini untuk mencatat temuan dari dua perspektif. Di kolom 'Lensa Sains', catat konsep gelombang bunyi (amplitudo, frekuensi, panjang gelombang, cepat rambat, refleksi, absorpsi, resonansi). Di kolom 'Lensa Etnosains', catat praktik akustik Randai (formasi lingkaran, posisi pemain, tempo tepuk, pemilihan lokasi)."
\item \textit{(Clue: Demonstrasikan pantulan bunyi di lorong/ruang kelas dan peredaman oleh bahan kain/karpet. Gunakan garpu tala/pipa untuk menunjukkan resonansi)}
\end{itemize}

\textbf{Aktivitas Siswa:}
\begin{itemize}
\item Melakukan investigasi dengan pembagian tugas dalam tim
\item Mencatat temuan dari kedua lensa pada LKPD
\item Mengumpulkan data tentang sifat gelombang bunyi dan kearifan tata panggung Randai
\end{itemize}

\subsubsection{Tahap 4: (A) Asimilasi Analitis (20 menit)}

\textbf{Aktivitas Guru:}
\begin{itemize}
\item \textbf{Pertanyaan Pancingan Kunci}: "Bagaimana pengaturan formasi, posisi, dan tempo dalam Randai memanfaatkan sifat gelombang bunyi untuk memperjelas suara tanpa mikrofon? Desain panggung seperti apa yang paling sesuai dengan prinsip-prinsip ini?"
\item \textit{(Clue: Fasilitasi siswa untuk menyusun peta konsep yang menghubungkan konsep fisika dengan praktik akustik Randai)}
\end{itemize}

\textbf{Aktivitas Siswa:}
\begin{itemize}
\item Berdiskusi intensif menyintesiskan temuan dari kedua lensa
\item Menghasilkan rekomendasi desain akustik sederhana untuk panggung Randai di sekolah
\item Menuliskan kesimpulan terpadu di kertas plano
\end{itemize}

\subsubsection{Tahap 5: (N) Nyatakan Pemahaman (15 menit)}

\textbf{Aktivitas Guru:}
\begin{itemize}
\item Memberikan studi kasus: "OSIS akan mengadakan pertunjukan Randai di lapangan sekolah. Anggaran minim, tanpa mikrofon. Berdasarkan pemahaman kalian tentang gelombang bunyi dan kearifan Randai, buatlah rekomendasi tata panggung, posisi pemain, dan bahan latar untuk hasil akustik terbaik."
\item \textbf{Instruksi Guru}: "Tuliskan jawaban individual 3-5 kalimat dengan bukti dari kedua lensa."
\end{itemize}

\textbf{Aktivitas Siswa:}
\begin{itemize}
\item Menyusun rekomendasi akustik untuk pertunjukan Randai di sekolah
\item Mengintegrasikan pemahaman dari perspektif sains dan etnosains
\item Memberikan justifikasi yang didukung bukti dari investigasi
\end{itemize}

\subsection{Kegiatan Penutup (15 Menit)}

\begin{enumerate}
\item \textbf{Presentasi \& Penguatan} (7 menit): Beberapa siswa mempresentasikan rekomendasi, guru memberikan penguatan konsep
\item \textbf{Refleksi} (5 menit): Melengkapi kolom L (Learned) pada K-W-L Chart, refleksi pembelajaran
\item \textbf{Tindak Lanjut \& Penutup} (3 menit): Preview pertemuan berikutnya tentang gelombang pada air dan tali, mengaitkan dengan permainan anak nagari
\end{enumerate}

\section{ASESMEN BERBASIS MODEL KESAN}

\subsection{Asesmen Diagnostik (Tahap K)}
\begin{itemize}
\item \textbf{Instrumen}: K-W-L Chart
\item \textbf{Tujuan}: Mengidentifikasi pengetahuan awal tentang gelombang bunyi dan Randai
\end{itemize}

\subsection{Asesmen Formatif (Tahap E-S-A)}
\begin{itemize}
\item \textbf{Tahap E}: Penilaian kualitas "Misi Penyelidikan" kelompok
\item \textbf{Tahap S}: Kelengkapan dan kualitas "Jurnal Investigasi Dual-Lensa"
\item \textbf{Tahap A}: Observasi diskusi dan kemampuan sintesis kelompok
\end{itemize}

\subsection{Asesmen Sumatif (Tahap N)}
\begin{itemize}
\item \textbf{Instrumen}: Rekomendasi desain akustik dengan rubrik 4 level
\item \textbf{Aspek}: Ketepatan konsep gelombang bunyi, keterkaitan etnosains, kelogisan dan kelayakan rekomendasi
\end{itemize}

\section{RUBRIK PENILAIAN STUDI KASUS}

\begin{longtable}{|p{3cm}|p{3cm}|p{3cm}|p{3cm}|p{3cm}|}
\hline
\textbf{Kriteria Penilaian} & \textbf{Skor 4 (Sangat Baik)} & \textbf{Skor 3 (Baik)} & \textbf{Skor 2 (Cukup)} & \textbf{Skor 1 (Kurang)} \\
\hline
\textbf{Ketepatan Konsep Gelombang Bunyi} & Menggunakan konsep amplitudo, frekuensi, panjang gelombang, cepat rambat, refleksi/absorpsi secara tepat dan relevan & Menggunakan sebagian besar konsep dengan tepat meski ada kekeliruan minor & Menggunakan konsep secara umum namun mengandung kekeliruan atau kurang relevan & Tidak menggunakan konsep yang tepat atau salah kaprah \\
\hline
\textbf{Keterkaitan dengan Etnosains} & Mengintegrasikan praktik akustik Randai secara jelas dan relevan dalam rekomendasi desain & Mengaitkan praktik Randai dengan rekomendasi meski kurang detail & Menyebutkan Randai tetapi integrasi masih superfisial & Tidak mengaitkan atau salah mengaitkan praktik Randai \\
\hline
\textbf{Kelogisan \& Kelayakan Rekomendasi} & Rekomendasi logis, realistis, dan mudah diterapkan di sekolah dengan sumber daya terbatas & Rekomendasi cukup logis dan layak dengan penyesuaian kecil & Rekomendasi sebagian layak namun kurang terperinci atau kurang realistis & Rekomendasi tidak logis atau tidak realistis untuk diterapkan \\
\hline
\end{longtable}

\vspace{1cm}

\section{DAFTAR PUSTAKA ETNOSAINS}

\begin{enumerate}
\item Navis, A.A. (1984). \textit{Alam Terkembang Jadi Guru: Adat dan Kebudayaan Minangkabau}. Jakarta: Grafiti Pers.
\item Kartomi, Margaret J. (2012). \textit{Musical Journeys in Sumatra}. Urbana: University of Illinois Press.
\item Amir, Adriyetti. (2013). \textit{Randai Minangkabau: Kajian Sastra Lisan}. Padang: UNP Press.
\item Dokumentasi Pertunjukan Randai. (2019). Balai Pelestarian Nilai Budaya Sumatra Barat.
\end{enumerate}

\end{document}