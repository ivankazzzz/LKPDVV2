% !TEX encoding = UTF-8

\documentclass[12pt,a4paper]{article}
\usepackage[margin=1.5cm]{geometry}
\usepackage[utf8]{inputenc}
\usepackage[T1]{fontenc}
\usepackage{times}
\usepackage{amsmath}
\usepackage{amsfonts}
\usepackage{amssymb}
\usepackage{graphicx}
\usepackage{xcolor}
\usepackage{tcolorbox}
\usepackage{enumitem}
\usepackage{multicol}
\usepackage{fancyhdr}
\usepackage{titlesec}

% Setup header and footer
\pagestyle{fancy}
\fancyhf{}
\fancyhead[L]{\textbf{RPP IPA Kelas VIII - Model KESAN}}
\fancyhead[R]{\textbf{Semester 2 - Pertemuan 14.2}}
\fancyfoot[C]{\thepage}

% Title formatting
\titleformat{\section}{\Large\bfseries}{}{0em}{}
\titleformat{\subsection}{\large\bfseries}{}{0em}{}
\titleformat{\subsubsection}{\normalsize\bfseries}{}{0em}{}

\begin{document}

\begin{center}
{\Huge\textbf{MODUL AJAR}}\\
\vspace{0.5cm}
{\Large\textbf{Kincir Air Minang \& Misteri Usaha dan Energi}}
\end{center}

\vspace{1cm}

\begin{tcolorbox}[colback=white,colframe=black,boxrule=1pt]
\textbf{Nama Penyusun:} Irfan Ananda\\
\textbf{Institusi:} SMP\\
\textbf{Mata Pelajaran:} Ilmu Pengetahuan Alam (IPA)\\
\textbf{Tahun Ajaran:} 2025/2026\\
\textbf{Semester:} Genap\\
\textbf{Jenjang Sekolah:} SMP\\
\textbf{Kelas/Fase:} VIII / D\\
\textbf{Alokasi waktu:} 3 x 40 Menit (1 Pertemuan)
\end{tcolorbox}

\section{DIMENSI PROFIL PELAJAR PANCASILA}
\textit{(Clue untuk Guru: Sebutkan dimensi ini secara eksplisit saat apersepsi agar siswa sadar tujuan non-akademis yang sedang mereka kembangkan).}

\begin{itemize}
\item \textbf{Berkebinekaan Global:} Mengenal dan menghargai budaya, khususnya menganalisis kearifan lokal (etnosains) Minangkabau dalam teknologi pengairan tradisional seperti kincir air dan sistem irigasi, lalu menghubungkannya dengan konsep sains universal tentang usaha dan energi.
\item \textbf{Bernalar Kritis:} Ditempa secara intensif saat menganalisis informasi dari sumber sains dan budaya (Tahap S), menyintesiskan kedua perspektif (Tahap A), dan menyusun argumen berbasis bukti (Tahap N).
\item \textbf{Gotong Royong:} Kemampuan untuk bekerja sama secara kolaboratif dalam kelompok untuk merumuskan masalah (Tahap E), melakukan investigasi (Tahap S), dan membangun pemahaman bersama (Tahap A).
\item \textbf{Kreatif:} Menghasilkan argumen atau solusi orisinal yang terintegrasi pada tahap akhir pembelajaran (Tahap N).
\end{itemize}

\section{Sarana dan Prasarana}

\begin{itemize}
\item \textbf{Media:} LKPD 17 - Jurnal Investigasi Dual-Lensa, video singkat kincir air tradisional Minangkabau dan sistem irigasi (misal dari YouTube, durasi 2-3 menit), gambar berbagai teknologi pengairan tradisional, artikel/infografis tentang usaha dan energi, diagram konversi energi pada kincir air.
\item \textbf{Alat:} Papan tulis/whiteboard, spidol, Proyektor \& Speaker, Kertas Plano atau Karton (1 per kelompok), sticky notes warna-warni, model kincir air sederhana (dari botol plastik dan sumpit), wadah air, beban kecil, penggaris, stopwatch.
\item \textbf{Sumber Belajar:} Buku ajar IPA kelas VIII, Tautan video animasi usaha dan energi (misal: bit.ly/animasi-usaha), tautan artikel teknologi pengairan tradisional Minangkabau (misal: bit.ly/irigasi-minang).
\end{itemize}

\section{Target Peserta didik}

\begin{itemize}
\item Peserta didik reguler kelas VIII (Fase D).
\end{itemize}

\section{Model Pembelajaran}

\begin{itemize}
\item Model Pembelajaran KESAN (Konektivitas Etnosains-Sains).
\end{itemize}

\section{Pemahaman Bermakna}
\textit{(Clue untuk Guru: Bacakan atau sampaikan narasi ini dengan intonasi yang menarik di akhir pembelajaran untuk mengikat semua pengalaman belajar siswa menjadi satu kesatuan yang bermakna).}

"Ananda Semua, hari ini kita telah mengungkap keajaiban teknologi nenek moyang Minangkabau dalam memanfaatkan energi air! Kita menemukan bahwa kincir air tradisional yang sederhana ternyata merupakan aplikasi brilian dari prinsip-prinsip usaha dan konversi energi. Setiap sudu kincir, setiap aliran air yang diarahkan, semuanya dirancang untuk memaksimalkan konversi energi kinetik air menjadi energi mekanik yang berguna. Nenek moyang kita telah menjadi insinyur energi sejati! Dengan ini, kita sadar bahwa teknologi ramah lingkungan dan berkelanjutan bukanlah hal baru, tapi sudah dipraktikkan oleh leluhur kita berabad-abad yang lalu."

\section{PERTEMUAN KEDUA: Usaha dan Energi dalam Teknologi Pengairan Tradisional Minangkabau}

\subsection{Capaian Pembelajaran (Fase D)}
Pada akhir Fase D, murid memiliki kemampuan [...] menganalisis gerak lurus, gerak melingkar, gaya, dan penerapannya pada gerak makhluk hidup dan gerak benda [...]

\subsection{Tujuan Pembelajaran (TP) Pertemuan 2:}
\textit{(Clue untuk Guru: Tujuan ini adalah kompas Anda. Pastikan setiap tahapan KESAN yang Anda lalui berkontribusi pada pencapaian tujuan-tujuan ini).}

Melalui model pembelajaran KESAN, peserta didik mampu:
\begin{itemize}
\item Menghubungkan fenomena kearifan lokal Minangkabau (penggunaan kincir air dan sistem irigasi tradisional) dengan konteks usaha dan energi. (Sintaks K)
\item Merumuskan pertanyaan investigatif mengenai cara kerja konversi energi pada kincir air dan efisiensi sistem pengairan tradisional. (Sintaks E)
\item Mengumpulkan informasi mengenai konsep usaha, energi kinetik, energi potensial, dan konversi energi dari sumber ilmiah serta teknologi pengairan tradisional Minangkabau dari sumber kultural. (Sintaks S)
\item Menganalisis dan menyintesiskan hubungan sebab-akibat antara konsep ilmiah (usaha, energi, konversi energi) dengan desain dan fungsi teknologi pengairan tradisional. (Sintaks A)
\item Menyusun sebuah penjelasan analitis yang logis mengenai bagaimana prinsip fisika usaha dan energi diterapkan dalam teknologi pengairan tradisional Minangkabau. (Sintaks N)
\end{itemize}

\subsection{Pertanyaan Pemantik}
\textit{(Clue untuk Guru: Ajukan dua pertanyaan ini secara berurutan, berikan jeda agar siswa berpikir. Jangan langsung minta jawaban, biarkan pertanyaan ini menggantung untuk memicu rasa ingin tahu).}

\begin{itemize}
\item "Pernahkah kalian melihat kincir air tradisional yang digunakan untuk mengangkat air atau menggiling padi? Dari mana energi untuk memutar kincir itu berasal dan bagaimana bisa menghasilkan kerja yang berguna?"
\item "Coba perhatikan sistem irigasi tradisional Minangkabau yang bisa mengalirkan air ke sawah-sawah di dataran tinggi. Bagaimana air bisa 'bekerja' untuk kepentingan pertanian tanpa menggunakan mesin modern?"
\end{itemize}

\section{Langkah-langkah Kegiatan Pembelajaran:}

\subsection{Kegiatan Pembuka (15 Menit)}
\begin{itemize}
\item Guru membuka pelajaran dengan salam, doa, dan memeriksa kehadiran.
\item \textbf{Asesmen Diagnostik Awal:} Guru membagikan lembar K-W-L.
    \begin{itemize}
    \item \textbf{Instruksi Guru:} "Ananda, setelah kita mempelajari gaya pada bajak tradisional, sekarang kita akan mengeksplorasi dunia usaha dan energi. Di lembar K-W-L ini, tulis di kolom \textbf{K (Tahu)} apa yang sudah kalian ketahui tentang energi ATAU tentang kincir air. Di kolom \textbf{W (Ingin Tahu)}, tulis apa yang membuat kalian penasaran tentang topik ini." \textit{(Clue: Ini membantu Anda melihat koneksi dengan pembelajaran sebelumnya dan pengalaman siswa dengan teknologi air).}
    \end{itemize}
\item \textbf{Apersepsi:}
    \begin{itemize}
    \item \textbf{Instruksi Guru:} "Hari ini kita akan menjadi insinyur energi tradisional! Kita akan menyelidiki bagaimana usaha dan energi bekerja dalam teknologi pengairan dan bagaimana nenek moyang kita menciptakan sistem yang berkelanjutan. Dalam investigasi ini, kita akan melatih kemampuan \textbf{Bernalar Kritis} kita, menghargai budaya lewat \textbf{Berkebinekaan Global}, dan bekerja sama dalam semangat \textbf{Gotong Royong}. Siap?"
    \end{itemize}
\end{itemize}

\subsection{Kegiatan Inti (90 Menit) - SINTAKS MODEL KESAN}

\subsubsection{Tahap 1: (K) Kaitkan Konteks Kultural (15 menit)}
\begin{itemize}
\item \textbf{Aktivitas Guru:}
    \begin{itemize}
    \item Menampilkan video singkat (2-3 menit) kincir air tradisional Minangkabau dan sistem irigasi yang masih berfungsi.
    \item Melakukan demonstrasi sederhana: menggunakan model kincir air sederhana untuk mengangkat beban kecil dengan aliran air.
    \item Mengajukan Pertanyaan Pemantik yang sudah disiapkan di atas.
    \item \textit{(Clue: Tujuan tahap ini adalah memvalidasi pengetahuan siswa tentang teknologi air dan memantik rasa heran tentang konversi energi. Biarkan siswa mengamati demonstrasi kincir air secara langsung. Tuliskan semua ide mereka dengan judul "\textbf{KEAJAIBAN TEKNOLOGI AIR KITA}").}
    \end{itemize}
\item \textbf{Aktivitas Siswa:} Mengamati demonstrasi dan video, mendengarkan pertanyaan, lalu secara sukarela berbagi pengalaman atau dugaan awal. Menuliskan minimal satu pertanyaan atau pengamatan di sticky notes dan menempelkannya di '\textbf{Papan Penasaran}'.
\end{itemize}

\subsubsection{Tahap 2: (E) Eksplorasi Enigma (15 menit)}
\begin{itemize}
\item \textbf{Aktivitas Guru:} Membentuk siswa menjadi kelompok (3-4 orang).
    \begin{itemize}
    \item \textbf{Instruksi Guru:} "Fenomena yang kalian amati tadi sangat menakjubkan! Sekarang, tugas kita sebagai insinyur energi adalah mengubah rasa takjub ini menjadi misi penelitian yang jelas. Dalam kelompok, diskusikan dan rumuskan minimal 3 pertanyaan kunci yang akan kita selidiki hari ini. Tuliskan dalam bentuk '\textbf{Misi Penyelidikan Tim [Nama Kelompok]}' di kertas plano yang Bapak/Ibu berikan."
    \item \textit{(Clue: Arahkan diskusi siswa agar pertanyaannya mencakup aspek 'bagaimana energi berubah bentuk' dan 'mengapa teknologi ini efisien'. Jika kelompok kesulitan, berikan pancingan: "Kira-kira, bagaimana energi air bisa diubah menjadi energi gerak? Dan mengapa kincir air bisa bekerja tanpa bahan bakar?").}
    \end{itemize}
\item \textbf{Aktivitas Siswa:} Berdiskusi dalam tim untuk merumuskan misi penyelidikan (daftar pertanyaan kunci) di kertas plano. \textit{(Contoh misi yang diharapkan: 1. Bagaimana energi air dikonversi menjadi energi mekanik? 2. Mengapa kincir air bisa mengangkat beban? 3. Bagaimana sistem irigasi tradisional memanfaatkan energi gravitasi?)}.
\end{itemize}

\subsubsection{Tahap 3: (S) Selidiki secara Sintetis (25 menit)}
\begin{itemize}
\item \textbf{Aktivitas Guru:} Membagikan "\textbf{LKPD 17 - Jurnal Investigasi Dual-Lensa}".
    \begin{itemize}
    \item \textbf{Instruksi Guru:} "Setiap tim akan melakukan investigasi dari dua lensa. Gunakan HP atau sumber yang disediakan untuk mencari jawabannya. Bagilah tugas dalam tim!"
    \item \textbf{Lensa Sains:} Buka link video bit.ly/animasi-usaha untuk memahami konsep usaha, energi kinetik, energi potensial, dan konversi energi. Lakukan juga eksperimen dengan model kincir air.
    \item \textbf{Lensa Etnosains/Kultural:} Buka link artikel bit.ly/irigasi-minang untuk memahami teknologi pengairan tradisional Minangkabau, sistem kerja kincir air, dan filosofi keberlanjutan dalam budaya Minang.
    \item \textit{(Clue: Pastikan sumber belajar sudah disiapkan dan model kincir air tersedia. Berkelilinglah untuk memastikan setiap kelompok melakukan eksperimen dengan benar dan mengamati konversi energi yang terjadi).}
    \end{itemize}
\item \textbf{Aktivitas Siswa:} Dalam kelompok, siswa berbagi tugas mencari informasi dari sumber yang diberikan dan mencatat temuan kunci di dua kolom terpisah pada "\textbf{LKPD 17 - Jurnal Investigasi Dual-Lensa}".
\end{itemize}

\subsubsection{Tahap 4: (A) Asimilasi Analitis (20 menit)}
\begin{itemize}
\item \textbf{Aktivitas Guru:} Memfasilitasi diskusi untuk menjembatani kedua lensa.
    \begin{itemize}
    \item \textbf{Pertanyaan Pancingan Kunci untuk Guru:}
        \begin{itemize}
        \item "Oke, dari Lensa Sains kita tahu energi dapat berubah bentuk dari kinetik ke potensial. Dari Lensa Etnosains, kita tahu kincir air bisa mengangkat air ke tempat tinggi. Nah, coba hubungkan! Bagaimana prinsip konversi energi diterapkan dalam kincir air?"
        \item "Dari Lensa Sains, kita tahu usaha = gaya × perpindahan. Dari Lensa Budaya, sistem irigasi tradisional bisa mengalirkan air tanpa pompa listrik. Apa hubungannya dengan efisiensi energi?"
        \end{itemize}
    \item \textit{(Clue: Fokuskan untuk membuat siswa 'menemukan' hubungannya sendiri. Gunakan kata "menurut kalian", "kira-kira bagaimana", "ada yang bisa menjelaskan?").}
    \end{itemize}
\item \textbf{Aktivitas Siswa:} Berdiskusi intensif untuk menghubungkan temuan sains dan budaya. Menuliskan kesimpulan terpadu (sintesis) mereka di kertas plano.
\end{itemize}

\subsubsection{Tahap 5: (N) Nyatakan Pemahaman (15 menit)}
\begin{itemize}
\item \textbf{Aktivitas Guru:} Memberikan studi kasus individual atau per kelompok.
    \begin{itemize}
    \item \textbf{Instruksi Guru (tuliskan di papan tulis):}
    
    "\textbf{STUDI KASUS UNTUK INSINYUR ENERGI TRADISIONAL:}
    
    Sebuah desa di Minangkabau ingin membangun sistem pengairan yang ramah lingkungan untuk sawah mereka di dataran tinggi. Mereka meminta bantuanmu untuk menjelaskan mengapa teknologi kincir air tradisional lebih berkelanjutan dibandingkan pompa listrik modern.
    
    Tugasmu: Tuliskan sebuah penjelasan singkat (3-5 kalimat) di buku latihanmu untuk membantu desa tersebut memahami keunggulan teknologi tradisional dari sudut pandang fisika energi. Gunakan pengetahuan gabungan dari sains (usaha, energi, konversi energi) dan kearifan lokal yang baru saja kamu pelajari."
    \end{itemize}
\item \textbf{Aktivitas Siswa:} Menyusun argumen tertulis untuk menjawab studi kasus yang diberikan, menggunakan bukti dari kedua lensa.
\end{itemize}

\subsection{Kegiatan Penutup (15 Menit)}
\begin{itemize}
\item \textbf{Presentasi \& Penguatan:} Guru meminta 2-3 siswa secara acak untuk membacakan jawaban studi kasus mereka. Guru memberikan pujian dan penguatan positif, menekankan betapa hebatnya argumen yang memadukan sains dan budaya.
\item \textbf{Refleksi:}
    \begin{itemize}
    \item \textbf{Instruksi Guru:} "Sekarang, kembali ke lembar K-W-L kalian. Lengkapi kolom terakhir, \textbf{L (Learned)}, dengan hal-hal baru yang paling berkesan yang kalian pelajari hari ini."
    \item \textbf{Instruksi Guru:} "Angkat tangan, siapa yang setelah belajar hari ini jadi lebih menghargai teknologi ramah lingkungan nenek moyang kita?"
    \end{itemize}
\item \textbf{Tindak Lanjut:}
    \begin{itemize}
    \item \textbf{Instruksi Guru:} "Hebat sekali, para insinyur energi! Hari ini kita sudah membongkar rahasia 'usaha dan energi dalam kincir air'. Pertemuan berikutnya, kita akan menyelidiki bagaimana konsep daya dan efisiensi diterapkan dalam teknologi pengolahan pangan tradisional Minangkabau. Siapa yang pernah melihat proses pembuatan rendang atau penggilingan padi tradisional? Kita akan temukan prinsip fisikanya!"
    \item \textit{(Clue: Kaitkan secara eksplisit dengan pembelajaran hari ini untuk membangun alur narasi yang berkelanjutan menuju pertemuan berikutnya tentang daya dan efisiensi).}
    \end{itemize}
\item Guru menutup pelajaran dengan doa dan salam.
\end{itemize}

\section{Asesmen (Penilaian)}

\begin{itemize}
\item \textbf{Asesmen Diagnostik (Awal):} Analisis lembar K-W-L. (Untuk mengetahui baseline siswa).
\item \textbf{Asesmen Formatif (Proses):} Observasi keaktifan diskusi (gotong royong) dan penilaian kelengkapan "\textbf{LKPD 17 - Jurnal Investigasi Dual-Lensa}".
\item \textbf{Asesmen Sumatif (Akhir Siklus):} Penilaian jawaban studi kasus (Tahap N) menggunakan rubrik.
\end{itemize}

\subsection{Rubrik Penilaian Jawaban Studi Kasus (Tahap N)}

\begin{center}
\begin{tabular}{|p{3cm}|p{3cm}|p{3cm}|p{3cm}|p{3cm}|}
\hline
\textbf{Kriteria Penilaian} & \textbf{Skor 4 (Sangat Baik)} & \textbf{Skor 3 (Baik)} & \textbf{Skor 2 (Cukup)} & \textbf{Skor 1 (Kurang)} \\
\hline
\textbf{Ketepatan Konsep Ilmiah} & Menggunakan istilah ilmiah (usaha, energi kinetik, energi potensial, konversi energi, efisiensi) dengan sangat tepat dan relevan dengan kasus. & Menggunakan istilah ilmiah dengan tepat, namun kurang relevan. & Menggunakan istilah ilmiah namun ada beberapa kesalahan konsep. & Tidak menggunakan istilah ilmiah atau salah total. \\
\hline
\textbf{Keterkaitan dengan Etnosains} & Mampu menghubungkan secara logis dan eksplisit antara teknologi pengairan tradisional dengan penjelasan ilmiahnya secara mendalam. & Mampu menghubungkan teknologi tradisional dengan penjelasan ilmiah, namun kurang mendalam. & Hanya menyebutkan teknologi tradisional tanpa menghubungkan dengan sains, atau sebaliknya. & Tidak ada keterkaitan antara sains dan budaya yang ditunjukkan. \\
\hline
\textbf{Kelogisan \& Struktur Argumen} & Penjelasan sangat logis, runtut, persuasif, dan mudah dipahami. & Penjelasan logis dan runtut, namun kurang persuasif. & Alur penjelasan kurang runtut atau sulit dipahami. & Penjelasan tidak logis dan tidak terstruktur. \\
\hline
\end{tabular}
\end{center}

\vspace{1cm}

\section{Daftar Pustaka Etnosains}
\begin{itemize}
\item Navis, A.A. (1984). \textit{Alam Terkembang Jadi Guru: Adat dan Kebudayaan Minangkabau}. Jakarta: Grafiti Pers.
\item Hakimy, I. (1994). \textit{Rangkaian Mustika Adat Basandi Syarak di Minangkabau}. Bandung: Remaja Rosdakarya.
\item Soekmono, R. (1973). \textit{Pengantar Sejarah Kebudayaan Indonesia}. Yogyakarta: Kanisius.
\item Couto, N. (2008). \textit{Teknologi Pengairan Tradisional Minangkabau}. Padang: Andalas University Press.
\item Amir, A. (2013). \textit{Kearifan Lokal dalam Teknologi Air Minangkabau}. Jakarta: Yayasan Pustaka Obor Indonesia.
\end{itemize}

\end{document}