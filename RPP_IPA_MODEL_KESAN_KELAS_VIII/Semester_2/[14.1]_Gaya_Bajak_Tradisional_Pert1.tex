% !TEX encoding = UTF-8

\documentclass[12pt,a4paper]{article}
\usepackage[margin=1.5cm]{geometry}
\usepackage[utf8]{inputenc}
\usepackage[T1]{fontenc}
\usepackage{times}
\usepackage{amsmath}
\usepackage{amsfonts}
\usepackage{amssymb}
\usepackage{graphicx}
\usepackage{xcolor}
\usepackage{tcolorbox}
\usepackage{enumitem}
\usepackage{multicol}
\usepackage{fancyhdr}
\usepackage{titlesec}

% Setup header and footer
\pagestyle{fancy}
\fancyhf{}
\fancyhead[L]{\textbf{RPP IPA Kelas VIII - Model KESAN}}
\fancyhead[R]{\textbf{Semester 2 - Pertemuan 14.1}}
\fancyfoot[C]{\thepage}

% Title formatting
\titleformat{\section}{\Large\bfseries}{}{0em}{}
\titleformat{\subsection}{\large\bfseries}{}{0em}{}
\titleformat{\subsubsection}{\normalsize\bfseries}{}{0em}{}

\begin{document}

\begin{center}
{\Huge\textbf{MODUL AJAR}}\\
\vspace{0.5cm}
{\Large\textbf{Bajak Tradisional Minang \& Rahasia Gaya dalam Pertanian}}
\end{center}

\vspace{1cm}

\begin{tcolorbox}[colback=white,colframe=black,boxrule=1pt]
\textbf{Nama Penyusun:} Irfan Ananda\\
\textbf{Institusi:} SMP\\
\textbf{Mata Pelajaran:} Ilmu Pengetahuan Alam (IPA)\\
\textbf{Tahun Ajaran:} 2025/2026\\
\textbf{Semester:} Genap\\
\textbf{Jenjang Sekolah:} SMP\\
\textbf{Kelas/Fase:} VIII / D\\
\textbf{Alokasi waktu:} 3 x 40 Menit (1 Pertemuan)
\end{tcolorbox}

\section{DIMENSI PROFIL PELAJAR PANCASILA}
\textit{(Clue untuk Guru: Sebutkan dimensi ini secara eksplisit saat apersepsi agar siswa sadar tujuan non-akademis yang sedang mereka kembangkan).}

\begin{itemize}
\item \textbf{Berkebinekaan Global:} Mengenal dan menghargai budaya, khususnya menganalisis kearifan lokal (etnosains) Minangkabau dalam teknologi pertanian tradisional seperti bajak dan sistem pengolahan tanah, lalu menghubungkannya dengan konsep sains universal tentang gaya dan gerak.
\item \textbf{Bernalar Kritis:} Ditempa secara intensif saat menganalisis informasi dari sumber sains dan budaya (Tahap S), menyintesiskan kedua perspektif (Tahap A), dan menyusun argumen berbasis bukti (Tahap N).
\item \textbf{Gotong Royong:} Kemampuan untuk bekerja sama secara kolaboratif dalam kelompok untuk merumuskan masalah (Tahap E), melakukan investigasi (Tahap S), dan membangun pemahaman bersama (Tahap A).
\item \textbf{Kreatif:} Menghasilkan argumen atau solusi orisinal yang terintegrasi pada tahap akhir pembelajaran (Tahap N).
\end{itemize}

\section{Sarana dan Prasarana}

\begin{itemize}
\item \textbf{Media:} LKPD 16 - Jurnal Investigasi Dual-Lensa, video singkat petani Minangkabau menggunakan bajak tradisional (misal dari YouTube, durasi 2-3 menit), gambar berbagai alat pertanian tradisional, artikel/infografis tentang gaya dan hukum Newton, diagram gaya pada bajak.
\item \textbf{Alat:} Papan tulis/whiteboard, spidol, Proyektor \& Speaker, Kertas Plano atau Karton (1 per kelompok), sticky notes warna-warni, balok kayu kecil, tali, dinamometer sederhana atau neraca pegas, permukaan dengan tekstur berbeda (halus dan kasar).
\item \textbf{Sumber Belajar:} Buku ajar IPA kelas VIII, Tautan video animasi gaya dan gerak (misal: bit.ly/animasi-gaya), tautan artikel teknologi pertanian tradisional Minangkabau (misal: bit.ly/pertanian-minang).
\end{itemize}

\section{Target Peserta didik}

\begin{itemize}
\item Peserta didik reguler kelas VIII (Fase D).
\end{itemize}

\section{Model Pembelajaran}

\begin{itemize}
\item Model Pembelajaran KESAN (Konektivitas Etnosains-Sains).
\end{itemize}

\section{Pemahaman Bermakna}
\textit{(Clue untuk Guru: Bacakan atau sampaikan narasi ini dengan intonasi yang menarik di akhir pembelajaran untuk mengikat semua pengalaman belajar siswa menjadi satu kesatuan yang bermakna).}

"Ananda Semua, hari ini kita telah membuka mata terhadap kearifan nenek moyang Minangkabau dalam bidang pertanian! Kita menemukan bahwa bajak tradisional yang sederhana ternyata merupakan aplikasi brilian dari prinsip-prinsip gaya dan gerak. Setiap sudut mata bajak, setiap bentuk pegangan, semuanya dirancang berdasarkan pemahaman intuitif tentang gaya gesek, gaya normal, dan efisiensi kerja. Petani Minangkabau telah menjadi fisikawan sejati tanpa mereka sadari! Dengan ini, kita sadar bahwa sains tidak hanya ada di buku teks, tapi juga terwujud dalam kearifan lokal yang telah membantu nenek moyang kita bertahan hidup dan berkembang selama berabad-abad."

\section{PERTEMUAN PERTAMA: Gaya dan Gerak dalam Teknologi Pertanian Tradisional Minangkabau}

\subsection{Capaian Pembelajaran (Fase D)}
Pada akhir Fase D, murid memiliki kemampuan [...] menganalisis gerak lurus, gerak melingkar, gaya, dan penerapannya pada gerak makhluk hidup dan gerak benda [...]

\subsection{Tujuan Pembelajaran (TP) Pertemuan 1:}
\textit{(Clue untuk Guru: Tujuan ini adalah kompas Anda. Pastikan setiap tahapan KESAN yang Anda lalui berkontribusi pada pencapaian tujuan-tujuan ini).}

Melalui model pembelajaran KESAN, peserta didik mampu:
\begin{itemize}
\item Menghubungkan fenomena kearifan lokal Minangkabau (penggunaan bajak tradisional dan teknik pengolahan tanah) dengan konteks gaya dan gerak. (Sintaks K)
\item Merumuskan pertanyaan investigatif mengenai cara kerja gaya pada alat pertanian tradisional dan efisiensi kerjanya. (Sintaks E)
\item Mengumpulkan informasi mengenai konsep gaya, hukum Newton, dan gaya gesek dari sumber ilmiah serta teknologi pertanian tradisional Minangkabau dari sumber kultural. (Sintaks S)
\item Menganalisis dan menyintesiskan hubungan sebab-akibat antara konsep ilmiah (gaya, gerak, gaya gesek) dengan desain dan fungsi alat pertanian tradisional. (Sintaks A)
\item Menyusun sebuah penjelasan analitis yang logis mengenai bagaimana prinsip fisika gaya dan gerak diterapkan dalam teknologi pertanian tradisional Minangkabau. (Sintaks N)
\end{itemize}

\subsection{Pertanyaan Pemantik}
\textit{(Clue untuk Guru: Ajukan dua pertanyaan ini secara berurutan, berikan jeda agar siswa berpikir. Jangan langsung minta jawaban, biarkan pertanyaan ini menggantung untuk memicu rasa ingin tahu).}

\begin{itemize}
\item "Pernahkah kalian melihat petani Minangkabau menggunakan bajak tradisional untuk mengolah sawah? Mengapa bajak itu bisa memotong dan membalik tanah dengan mudah, padahal hanya ditarik oleh kerbau?"
\item "Coba perhatikan bentuk mata bajak yang runcing dan melengkung. Mengapa nenek moyang kita mendesain bajak dengan bentuk seperti itu? Adakah hubungannya dengan prinsip fisika?"
\end{itemize}

\section{Langkah-langkah Kegiatan Pembelajaran:}

\subsection{Kegiatan Pembuka (15 Menit)}
\begin{itemize}
\item Guru membuka pelajaran dengan salam, doa, dan memeriksa kehadiran.
\item \textbf{Asesmen Diagnostik Awal:} Guru membagikan lembar K-W-L.
    \begin{itemize}
    \item \textbf{Instruksi Guru:} "Ananda, setelah kita mempelajari cahaya dan spektrum warna, sekarang kita akan mengeksplorasi dunia gaya dan gerak. Di lembar K-W-L ini, tulis di kolom \textbf{K (Tahu)} apa yang sudah kalian ketahui tentang gaya ATAU tentang alat pertanian tradisional. Di kolom \textbf{W (Ingin Tahu)}, tulis apa yang membuat kalian penasaran tentang topik ini." \textit{(Clue: Ini membantu Anda melihat koneksi dengan pembelajaran sebelumnya dan pengalaman siswa dengan pertanian).}
    \end{itemize}
\item \textbf{Apersepsi:}
    \begin{itemize}
    \item \textbf{Instruksi Guru:} "Hari ini kita akan menjadi fisikawan pertanian! Kita akan menyelidiki bagaimana gaya bekerja dalam teknologi pertanian tradisional dan bagaimana nenek moyang kita menerapkan prinsip fisika tanpa mereka sadari. Dalam investigasi ini, kita akan melatih kemampuan \textbf{Bernalar Kritis} kita, menghargai budaya lewat \textbf{Berkebinekaan Global}, dan bekerja sama dalam semangat \textbf{Gotong Royong}. Siap?"
    \end{itemize}
\end{itemize}

\subsection{Kegiatan Inti (90 Menit) - SINTAKS MODEL KESAN}

\subsubsection{Tahap 1: (K) Kaitkan Konteks Kultural (15 menit)}
\begin{itemize}
\item \textbf{Aktivitas Guru:}
    \begin{itemize}
    \item Menampilkan video singkat (2-3 menit) petani Minangkabau menggunakan bajak tradisional di sawah.
    \item Melakukan demonstrasi sederhana: menarik balok kayu di atas permukaan halus dan kasar untuk menunjukkan perbedaan gaya gesek.
    \item Mengajukan Pertanyaan Pemantik yang sudah disiapkan di atas.
    \item \textit{(Clue: Tujuan tahap ini adalah memvalidasi pengetahuan siswa tentang pertanian dan memantik rasa heran tentang prinsip fisika di baliknya. Biarkan siswa mengamati demonstrasi gaya gesek secara langsung. Tuliskan semua ide mereka dengan judul "\textbf{KEARIFAN PETANI KITA}").}
    \end{itemize}
\item \textbf{Aktivitas Siswa:} Mengamati demonstrasi dan video, mendengarkan pertanyaan, lalu secara sukarela berbagi pengalaman atau dugaan awal. Menuliskan minimal satu pertanyaan atau pengamatan di sticky notes dan menempelkannya di '\textbf{Papan Penasaran}'.
\end{itemize}

\subsubsection{Tahap 2: (E) Eksplorasi Enigma (15 menit)}
\begin{itemize}
\item \textbf{Aktivitas Guru:} Membentuk siswa menjadi kelompok (3-4 orang).
    \begin{itemize}
    \item \textbf{Instruksi Guru:} "Fenomena yang kalian amati tadi sangat menarik! Sekarang, tugas kita sebagai fisikawan pertanian adalah mengubah rasa penasaran ini menjadi misi penelitian yang jelas. Dalam kelompok, diskusikan dan rumuskan minimal 3 pertanyaan kunci yang akan kita selidiki hari ini. Tuliskan dalam bentuk '\textbf{Misi Penyelidikan Tim [Nama Kelompok]}' di kertas plano yang Bapak/Ibu berikan."
    \item \textit{(Clue: Arahkan diskusi siswa agar pertanyaannya mencakup aspek 'bagaimana gaya bekerja' dan 'mengapa desain alat seperti itu'. Jika kelompok kesulitan, berikan pancingan: "Kira-kira, mengapa bajak bisa memotong tanah dengan mudah? Dan bagaimana gaya gesek mempengaruhi kerja bajak?").}
    \end{itemize}
\item \textbf{Aktivitas Siswa:} Berdiskusi dalam tim untuk merumuskan misi penyelidikan (daftar pertanyaan kunci) di kertas plano. \textit{(Contoh misi yang diharapkan: 1. Bagaimana gaya bekerja pada bajak tradisional? 2. Mengapa bentuk mata bajak runcing dan melengkung? 3. Bagaimana gaya gesek mempengaruhi efisiensi bajak?)}.
\end{itemize}

\subsubsection{Tahap 3: (S) Selidiki secara Sintetis (25 menit)}
\begin{itemize}
\item \textbf{Aktivitas Guru:} Membagikan "\textbf{LKPD 16 - Jurnal Investigasi Dual-Lensa}".
    \begin{itemize}
    \item \textbf{Instruksi Guru:} "Setiap tim akan melakukan investigasi dari dua lensa. Gunakan HP atau sumber yang disediakan untuk mencari jawabannya. Bagilah tugas dalam tim!"
    \item \textbf{Lensa Sains:} Buka link video bit.ly/animasi-gaya untuk memahami konsep gaya, hukum Newton, gaya gesek, dan aplikasinya. Lakukan juga eksperimen sederhana dengan balok dan dinamometer.
    \item \textbf{Lensa Etnosains/Kultural:} Buka link artikel bit.ly/pertanian-minang untuk memahami teknologi pertanian tradisional Minangkabau, desain alat, dan filosofi kerja dalam budaya Minang.
    \item \textit{(Clue: Pastikan sumber belajar sudah disiapkan dan alat eksperimen tersedia. Berkelilinglah untuk memastikan setiap kelompok melakukan eksperimen dengan benar dan mengamati perbedaan gaya pada permukaan yang berbeda).}
    \end{itemize}
\item \textbf{Aktivitas Siswa:} Dalam kelompok, siswa berbagi tugas mencari informasi dari sumber yang diberikan dan mencatat temuan kunci di dua kolom terpisah pada "\textbf{LKPD 16 - Jurnal Investigasi Dual-Lensa}".
\end{itemize}

\subsubsection{Tahap 4: (A) Asimilasi Analitis (20 menit)}
\begin{itemize}
\item \textbf{Aktivitas Guru:} Memfasilitasi diskusi untuk menjembatani kedua lensa.
    \begin{itemize}
    \item \textbf{Pertanyaan Pancingan Kunci untuk Guru:}
        \begin{itemize}
        \item "Oke, dari Lensa Sains kita tahu gaya gesek mempengaruhi gerak benda. Dari Lensa Etnosains, kita tahu bajak dirancang dengan bentuk tertentu. Nah, coba hubungkan! Bagaimana desain bajak meminimalkan gaya gesek yang tidak perlu?"
        \item "Dari Lensa Sains, kita tahu hukum Newton tentang aksi-reaksi. Dari Lensa Budaya, petani Minang menggunakan kerbau untuk menarik bajak. Apa hubungannya dengan efisiensi kerja?"
        \end{itemize}
    \item \textit{(Clue: Fokuskan untuk membuat siswa 'menemukan' hubungannya sendiri. Gunakan kata "menurut kalian", "kira-kira bagaimana", "ada yang bisa menjelaskan?").}
    \end{itemize}
\item \textbf{Aktivitas Siswa:} Berdiskusi intensif untuk menghubungkan temuan sains dan budaya. Menuliskan kesimpulan terpadu (sintesis) mereka di kertas plano.
\end{itemize}

\subsubsection{Tahap 5: (N) Nyatakan Pemahaman (15 menit)}
\begin{itemize}
\item \textbf{Aktivitas Guru:} Memberikan studi kasus individual atau per kelompok.
    \begin{itemize}
    \item \textbf{Instruksi Guru (tuliskan di papan tulis):}
    
    "\textbf{STUDI KASUS UNTUK FISIKAWAN PERTANIAN:}
    
    Seorang petani muda di Minangkabau ingin memodernisasi alat pertaniannya namun tetap mempertahankan prinsip-prinsip tradisional yang efisien. Dia meminta bantuanmu untuk menjelaskan mengapa bajak tradisional Minangkabau sangat efisien dari sudut pandang fisika.
    
    Tugasmu: Tuliskan sebuah penjelasan singkat (3-5 kalimat) di buku latihanmu untuk membantu petani tersebut memahami prinsip fisika di balik efisiensi bajak tradisional. Gunakan pengetahuan gabungan dari sains (gaya, gaya gesek, hukum Newton) dan kearifan lokal yang baru saja kamu pelajari."
    \end{itemize}
\item \textbf{Aktivitas Siswa:} Menyusun argumen tertulis untuk menjawab studi kasus yang diberikan, menggunakan bukti dari kedua lensa.
\end{itemize}

\subsection{Kegiatan Penutup (15 Menit)}
\begin{itemize}
\item \textbf{Presentasi \& Penguatan:} Guru meminta 2-3 siswa secara acak untuk membacakan jawaban studi kasus mereka. Guru memberikan pujian dan penguatan positif, menekankan betapa hebatnya argumen yang memadukan sains dan budaya.
\item \textbf{Refleksi:}
    \begin{itemize}
    \item \textbf{Instruksi Guru:} "Sekarang, kembali ke lembar K-W-L kalian. Lengkapi kolom terakhir, \textbf{L (Learned)}, dengan hal-hal baru yang paling berkesan yang kalian pelajari hari ini."
    \item \textbf{Instruksi Guru:} "Angkat tangan, siapa yang setelah belajar hari ini jadi lebih menghargai kearifan petani tradisional?"
    \end{itemize}
\item \textbf{Tindak Lanjut:}
    \begin{itemize}
    \item \textbf{Instruksi Guru:} "Luar biasa, para fisikawan pertanian! Hari ini kita sudah membongkar rahasia 'gaya dalam bajak tradisional'. Pertemuan berikutnya, kita akan menyelidiki bagaimana konsep usaha dan energi diterapkan dalam teknologi pengairan tradisional Minangkabau. Siapa yang pernah melihat kincir air atau sistem irigasi tradisional? Kita akan temukan prinsip fisikanya!"
    \item \textit{(Clue: Kaitkan secara eksplisit dengan pembelajaran hari ini untuk membangun alur narasi yang berkelanjutan menuju pertemuan berikutnya tentang usaha dan energi).}
    \end{itemize}
\item Guru menutup pelajaran dengan doa dan salam.
\end{itemize}

\section{Asesmen (Penilaian)}

\begin{itemize}
\item \textbf{Asesmen Diagnostik (Awal):} Analisis lembar K-W-L. (Untuk mengetahui baseline siswa).
\item \textbf{Asesmen Formatif (Proses):} Observasi keaktifan diskusi (gotong royong) dan penilaian kelengkapan "\textbf{LKPD 16 - Jurnal Investigasi Dual-Lensa}".
\item \textbf{Asesmen Sumatif (Akhir Siklus):} Penilaian jawaban studi kasus (Tahap N) menggunakan rubrik.
\end{itemize}

\subsection{Rubrik Penilaian Jawaban Studi Kasus (Tahap N)}

\begin{center}
\begin{tabular}{|p{3cm}|p{3cm}|p{3cm}|p{3cm}|p{3cm}|}
\hline
\textbf{Kriteria Penilaian} & \textbf{Skor 4 (Sangat Baik)} & \textbf{Skor 3 (Baik)} & \textbf{Skor 2 (Cukup)} & \textbf{Skor 1 (Kurang)} \\
\hline
\textbf{Ketepatan Konsep Ilmiah} & Menggunakan istilah ilmiah (gaya, gaya gesek, hukum Newton, efisiensi) dengan sangat tepat dan relevan dengan kasus. & Menggunakan istilah ilmiah dengan tepat, namun kurang relevan. & Menggunakan istilah ilmiah namun ada beberapa kesalahan konsep. & Tidak menggunakan istilah ilmiah atau salah total. \\
\hline
\textbf{Keterkaitan dengan Etnosains} & Mampu menghubungkan secara logis dan eksplisit antara teknologi pertanian tradisional dengan penjelasan ilmiahnya secara mendalam. & Mampu menghubungkan teknologi tradisional dengan penjelasan ilmiah, namun kurang mendalam. & Hanya menyebutkan teknologi tradisional tanpa menghubungkan dengan sains, atau sebaliknya. & Tidak ada keterkaitan antara sains dan budaya yang ditunjukkan. \\
\hline
\textbf{Kelogisan \& Struktur Argumen} & Penjelasan sangat logis, runtut, persuasif, dan mudah dipahami. & Penjelasan logis dan runtut, namun kurang persuasif. & Alur penjelasan kurang runtut atau sulit dipahami. & Penjelasan tidak logis dan tidak terstruktur. \\
\hline
\end{tabular}
\end{center}

\vspace{1cm}

\section{Daftar Pustaka Etnosains}
\begin{itemize}
\item Navis, A.A. (1984). \textit{Alam Terkembang Jadi Guru: Adat dan Kebudayaan Minangkabau}. Jakarta: Grafiti Pers.
\item Hakimy, I. (1994). \textit{Rangkaian Mustika Adat Basandi Syarak di Minangkabau}. Bandung: Remaja Rosdakarya.
\item Soekmono, R. (1973). \textit{Pengantar Sejarah Kebudayaan Indonesia}. Yogyakarta: Kanisius.
\item Couto, N. (2008). \textit{Teknologi Pertanian Tradisional Minangkabau}. Padang: Andalas University Press.
\item Amir, A. (2013). \textit{Kearifan Lokal dalam Pertanian Minangkabau}. Jakarta: Yayasan Pustaka Obor Indonesia.
\end{itemize}

\end{document}