% Dokumen LaTeX untuk kompilasi dengan pdfLaTeX

\documentclass[12pt,a4paper]{article}
\usepackage[margin=1.5cm]{geometry}
\usepackage[utf8]{inputenc}
\usepackage[T1]{fontenc}
\usepackage{times}
\usepackage{amsmath}
\usepackage{amsfonts}
\usepackage{amssymb}
\usepackage{graphicx}
\usepackage{xcolor}
\usepackage{tcolorbox}
\usepackage{enumitem}
\usepackage{multicol}
\usepackage{fancyhdr}
\usepackage{titlesec}

% Setup header and footer
\pagestyle{fancy}
\fancyhf{}
\fancyhead[L]{\textbf{RPP IPA Kelas VIII - Model KESAN}}
\fancyhead[R]{\textbf{Semester 2 - Pertemuan 13.3}}
\fancyfoot[C]{\thepage}

% Title formatting
\titleformat{\section}{\Large\bfseries}{}{0em}{}
\titleformat{\subsection}{\large\bfseries}{}{0em}{}
\titleformat{\subsubsection}{\normalsize\bfseries}{}{0em}{}

\begin{document}

\begin{center}
{\Huge\textbf{MODUL AJAR}}\\
\vspace{0.5cm}
{\Large\textbf{Pelangi Danau Maninjau \& Spektrum Cahaya Tersembunyi}}
\end{center}

\vspace{1cm}

\begin{tcolorbox}[colback=white,colframe=black,boxrule=1pt]
\textbf{Nama Penyusun:} Irfan Ananda\\
\textbf{Institusi:} SMP\\
\textbf{Mata Pelajaran:} Ilmu Pengetahuan Alam (IPA)\\
\textbf{Tahun Ajaran:} 2025/2026\\
\textbf{Semester:} Genap\\
\textbf{Jenjang Sekolah:} SMP\\
\textbf{Kelas/Fase:} VIII / D\\
\textbf{Alokasi waktu:} 3 x 40 Menit (1 Pertemuan)
\end{tcolorbox}

\section{DIMENSI PROFIL PELAJAR PANCASILA}
\textit{(Clue untuk Guru: Sebutkan dimensi ini secara eksplisit saat apersepsi agar siswa sadar tujuan non-akademis yang sedang mereka kembangkan).}

\begin{itemize}
\item \textbf{Berkebinekaan Global:} Mengenal dan menghargai budaya, khususnya menganalisis kearifan lokal (etnosains) Minangkabau dalam memahami fenomena alam seperti pelangi dan penggunaan warna dalam seni tradisional, lalu menghubungkannya dengan konsep sains universal tentang dispersi cahaya.
\item \textbf{Bernalar Kritis:} Ditempa secara intensif saat menganalisis informasi dari sumber sains dan budaya (Tahap S), menyintesiskan kedua perspektif (Tahap A), dan menyusun argumen berbasis bukti (Tahap N).
\item \textbf{Gotong Royong:} Kemampuan untuk bekerja sama secara kolaboratif dalam kelompok untuk merumuskan masalah (Tahap E), melakukan investigasi (Tahap S), dan membangun pemahaman bersama (Tahap A).
\item \textbf{Kreatif:} Menghasilkan argumen atau solusi orisinal yang terintegrasi pada tahap akhir pembelajaran (Tahap N).
\end{itemize}

\section{Sarana dan Prasarana}

\begin{itemize}
\item \textbf{Media:} LKPD 15 - Jurnal Investigasi Dual-Lensa, video singkat keindahan Danau Maninjau dan fenomena pelangi (misal dari YouTube, durasi 1-2 menit), gambar songket dengan berbagai warna, artikel/infografis tentang dispersi cahaya dan spektrum elektromagnetik, diagram pembentukan pelangi.
\item \textbf{Alat:} Papan tulis/whiteboard, spidol, Proyektor \& Speaker, Kertas Plano atau Karton (1 per kelompok), sticky notes warna-warni, prisma kaca atau CD/DVD bekas, senter/laser pointer, semprotan air halus.
\item \textbf{Sumber Belajar:} Buku ajar IPA kelas VIII, Tautan video animasi dispersi cahaya (misal: bit.ly/animasi-dispersi), tautan artikel filosofi warna dalam budaya Minangkabau (misal: bit.ly/warna-minang).
\end{itemize}

\section{Target Peserta didik}

\begin{itemize}
\item Peserta didik reguler kelas VIII (Fase D).
\end{itemize}

\section{Model Pembelajaran}

\begin{itemize}
\item Model Pembelajaran KESAN (Konektivitas Etnosains-Sains).
\end{itemize}

\section{Pemahaman Bermakna}
\textit{(Clue untuk Guru: Bacakan atau sampaikan narasi ini dengan intonasi yang menarik di akhir pembelajaran untuk mengikat semua pengalaman belajar siswa menjadi satu kesatuan yang bermakna).}

"Ananda Semua, hari ini kita telah mengungkap rahasia di balik keindahan pelangi yang memukau di atas Danau Maninjau! Kita menemukan bahwa cahaya putih yang tampak sederhana sebenarnya menyimpan spektrum warna yang menakjubkan. Nenek moyang Minangkabau telah memahami keindahan ini dan mengaplikasikannya dalam seni songket dengan perpaduan warna yang harmonis. Setiap benang emas, setiap gradasi warna dalam songket, semuanya mencerminkan pemahaman intuitif tentang spektrum cahaya. Dengan ini, kita sadar bahwa sains tentang cahaya tidak hanya ada di laboratorium, tapi juga terwujud dalam keindahan alam dan karya seni leluhur kita."

\section{PERTEMUAN KETIGA: Dispersi Cahaya dan Spektrum Warna dalam Budaya Minangkabau}

\subsection{Capaian Pembelajaran (Fase D)}
Pada akhir Fase D, murid memiliki kemampuan [...] menganalisis gelombang dan pemanfaatannya dalam kehidupan sehari-hari [...]

\subsection{Tujuan Pembelajaran (TP) Pertemuan 3:}
\textit{(Clue untuk Guru: Tujuan ini adalah kompas Anda. Pastikan setiap tahapan KESAN yang Anda lalui berkontribusi pada pencapaian tujuan-tujuan ini).}

Melalui model pembelajaran KESAN, peserta didik mampu:
\begin{itemize}
\item Menghubungkan fenomena kearifan lokal Minangkabau (keindahan pelangi di Danau Maninjau dan penggunaan warna dalam songket) dengan konteks dispersi cahaya. (Sintaks K)
\item Merumuskan pertanyaan investigatif mengenai cara kerja dispersi cahaya dan pembentukan spektrum warna. (Sintaks E)
\item Mengumpulkan informasi mengenai dispersi cahaya dan spektrum elektromagnetik dari sumber ilmiah serta filosofi warna dalam budaya Minangkabau dari sumber kultural. (Sintaks S)
\item Menganalisis dan menyintesiskan hubungan sebab-akibat antara konsep ilmiah (dispersi, spektrum, panjang gelombang) dengan praktik budaya (pemilihan warna dalam seni tradisional). (Sintaks A)
\item Menyusun sebuah penjelasan analitis yang logis mengenai bagaimana kearifan lokal dalam seni warna Minangkabau dapat dibenarkan secara saintifik untuk memahami spektrum cahaya. (Sintaks N)
\end{itemize}

\subsection{Pertanyaan Pemantik}
\textit{(Clue untuk Guru: Ajukan dua pertanyaan ini secara berurutan, berikan jeda agar siswa berpikir. Jangan langsung minta jawaban, biarkan pertanyaan ini menggantung untuk memicu rasa ingin tahu).}

\begin{itemize}
\item "Pernahkah kalian melihat pelangi yang indah di atas Danau Maninjau setelah hujan? Dari mana sebenarnya warna-warna cantik itu berasal, padahal cahaya matahari tampak putih?"
\item "Coba perhatikan songket Minangkabau dengan perpaduan warna emasnya yang memukau. Mengapa nenek moyang kita memilih kombinasi warna tertentu yang selalu tampak harmonis dan indah di mata? Adakah hubungannya dengan sains?"
\end{itemize}

\section{Langkah-langkah Kegiatan Pembelajaran:}

\subsection{Kegiatan Pembuka (15 Menit)}
\begin{itemize}
\item Guru membuka pelajaran dengan salam, doa, dan memeriksa kehadiran.
\item \textbf{Asesmen Diagnostik Awal:} Guru membagikan lembar K-W-L.
    \begin{itemize}
    \item \textbf{Instruksi Guru:} "Ananda, setelah kita mempelajari pemantulan dan pembiasan cahaya, sekarang kita akan mengeksplorasi rahasia warna. Di lembar K-W-L ini, tulis di kolom \textbf{K (Tahu)} apa yang sudah kalian ketahui tentang pelangi ATAU tentang warna dalam songket. Di kolom \textbf{W (Ingin Tahu)}, tulis apa yang membuat kalian penasaran tentang topik ini." \textit{(Clue: Ini membantu Anda melihat koneksi dengan pembelajaran sebelumnya dan apresiasi siswa terhadap seni tradisional).}
    \end{itemize}
\item \textbf{Apersepsi:}
    \begin{itemize}
    \item \textbf{Instruksi Guru:} "Hari ini kita akan menjadi seniman sains dan budaya. Kita akan menyelidiki bagaimana cahaya putih menyimpan spektrum warna yang menakjubkan dan bagaimana nenek moyang kita memahami keindahan ini. Dalam investigasi ini, kita akan melatih kemampuan \textbf{Bernalar Kritis} kita, menghargai budaya lewat \textbf{Berkebinekaan Global}, dan bekerja sama dalam semangat \textbf{Gotong Royong}. Siap?"
    \end{itemize}
\end{itemize}

\subsection{Kegiatan Inti (90 Menit) - SINTAKS MODEL KESAN}

\subsubsection{Tahap 1: (K) Kaitkan Konteks Kultural (15 menit)}
\begin{itemize}
\item \textbf{Aktivitas Guru:}
    \begin{itemize}
    \item Menampilkan video singkat (1-2 menit) keindahan Danau Maninjau dengan fenomena pelangi dan songket Minangkabau dengan berbagai warna.
    \item Melakukan demonstrasi sederhana: menyinari prisma atau CD dengan senter untuk menghasilkan spektrum warna.
    \item Mengajukan Pertanyaan Pemantik yang sudah disiapkan di atas.
    \item \textit{(Clue: Tujuan tahap ini adalah memvalidasi pengetahuan siswa dan memantik rasa heran. Biarkan siswa mengamati fenomena dispersi secara langsung. Tuliskan semua ide mereka dengan judul "\textbf{KEAJAIBAN WARNA KITA}").}
    \end{itemize}
\item \textbf{Aktivitas Siswa:} Mengamati demonstrasi dan video, mendengarkan pertanyaan, lalu secara sukarela berbagi pengalaman atau dugaan awal. Menuliskan minimal satu pertanyaan atau pengamatan di sticky notes dan menempelkannya di '\textbf{Papan Penasaran}'.
\end{itemize}

\subsubsection{Tahap 2: (E) Eksplorasi Enigma (15 menit)}
\begin{itemize}
\item \textbf{Aktivitas Guru:} Membentuk siswa menjadi kelompok (3-4 orang).
    \begin{itemize}
    \item \textbf{Instruksi Guru:} "Fenomena warna yang kalian amati tadi sangat menakjubkan! Sekarang, tugas kita sebagai seniman sains adalah mengubah rasa takjub ini menjadi misi penelitian yang jelas. Dalam kelompok, diskusikan dan rumuskan minimal 3 pertanyaan kunci yang akan kita selidiki hari ini. Tuliskan dalam bentuk '\textbf{Misi Penyelidikan Tim [Nama Kelompok]}' di kertas plano yang Bapak/Ibu berikan."
    \item \textit{(Clue: Arahkan diskusi siswa agar pertanyaannya mencakup aspek 'bagaimana warna terbentuk' dan 'bagaimana budaya memanfaatkan warna'. Jika kelompok kesulitan, berikan pancingan: "Kira-kira, mengapa cahaya putih bisa menghasilkan banyak warna? Dan bagaimana nenek moyang kita memilih warna untuk songket?").}
    \end{itemize}
\item \textbf{Aktivitas Siswa:} Berdiskusi dalam tim untuk merumuskan misi penyelidikan (daftar pertanyaan kunci) di kertas plano. \textit{(Contoh misi yang diharapkan: 1. Bagaimana cahaya putih bisa terurai menjadi spektrum warna? 2. Apa hubungan antara panjang gelombang dan warna? 3. Bagaimana filosofi warna dalam songket Minangkabau?)}.
\end{itemize}

\subsubsection{Tahap 3: (S) Selidiki secara Sintetis (25 menit)}
\begin{itemize}
\item \textbf{Aktivitas Guru:} Membagikan "\textbf{LKPD 15 - Jurnal Investigasi Dual-Lensa}".
    \begin{itemize}
    \item \textbf{Instruksi Guru:} "Setiap tim akan melakukan investigasi dari dua lensa. Gunakan HP atau sumber yang disediakan untuk mencari jawabannya. Bagilah tugas dalam tim!"
    \item \textbf{Lensa Sains:} Buka link video bit.ly/animasi-dispersi untuk memahami dispersi cahaya, spektrum elektromagnetik, dan hubungan panjang gelombang dengan warna. Lakukan juga eksperimen dengan prisma atau CD.
    \item \textbf{Lensa Etnosains/Kultural:} Buka link artikel bit.ly/warna-minang untuk memahami filosofi dan makna warna dalam songket serta hubungannya dengan alam Minangkabau.
    \item \textit{(Clue: Pastikan sumber belajar sudah disiapkan dan alat eksperimen tersedia. Berkelilinglah untuk memastikan setiap kelompok melakukan eksperimen dengan benar dan mengamati spektrum warna yang terbentuk).}
    \end{itemize}
\item \textbf{Aktivitas Siswa:} Dalam kelompok, siswa berbagi tugas mencari informasi dari sumber yang diberikan dan mencatat temuan kunci di dua kolom terpisah pada "\textbf{LKPD 15 - Jurnal Investigasi Dual-Lensa}".
\end{itemize}

\subsubsection{Tahap 4: (A) Asimilasi Analitis (20 menit)}
\begin{itemize}
\item \textbf{Aktivitas Guru:} Memfasilitasi diskusi untuk menjembatani kedua lensa.
    \begin{itemize}
    \item \textbf{Pertanyaan Pancingan Kunci untuk Guru:}
        \begin{itemize}
        \item "Oke, dari Lensa Sains kita tahu setiap warna memiliki panjang gelombang berbeda. Dari Lensa Etnosains, kita tahu songket menggunakan kombinasi warna tertentu yang harmonis. Nah, coba hubungkan! Bagaimana pemahaman tentang spektrum warna membantu menciptakan harmoni dalam seni?"
        \item "Dari Lensa Sains, kita tahu pelangi terbentuk karena dispersi cahaya oleh tetesan air. Dari Lensa Budaya, Danau Maninjau sering dijadikan inspirasi dalam seni dan sastra Minang. Apa hubungannya dengan apresiasi terhadap fenomena alam?"
        \end{itemize}
    \item \textit{(Clue: Fokuskan untuk membuat siswa 'menemukan' hubungannya sendiri. Gunakan kata "menurut kalian", "kira-kira bagaimana", "ada yang bisa menjelaskan?").}
    \end{itemize}
\item \textbf{Aktivitas Siswa:} Berdiskusi intensif untuk menghubungkan temuan sains dan budaya. Menuliskan kesimpulan terpadu (sintesis) mereka di kertas plano.
\end{itemize}

\subsubsection{Tahap 5: (N) Nyatakan Pemahaman (15 menit)}
\begin{itemize}
\item \textbf{Aktivitas Guru:} Memberikan studi kasus individual atau per kelompok.
    \begin{itemize}
    \item \textbf{Instruksi Guru (tuliskan di papan tulis):}
    
    "\textbf{STUDI KASUS UNTUK SENIMAN SAINS-BUDAYA:}
    
    Sekolahmu akan mengadakan pameran seni tradisional dan kamu ditugaskan untuk membuat display tentang songket Minangkabau. Kamu ingin menjelaskan kepada pengunjung mengapa kombinasi warna dalam songket selalu tampak indah dan harmonis, dengan menggunakan pendekatan sains.
    
    Tugasmu: Tuliskan sebuah penjelasan singkat (3-5 kalimat) di buku latihanmu untuk display pameran tersebut. Gunakan pengetahuan gabungan dari sains (dispersi cahaya, spektrum warna, panjang gelombang) dan kearifan lokal yang baru saja kamu pelajari untuk menjelaskan keindahan harmoni warna dalam songket."
    \end{itemize}
\item \textbf{Aktivitas Siswa:} Menyusun argumen tertulis untuk menjawab studi kasus yang diberikan, menggunakan bukti dari kedua lensa.
\end{itemize}

\subsection{Kegiatan Penutup (15 Menit)}
\begin{itemize}
\item \textbf{Presentasi \& Penguatan:} Guru meminta 2-3 siswa secara acak untuk membacakan jawaban studi kasus mereka. Guru memberikan pujian dan penguatan positif, menekankan betapa hebatnya argumen yang memadukan sains dan budaya.
\item \textbf{Refleksi:}
    \begin{itemize}
    \item \textbf{Instruksi Guru:} "Sekarang, kembali ke lembar K-W-L kalian. Lengkapi kolom terakhir, \textbf{L (Learned)}, dengan hal-hal baru yang paling berkesan yang kalian pelajari hari ini."
    \item \textbf{Instruksi Guru:} "Angkat tangan, siapa yang setelah belajar hari ini jadi lebih menghargai keindahan pelangi dan songket?"
    \end{itemize}
\item \textbf{Tindak Lanjut:}
    \begin{itemize}
    \item \textbf{Instruksi Guru:} "Fantastis, para seniman sains! Hari ini kita sudah membongkar rahasia 'spektrum warna tersembunyi'. Pertemuan berikutnya, kita akan menyelidiki bagaimana gaya dan usaha bekerja dalam teknologi pertanian tradisional Minangkabau. Siapa yang pernah melihat kincir air atau bajak tradisional? Kita akan temukan prinsip fisikanya!"
    \item \textit{(Clue: Kaitkan secara eksplisit dengan pembelajaran hari ini untuk membangun alur narasi yang berkelanjutan menuju topik berikutnya tentang gaya dan usaha).}
    \end{itemize}
\item Guru menutup pelajaran dengan doa dan salam.
\end{itemize}

\section{Asesmen (Penilaian)}

\begin{itemize}
\item \textbf{Asesmen Diagnostik (Awal):} Analisis lembar K-W-L. (Untuk mengetahui baseline siswa).
\item \textbf{Asesmen Formatif (Proses):} Observasi keaktifan diskusi (gotong royong) dan penilaian kelengkapan "\textbf{LKPD 15 - Jurnal Investigasi Dual-Lensa}".
\item \textbf{Asesmen Sumatif (Akhir Siklus):} Penilaian jawaban studi kasus (Tahap N) menggunakan rubrik.
\end{itemize}

\subsection{Rubrik Penilaian Jawaban Studi Kasus (Tahap N)}

\begin{center}
\begin{tabular}{|p{3cm}|p{3cm}|p{3cm}|p{3cm}|p{3cm}|}
\hline
\textbf{Kriteria Penilaian} & \textbf{Skor 4 (Sangat Baik)} & \textbf{Skor 3 (Baik)} & \textbf{Skor 2 (Cukup)} & \textbf{Skor 1 (Kurang)} \\
\hline
\textbf{Ketepatan Konsep Ilmiah} & Menggunakan istilah ilmiah (dispersi, spektrum, panjang gelombang, frekuensi) dengan sangat tepat dan relevan dengan kasus. & Menggunakan istilah ilmiah dengan tepat, namun kurang relevan. & Menggunakan istilah ilmiah namun ada beberapa kesalahan konsep. & Tidak menggunakan istilah ilmiah atau salah total. \\
\hline
\textbf{Keterkaitan dengan Etnosains} & Mampu menghubungkan secara logis dan eksplisit antara praktik budaya (seni warna songket) dengan penjelasan ilmiahnya secara mendalam. & Mampu menghubungkan praktik budaya dengan penjelasan ilmiah, namun kurang mendalam. & Hanya menyebutkan praktik budaya tanpa menghubungkan dengan sains, atau sebaliknya. & Tidak ada keterkaitan antara sains dan budaya yang ditunjukkan. \\
\hline
\textbf{Kelogisan \& Struktur Argumen} & Penjelasan sangat logis, runtut, persuasif, dan mudah dipahami. & Penjelasan logis dan runtut, namun kurang persuasif. & Alur penjelasan kurang runtut atau sulit dipahami. & Penjelasan tidak logis dan tidak terstruktur. \\
\hline
\end{tabular}
\end{center}

\vspace{1cm}

\section{Daftar Pustaka Etnosains}
\begin{itemize}
\item Navis, A.A. (1984). \textit{Alam Terkembang Jadi Guru: Adat dan Kebudayaan Minangkabau}. Jakarta: Grafiti Pers.
\item Hakimy, I. (1994). \textit{Rangkaian Mustika Adat Basandi Syarak di Minangkabau}. Bandung: Remaja Rosdakarya.
\item Soekmono, R. (1973). \textit{Pengantar Sejarah Kebudayaan Indonesia}. Yogyakarta: Kanisius.
\item Couto, N. (2008). \textit{Budaya Visual dan Estetika Warna Minangkabau}. Padang: Andalas University Press.
\item Amir, A. (2013). \textit{Sastra Lisan Minangkabau}. Jakarta: Yayasan Pustaka Obor Indonesia.
\end{itemize}

\end{document}