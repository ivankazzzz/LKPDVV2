% Dokumen LaTeX untuk kompilasi dengan pdfLaTeX

\documentclass[12pt,a4paper]{article}
\usepackage[margin=1.5cm]{geometry}
\usepackage[utf8]{inputenc}
\usepackage[T1]{fontenc}
\usepackage{times}
\usepackage{amsmath}
\usepackage{amsfonts}
\usepackage{amssymb}
\usepackage{graphicx}
\usepackage{xcolor}
\usepackage{tcolorbox}
\usepackage{enumitem}
\usepackage{multicol}
\usepackage{fancyhdr}
\usepackage{titlesec}

% Setup header and footer
\pagestyle{fancy}
\fancyhf{}
\fancyhead[L]{\textbf{RPP IPA Kelas VIII - Model KESAN}}
\fancyhead[R]{\textbf{Semester 2 - Pertemuan 13.1}}
\fancyfoot[C]{\thepage}

% Title formatting
\titleformat{\section}{\Large\bfseries}{}{0em}{}
\titleformat{\subsection}{\large\bfseries}{}{0em}{}
\titleformat{\subsubsection}{\normalsize\bfseries}{}{0em}{}

\begin{document}

\begin{center}
{\Huge\textbf{MODUL AJAR}}\\
\vspace{0.5cm}
{\Large\textbf{Rahasia Cermin Rumah Gadang \& Perjalanan Cahaya di Sekitar Kita}}
\end{center}

\vspace{1cm}

\begin{tcolorbox}[colback=white,colframe=black,boxrule=1pt]
\textbf{Nama Penyusun:} Irfan Ananda\\
\textbf{Institusi:} SMP\\
\textbf{Mata Pelajaran:} Ilmu Pengetahuan Alam (IPA)\\
\textbf{Tahun Ajaran:} 2025/2026\\
\textbf{Semester:} Genap\\
\textbf{Jenjang Sekolah:} SMP\\
\textbf{Kelas/Fase:} VIII / D\\
\textbf{Alokasi waktu:} 3 x 40 Menit (1 Pertemuan)
\end{tcolorbox}

\section{DIMENSI PROFIL PELAJAR PANCASILA}
\textit{(Clue untuk Guru: Sebutkan dimensi ini secara eksplisit saat apersepsi agar siswa sadar tujuan non-akademis yang sedang mereka kembangkan).}

\begin{itemize}
\item \textbf{Berkebinekaan Global:} Mengenal dan menghargai budaya, khususnya menganalisis kearifan lokal (etnosains) Minangkabau dalam arsitektur rumah gadang dan penggunaan cermin tradisional, lalu menghubungkannya dengan konsep sains universal tentang cahaya.
\item \textbf{Bernalar Kritis:} Ditempa secara intensif saat menganalisis informasi dari sumber sains dan budaya (Tahap S), menyintesiskan kedua perspektif (Tahap A), dan menyusun argumen berbasis bukti (Tahap N).
\item \textbf{Gotong Royong:} Kemampuan untuk bekerja sama secara kolaboratif dalam kelompok untuk merumuskan masalah (Tahap E), melakukan investigasi (Tahap S), dan membangun pemahaman bersama (Tahap A).
\item \textbf{Kreatif:} Menghasilkan argumen atau solusi orisinal yang terintegrasi pada tahap akhir pembelajaran (Tahap N).
\end{itemize}

\section{Sarana dan Prasarana}

\begin{itemize}
\item \textbf{Media:} LKPD 13 - Jurnal Investigasi Dual-Lensa, video singkat arsitektur rumah gadang (misal dari YouTube, durasi 1-2 menit), gambar cermin tradisional dan kaca jendela rumah gadang, artikel/infografis tentang pemantulan dan pembiasan cahaya, diagram sinar cahaya.
\item \textbf{Alat:} Papan tulis/whiteboard, spidol, Proyektor \& Speaker, Kertas Plano atau Karton (1 per kelompok), sticky notes warna-warni, cermin datar, senter/laser pointer, gelas bening berisi air.
\item \textbf{Sumber Belajar:} Buku ajar IPA kelas VIII, Tautan video animasi pemantulan cahaya (misal: bit.ly/animasi-cahaya), tautan artikel arsitektur rumah gadang (misal: bit.ly/rumah-gadang).
\end{itemize}

\section{Target Peserta didik}

\begin{itemize}
\item Peserta didik reguler kelas VIII (Fase D).
\end{itemize}

\section{Model Pembelajaran}

\begin{itemize}
\item Model Pembelajaran KESAN (Konektivitas Etnosains-Sains).
\end{itemize}

\section{Pemahaman Bermakna}
\textit{(Clue untuk Guru: Bacakan atau sampaikan narasi ini dengan intonasi yang menarik di akhir pembelajaran untuk mengikat semua pengalaman belajar siswa menjadi satu kesatuan yang bermakna).}

"Ananda Semua, hari ini kita telah membongkar rahasia di balik keindahan rumah gadang yang memukau! Kita menemukan bahwa arsitektur tradisional Minangkabau tidak hanya soal estetika, tetapi juga penerapan ilmu cahaya yang sangat canggih. Setiap cermin, setiap kaca jendela, setiap sudut atap yang dirancang dengan cermat - semuanya mengoptimalkan pencahayaan alami dan menciptakan suasana yang nyaman. Dengan ini, kita sadar bahwa sains cahaya tidak hanya ada di laboratorium, tapi juga terwujud dalam karya arsitektur leluhur kita yang menakjubkan."

\section{PERTEMUAN PERTAMA: Cahaya dan Kearifan Lokal Arsitektur Minangkabau}

\subsection{Capaian Pembelajaran (Fase D)}
Pada akhir Fase D, murid memiliki kemampuan [...] menganalisis gelombang dan pemanfaatannya dalam kehidupan sehari-hari [...]

\subsection{Tujuan Pembelajaran (TP) Pertemuan 1:}
\textit{(Clue untuk Guru: Tujuan ini adalah kompas Anda. Pastikan setiap tahapan KESAN yang Anda lalui berkontribusi pada pencapaian tujuan-tujuan ini).}

Melalui model pembelajaran KESAN, peserta didik mampu:
\begin{itemize}
\item Menghubungkan fenomena kearifan lokal Minangkabau (penggunaan cermin dan kaca dalam rumah gadang) dengan konteks sifat-sifat cahaya. (Sintaks K)
\item Merumuskan pertanyaan investigatif mengenai cara kerja pemantulan dan pembiasan cahaya dalam arsitektur tradisional. (Sintaks E)
\item Mengumpulkan informasi mengenai sifat-sifat cahaya dari sumber ilmiah serta fungsi cermin dan kaca dalam rumah gadang dari sumber kultural. (Sintaks S)
\item Menganalisis dan menyintesiskan hubungan sebab-akibat antara konsep ilmiah (pemantulan, pembiasan, dispersi) dengan praktik budaya (desain pencahayaan rumah gadang). (Sintaks A)
\item Menyusun sebuah penjelasan analitis yang logis mengenai bagaimana kearifan lokal dalam arsitektur Minangkabau dapat dibenarkan secara saintifik untuk mengoptimalkan pencahayaan alami. (Sintaks N)
\end{itemize}

\subsection{Pertanyaan Pemantik}
\textit{(Clue untuk Guru: Ajukan dua pertanyaan ini secara berurutan, berikan jeda agar siswa berpikir. Jangan langsung minta jawaban, biarkan pertanyaan ini menggantung untuk memicu rasa ingin tahu).}

\begin{itemize}
\item "Pernahkah kalian memperhatikan rumah gadang yang megah? Mengapa di dalam rumah gadang terasa terang meskipun tidak banyak lampu, padahal strukturnya begitu besar dan tinggi?"
\item "Coba perhatikan cermin dan kaca jendela di rumah gadang. Mengapa nenek moyang kita menempatkan cermin di posisi-posisi tertentu? Apa hubungannya dengan cahaya matahari yang masuk ke dalam rumah?"
\end{itemize}

\section{Langkah-langkah Kegiatan Pembelajaran:}

\subsection{Kegiatan Pembuka (15 Menit)}
\begin{itemize}
\item Guru membuka pelajaran dengan salam, doa, dan memeriksa kehadiran.
\item \textbf{Asesmen Diagnostik Awal:} Guru membagikan lembar K-W-L.
    \begin{itemize}
    \item \textbf{Instruksi Guru:} "Ananda, sebelum kita mulai menjelajahi dunia cahaya, tolong isi dua kolom pertama di lembar ini. Di kolom \textbf{K (Tahu)}, tulis apa saja yang sudah kalian ketahui tentang cahaya ATAU tentang rumah gadang. Di kolom \textbf{W (Ingin Tahu)}, tulis apa yang membuat kalian penasaran tentang topik ini." \textit{(Clue: Ini membantu Anda memetakan pengetahuan awal dan minat siswa secara cepat).}
    \end{itemize}
\item \textbf{Apersepsi:}
    \begin{itemize}
    \item \textbf{Instruksi Guru:} "Hari ini kita akan menjadi arsitek sains dan budaya. Kita akan menyelidiki bagaimana rumah gadang yang megah menggunakan prinsip-prinsip cahaya untuk menciptakan kenyamanan. Dalam investigasi ini, kita akan melatih kemampuan \textbf{Bernalar Kritis} kita, menghargai budaya lewat \textbf{Berkebinekaan Global}, dan bekerja sama dalam semangat \textbf{Gotong Royong}. Siap?"
    \end{itemize}
\end{itemize}

\subsection{Kegiatan Inti (90 Menit) - SINTAKS MODEL KESAN}

\subsubsection{Tahap 1: (K) Kaitkan Konteks Kultural (15 menit)}
\begin{itemize}
\item \textbf{Aktivitas Guru:}
    \begin{itemize}
    \item Menampilkan video singkat (1-2 menit) arsitektur rumah gadang yang menyorot penggunaan cermin dan kaca jendela.
    \item Mengajukan Pertanyaan Pemantik yang sudah disiapkan di atas.
    \item \textit{(Clue: Tujuan tahap ini adalah memvalidasi pengetahuan siswa dan memantik rasa heran, bukan mencari jawaban benar. Sediakan spidol dan papan tulis/whiteboard. Saat siswa menjawab, tuliskan semua ide mereka, bahkan yang keliru sekalipun, dengan judul "\textbf{IDE AWAL KITA}". Ini menunjukkan bahwa semua pemikiran dihargai).}
    \end{itemize}
\item \textbf{Aktivitas Siswa:} Mengamati video, mendengarkan pertanyaan, lalu secara sukarela berbagi pengalaman atau dugaan awal. Menuliskan minimal satu pertanyaan atau pengalaman di sticky notes dan menempelkannya di '\textbf{Papan Penasaran}'.
\end{itemize}

\subsubsection{Tahap 2: (E) Eksplorasi Enigma (15 menit)}
\begin{itemize}
\item \textbf{Aktivitas Guru:} Membentuk siswa menjadi kelompok (3-4 orang).
    \begin{itemize}
    \item \textbf{Instruksi Guru:} "Baik, rasa penasaran kalian luar biasa! Sekarang, tugas kita sebagai arsitek adalah mengubah rasa penasaran ini menjadi misi yang jelas. Dalam kelompok, diskusikan dan rumuskan minimal 3 pertanyaan kunci yang akan kita selidiki hari ini. Tuliskan dalam bentuk '\textbf{Misi Penyelidikan Tim [Nama Kelompok]}' di kertas plano yang Bapak/Ibu berikan."
    \item \textit{(Clue: Arahkan diskusi siswa agar pertanyaannya mencakup aspek 'bagaimana cahaya berperilaku' dan 'bagaimana rumah gadang memanfaatkannya'. Jika kelompok kesulitan, berikan pancingan: "Kira-kira, apa dulu yang perlu kita tahu? Sifat cahayanya atau fungsi cerminnya?").}
    \end{itemize}
\item \textbf{Aktivitas Siswa:} Berdiskusi dalam tim untuk merumuskan misi penyelidikan (daftar pertanyaan kunci) di kertas plano. \textit{(Contoh misi yang diharapkan: 1. Bagaimana cahaya bisa dipantulkan dan dibiaskan? 2. Mengapa cermin ditempatkan di posisi tertentu dalam rumah gadang? 3. Bagaimana desain jendela rumah gadang mengoptimalkan pencahayaan?)}.
\end{itemize}

\subsubsection{Tahap 3: (S) Selidiki secara Sintetis (25 menit)}
\begin{itemize}
\item \textbf{Aktivitas Guru:} Membagikan "\textbf{LKPD 13 - Jurnal Investigasi Dual-Lensa}".
    \begin{itemize}
    \item \textbf{Instruksi Guru:} "Setiap tim akan melakukan investigasi dari dua lensa. Gunakan HP atau sumber yang disediakan untuk mencari jawabannya. Bagilah tugas dalam tim!"
    \item \textbf{Lensa Sains:} Buka link video bit.ly/animasi-cahaya untuk memahami pemantulan, pembiasan, dan dispersi cahaya. Lakukan juga eksperimen sederhana dengan cermin dan senter.
    \item \textbf{Lensa Etnosains/Kultural:} Buka link artikel bit.ly/rumah-gadang untuk memahami filosofi dan fungsi cermin serta jendela dalam arsitektur tradisional Minangkabau.
    \item \textit{(Clue: Pastikan sumber belajar sudah disiapkan dan link bisa diakses. Berkelilinglah untuk memastikan setiap kelompok membagi tugas dan tidak hanya fokus pada satu lensa saja).}
    \end{itemize}
\item \textbf{Aktivitas Siswa:} Dalam kelompok, siswa berbagi tugas mencari informasi dari sumber yang diberikan dan mencatat temuan kunci di dua kolom terpisah pada "\textbf{LKPD 13 - Jurnal Investigasi Dual-Lensa}".
\end{itemize}

\subsubsection{Tahap 4: (A) Asimilasi Analitis (20 menit)}
\begin{itemize}
\item \textbf{Aktivitas Guru:} Memfasilitasi diskusi untuk menjembatani kedua lensa.
    \begin{itemize}
    \item \textbf{Pertanyaan Pancingan Kunci untuk Guru:}
        \begin{itemize}
        \item "Oke, dari Lensa Sains kita tahu cahaya bisa dipantulkan oleh cermin. Dari Lensa Etnosains, kita tahu cermin di rumah gadang ditempatkan strategis. Nah, coba hubungkan! Bagaimana penempatan cermin tersebut membantu menerangi ruangan?"
        \item "Dari Lensa Sains, kita tahu cahaya bisa dibiaskan saat melewati kaca. Dari Lensa Budaya, jendela rumah gadang memiliki desain khusus. Apa hubungannya dengan kenyamanan pencahayaan di dalam rumah?"
        \end{itemize}
    \item \textit{(Clue: Fokuskan untuk membuat siswa 'menemukan' hubungannya sendiri, bukan diberitahu. Gunakan kata "menurut kalian", "kira-kira kenapa", "ada yang punya ide?").}
    \end{itemize}
\item \textbf{Aktivitas Siswa:} Berdiskusi intensif untuk menghubungkan temuan sains dan budaya. Menuliskan kesimpulan terpadu (sintesis) mereka di kertas plano.
\end{itemize}

\subsubsection{Tahap 5: (N) Nyatakan Pemahaman (15 menit)}
\begin{itemize}
\item \textbf{Aktivitas Guru:} Memberikan studi kasus individual atau per kelompok.
    \begin{itemize}
    \item \textbf{Instruksi Guru (tuliskan di papan tulis):}
    
    "\textbf{STUDI KASUS UNTUK ARSITEK SAINS-BUDAYA:}
    
    Keluargamu berencana merenovasi rumah dan ingin menerapkan prinsip pencahayaan alami seperti rumah gadang. Ruang tamu kalian menghadap utara (kurang sinar matahari langsung) dan terasa gelap di siang hari. Ayahmu ingin memasang lampu banyak, tapi ibumu ingin solusi hemat listrik.
    
    Tugasmu: Tuliskan sebuah saran singkat (3-5 kalimat) di buku latihanmu untuk keluargamu. Gunakan pengetahuan gabungan dari sains (pemantulan, pembiasan cahaya) dan kearifan lokal rumah gadang yang baru saja kamu pelajari untuk memberikan solusi pencahayaan alami yang efektif."
    \end{itemize}
\item \textbf{Aktivitas Siswa:} Menyusun argumen tertulis untuk menjawab studi kasus yang diberikan, menggunakan bukti dari kedua lensa.
\end{itemize}

\subsection{Kegiatan Penutup (15 Menit)}
\begin{itemize}
\item \textbf{Presentasi \& Penguatan:} Guru meminta 2-3 siswa secara acak untuk membacakan jawaban studi kasus mereka. Guru memberikan pujian dan penguatan positif, menekankan betapa hebatnya argumen yang memadukan sains dan budaya.
\item \textbf{Refleksi:}
    \begin{itemize}
    \item \textbf{Instruksi Guru:} "Sekarang, kembali ke lembar K-W-L kalian. Lengkapi kolom terakhir, \textbf{L (Learned)}, dengan hal-hal baru yang paling berkesan yang kalian pelajari hari ini."
    \item \textbf{Instruksi Guru:} "Angkat tangan, siapa yang setelah belajar hari ini jadi melihat rumah dan bangunan di sekitar dengan cara yang berbeda?"
    \end{itemize}
\item \textbf{Tindak Lanjut:}
    \begin{itemize}
    \item \textbf{Instruksi Guru:} "Kerja bagus, para arsitek! Hari ini kita sudah membongkar rahasia 'pencahayaan alami'. Pertemuan berikutnya, kita akan menyelidiki bagaimana cahaya bisa 'dibelokkan' dan 'diperbesar' melalui alat-alat optik tradisional. Siapa tahu alat optik apa yang pernah digunakan nenek moyang kita? Kita akan temukan jawabannya!"
    \item \textit{(Clue: Kaitkan secara eksplisit dengan pembelajaran hari ini untuk membangun alur narasi yang berkelanjutan. Ini membuat siswa merasa belajar IPA seperti mengikuti sebuah cerita bersambung).}
    \end{itemize}
\item Guru menutup pelajaran dengan doa dan salam.
\end{itemize}

\section{Asesmen (Penilaian)}

\begin{itemize}
\item \textbf{Asesmen Diagnostik (Awal):} Analisis lembar K-W-L. (Untuk mengetahui baseline siswa).
\item \textbf{Asesmen Formatif (Proses):} Observasi keaktifan diskusi (gotong royong) dan penilaian kelengkapan "\textbf{LKPD 13 - Jurnal Investigasi Dual-Lensa}".
\item \textbf{Asesmen Sumatif (Akhir Siklus):} Penilaian jawaban studi kasus (Tahap N) menggunakan rubrik.
\end{itemize}

\subsection{Rubrik Penilaian Jawaban Studi Kasus (Tahap N)}

\begin{center}
\begin{tabular}{|p{3cm}|p{3cm}|p{3cm}|p{3cm}|p{3cm}|}
\hline
\textbf{Kriteria Penilaian} & \textbf{Skor 4 (Sangat Baik)} & \textbf{Skor 3 (Baik)} & \textbf{Skor 2 (Cukup)} & \textbf{Skor 1 (Kurang)} \\
\hline
\textbf{Ketepatan Konsep Ilmiah} & Menggunakan istilah ilmiah (pemantulan, pembiasan, dispersi, sudut datang) dengan sangat tepat dan relevan dengan kasus. & Menggunakan istilah ilmiah dengan tepat, namun kurang relevan. & Menggunakan istilah ilmiah namun ada beberapa kesalahan konsep. & Tidak menggunakan istilah ilmiah atau salah total. \\
\hline
\textbf{Keterkaitan dengan Etnosains} & Mampu menghubungkan secara logis dan eksplisit antara praktik budaya (desain rumah gadang) dengan penjelasan ilmiahnya secara mendalam. & Mampu menghubungkan praktik budaya dengan penjelasan ilmiah, namun kurang mendalam. & Hanya menyebutkan praktik budaya tanpa menghubungkan dengan sains, atau sebaliknya. & Tidak ada keterkaitan antara sains dan budaya yang ditunjukkan. \\
\hline
\textbf{Kelogisan \& Struktur Argumen} & Penjelasan sangat logis, runtut, persuasif, dan mudah dipahami. & Penjelasan logis dan runtut, namun kurang persuasif. & Alur penjelasan kurang runtut atau sulit dipahami. & Penjelasan tidak logis dan tidak terstruktur. \\
\hline
\end{tabular}
\end{center}

\vspace{1cm}

\section{Daftar Pustaka Etnosains}
\begin{itemize}
\item Navis, A.A. (1984). \textit{Alam Terkembang Jadi Guru: Adat dan Kebudayaan Minangkabau}. Jakarta: Grafiti Pers.
\item Hakimy, I. (1994). \textit{Rangkaian Mustika Adat Basandi Syarak di Minangkabau}. Bandung: Remaja Rosdakarya.
\item Couto, N. (2008). \textit{Budaya Visual Rumah Gadang}. Padang: Andalas University Press.
\item Asnan, G. (2007). \textit{Memikir Ulang Regionalisme: Sumatera Barat Tahun 1950-an}. Jakarta: Yayasan Obor Indonesia.
\end{itemize}

\end{document}