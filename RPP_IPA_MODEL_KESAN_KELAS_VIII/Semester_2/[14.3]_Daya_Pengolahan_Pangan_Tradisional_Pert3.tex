% !TEX encoding = UTF-8

\documentclass[12pt,a4paper]{article}
\usepackage[margin=1.5cm]{geometry}
\usepackage[utf8]{inputenc}
\usepackage[T1]{fontenc}
\usepackage{times}
\usepackage{amsmath}
\usepackage{amsfonts}
\usepackage{amssymb}
\usepackage{graphicx}
\usepackage{xcolor}
\usepackage{tcolorbox}
\usepackage{enumitem}
\usepackage{multicol}
\usepackage{fancyhdr}
\usepackage{titlesec}

% Setup header and footer
\pagestyle{fancy}
\fancyhf{}
\fancyhead[L]{\textbf{RPP IPA Kelas VIII - Model KESAN}}
\fancyhead[R]{\textbf{Semester 2 - Pertemuan 14.3}}
\fancyfoot[C]{\thepage}

% Title formatting
\titleformat{\section}{\Large\bfseries}{}{0em}{}
\titleformat{\subsection}{\large\bfseries}{}{0em}{}
\titleformat{\subsubsection}{\normalsize\bfseries}{}{0em}{}

\begin{document}

\begin{center}
{\Huge\textbf{MODUL AJAR}}\\n\vspace{0.5cm}
{\Large\textbf{Pengolahan Pangan Tradisional \& Konsep Daya}}
\end{center}

\vspace{1cm}

\begin{tcolorbox}[colback=white,colframe=black,boxrule=1pt]
\textbf{Nama Penyusun:} Irfan Ananda\\
\textbf{Institusi:} SMP\\
\textbf{Mata Pelajaran:} Ilmu Pengetahuan Alam (IPA)\\
\textbf{Tahun Ajaran:} 2025/2026\\
\textbf{Semester:} Genap\\
\textbf{Jenjang Sekolah:} SMP\\
\textbf{Kelas/Fase:} VIII / D\\
\textbf{Alokasi waktu:} 3 x 40 Menit (1 Pertemuan)
\end{tcolorbox}

\section*{A. PROFIL PELAJAR PANCASILA}
\begin{itemize}[leftmargin=*]
    \item Bernalar kritis: Memproses informasi secara objektif, membangun keterkaitan antara berbagai informasi, menganalisis dan mengevaluasi penalaran.
    \item Kreatif: Menghasilkan gagasan yang orisinal, menghasilkan karya dan tindakan yang orisinal, memiliki keluwesan berpikir dalam mencari alternatif solusi permasalahan.
    \item Mandiri: Mengenali kualitas dan minat diri serta tantangan yang dihadapi, mengembangkan diri dan situasi yang dihadapi.
    \item Gotong royong: Kolaborasi, kepedulian, berbagi.
\end{itemize}

\section*{B. SARANA DAN PRASARANA}
\begin{itemize}[leftmargin=*]
    \item Alat dan bahan praktikum: Alat pengolahan pangan tradisional Minangkabau (lesung, alu, tungku tradisional), bahan makanan lokal (beras, singkong, kelapa).
    \item Media pembelajaran: Video proses pengolahan pangan tradisional, poster/infografis konsep daya dalam fisika.
    \item Lembar Kerja Peserta Didik (LKPD) tentang pengukuran daya pada proses pengolahan pangan tradisional.
\end{itemize}

\section*{C. TARGET PESERTA DIDIK}
\begin{itemize}[leftmargin=*]
    \item Peserta didik kelas VIII SMP dengan kemampuan beragam.
    \item Peserta didik memiliki pengetahuan awal tentang konsep usaha dan energi dari pertemuan sebelumnya.
    \item Peserta didik memiliki pengalaman sehari-hari terkait pengolahan pangan tradisional di lingkungan Minangkabau.
\end{itemize}

\section*{D. MODEL PEMBELAJARAN KESAN}
\begin{tcolorbox}[colback=white,colframe=black,title=Model KESAN (Kontekstual Etnosains),boxrule=1pt]
Model pembelajaran KESAN (Kontekstual Etnosains) mengintegrasikan kearifan lokal Minangkabau dengan konsep sains, melalui tahapan:
\begin{enumerate}[leftmargin=*]
    \item \textbf{Kontekstualisasi}: Mengaitkan konsep daya dengan proses pengolahan pangan tradisional Minangkabau.
    \item \textbf{Eksplorasi}: Menggali pengetahuan tentang teknik pengolahan pangan tradisional dan konsep daya yang terlibat.
    \item \textbf{Saintifikasi}: Menjelaskan prinsip fisika (daya) dalam proses pengolahan pangan tradisional.
    \item \textbf{Aktualisasi}: Menerapkan konsep daya untuk mengoptimalkan proses pengolahan pangan tradisional.
    \item \textbf{Naturalisasi}: Mengintegrasikan pemahaman sains dengan kearifan lokal untuk pelestarian budaya dan lingkungan.
\end{enumerate}
\end{tcolorbox}

\section*{E. PEMAHAMAN BERMAKNA}
\begin{tcolorbox}[colback=white,colframe=black,boxrule=1pt]
Peserta didik akan memahami bahwa:
\begin{itemize}[leftmargin=*]
    \item Konsep daya dalam fisika dapat menjelaskan efisiensi proses pengolahan pangan tradisional.
    \item Kearifan lokal dalam teknik pengolahan pangan tradisional Minangkabau mengandung prinsip-prinsip sains yang dapat divalidasi secara ilmiah.
    \item Pemahaman konsep daya dapat digunakan untuk mengoptimalkan proses pengolahan pangan tradisional tanpa menghilangkan nilai-nilai budaya.
\end{itemize}
\end{tcolorbox}

\section*{F. TUJUAN PEMBELAJARAN}
Setelah mengikuti pembelajaran, peserta didik dapat:
\begin{enumerate}[leftmargin=*]
    \item Menjelaskan konsep daya dalam fisika dan aplikasinya dalam kehidupan sehari-hari.
    \item Menganalisis prinsip daya yang terlibat dalam proses pengolahan pangan tradisional Minangkabau.
    \item Mengukur daya yang digunakan dalam berbagai teknik pengolahan pangan tradisional.
    \item Membandingkan efisiensi daya antara metode pengolahan pangan tradisional dan modern.
    \item Merancang perbaikan proses pengolahan pangan tradisional berdasarkan konsep daya untuk meningkatkan efisiensi.
\end{enumerate}

\section*{G. PERTANYAAN PEMANTIK}
\begin{tcolorbox}[colback=white,colframe=black,boxrule=1pt]
\begin{enumerate}[leftmargin=*]
    \item Mengapa beberapa proses pengolahan pangan tradisional membutuhkan waktu dan tenaga yang lebih besar dibandingkan metode modern?
    \item Bagaimana konsep daya dapat menjelaskan efisiensi proses pengolahan pangan tradisional?
    \item Apakah ada kearifan lokal dalam teknik pengolahan pangan tradisional yang secara tidak langsung telah menerapkan prinsip efisiensi daya?
    \item Bagaimana kita dapat meningkatkan efisiensi pengolahan pangan tradisional tanpa menghilangkan nilai budayanya?
\end{enumerate}
\end{tcolorbox}

\section*{H. KEGIATAN PEMBELAJARAN}

\subsection*{1. Kegiatan Pendahuluan (15 menit)}
\begin{itemize}[leftmargin=*]
    \item Guru membuka pembelajaran dengan salam dan doa.
    \item Guru memeriksa kehadiran peserta didik dan menyiapkan fisik dan psikis peserta didik.
    \item Guru menayangkan video singkat tentang proses pengolahan pangan tradisional Minangkabau (pembuatan rendang, ampiang dadiah, atau sala lauak).
    \item Guru mengajukan pertanyaan: "Berapa lama waktu yang dibutuhkan untuk membuat rendang secara tradisional? Mengapa membutuhkan waktu yang lama?"
    \item Guru menyampaikan tujuan pembelajaran dan manfaat mempelajari konsep daya dalam konteks pengolahan pangan tradisional.
\end{itemize}

\subsection*{2. Kegiatan Inti dengan Sintaks KESAN (90 menit)}

\subsubsection*{a. Kontekstualisasi (15 menit)}
\begin{itemize}[leftmargin=*]
    \item Peserta didik mengamati berbagai alat pengolahan pangan tradisional Minangkabau yang telah disiapkan (lesung dan alu, tungku tradisional, parutan kelapa tradisional).
    \item Peserta didik berbagi pengalaman tentang proses pengolahan pangan tradisional yang pernah mereka lihat atau lakukan di rumah atau lingkungan sekitar.
    \item Guru menjelaskan bahwa proses pengolahan pangan tradisional dapat dikaji dari perspektif sains, khususnya konsep daya dalam fisika.
\end{itemize}

\subsubsection*{b. Eksplorasi (20 menit)}
\begin{itemize}[leftmargin=*]
    \item Peserta didik dibagi menjadi beberapa kelompok untuk mengeksplorasi berbagai teknik pengolahan pangan tradisional Minangkabau.
    \item Setiap kelompok mendapatkan LKPD yang berisi panduan untuk mengidentifikasi proses pengolahan pangan tradisional dan menganalisis konsep daya yang terlibat.
    \item Peserta didik melakukan studi literatur dan diskusi kelompok untuk mengidentifikasi faktor-faktor yang mempengaruhi daya dalam proses pengolahan pangan tradisional.
\end{itemize}

\subsubsection*{c. Saintifikasi (25 menit)}
\begin{itemize}[leftmargin=*]
    \item Guru menjelaskan konsep daya dalam fisika: $P = \frac{W}{t}$ atau $P = F \cdot v$, dengan contoh-contoh aplikasi dalam kehidupan sehari-hari.
    \item Peserta didik melakukan praktikum sederhana untuk mengukur daya yang digunakan dalam proses pengolahan pangan tradisional, misalnya:
    \begin{itemize}
        \item Mengukur daya saat menumbuk beras menggunakan lesung dan alu.
        \item Mengukur daya pada proses pengadukan rendang secara tradisional.
        \item Membandingkan daya yang digunakan pada tungku tradisional dan kompor modern.
    \end{itemize}
    \item Peserta didik mencatat hasil pengukuran dan melakukan analisis data untuk menghitung daya pada setiap proses.
\end{itemize}

\subsubsection*{d. Aktualisasi (20 menit)}
\begin{itemize}[leftmargin=*]
    \item Berdasarkan hasil pengukuran dan analisis, peserta didik mengidentifikasi faktor-faktor yang mempengaruhi efisiensi daya dalam proses pengolahan pangan tradisional.
    \item Peserta didik merancang perbaikan pada proses pengolahan pangan tradisional untuk meningkatkan efisiensi daya tanpa menghilangkan nilai budayanya.
    \item Setiap kelompok mempresentasikan hasil rancangan perbaikan dan mendapatkan umpan balik dari kelompok lain dan guru.
\end{itemize}

\subsubsection*{e. Naturalisasi (10 menit)}
\begin{itemize}[leftmargin=*]
    \item Peserta didik mendiskusikan bagaimana pemahaman konsep daya dapat membantu melestarikan teknik pengolahan pangan tradisional dengan adaptasi yang sesuai dengan kebutuhan modern.
    \item Peserta didik merefleksikan nilai-nilai kearifan lokal dalam teknik pengolahan pangan tradisional yang telah secara tidak langsung menerapkan prinsip efisiensi daya.
    \item Guru memfasilitasi diskusi tentang pentingnya melestarikan teknik pengolahan pangan tradisional sebagai bagian dari identitas budaya, dengan tetap mengadopsi pemahaman sains untuk peningkatan efisiensi.
\end{itemize}

\subsection*{3. Kegiatan Penutup (15 menit)}
\begin{itemize}[leftmargin=*]
    \item Peserta didik dengan bimbingan guru menyimpulkan konsep daya dan aplikasinya dalam proses pengolahan pangan tradisional.
    \item Guru memberikan penguatan tentang pentingnya memahami konsep daya untuk mengoptimalkan berbagai proses dalam kehidupan sehari-hari.
    \item Peserta didik mengerjakan kuis singkat tentang konsep daya dan aplikasinya.
    \item Guru memberikan tugas proyek: "Menganalisis dan Meningkatkan Efisiensi Daya pada Satu Proses Pengolahan Pangan Tradisional di Rumah".
    \item Guru menyampaikan informasi tentang materi yang akan dipelajari pada pertemuan berikutnya.
    \item Guru menutup pembelajaran dengan doa dan salam.
\end{itemize}

\section*{I. ASESMEN}
\begin{tcolorbox}[colback=white,colframe=black,title=Asesmen Formatif,boxrule=1pt]
\begin{enumerate}[leftmargin=*]
    \item \textbf{Observasi}: Mengamati keterlibatan dan kontribusi peserta didik selama diskusi kelompok dan praktikum.
    \item \textbf{Tes Tertulis}: Kuis singkat tentang konsep daya dan aplikasinya dalam proses pengolahan pangan tradisional.
    \item \textbf{Produk}: Laporan hasil pengukuran dan analisis daya pada proses pengolahan pangan tradisional.
    \item \textbf{Proyek}: Rancangan perbaikan proses pengolahan pangan tradisional untuk meningkatkan efisiensi daya.
\end{enumerate}
\end{tcolorbox}

\begin{tcolorbox}[colback=white,colframe=black,title=Asesmen Sumatif,boxrule=1pt]
\begin{enumerate}[leftmargin=*]
    \item \textbf{Tes Tertulis}: Soal-soal tentang konsep daya dan aplikasinya dalam berbagai konteks.
    \item \textbf{Proyek Akhir}: Menganalisis dan meningkatkan efisiensi daya pada satu proses pengolahan pangan tradisional di rumah, dengan dokumentasi proses dan hasil.
\end{enumerate}
\end{tcolorbox}

\section*{J. RUBRIK PENILAIAN STUDI KASUS}
\begin{tcolorbox}[colback=white,colframe=black,boxrule=1pt]
\textbf{Studi Kasus}: Analisis Efisiensi Daya pada Proses Pembuatan Rendang Tradisional

\begin{tabular}{|p{3cm}|p{3.5cm}|p{3.5cm}|p{3.5cm}|}
\hline
\textbf{Aspek} & \textbf{Baik (3)} & \textbf{Cukup (2)} & \textbf{Perlu Perbaikan (1)} \\ \hline
\textbf{Identifikasi Proses} & Mengidentifikasi semua tahapan proses pembuatan rendang dengan detail dan akurat & Mengidentifikasi sebagian besar tahapan proses dengan cukup detail & Identifikasi tahapan proses tidak lengkap atau tidak akurat \\ \hline
\textbf{Pengukuran Daya} & Pengukuran daya dilakukan dengan metode yang tepat dan hasil yang akurat pada setiap tahapan & Pengukuran daya dilakukan dengan metode yang cukup tepat pada sebagian tahapan & Pengukuran daya tidak tepat atau tidak dilakukan pada sebagian besar tahapan \\ \hline
\textbf{Analisis Data} & Analisis data komprehensif, menggunakan konsep daya dengan tepat, dan mengidentifikasi faktor-faktor yang mempengaruhi efisiensi & Analisis data cukup baik, menggunakan konsep daya dengan cukup tepat & Analisis data minimal, konsep daya tidak diterapkan dengan tepat \\ \hline
\textbf{Rancangan Perbaikan} & Rancangan perbaikan inovatif, berdasarkan data, dan mempertimbangkan pelestarian nilai budaya & Rancangan perbaikan cukup baik dan mempertimbangkan beberapa aspek budaya & Rancangan perbaikan minimal atau tidak mempertimbangkan pelestarian nilai budaya \\ \hline
\end{tabular}
\end{tcolorbox}

\section*{K. DAFTAR PUSTAKA ETNOSAINS}
\begin{enumerate}[leftmargin=*]
    \item Anwar, R. (2023). \textit{Fisika dalam Kuliner Tradisional Minangkabau}. Padang: Universitas Andalas Press.
    \item Ismail, I.A. (2024). \textit{Integrasi Etnosains dalam Pembelajaran IPA: Model KESAN}. Jakarta: Kementerian Pendidikan dan Kebudayaan.
    \item Yuliana, S. (2022). \textit{Kearifan Lokal dalam Pengolahan Pangan Tradisional Minangkabau}. Jurnal Etnosains Indonesia, 5(2), 78-92.
    \item Zulkifli, M. (2023). \textit{Analisis Efisiensi Energi pada Proses Pengolahan Pangan Tradisional di Sumatera Barat}. Jurnal Fisika Terapan, 8(1), 45-59.
\end{enumerate}

\end{document}