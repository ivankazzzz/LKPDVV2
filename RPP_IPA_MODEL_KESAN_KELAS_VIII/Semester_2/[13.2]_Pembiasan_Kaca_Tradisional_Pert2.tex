% Dokumen LaTeX untuk kompilasi dengan pdfLaTeX

\documentclass[12pt,a4paper]{article}
\usepackage[margin=1.5cm]{geometry}
\usepackage[utf8]{inputenc}
\usepackage[T1]{fontenc}
\usepackage{times}
\usepackage{amsmath}
\usepackage{amsfonts}
\usepackage{amssymb}
\usepackage{graphicx}
\usepackage{xcolor}
\usepackage{tcolorbox}
\usepackage{enumitem}
\usepackage{multicol}
\usepackage{fancyhdr}
\usepackage{titlesec}

% Setup header and footer
\pagestyle{fancy}
\fancyhf{}
\fancyhead[L]{\textbf{RPP IPA Kelas VIII - Model KESAN}}
\fancyhead[R]{\textbf{Semester 2 - Pertemuan 13.2}}
\fancyfoot[C]{\thepage}

% Title formatting
\titleformat{\section}{\Large\bfseries}{}{0em}{}
\titleformat{\subsection}{\large\bfseries}{}{0em}{}
\titleformat{\subsubsection}{\normalsize\bfseries}{}{0em}{}

\begin{document}

\begin{center}
{\Huge\textbf{MODUL AJAR}}\\
\vspace{0.5cm}
{\Large\textbf{Misteri Kaca Tradisional \& Keajaiban Pembiasan Cahaya}}
\end{center}

\vspace{1cm}

\begin{tcolorbox}[colback=white,colframe=black,boxrule=1pt]
\textbf{Nama Penyusun:} Irfan Ananda\\
\textbf{Institusi:} SMP\\
\textbf{Mata Pelajaran:} Ilmu Pengetahuan Alam (IPA)\\
\textbf{Tahun Ajaran:} 2025/2026\\
\textbf{Semester:} Genap\\
\textbf{Jenjang Sekolah:} SMP\\
\textbf{Kelas/Fase:} VIII / D\\
\textbf{Alokasi waktu:} 3 x 40 Menit (1 Pertemuan)
\end{tcolorbox}

\section{DIMENSI PROFIL PELAJAR PANCASILA}
\textit{(Clue untuk Guru: Sebutkan dimensi ini secara eksplisit saat apersepsi agar siswa sadar tujuan non-akademis yang sedang mereka kembangkan).}

\begin{itemize}
\item \textbf{Berkebinekaan Global:} Mengenal dan menghargai budaya, khususnya menganalisis kearifan lokal (etnosains) Minangkabau dalam pembuatan dan penggunaan kaca tradisional serta alat optik sederhana, lalu menghubungkannya dengan konsep sains universal tentang pembiasan cahaya.
\item \textbf{Bernalar Kritis:} Ditempa secara intensif saat menganalisis informasi dari sumber sains dan budaya (Tahap S), menyintesiskan kedua perspektif (Tahap A), dan menyusun argumen berbasis bukti (Tahap N).
\item \textbf{Gotong Royong:} Kemampuan untuk bekerja sama secara kolaboratif dalam kelompok untuk merumuskan masalah (Tahap E), melakukan investigasi (Tahap S), dan membangun pemahaman bersama (Tahap A).
\item \textbf{Kreatif:} Menghasilkan argumen atau solusi orisinal yang terintegrasi pada tahap akhir pembelajaran (Tahap N).
\end{itemize}

\section{Sarana dan Prasarana}

\begin{itemize}
\item \textbf{Media:} LKPD 14 - Jurnal Investigasi Dual-Lensa, video singkat pembuatan kaca tradisional (misal dari YouTube, durasi 1-2 menit), gambar alat optik tradisional dan kaca pembesar, artikel/infografis tentang pembiasan cahaya dan indeks bias, diagram jalannya sinar pada lensa.
\item \textbf{Alat:} Papan tulis/whiteboard, spidol, Proyektor \& Speaker, Kertas Plano atau Karton (1 per kelompok), sticky notes warna-warni, gelas bening berisi air, pensil, lensa cembung sederhana, senter/laser pointer.
\item \textbf{Sumber Belajar:} Buku ajar IPA kelas VIII, Tautan video animasi pembiasan cahaya (misal: bit.ly/animasi-pembiasan), tautan artikel kaca tradisional Minangkabau (misal: bit.ly/kaca-tradisional).
\end{itemize}

\section{Target Peserta didik}

\begin{itemize}
\item Peserta didik reguler kelas VIII (Fase D).
\end{itemize}

\section{Model Pembelajaran}

\begin{itemize}
\item Model Pembelajaran KESAN (Konektivitas Etnosains-Sains).
\end{itemize}

\section{Pemahaman Bermakna}
\textit{(Clue untuk Guru: Bacakan atau sampaikan narasi ini dengan intonasi yang menarik di akhir pembelajaran untuk mengikat semua pengalaman belajar siswa menjadi satu kesatuan yang bermakna).}

"Ananda Semua, hari ini kita telah mengungkap keajaiban di balik kaca tradisional yang sederhana namun menakjubkan! Kita menemukan bahwa nenek moyang Minangkabau telah memahami secara intuitif bagaimana cahaya dapat 'dibelokkan' dan 'diperbesar' melalui berbagai media transparan. Setiap kaca jendela, setiap wadah air yang jernih, bahkan tetesan embun di daun - semuanya adalah laboratorium alami untuk memahami pembiasan cahaya. Dengan ini, kita sadar bahwa prinsip-prinsip optik tidak hanya ada di kacamata modern, tapi sudah lama dipraktikkan dalam kehidupan sehari-hari leluhur kita."

\section{PERTEMUAN KEDUA: Pembiasan Cahaya dan Alat Optik Tradisional}

\subsection{Capaian Pembelajaran (Fase D)}
Pada akhir Fase D, murid memiliki kemampuan [...] menganalisis gelombang dan pemanfaatannya dalam kehidupan sehari-hari [...]

\subsection{Tujuan Pembelajaran (TP) Pertemuan 2:}
\textit{(Clue untuk Guru: Tujuan ini adalah kompas Anda. Pastikan setiap tahapan KESAN yang Anda lalui berkontribusi pada pencapaian tujuan-tujuan ini).}

Melalui model pembelajaran KESAN, peserta didik mampu:
\begin{itemize}
\item Menghubungkan fenomena kearifan lokal Minangkabau (penggunaan kaca dan alat optik sederhana) dengan konteks pembiasan cahaya. (Sintaks K)
\item Merumuskan pertanyaan investigatif mengenai cara kerja pembiasan cahaya dan aplikasinya dalam alat optik tradisional. (Sintaks E)
\item Mengumpulkan informasi mengenai hukum pembiasan dan indeks bias dari sumber ilmiah serta penggunaan alat optik dalam budaya Minangkabau dari sumber kultural. (Sintaks S)
\item Menganalisis dan menyintesiskan hubungan sebab-akibat antara konsep ilmiah (hukum Snellius, indeks bias, lensa) dengan praktik budaya (pembuatan dan penggunaan alat optik tradisional). (Sintaks A)
\item Menyusun sebuah penjelasan analitis yang logis mengenai bagaimana kearifan lokal dalam teknologi optik Minangkabau dapat dibenarkan secara saintifik untuk memahami sifat cahaya. (Sintaks N)
\end{itemize}

\subsection{Pertanyaan Pemantik}
\textit{(Clue untuk Guru: Ajukan dua pertanyaan ini secara berurutan, berikan jeda agar siswa berpikir. Jangan langsung minta jawaban, biarkan pertanyaan ini menggantung untuk memicu rasa ingin tahu).}

\begin{itemize}
\item "Pernahkah kalian memperhatikan saat memasukkan sendok ke dalam gelas berisi air, sendok tersebut tampak 'patah' atau bengkok? Mengapa hal ini bisa terjadi?"
\item "Nenek moyang kita sudah lama menggunakan kaca pembesar sederhana untuk melihat benda-benda kecil, seperti saat memeriksa kualitas benang emas untuk songket. Bagaimana sebenarnya kaca tersebut bisa membuat benda tampak lebih besar?"
\end{itemize}

\section{Langkah-langkah Kegiatan Pembelajaran:}

\subsection{Kegiatan Pembuka (15 Menit)}
\begin{itemize}
\item Guru membuka pelajaran dengan salam, doa, dan memeriksa kehadiran.
\item \textbf{Asesmen Diagnostik Awal:} Guru membagikan lembar K-W-L.
    \begin{itemize}
    \item \textbf{Instruksi Guru:} "Ananda, setelah kemarin kita belajar tentang pemantulan cahaya, sekarang kita akan mendalami pembiasan. Di lembar K-W-L ini, tulis di kolom \textbf{K (Tahu)} apa yang sudah kalian ketahui tentang pembiasan cahaya ATAU tentang kaca pembesar. Di kolom \textbf{W (Ingin Tahu)}, tulis apa yang membuat kalian penasaran tentang topik ini." \textit{(Clue: Ini membantu Anda melihat koneksi dengan pembelajaran sebelumnya dan minat siswa).}
    \end{itemize}
\item \textbf{Apersepsi:}
    \begin{itemize}
    \item \textbf{Instruksi Guru:} "Hari ini kita akan menjadi ahli optik tradisional. Kita akan menyelidiki bagaimana cahaya dapat 'dibelokkan' dan bagaimana nenek moyang kita memanfaatkan fenomena ini. Dalam investigasi ini, kita akan melatih kemampuan \textbf{Bernalar Kritis} kita, menghargai budaya lewat \textbf{Berkebinekaan Global}, dan bekerja sama dalam semangat \textbf{Gotong Royong}. Siap?"
    \end{itemize}
\end{itemize}

\subsection{Kegiatan Inti (90 Menit) - SINTAKS MODEL KESAN}

\subsubsection{Tahap 1: (K) Kaitkan Konteks Kultural (15 menit)}
\begin{itemize}
\item \textbf{Aktivitas Guru:}
    \begin{itemize}
    \item Menampilkan video singkat (1-2 menit) tentang pembuatan kaca tradisional dan penggunaan alat optik sederhana dalam kerajinan Minangkabau.
    \item Melakukan demonstrasi sederhana: memasukkan pensil ke dalam gelas berisi air.
    \item Mengajukan Pertanyaan Pemantik yang sudah disiapkan di atas.
    \item \textit{(Clue: Tujuan tahap ini adalah memvalidasi pengetahuan siswa dan memantik rasa heran. Biarkan siswa mengamati fenomena pembiasan secara langsung. Tuliskan semua ide mereka dengan judul "\textbf{PENGAMATAN AWAL KITA}").}
    \end{itemize}
\item \textbf{Aktivitas Siswa:} Mengamati demonstrasi dan video, mendengarkan pertanyaan, lalu secara sukarela berbagi pengalaman atau dugaan awal. Menuliskan minimal satu pertanyaan atau pengamatan di sticky notes dan menempelkannya di '\textbf{Papan Penasaran}'.
\end{itemize}

\subsubsection{Tahap 2: (E) Eksplorasi Enigma (15 menit)}
\begin{itemize}
\item \textbf{Aktivitas Guru:} Membentuk siswa menjadi kelompok (3-4 orang).
    \begin{itemize}
    \item \textbf{Instruksi Guru:} "Fenomena yang kalian amati tadi sangat menarik! Sekarang, tugas kita sebagai ahli optik adalah mengubah rasa penasaran ini menjadi misi penelitian yang jelas. Dalam kelompok, diskusikan dan rumuskan minimal 3 pertanyaan kunci yang akan kita selidiki hari ini. Tuliskan dalam bentuk '\textbf{Misi Penyelidikan Tim [Nama Kelompok]}' di kertas plano yang Bapak/Ibu berikan."
    \item \textit{(Clue: Arahkan diskusi siswa agar pertanyaannya mencakup aspek 'mengapa cahaya berbelok' dan 'bagaimana alat optik bekerja'. Jika kelompok kesulitan, berikan pancingan: "Kira-kira, apa yang menyebabkan pensil tampak bengkok? Dan bagaimana kaca pembesar bisa memperbesar benda?").}
    \end{itemize}
\item \textbf{Aktivitas Siswa:} Berdiskusi dalam tim untuk merumuskan misi penyelidikan (daftar pertanyaan kunci) di kertas plano. \textit{(Contoh misi yang diharapkan: 1. Mengapa cahaya berbelok saat melewati air atau kaca? 2. Bagaimana cara kerja lensa cembung dan cekung? 3. Bagaimana nenek moyang kita membuat dan menggunakan alat optik sederhana?)}.
\end{itemize}

\subsubsection{Tahap 3: (S) Selidiki secara Sintetis (25 menit)}
\begin{itemize}
\item \textbf{Aktivitas Guru:} Membagikan "\textbf{LKPD 14 - Jurnal Investigasi Dual-Lensa}".
    \begin{itemize}
    \item \textbf{Instruksi Guru:} "Setiap tim akan melakukan investigasi dari dua lensa. Gunakan HP atau sumber yang disediakan untuk mencari jawabannya. Bagilah tugas dalam tim!"
    \item \textbf{Lensa Sains:} Buka link video bit.ly/animasi-pembiasan untuk memahami hukum Snellius, indeks bias, dan cara kerja lensa. Lakukan juga eksperimen sederhana dengan gelas air dan lensa cembung.
    \item \textbf{Lensa Etnosains/Kultural:} Buka link artikel bit.ly/kaca-tradisional untuk memahami sejarah dan penggunaan alat optik dalam kerajinan dan kehidupan sehari-hari masyarakat Minangkabau.
    \item \textit{(Clue: Pastikan sumber belajar sudah disiapkan dan alat eksperimen tersedia. Berkelilinglah untuk memastikan setiap kelompok melakukan eksperimen dengan benar dan mencatat hasil pengamatan).}
    \end{itemize}
\item \textbf{Aktivitas Siswa:} Dalam kelompok, siswa berbagi tugas mencari informasi dari sumber yang diberikan dan mencatat temuan kunci di dua kolom terpisah pada "\textbf{LKPD 14 - Jurnal Investigasi Dual-Lensa}".
\end{itemize}

\subsubsection{Tahap 4: (A) Asimilasi Analitis (20 menit)}
\begin{itemize}
\item \textbf{Aktivitas Guru:} Memfasilitasi diskusi untuk menjembatani kedua lensa.
    \begin{itemize}
    \item \textbf{Pertanyaan Pancingan Kunci untuk Guru:}
        \begin{itemize}
        \item "Oke, dari Lensa Sains kita tahu cahaya berbelok karena perbedaan kecepatan di medium berbeda. Dari Lensa Etnosains, kita tahu nenek moyang kita menggunakan air jernih dan kaca untuk memperbesar benda. Nah, coba hubungkan! Bagaimana prinsip pembiasan ini diterapkan dalam alat optik tradisional?"
        \item "Dari Lensa Sains, kita tahu lensa cembung mengumpulkan cahaya. Dari Lensa Budaya, pengrajin songket menggunakan kaca pembesar untuk melihat detail benang. Apa hubungannya dengan kualitas hasil kerajinan mereka?"
        \end{itemize}
    \item \textit{(Clue: Fokuskan untuk membuat siswa 'menemukan' hubungannya sendiri. Gunakan kata "menurut kalian", "kira-kira bagaimana", "ada yang bisa menjelaskan?").}
    \end{itemize}
\item \textbf{Aktivitas Siswa:} Berdiskusi intensif untuk menghubungkan temuan sains dan budaya. Menuliskan kesimpulan terpadu (sintesis) mereka di kertas plano.
\end{itemize}

\subsubsection{Tahap 5: (N) Nyatakan Pemahaman (15 menit)}
\begin{itemize}
\item \textbf{Aktivitas Guru:} Memberikan studi kasus individual atau per kelompok.
    \begin{itemize}
    \item \textbf{Instruksi Guru (tuliskan di papan tulis):}
    
    "\textbf{STUDI KASUS UNTUK AHLI OPTIK SAINS-BUDAYA:}
    
    Kakekmu adalah seorang pengrajin emas yang sudah tua dan matanya mulai rabun. Dia kesulitan melihat detail halus saat membuat perhiasan tradisional. Kakekmu meminta bantuanmu untuk mencari solusi agar bisa tetap berkarya dengan kualitas terbaik, tapi dia tidak mau menggunakan kacamata modern.
    
    Tugasmu: Tuliskan sebuah saran singkat (3-5 kalimat) di buku latihanmu untuk kakekmu. Gunakan pengetahuan gabungan dari sains (pembiasan cahaya, lensa cembung) dan kearifan lokal yang baru saja kamu pelajari untuk memberikan solusi optik tradisional yang efektif."
    \end{itemize}
\item \textbf{Aktivitas Siswa:} Menyusun argumen tertulis untuk menjawab studi kasus yang diberikan, menggunakan bukti dari kedua lensa.
\end{itemize}

\subsection{Kegiatan Penutup (15 Menit)}
\begin{itemize}
\item \textbf{Presentasi \& Penguatan:} Guru meminta 2-3 siswa secara acak untuk membacakan jawaban studi kasus mereka. Guru memberikan pujian dan penguatan positif, menekankan betapa hebatnya argumen yang memadukan sains dan budaya.
\item \textbf{Refleksi:}
    \begin{itemize}
    \item \textbf{Instruksi Guru:} "Sekarang, kembali ke lembar K-W-L kalian. Lengkapi kolom terakhir, \textbf{L (Learned)}, dengan hal-hal baru yang paling berkesan yang kalian pelajari hari ini."
    \item \textbf{Instruksi Guru:} "Angkat tangan, siapa yang setelah belajar hari ini jadi lebih memahami mengapa benda di dalam air tampak berbeda?"
    \end{itemize}
\item \textbf{Tindak Lanjut:}
    \begin{itemize}
    \item \textbf{Instruksi Guru:} "Luar biasa, para ahli optik! Hari ini kita sudah membongkar rahasia 'pembelokan cahaya'. Pertemuan berikutnya, kita akan menyelidiki bagaimana cahaya putih sebenarnya tersusun dari berbagai warna, dan bagaimana fenomena ini tercermin dalam keindahan alam Minangkabau. Siapa yang pernah melihat pelangi di atas Danau Maninjau? Kita akan temukan rahasianya!"
    \item \textit{(Clue: Kaitkan secara eksplisit dengan pembelajaran hari ini untuk membangun alur narasi yang berkelanjutan. Ini membuat siswa merasa belajar IPA seperti mengikuti sebuah cerita bersambung).}
    \end{itemize}
\item Guru menutup pelajaran dengan doa dan salam.
\end{itemize}

\section{Asesmen (Penilaian)}

\begin{itemize}
\item \textbf{Asesmen Diagnostik (Awal):} Analisis lembar K-W-L. (Untuk mengetahui baseline siswa).
\item \textbf{Asesmen Formatif (Proses):} Observasi keaktifan diskusi (gotong royong) dan penilaian kelengkapan "\textbf{LKPD 14 - Jurnal Investigasi Dual-Lensa}".
\item \textbf{Asesmen Sumatif (Akhir Siklus):} Penilaian jawaban studi kasus (Tahap N) menggunakan rubrik.
\end{itemize}

\subsection{Rubrik Penilaian Jawaban Studi Kasus (Tahap N)}

\begin{center}
\begin{tabular}{|p{3cm}|p{3cm}|p{3cm}|p{3cm}|p{3cm}|}
\hline
\textbf{Kriteria Penilaian} & \textbf{Skor 4 (Sangat Baik)} & \textbf{Skor 3 (Baik)} & \textbf{Skor 2 (Cukup)} & \textbf{Skor 1 (Kurang)} \\
\hline
\textbf{Ketepatan Konsep Ilmiah} & Menggunakan istilah ilmiah (pembiasan, indeks bias, lensa cembung, titik fokus) dengan sangat tepat dan relevan dengan kasus. & Menggunakan istilah ilmiah dengan tepat, namun kurang relevan. & Menggunakan istilah ilmiah namun ada beberapa kesalahan konsep. & Tidak menggunakan istilah ilmiah atau salah total. \\
\hline
\textbf{Keterkaitan dengan Etnosains} & Mampu menghubungkan secara logis dan eksplisit antara praktik budaya (alat optik tradisional) dengan penjelasan ilmiahnya secara mendalam. & Mampu menghubungkan praktik budaya dengan penjelasan ilmiah, namun kurang mendalam. & Hanya menyebutkan praktik budaya tanpa menghubungkan dengan sains, atau sebaliknya. & Tidak ada keterkaitan antara sains dan budaya yang ditunjukkan. \\
\hline
\textbf{Kelogisan \& Struktur Argumen} & Penjelasan sangat logis, runtut, persuasif, dan mudah dipahami. & Penjelasan logis dan runtut, namun kurang persuasif. & Alur penjelasan kurang runtut atau sulit dipahami. & Penjelasan tidak logis dan tidak terstruktur. \\
\hline
\end{tabular}
\end{center}

\vspace{1cm}

\section{Daftar Pustaka Etnosains}
\begin{itemize}
\item Navis, A.A. (1984). \textit{Alam Terkembang Jadi Guru: Adat dan Kebudayaan Minangkabau}. Jakarta: Grafiti Pers.
\item Hakimy, I. (1994). \textit{Rangkaian Mustika Adat Basandi Syarak di Minangkabau}. Bandung: Remaja Rosdakarya.
\item Soekmono, R. (1973). \textit{Pengantar Sejarah Kebudayaan Indonesia}. Yogyakarta: Kanisius.
\item Couto, N. (2008). \textit{Budaya Visual dan Teknologi Tradisional Minangkabau}. Padang: Andalas University Press.
\end{itemize}

\end{document}