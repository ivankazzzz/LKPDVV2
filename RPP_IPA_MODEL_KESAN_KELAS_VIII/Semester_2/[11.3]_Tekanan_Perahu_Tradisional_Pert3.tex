\documentclass[12pt,a4paper]{article}
\usepackage[margin=1.5cm]{geometry}
\usepackage[utf8]{inputenc}
\usepackage{amsmath}
\usepackage{amsfonts}
\usepackage{amssymb}
\usepackage{graphicx}
\usepackage{xcolor}
\usepackage{tcolorbox}
\usepackage{enumitem}
\usepackage{multicol}
\usepackage{fancyhdr}
\usepackage{titlesec}
\usepackage{array}
\usepackage{longtable}
\usepackage{booktabs}

% Setup header and footer
\pagestyle{fancy}
\fancyhf{}
\fancyhead[L]{\textbf{RPP IPA Kelas VIII - Model KESAN}}
\fancyhead[R]{\textbf{Semester Genap 2025/2026}}
\fancyfoot[C]{\thepage}

% Define colors
\definecolor{primarycolor}{RGB}{0,0,0}
\definecolor{secondarycolor}{RGB}{255,255,255}

% Custom tcolorbox styles
\tcbset{
    mystyle/.style={
        colback=white,
        colframe=black,
        boxrule=1pt,
        arc=2pt,
        left=3pt,
        right=3pt,
        top=3pt,
        bottom=3pt
    }
}

\begin{document}

\begin{center}
\textbf{\Large MODUL AJAR: Kearifan Perahu Tradisional \& Hukum Pascal dalam Navigasi}
\end{center}

\vspace{0.5cm}

\begin{tcolorbox}[mystyle]
\textbf{Nama Penyusun:} Irfan Ananda \\
\textbf{Institusi:} SMP \\
\textbf{Mata Pelajaran:} Ilmu Pengetahuan Alam (IPA) \\
\textbf{Tahun Ajaran:} 2025/2026 \\
\textbf{Semester:} Genap \\
\textbf{Jenjang Sekolah:} SMP \\
\textbf{Kelas/Fase:} VIII / D \\
\textbf{Alokasi waktu:} 3 x 40 Menit (1 Pertemuan)
\end{tcolorbox}

\section{DIMENSI PROFIL PELAJAR PANCASILA}
\textit{(Clue untuk Guru: Sebutkan dimensi ini secara eksplisit saat apersepsi agar siswa sadar tujuan non-akademis yang sedang mereka kembangkan).}

\begin{itemize}
\item \textbf{Berkebinekaan Global:} Mengenal dan menghargai budaya, khususnya menganalisis kearifan lokal pembuatan perahu tradisional Minangkabau dalam menerapkan prinsip-prinsip tekanan, hukum Pascal, dan gaya apung, lalu menghubungkannya dengan konsep fisika universal dalam navigasi.
\item \textbf{Bernalar Kritis:} Ditempa secara intensif saat menganalisis informasi dari sumber sains dan budaya (Tahap S), menyintesiskan kedua perspektif (Tahap A), dan menyusun argumen berbasis bukti (Tahap N).
\item \textbf{Gotong Royong:} Kemampuan untuk bekerja sama secara kolaboratif dalam kelompok untuk merumuskan masalah (Tahap E), melakukan investigasi (Tahap S), dan membangun pemahaman bersama (Tahap A).
\item \textbf{Kreatif:} Menghasilkan argumen atau solusi orisinal yang terintegrasi pada tahap akhir pembelajaran (Tahap N).
\end{itemize}

\section{Sarana dan Prasarana}

\begin{itemize}
\item \textbf{Media:} LKPD 13 - Jurnal Investigasi Dual-Lensa, video singkat pembuatan perahu tradisional Minangkabau (durasi 2-3 menit), gambar/diagram bentuk perahu dan prinsip gaya apung, artikel/infografis hukum Pascal dan tekanan fluida, model perahu sederhana untuk demonstrasi.
\item \textbf{Alat:} Papan tulis/whiteboard, spidol, Proyektor \& Speaker, Kertas Plano atau Karton (1 per kelompok), sticky notes warna-warni, bak air, perahu mainan, beban kecil untuk eksperimen.
\item \textbf{Sumber Belajar:} Buku ajar IPA kelas VIII, Tautan video animasi hukum Pascal (misal: bit.ly/pascal-law), tautan artikel/video tradisi bahari Minangkabau (misal: bit.ly/perahu-minang).
\end{itemize}

\section{Target Peserta didik}

\begin{itemize}
\item Peserta didik reguler kelas VIII (Fase D).
\end{itemize}

\section{Model Pembelajaran}

\begin{itemize}
\item Model Pembelajaran KESAN (Konektivitas Etnosains-Sains).
\end{itemize}

\section{Pemahaman Bermakna}
\textit{(Clue untuk Guru: Bacakan atau sampaikan narasi ini dengan intonasi yang menarik di akhir pembelajaran untuk mengikat semua pengalaman belajar siswa menjadi satu kesatuan yang bermakna).}

\begin{tcolorbox}[mystyle]
"Ananda Semua, hari ini kita telah menyelami kearifan para pembuat perahu tradisional Minangkabau! Kita menemukan bahwa mereka telah menguasai hukum Pascal, prinsip gaya apung Archimedes, dan distribusi tekanan fluida dengan sangat mahir. Desain perahu yang ramping di depan dan melebar di tengah, penggunaan kayu yang tepat, hingga teknik menyeimbangkan muatan—semuanya didasarkan pada pemahaman intuitif tentang fisika fluida. Nenek moyang kita telah menjadi 'fisikawan air' yang handal tanpa pernah membuka buku fisika! Dengan ini, kita semakin yakin bahwa sains sejati lahir dari pengamatan alam dan pengalaman hidup yang mendalam."
\end{tcolorbox}

\section{PERTEMUAN 3: Hukum Pascal dalam Desain Perahu Tradisional}

\subsection{Capaian Pembelajaran (Fase D)}
Menganalisis ragam gerak, gaya, dan tekanan dalam konteks kehidupan sehari-hari dan fenomena alam, khususnya hukum Pascal dan aplikasinya dalam teknologi tradisional.

\subsection{Tujuan Pembelajaran Berbasis Sintaks KESAN}
\textit{(Clue untuk Guru: Setiap TP ini adalah checkpoint yang harus Anda pastikan tercapai di setiap tahap)}

Melalui Model Pembelajaran KESAN, peserta didik mampu:

\begin{enumerate}
\item \textbf{(Tahap K)} Mengidentifikasi dan mengapresiasi kearifan pembuatan perahu tradisional Minangkabau dalam menerapkan prinsip-prinsip hukum Pascal, gaya apung, dan distribusi tekanan fluida.
\item \textbf{(Tahap E)} Merumuskan pertanyaan investigatif tentang hubungan antara desain perahu tradisional dan konsep hukum Pascal dalam fisika fluida.
\item \textbf{(Tahap S)} Menganalisis secara sistematis prinsip hukum Pascal dari perspektif sains modern dan kearifan pembuatan perahu tradisional Minangkabau.
\item \textbf{(Tahap A)} Menyintesiskan pemahaman tentang konsep hukum Pascal dengan mengintegrasikan pengetahuan fisika dan kearifan teknologi bahari tradisional.
\item \textbf{(Tahap N)} Menerapkan pemahaman terintegrasi tentang hukum Pascal untuk menganalisis dan merancang teknologi transportasi air yang efisien dalam konteks modern.
\end{enumerate}

\subsection{Pertanyaan Pemantik Berbasis Model KESAN}
\textit{(Clue untuk Guru: Ajukan secara berurutan dengan jeda reflektif. Biarkan ketegangan kognitif terbangun sebelum masuk ke tahap investigasi)}

\begin{itemize}
\item "Mengapa perahu tradisional Minangkabau yang dibuat tanpa teknologi modern bisa berlayar dengan stabil dan menahan beban berat tanpa tenggelam?"
\item "Apa rahasia di balik bentuk perahu yang ramping di depan dan melebar di tengah? Bagaimana hal ini berhubungan dengan prinsip tekanan dan gaya dalam fluida?"
\end{itemize}

\section{LANGKAH-LANGKAH KEGIATAN PEMBELAJARAN BERBASIS SINTAKS KESAN}

\subsection{Kegiatan Pembuka (15 Menit)}

\begin{enumerate}
\item \textbf{Orientasi} (3 menit): Salam, doa, presensi, dan pengkondisian kelas
\item \textbf{Asesmen Diagnostik} (7 menit): K-W-L Chart untuk mengidentifikasi prior knowledge tentang hukum Pascal dan teknologi perahu
   \begin{itemize}
   \item \textbf{Instruksi Guru}: "Anak-anak, mari kita isi kolom K (Know) dengan apa yang kalian ketahui tentang perahu dan prinsip mengapung. Lalu isi kolom W (Want to know) dengan apa yang ingin kalian ketahui tentang hukum Pascal dan teknologi bahari tradisional."
   \end{itemize}
\item \textbf{Apersepsi \& Motivasi} (5 menit): Pengenalan dimensi PPP dan misi pembelajaran
\end{enumerate}

\subsection{Kegiatan Inti (90 Menit) - IMPLEMENTASI SINTAKS MODEL KESAN}

\subsubsection{Tahap 1: (K) Kaitkan Konteks Kultural (15 menit)}

\textbf{Aktivitas Guru:}
\begin{itemize}
\item Menampilkan video singkat tentang pembuatan dan penggunaan perahu tradisional Minangkabau
\item Mengajukan pertanyaan pemantik tentang kecanggihan teknologi bahari tradisional
\item Memfasilitasi sharing pengalaman siswa tentang perahu atau transportasi air yang pernah mereka gunakan
\item \textit{(Clue: Dorong siswa untuk berbagi pengalaman tentang naik perahu, kapal, atau mengamati kapal nelayan. Ciptakan rasa kagum terhadap kearifan bahari nenek moyang)}
\end{itemize}

\textbf{Aktivitas Siswa:}
\begin{itemize}
\item Mengamati video dan gambar perahu tradisional yang disajikan
\item Berbagi pengalaman pribadi tentang transportasi air yang pernah dialami
\item Menuliskan pertanyaan awal di sticky notes untuk "Papan Penasaran"
\end{itemize}

\subsubsection{Tahap 2: (E) Eksplorasi Enigma (15 menit)}

\textbf{Aktivitas Guru:}
\begin{itemize}
\item Membentuk kelompok heterogen (3-4 siswa)
\item \textbf{Instruksi Guru}: "Sekarang, dalam kelompok kalian, diskusikan dan rumuskan 'Misi Penyelidikan Tim' kalian. Apa yang ingin kalian selidiki tentang hubungan antara desain perahu tradisional dan konsep hukum Pascal? Tuliskan minimal 3 pertanyaan kunci yang akan menjadi fokus investigasi kelompok kalian."
\item \textit{(Clue: Pastikan pertanyaan yang dirumuskan menghubungkan desain perahu dengan prinsip fisika. Arahkan pada aspek stabilitas, daya apung, dan distribusi tekanan)}
\end{itemize}

\textbf{Aktivitas Siswa:}
\begin{itemize}
\item Berdiskusi dalam kelompok untuk merumuskan misi investigasi
\item Menuliskan minimal 3 pertanyaan kunci di kertas plano
\item Mempresentasikan rumusan masalah secara singkat
\end{itemize}

\subsubsection{Tahap 3: (S) Selidiki secara Sintetis (25 menit)}

\textbf{Aktivitas Guru:}
\begin{itemize}
\item Membagikan "LKPD 13 - Jurnal Investigasi Dual-Lensa"
\item \textbf{Instruksi Guru}: "Sekarang saatnya investigasi mendalam! Gunakan LKPD ini untuk mencatat temuan kalian dari dua perspektif. Di kolom 'Lensa Sains', catat semua informasi tentang hukum Pascal, gaya apung, dan prinsip fluida dari sumber-sumber ilmiah. Di kolom 'Lensa Etnosains', catat semua informasi tentang kearifan pembuatan perahu tradisional Minangkabau dari sumber-sumber budaya."
\item \textit{(Clue: Demonstrasikan hukum Pascal dengan suntikan dan air. Tunjukkan juga percobaan gaya apung dengan memasukkan benda ke air. Pastikan siswa memahami konsep sebelum menganalisis aplikasinya)}
\end{itemize}

\textbf{Aktivitas Siswa:}
\begin{itemize}
\item Melakukan investigasi dengan pembagian tugas dalam tim
\item Mencatat temuan dari perspektif sains dan etnosains secara terpisah pada LKPD
\item Mengumpulkan data tentang prinsip hukum Pascal dan kearifan pembuatan perahu tradisional
\end{itemize}

\subsubsection{Tahap 4: (A) Asimilasi Analitis (20 menit)}

\textbf{Aktivitas Guru:}
\begin{itemize}
\item \textbf{Pertanyaan Pancingan Kunci}: "Bagaimana hukum Pascal dan prinsip gaya apung yang kalian pelajari dapat menjelaskan mengapa desain perahu tradisional begitu efektif? Apa kesamaan prinsip antara teknologi perahu modern dan tradisional dalam memanfaatkan sifat-sifat fluida?"
\item \textit{(Clue: Dorong siswa untuk mengidentifikasi aplikasi hukum Pascal dalam desain lambung perahu, distribusi muatan, dan sistem navigasi. Bantu mereka melihat bahwa prinsip fisika yang sama berlaku dalam kedua teknologi)}
\end{itemize}

\textbf{Aktivitas Siswa:}
\begin{itemize}
\item Berdiskusi intensif untuk menghubungkan temuan dari kedua lensa
\item Mengidentifikasi kesamaan prinsip fisika dalam teknologi tradisional dan modern
\item Menuliskan kesimpulan terpadu di kertas plano
\end{itemize}

\subsubsection{Tahap 5: (N) Nyatakan Pemahaman (15 menit)}

\textbf{Aktivitas Guru:}
\begin{itemize}
\item Memberikan studi kasus: "Sebuah perusahaan transportasi air ingin merancang kapal penumpang yang efisien dan ramah lingkungan untuk transportasi antar pulau. Mereka ingin mengadaptasi kearifan desain perahu tradisional dengan teknologi modern. Berdasarkan pemahaman kalian tentang hukum Pascal dan kearifan pembuatan perahu tradisional, berikan rekomendasi desain yang menggabungkan kedua pendekatan tersebut."
\item \textbf{Instruksi Guru}: "Tuliskan jawaban individual kalian dalam 3-5 kalimat. Gunakan bukti dari kedua lensa yang telah kalian investigasi untuk mendukung rekomendasi kalian."
\end{itemize}

\textbf{Aktivitas Siswa:}
\begin{itemize}
\item Menyusun rekomendasi desain kapal berdasarkan studi kasus
\item Mengintegrasikan pemahaman dari perspektif sains dan etnosains
\item Memberikan justifikasi yang didukung bukti dari investigasi
\end{itemize}

\subsection{Kegiatan Penutup (15 Menit)}

\begin{enumerate}
\item \textbf{Presentasi \& Penguatan} (7 menit): Beberapa siswa mempresentasikan rekomendasi mereka, guru memberikan penguatan konsep
\item \textbf{Refleksi} (5 menit): Melengkapi kolom L (Learned) pada K-W-L Chart, refleksi pembelajaran
\item \textbf{Tindak Lanjut \& Penutup} (3 menit): Preview pembelajaran berikutnya tentang getaran dan gelombang dengan narasi berkelanjutan tentang perahu dan gelombang laut
\end{enumerate}

\section{ASESMEN BERBASIS MODEL KESAN}

\subsection{Asesmen Diagnostik (Tahap K)}
\begin{itemize}
\item \textbf{Instrumen}: K-W-L Chart
\item \textbf{Tujuan}: Mengidentifikasi pengetahuan awal siswa tentang hukum Pascal dan teknologi perahu
\end{itemize}

\subsection{Asesmen Formatif (Tahap E-S-A)}
\begin{itemize}
\item \textbf{Tahap E}: Penilaian kualitas "Misi Penyelidikan" kelompok
\item \textbf{Tahap S}: Kelengkapan dan kualitas "Jurnal Investigasi Dual-Lensa"
\item \textbf{Tahap A}: Observasi diskusi dan kemampuan sintesis kelompok
\end{itemize}

\subsection{Asesmen Sumatif (Tahap N)}
\begin{itemize}
\item \textbf{Instrumen}: Rekomendasi desain kapal dengan rubrik 4 level
\item \textbf{Aspek}: Ketepatan konsep hukum Pascal, keterkaitan etnosains, kreativitas solusi
\end{itemize}

\section{RUBRIK PENILAIAN STUDI KASUS}

\begin{longtable}{|p{3cm}|p{3cm}|p{3cm}|p{3cm}|p{3cm}|}
\hline
\textbf{Kriteria Penilaian} & \textbf{Skor 4 (Sangat Baik)} & \textbf{Skor 3 (Baik)} & \textbf{Skor 2 (Cukup)} & \textbf{Skor 1 (Kurang)} \\
\hline
\textbf{Ketepatan Konsep Hukum Pascal} & Menjelaskan hukum Pascal dengan tepat dan mengaitkannya secara akurat dengan desain kapal yang direkomendasikan & Menjelaskan hukum Pascal dengan cukup tepat dan mengaitkannya dengan desain kapal meski ada beberapa kekeliruan minor & Menjelaskan hukum Pascal secara umum namun kaitan dengan desain kapal kurang jelas atau mengandung kesalahan konsep & Penjelasan hukum Pascal tidak tepat atau tidak dikaitkan dengan konteks desain kapal \\
\hline
\textbf{Keterkaitan dengan Etnosains} & Mengidentifikasi dengan jelas kearifan pembuatan perahu tradisional dan mengintegrasikannya secara relevan dalam rekomendasi modern & Mengidentifikasi kearifan perahu tradisional dan cukup berhasil mengintegrasikannya dalam rekomendasi meski kurang detail & Menyebutkan aspek tradisional namun integrasinya dengan rekomendasi modern masih superfisial & Tidak mengaitkan atau salah mengaitkan kearifan tradisional dengan rekomendasi modern \\
\hline
\textbf{Kreativitas \& Kelayakan Solusi} & Rekomendasi sangat kreatif, inovatif, dan realistis untuk diimplementasikan dengan mempertimbangkan aspek teknis dan budaya & Rekomendasi cukup kreatif dan layak dengan pertimbangan teknis yang memadai & Rekomendasi standar dengan kreativitas terbatas namun masih layak untuk dipertimbangkan & Rekomendasi tidak kreatif, tidak realistis, atau tidak layak untuk diimplementasikan \\
\hline
\end{longtable}

\vspace{1cm}

\section{DAFTAR PUSTAKA ETNOSAINS}

\begin{enumerate}
\item Navis, A.A. (1984). \textit{Alam Terkembang Jadi Guru: Adat dan Kebudayaan Minangkabau}. Jakarta: Grafiti Pers.
\item Kato, Tsuyoshi. (2005). \textit{Adat Minangkabau dan Merantau dalam Perspektif Sejarah}. Jakarta: Balai Pustaka.
\item Graves, Elizabeth E. (2007). \textit{Asal-usul Elite Minangkabau Modern: Respons terhadap Kolonial Belanda Abad XIX/XX}. Jakarta: Yayasan Obor Indonesia.
\item Mansoer, M.D., dkk. (1970). \textit{Sejarah Minangkabau}. Jakarta: Bhratara.
\item Dokumentasi Etnografis Perahu Tradisional Sumatra Barat. (2018). Museum Adityawarman Padang.
\end{enumerate}

\end{document}