\documentclass[12pt,a4paper]{article}
\usepackage[margin=1.5cm]{geometry}
\usepackage[utf8]{inputenc}
\usepackage{amsmath}
\usepackage{amsfonts}
\usepackage{amssymb}
\usepackage{graphicx}
\usepackage{xcolor}
\usepackage{tcolorbox}
\usepackage{enumitem}
\usepackage{multicol}
\usepackage{fancyhdr}
\usepackage{titlesec}
\usepackage{array}
\usepackage{longtable}
\usepackage{booktabs}

% Setup header and footer
\pagestyle{fancy}
\fancyhf{}
\fancyhead[L]{\textbf{RPP IPA Kelas VIII - Model KESAN}}
\fancyhead[R]{\textbf{Semester Genap 2025/2026}}
\fancyfoot[C]{\thepage}

% Define colors
\definecolor{primarycolor}{RGB}{0,0,0}
\definecolor{secondarycolor}{RGB}{255,255,255}

% Custom tcolorbox styles
\tcbset{
    mystyle/.style={
        colback=white,
        colframe=black,
        boxrule=1pt,
        arc=2pt,
        left=3pt,
        right=3pt,
        top=3pt,
        bottom=3pt
    }
}

\begin{document}

\begin{center}
\textbf{\Large MODUL AJAR: Riak di Kolam \& Misteri Gelombang Air dalam Permainan Tradisional}
\end{center}

\vspace{0.5cm}

\begin{tcolorbox}[mystyle]
\textbf{Nama Penyusun:} Irfan Ananda \\
\textbf{Institusi:} SMP \\
\textbf{Mata Pelajaran:} Ilmu Pengetahuan Alam (IPA) \\
\textbf{Tahun Ajaran:} 2025/2026 \\
\textbf{Semester:} Genap \\
\textbf{Jenjang Sekolah:} SMP \\
\textbf{Kelas/Fase:} VIII / D \\
\textbf{Alokasi waktu:} 3 x 40 Menit (1 Pertemuan)
\end{tcolorbox}

\section{DIMENSI PROFIL PELAJAR PANCASILA}
\textit{(Clue untuk Guru: Sebutkan dimensi ini secara eksplisit saat apersepsi agar siswa sadar tujuan non-akademis yang sedang mereka kembangkan).}

\begin{itemize}
\item \textbf{Berkebinekaan Global:} Mengenal dan menghargai budaya, khususnya menganalisis kearifan lokal permainan tradisional anak nagari Minangkabau yang melibatkan air (seperti bermain batu dan gelombang di sungai/kolam) dalam memahami prinsip-prinsip gelombang transversal dan longitudinal, lalu menghubungkannya dengan konsep fisika universal.
\item \textbf{Bernalar Kritis:} Ditempa secara intensif saat menganalisis informasi dari sumber sains dan budaya (Tahap S), menyintesiskan kedua perspektif (Tahap A), dan menyusun argumen berbasis bukti (Tahap N).
\item \textbf{Gotong Royong:} Kemampuan untuk bekerja sama secara kolaboratif dalam kelompok untuk merumuskan masalah (Tahap E), melakukan investigasi (Tahap S), dan membangun pemahaman bersama (Tahap A).
\item \textbf{Kreatif:} Menghasilkan argumen atau solusi orisinal yang terintegrasi pada tahap akhir pembelajaran (Tahap N).
\end{itemize}

\section{Sarana dan Prasarana}

\begin{itemize}
\item \textbf{Media:} LKPD 16 - Jurnal Investigasi Dual-Lensa, video singkat permainan anak nagari di sungai/kolam dengan fokus pada riak air (durasi 2-3 menit), gambar/diagram gelombang transversal dan longitudinal, artikel/infografis sifat-sifat gelombang (amplitudo, frekuensi, panjang gelombang, cepat rambat), baskom air untuk demonstrasi.
\item \textbf{Alat:} Papan tulis/whiteboard, spidol, Proyektor \& Speaker, Kertas Plano atau Karton (1 per kelompok), sticky notes warna-warni, baskom/nampan besar berisi air, batu kecil, penggaris, tali/slinki, stopwatch (smartphone).
\item \textbf{Sumber Belajar:} Buku ajar IPA kelas VIII, Tautan video animasi gelombang air (misal: bit.ly/animasi-gelombang), tautan artikel/video permainan tradisional anak nagari Minangkabau (misal: bit.ly/permainan-nagari).
\end{itemize}

\section{Target Peserta didik}

\begin{itemize}
\item Peserta didik reguler kelas VIII (Fase D).
\end{itemize}

\section{Model Pembelajaran}

\begin{itemize}
\item Model Pembelajaran KESAN (Konektivitas Etnosains-Sains).
\end{itemize}

\section{Pemahaman Bermakna}
\textit{(Clue untuk Guru: Bacakan atau sampaikan narasi ini dengan intonasi yang menarik di akhir pembelajaran untuk mengikat semua pengalaman belajar siswa menjadi satu kesatuan yang bermakna).}

\begin{tcolorbox}[mystyle]
"Ananda Semua, hari ini kita telah mengungkap keajaiban di balik permainan sederhana anak nagari yang melibatkan air! Kita menemukan bahwa ketika anak-anak melempar batu ke sungai atau kolam, mereka sebenarnya sedang melakukan eksperimen fisika yang luar biasa. Riak air yang mereka ciptakan mengandung semua prinsip gelombang: amplitudo, frekuensi, panjang gelombang, dan cepat rambat. Para leluhur Minangkabau tanpa sadar telah mengajarkan anak-anak mereka memahami alam melalui permainan yang mengandung sains. Dengan ini, kita sadar bahwa pembelajaran fisika tidak hanya ada di laboratorium, tetapi juga tersembunyi dalam kearifan permainan tradisional yang mengajarkan anak-anak mengamati dan memahami fenomena alam!"
\end{tcolorbox}

\section{PERTEMUAN 3: Gelombang Air dalam Permainan Tradisional Anak Nagari}

\subsection{Capaian Pembelajaran (Fase D)}
Menganalisis getaran, gelombang, dan bunyi dalam konteks teknologi dan budaya, termasuk karakteristik gelombang dan perambatannya pada berbagai media.

\subsection{Tujuan Pembelajaran Berbasis Sintaks KESAN}
\textit{(Clue untuk Guru: Setiap TP ini adalah checkpoint yang harus Anda pastikan tercapai di setiap tahap)}

Melalui Model Pembelajaran KESAN, peserta didik mampu:

\begin{enumerate}
\item \textbf{(Tahap K)} Mengidentifikasi dan mengapresiasi kearifan permainan tradisional anak nagari Minangkabau yang melibatkan fenomena gelombang air (melempar batu, bermain perahu kertas) sebagai sarana pembelajaran alam.
\item \textbf{(Tahap E)} Merumuskan pertanyaan investigatif tentang hubungan antara permainan tradisional dengan air dan konsep gelombang (amplitudo, frekuensi, panjang gelombang, cepat rambat).
\item \textbf{(Tahap S)} Menganalisis secara sistematis sifat-sifat gelombang air dari perspektif sains modern dan kearifan permainan tradisional anak nagari Minangkabau.
\item \textbf{(Tahap A)} Menyintesiskan pemahaman tentang gelombang dengan mengintegrasikan konsep fisika dan nilai edukatif permainan tradisional.
\item \textbf{(Tahap N)} Menerapkan pemahaman untuk menganalisis fenomena gelombang dalam aktivitas sehari-hari dan merancang permainan edukatif berbasis sains gelombang.
\end{enumerate}

\subsection{Pertanyaan Pemantik Berbasis Model KESAN}
\textit{(Clue untuk Guru: Ajukan secara berurutan dengan jeda reflektif. Biarkan ketegangan kognitif terbangun sebelum masuk ke tahap investigasi)}

\begin{itemize}
\item "Mengapa riak air yang dihasilkan dari lemparan batu membentuk lingkaran yang semakin membesar? Apa yang terjadi pada kekuatan riak saat menjauhi titik asal?"
\item "Pernahkah kalian memperhatikan bahwa batu besar dan batu kecil menghasilkan riak yang berbeda? Apa hubungannya dengan energi dan sifat gelombang?"
\end{itemize}

\section{LANGKAH-LANGKAH KEGIATAN PEMBELAJARAN BERBASIS SINTAKS KESAN}

\subsection{Kegiatan Pembuka (15 Menit)}

\begin{enumerate}
\item \textbf{Orientasi} (3 menit): Salam, doa, presensi, dan pengkondisian kelas
\item \textbf{Asesmen Diagnostik} (7 menit): K-W-L Chart untuk mengidentifikasi prior knowledge tentang gelombang air dan permainan tradisional
   \begin{itemize}
   \item \textbf{Instruksi Guru}: "Anak-anak, mari kita isi kolom K (Know) dengan apa yang kalian ketahui tentang gelombang air dan permainan di sungai atau kolam. Lalu isi kolom W (Want to know) dengan apa yang ingin kalian ketahui tentang riak air dan bagaimana gelombang merambat."
   \end{itemize}
\item \textbf{Apersepsi \& Motivasi} (5 menit): Pengenalan dimensi PPP dan misi pembelajaran
\end{enumerate}

\subsection{Kegiatan Inti (90 Menit) - IMPLEMENTASI SINTAKS MODEL KESAN}

\subsubsection{Tahap 1: (K) Kaitkan Konteks Kultural (15 menit)}

\textbf{Aktivitas Guru:}
\begin{itemize}
\item Menampilkan video singkat permainan anak nagari di sungai/kolam dengan fokus pada riak air yang terbentuk
\item Mengajukan pertanyaan pemantik tentang fenomena riak air dalam permainan tradisional
\item Memfasilitasi sharing pengalaman siswa tentang bermain di sungai, kolam, atau bahkan di bathtub
\item \textit{(Clue: Buat demonstrasi langsung dengan baskom air dan batu kecil. Minta siswa mengamati pola riak yang terbentuk dan seberapa jauh riak merambat)}
\end{itemize}

\textbf{Aktivitas Siswa:}
\begin{itemize}
\item Mengamati video dan demonstrasi riak air yang disajikan
\item Berbagi pengalaman pribadi tentang bermain air dan mengamati riak
\item Menuliskan observasi awal dan pertanyaan di sticky notes untuk "Papan Penasaran"
\end{itemize}

\subsubsection{Tahap 2: (E) Eksplorasi Enigma (15 menit)}

\textbf{Aktivitas Guru:}
\begin{itemize}
\item Membentuk kelompok heterogen (3-4 siswa)
\item \textbf{Instruksi Guru}: "Sekarang, dalam kelompok kalian, rumuskan 'Misi Penyelidikan Tim' tentang bagaimana sifat-sifat gelombang air dapat dijelaskan melalui permainan tradisional yang melibatkan air. Tuliskan minimal 3 pertanyaan kunci yang akan kalian investigasi."
\item \textit{(Clue: Arahkan pertanyaan ke hubungan antara ukuran gangguan dengan amplitudo gelombang, kecepatan rambat gelombang, dan energi yang dibawa oleh gelombang)}
\end{itemize}

\textbf{Aktivitas Siswa:}
\begin{itemize}
\item Berdiskusi dalam kelompok untuk merumuskan misi investigasi
\item Menuliskan minimal 3 pertanyaan kunci di kertas plano
\item Mempresentasikan rumusan masalah secara singkat
\end{itemize}

\subsubsection{Tahap 3: (S) Selidiki secara Sintetis (25 menit)}

\textbf{Aktivitas Guru:}
\begin{itemize}
\item Membagikan "LKPD 16 - Jurnal Investigasi Dual-Lensa"
\item \textbf{Instruksi Guru}: "Gunakan LKPD ini untuk mencatat temuan dari dua perspektif. Di kolom 'Lensa Sains', catat konsep gelombang (amplitudo, frekuensi, panjang gelombang, cepat rambat, interferensi, difraksi). Di kolom 'Lensa Etnosains', catat kearifan permainan tradisional anak nagari yang melibatkan air dan observasi fenomena alam."
\item \textit{(Clue: Lakukan eksperimen sederhana dengan baskom air. Buat gelombang dengan frekuensi berbeda, amati interferensi saat dua gelombang bertemu, dan tunjukkan difraksi gelombang melewati rintangan)}
\end{itemize}

\textbf{Aktivitas Siswa:}
\begin{itemize}
\item Melakukan investigasi dalam kelompok dengan pembagian tugas
\item Mengumpulkan data tentang sifat gelombang air dan kearifan permainan tradisional
\item Mencatat temuan di dua kolom terpisah pada LKPD
\end{itemize}

\subsubsection{Tahap 4: (A) Asimilasi Analitis (20 menit)}

\textbf{Aktivitas Guru:}
\begin{itemize}
\item Memfasilitasi diskusi untuk menjembatani kedua lensa
\item \textbf{Pertanyaan Pancingan Kunci}: "Bagaimana konsep amplitudo, frekuensi, dan energi gelombang yang kalian pelajari dapat menjelaskan mengapa riak dari batu besar berbeda dengan riak dari batu kecil? Apa nilai edukatif dari permainan tradisional ini dalam mengajarkan anak-anak memahami alam?"
\item \textit{(Clue: Bantu siswa menghubungkan bahwa permainan tradisional sebenarnya mengajarkan pengamatan ilmiah secara alami. Dorong mereka mengidentifikasi bahwa leluhur sudah memahami prinsip sebab-akibat dalam fenomena alam)}
\end{itemize}

\textbf{Aktivitas Siswa:}
\begin{itemize}
\item Berdiskusi intensif untuk mengintegrasikan temuan dari kedua perspektif
\item Menganalisis hubungan antara konsep fisika dan kearifan permainan tradisional
\item Menuliskan kesimpulan terpadu di kertas plano
\end{itemize}

\subsubsection{Tahap 5: (N) Nyatakan Pemahaman (15 menit)}

\textbf{Aktivitas Guru:}
\begin{itemize}
\item Memberikan studi kasus kontekstual untuk aplikasi pemahaman
\item \textbf{Instruksi Guru}: "Berdasarkan pemahaman kalian tentang gelombang dan kearifan permainan tradisional, rancang sebuah permainan edukatif untuk anak-anak SD yang dapat mengajarkan konsep gelombang. Jelaskan aturan permainan dan konsep fisika apa yang bisa dipelajari."
\end{itemize}

\textbf{Aktivitas Siswa:}
\begin{itemize}
\item Menyusun rancangan permainan edukatif berbasis konsep gelombang
\item Menuliskan argumen ilmiah yang mendasari rancangan permainan
\item Mempresentasikan ide kreatif mereka
\end{itemize}

\subsection{Kegiatan Penutup (15 Menit)}

\begin{enumerate}
\item \textbf{Presentasi \& Penguatan} (7 menit): Beberapa kelompok mempresentasikan rancangan permainan edukatif, guru memberikan penguatan tentang integrasi sains dan budaya
\item \textbf{Refleksi} (5 menit): Melengkapi K-W-L Chart dan refleksi pembelajaran
\item \textbf{Tindak Lanjut \& Penutup} (3 menit): Preview pembelajaran berikutnya tentang cahaya dengan narasi berkelanjutan tentang pantulan cahaya di permukaan air
\end{enumerate}

\section{ASESMEN BERBASIS MODEL KESAN}

\subsection{1. Asesmen Diagnostik (Tahap K)}
\begin{itemize}
\item \textbf{Instrumen}: K-W-L Chart
\item \textbf{Tujuan}: Mengidentifikasi pengetahuan awal siswa tentang gelombang air dan permainan tradisional
\end{itemize}

\subsection{2. Asesmen Formatif (Tahap E-S-A)}
\begin{itemize}
\item \textbf{Tahap E}: Penilaian kualitas rumusan "Misi Penyelidikan Tim"
\item \textbf{Tahap S}: Kelengkapan "LKPD 16 - Jurnal Investigasi Dual-Lensa"  
\item \textbf{Tahap A}: Observasi diskusi dan kemampuan mengintegrasikan kedua perspektif
\end{itemize}

\subsection{3. Asesmen Sumatif (Tahap N)}
\begin{itemize}
\item \textbf{Instrumen}: Rancangan permainan edukatif dengan rubrik penilaian
\item \textbf{Aspek}: Ketepatan konsep gelombang, kreativitas rancangan, keterkaitan etnosains
\end{itemize}

\section{RUBRIK PENILAIAN RANCANGAN PERMAINAN EDUKATIF (TAHAP N)}

\begin{longtable}{|p{3cm}|p{3cm}|p{3cm}|p{3cm}|p{3cm}|}
\hline
\textbf{Kriteria Penilaian} & \textbf{Skor 4 (Sangat Baik)} & \textbf{Skor 3 (Baik)} & \textbf{Skor 2 (Cukup)} & \textbf{Skor 1 (Kurang)} \\
\hline
\textbf{Ketepatan Konsep Gelombang} & Mengintegrasikan konsep amplitudo, frekuensi, panjang gelombang, dan cepat rambat dengan sangat tepat dalam rancangan permainan & Menggunakan sebagian besar konsep gelombang dengan tepat meski ada kekeliruan minor & Menggunakan konsep gelombang secara umum namun kurang mendalam atau mengandung kesalahan & Tidak menggunakan konsep gelombang yang tepat atau salah total \\
\hline
\textbf{Kreativitas \& Inovasi Rancangan} & Rancangan permainan sangat kreatif, inovatif, dan menarik untuk anak-anak dengan aturan yang jelas dan mudah dipahami & Rancangan permainan cukup kreatif dan menarik dengan aturan yang jelas & Rancangan permainan sederhana namun masih dapat dijalankan dengan beberapa perbaikan aturan & Rancangan permainan tidak jelas, sulit dipahami, atau tidak dapat dijalankan \\
\hline
\textbf{Keterkaitan dengan Etnosains} & Sangat jelas mengintegrasikan nilai-nilai permainan tradisional dengan konsep modern, menunjukkan apresiasi mendalam terhadap kearifan lokal & Cukup baik mengintegrasikan aspek tradisional dengan konsep modern & Hanya menyebutkan aspek tradisional tanpa integrasi yang kuat dengan konsep sains & Tidak ada keterkaitan dengan permainan tradisional atau etnosains \\
\hline
\end{longtable}

\section{DAFTAR PUSTAKA ETNOSAINS MINANGKABAU}

\begin{enumerate}
\item Navis, A.A. (1984). \textit{Alam Terkembang Jadi Guru: Adat dan Kebudayaan Minangkabau}. Jakarta: Grafiti Pers.
\item Graves, Elizabeth E. (2007). \textit{Asal Usul Elite Minangkabau Modern: Respons Terhadap Kolonial Belanda Abad XIX/XX}. Jakarta: Yayasan Obor Indonesia.
\item Kato, Tsuyoshi. (2005). \textit{Adat Minangkabau dan Merantau dalam Perspektif Sejarah}. Jakarta: Balai Pustaka.
\item Hakimy, Idrus. (1994). \textit{Rangkaian Mustika Adat Basandi Syarak di Minangkabau}. Bandung: Remaja Rosdakarya.
\item Dokumentasi Permainan Tradisional Anak Nagari Minangkabau. (2020). \textit{Balai Pelestarian Nilai Budaya Sumatera Barat}.
\end{enumerate}

\end{document}