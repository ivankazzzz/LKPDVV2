\documentclass[12pt,a4paper]{article}
\usepackage[margin=1.5cm]{geometry}
\usepackage[utf8]{inputenc}
\usepackage{amsmath}
\usepackage{amsfonts}
\usepackage{amssymb}
\usepackage{graphicx}
\usepackage{xcolor}
\usepackage{tcolorbox}
\usepackage{enumitem}
\usepackage{multicol}
\usepackage{fancyhdr}
\usepackage{titlesec}
\usepackage{array}
\usepackage{longtable}
\usepackage{booktabs}

% Setup header and footer
\pagestyle{fancy}
\fancyhf{}
\fancyhead[L]{\textbf{RPP IPA Kelas VIII - Model KESAN}}
\fancyhead[R]{\textbf{Semester Genap 2025/2026}}
\fancyfoot[C]{\thepage}

% Define colors
\definecolor{primarycolor}{RGB}{0,0,0}
\definecolor{secondarycolor}{RGB}{255,255,255}

% Custom tcolorbox styles
\tcbset{
    mystyle/.style={
        colback=white,
        colframe=black,
        boxrule=1pt,
        arc=2pt,
        left=3pt,
        right=3pt,
        top=3pt,
        bottom=3pt
    }
}

\begin{document}

\begin{center}
\textbf{\Large MODUL AJAR: Rahasia Kokoh Rumah Gadang \& Misteri Tekanan di Sekitar Kita}
\end{center}

\vspace{0.5cm}

\begin{tcolorbox}[mystyle]
\textbf{Nama Penyusun:} Irfan Ananda \\
\textbf{Institusi:} SMP \\
\textbf{Mata Pelajaran:} Ilmu Pengetahuan Alam (IPA) \\
\textbf{Tahun Ajaran:} 2025/2026 \\
\textbf{Semester:} Genap \\
\textbf{Jenjang Sekolah:} SMP \\
\textbf{Kelas/Fase:} VIII / D \\
\textbf{Alokasi waktu:} 3 x 40 Menit (1 Pertemuan)
\end{tcolorbox}

\section{DIMENSI PROFIL PELAJAR PANCASILA}
\textit{(Clue untuk Guru: Sebutkan dimensi ini secara eksplisit saat apersepsi agar siswa sadar tujuan non-akademis yang sedang mereka kembangkan).}

\begin{itemize}
\item \textbf{Berkebinekaan Global:} Mengenal dan menghargai budaya, khususnya menganalisis kearifan lokal arsitektur Rumah Gadang Minangkabau dalam menerapkan prinsip-prinsip tekanan dan distribusi beban, lalu menghubungkannya dengan konsep fisika universal.
\item \textbf{Bernalar Kritis:} Ditempa secara intensif saat menganalisis informasi dari sumber sains dan budaya (Tahap S), menyintesiskan kedua perspektif (Tahap A), dan menyusun argumen berbasis bukti (Tahap N).
\item \textbf{Gotong Royong:} Kemampuan untuk bekerja sama secara kolaboratif dalam kelompok untuk merumuskan masalah (Tahap E), melakukan investigasi (Tahap S), dan membangun pemahaman bersama (Tahap A).
\item \textbf{Kreatif:} Menghasilkan argumen atau solusi orisinal yang terintegrasi pada tahap akhir pembelajaran (Tahap N).
\end{itemize}

\section{Sarana dan Prasarana}

\begin{itemize}
\item \textbf{Media:} LKPD 11 - Jurnal Investigasi Dual-Lensa, video singkat konstruksi Rumah Gadang (durasi 2-3 menit), gambar/diagram struktur Rumah Gadang, artikel/infografis prinsip tekanan dalam arsitektur tradisional, model/gambar tiang pancang dan pondasi tradisional.
\item \textbf{Alat:} Papan tulis/whiteboard, spidol, Proyektor \& Speaker, Kertas Plano atau Karton (1 per kelompok), sticky notes warna-warni, penggaris, kalkulator sederhana.
\item \textbf{Sumber Belajar:} Buku ajar IPA kelas VIII, Tautan video animasi konsep tekanan (misal: bit.ly/animasi-tekanan), tautan artikel/video filosofi arsitektur Rumah Gadang (misal: bit.ly/rumah-gadang-arsitektur).
\end{itemize}

\section{Target Peserta didik}

\begin{itemize}
\item Peserta didik reguler kelas VIII (Fase D).
\end{itemize}

\section{Model Pembelajaran}

\begin{itemize}
\item Model Pembelajaran KESAN (Konektivitas Etnosains-Sains).
\end{itemize}

\section{Pemahaman Bermakna}
\textit{(Clue untuk Guru: Bacakan atau sampaikan narasi ini dengan intonasi yang menarik di akhir pembelajaran untuk mengikat semua pengalaman belajar siswa menjadi satu kesatuan yang bermakna).}

\begin{tcolorbox}[mystyle]
"Ananda Semua, hari ini kita telah membongkar rahasia mengapa Rumah Gadang bisa berdiri kokoh selama ratusan tahun tanpa menggunakan paku atau semen modern! Kita menemukan bahwa nenek moyang Minangkabau telah menerapkan prinsip-prinsip tekanan, distribusi beban, dan gaya dengan sangat cerdas. Setiap tiang, setiap sambungan, setiap sudut atap dirancang berdasarkan pemahaman mendalam tentang fisika. Dengan ini, kita sadar bahwa sains tidak hanya ada di buku pelajaran, tapi juga terejawantah dalam karya arsitektur warisan budaya kita yang menakjubkan."
\end{tcolorbox}

\section{PERTEMUAN 1: Tekanan dalam Arsitektur Rumah Gadang}

\subsection{Capaian Pembelajaran (Fase D)}
Menganalisis ragam gerak, gaya, dan tekanan dalam konteks kehidupan sehari-hari dan fenomena alam.

\subsection{Tujuan Pembelajaran Berbasis Sintaks KESAN}
\textit{(Clue untuk Guru: Setiap TP ini adalah checkpoint yang harus Anda pastikan tercapai di setiap tahap)}

Melalui Model Pembelajaran KESAN, peserta didik mampu:

\begin{enumerate}
\item \textbf{(Tahap K)} Mengidentifikasi dan mengapresiasi kearifan arsitektur Rumah Gadang Minangkabau dalam menerapkan prinsip-prinsip tekanan dan distribusi beban.
\item \textbf{(Tahap E)} Merumuskan pertanyaan investigatif tentang hubungan antara desain arsitektur tradisional dan konsep tekanan dalam fisika.
\item \textbf{(Tahap S)} Menganalisis secara sistematis prinsip tekanan dari perspektif sains modern dan kearifan arsitektur tradisional Minangkabau.
\item \textbf{(Tahap A)} Menyintesiskan pemahaman tentang konsep tekanan dengan mengintegrasikan pengetahuan fisika dan kearifan arsitektur tradisional.
\item \textbf{(Tahap N)} Menerapkan pemahaman terintegrasi tentang tekanan untuk menganalisis dan mengevaluasi desain struktur bangunan dalam konteks baru.
\end{enumerate}

\subsection{Pertanyaan Pemantik Berbasis Model KESAN}
\textit{(Clue untuk Guru: Ajukan secara berurutan dengan jeda reflektif. Biarkan ketegangan kognitif terbangun sebelum masuk ke tahap investigasi)}

\begin{itemize}
\item "Mengapa Rumah Gadang yang dibangun ratusan tahun lalu tanpa teknologi modern bisa tetap berdiri kokoh, sementara beberapa bangunan modern justru mudah roboh?"
\item "Apa rahasia di balik bentuk atap Rumah Gadang yang melengkung dan runcing? Apakah ada hubungannya dengan konsep tekanan yang kita pelajari di fisika?"
\end{itemize}

\section{LANGKAH-LANGKAH KEGIATAN PEMBELAJARAN BERBASIS SINTAKS KESAN}

\subsection{Kegiatan Pembuka (15 Menit)}

\begin{enumerate}
\item \textbf{Orientasi} (3 menit): Salam, doa, presensi, dan pengkondisian kelas
\item \textbf{Asesmen Diagnostik} (7 menit): K-W-L Chart untuk mengidentifikasi prior knowledge tentang tekanan dan arsitektur tradisional
   \begin{itemize}
   \item \textbf{Instruksi Guru}: "Anak-anak, mari kita isi kolom K (Know) dengan apa yang kalian ketahui tentang Rumah Gadang dan konsep tekanan. Lalu isi kolom W (Want to know) dengan apa yang ingin kalian ketahui tentang keduanya."
   \end{itemize}
\item \textbf{Apersepsi \& Motivasi} (5 menit): Pengenalan dimensi PPP dan misi pembelajaran
\end{enumerate}

\subsection{Kegiatan Inti (90 Menit) - IMPLEMENTASI SINTAKS MODEL KESAN}

\subsubsection{Tahap 1: (K) Kaitkan Konteks Kultural (15 menit)}

\textbf{Aktivitas Guru:}
\begin{itemize}
\item Menampilkan video singkat tentang konstruksi dan keunikan Rumah Gadang
\item Mengajukan pertanyaan pemantik tentang kekokohan arsitektur tradisional
\item Memfasilitasi sharing pengalaman siswa tentang bangunan tradisional yang pernah mereka lihat
\item \textit{(Clue: Dorong siswa untuk berbagi pengalaman tentang rumah adat atau bangunan tua yang mereka kenal. Ciptakan suasana kagum terhadap kearifan nenek moyang)}
\end{itemize}

\textbf{Aktivitas Siswa:}
\begin{itemize}
\item Mengamati video dan gambar Rumah Gadang yang disajikan
\item Berbagi pengalaman pribadi tentang bangunan tradisional yang pernah dilihat
\item Menuliskan pertanyaan awal di sticky notes untuk "Papan Penasaran"
\end{itemize}

\subsubsection{Tahap 2: (E) Eksplorasi Enigma (15 menit)}

\textbf{Aktivitas Guru:}
\begin{itemize}
\item Membentuk kelompok heterogen (3-4 siswa)
\item \textbf{Instruksi Guru}: "Sekarang, dalam kelompok kalian, diskusikan dan rumuskan 'Misi Penyelidikan Tim' kalian. Apa yang ingin kalian selidiki tentang hubungan antara desain Rumah Gadang dan konsep tekanan? Tuliskan minimal 3 pertanyaan kunci yang akan menjadi fokus investigasi kelompok kalian."
\item \textit{(Clue: Berkeliling dan pastikan setiap kelompok merumuskan pertanyaan yang investigatable, bukan hanya pertanyaan faktual)}
\end{itemize}

\textbf{Aktivitas Siswa:}
\begin{itemize}
\item Berdiskusi dalam kelompok untuk merumuskan misi investigasi
\item Menuliskan minimal 3 pertanyaan kunci di kertas plano
\item Mempresentasikan rumusan masalah secara singkat
\end{itemize}

\subsubsection{Tahap 3: (S) Selidiki secara Sintetis (25 menit)}

\textbf{Aktivitas Guru:}
\begin{itemize}
\item Membagikan "LKPD 11 - Jurnal Investigasi Dual-Lensa"
\item \textbf{Instruksi Guru}: "Sekarang saatnya investigasi! Gunakan LKPD ini untuk mencatat temuan kalian dari dua perspektif. Di kolom 'Lensa Sains', catat semua informasi tentang konsep tekanan, gaya, dan distribusi beban dari sumber-sumber ilmiah. Di kolom 'Lensa Etnosains', catat semua informasi tentang kearifan arsitektur Rumah Gadang dari sumber-sumber budaya."
\item \textit{(Clue: Sediakan sumber yang terkurasi untuk kedua lensa. Pastikan siswa tidak hanya copy-paste tapi benar-benar memahami)}
\end{itemize}

\textbf{Aktivitas Siswa:}
\begin{itemize}
\item Melakukan investigasi dengan pembagian tugas dalam tim
\item Mencatat temuan dari perspektif sains dan etnosains secara terpisah pada LKPD
\item Mengumpulkan data tentang prinsip tekanan dan kearifan arsitektur tradisional
\end{itemize}

\subsubsection{Tahap 4: (A) Asimilasi Analitis (20 menit)}

\textbf{Aktivitas Guru:}
\begin{itemize}
\item \textbf{Pertanyaan Pancingan Kunci}: "Bagaimana prinsip-prinsip tekanan yang kalian pelajari dari sains dapat menjelaskan mengapa desain Rumah Gadang begitu efektif? Apa kesamaan dan perbedaan antara pendekatan tradisional dan modern dalam mengatasi masalah tekanan pada bangunan?"
\item \textit{(Clue: Dorong siswa untuk menemukan titik-titik koneksi antara kedua perspektif. Jangan langsung memberikan jawaban, biarkan mereka berdiskusi dan menemukan sendiri)}
\end{itemize}

\textbf{Aktivitas Siswa:}
\begin{itemize}
\item Berdiskusi intensif untuk menghubungkan temuan dari kedua lensa
\item Mengidentifikasi kesamaan prinsip antara sains modern dan kearifan tradisional
\item Menuliskan kesimpulan terpadu di kertas plano
\end{itemize}

\subsubsection{Tahap 5: (N) Nyatakan Pemahaman (15 menit)}

\textbf{Aktivitas Guru:}
\begin{itemize}
\item Memberikan studi kasus: "Sebuah arsitek modern ingin merancang gedung sekolah yang tahan gempa dengan mengadaptasi prinsip-prinsip Rumah Gadang. Berdasarkan pemahaman kalian tentang tekanan dan kearifan arsitektur tradisional, berikan rekomendasi desain yang menggabungkan kedua pendekatan tersebut."
\item \textbf{Instruksi Guru}: "Tuliskan jawaban individual kalian dalam 3-5 kalimat. Gunakan bukti dari kedua lensa yang telah kalian investigasi untuk mendukung argumen kalian."
\end{itemize}

\textbf{Aktivitas Siswa:}
\begin{itemize}
\item Menyusun argumen tertulis individual berdasarkan studi kasus
\item Mengintegrasikan pemahaman dari perspektif sains dan etnosains
\item Memberikan rekomendasi yang didukung bukti dari investigasi
\end{itemize}

\subsection{Kegiatan Penutup (15 Menit)}

\begin{enumerate}
\item \textbf{Presentasi \& Penguatan} (7 menit): Beberapa siswa membacakan jawaban studi kasus, guru memberikan penguatan konsep
\item \textbf{Refleksi} (5 menit): Melengkapi kolom L (Learned) pada K-W-L Chart, refleksi pembelajaran
\item \textbf{Tindak Lanjut \& Penutup} (3 menit): Preview pembelajaran berikutnya tentang aplikasi tekanan dalam teknologi modern dengan narasi berkelanjutan
\end{enumerate}

\section{ASESMEN BERBASIS MODEL KESAN}

\subsection{Asesmen Diagnostik (Tahap K)}
\begin{itemize}
\item \textbf{Instrumen}: K-W-L Chart
\item \textbf{Tujuan}: Mengidentifikasi pengetahuan awal siswa tentang tekanan dan arsitektur tradisional
\end{itemize}

\subsection{Asesmen Formatif (Tahap E-S-A)}
\begin{itemize}
\item \textbf{Tahap E}: Penilaian kualitas "Misi Penyelidikan" kelompok
\item \textbf{Tahap S}: Kelengkapan dan kualitas "Jurnal Investigasi Dual-Lensa"
\item \textbf{Tahap A}: Observasi diskusi dan kemampuan sintesis kelompok
\end{itemize}

\subsection{Asesmen Sumatif (Tahap N)}
\begin{itemize}
\item \textbf{Instrumen}: Jawaban studi kasus dengan rubrik 4 level
\item \textbf{Aspek}: Ketepatan konsep tekanan, keterkaitan etnosains, kelogisan argumen
\end{itemize}

\section{RUBRIK PENILAIAN STUDI KASUS}

\begin{longtable}{|p{3cm}|p{3cm}|p{3cm}|p{3cm}|p{3cm}|}
\hline
\textbf{Kriteria Penilaian} & \textbf{Skor 4 (Sangat Baik)} & \textbf{Skor 3 (Baik)} & \textbf{Skor 2 (Cukup)} & \textbf{Skor 1 (Kurang)} \\
\hline
\textbf{Ketepatan Konsep Tekanan} & Menjelaskan konsep tekanan, gaya, dan distribusi beban dengan sangat tepat dan lengkap, serta mampu mengaplikasikannya dalam konteks arsitektur & Menjelaskan konsep tekanan dengan tepat dan dapat mengaplikasikannya dengan baik & Menjelaskan konsep tekanan dengan cukup tepat namun aplikasi masih terbatas & Penjelasan konsep tekanan kurang tepat dan aplikasi tidak sesuai \\
\hline
\textbf{Keterkaitan dengan Etnosains} & Mengintegrasikan kearifan arsitektur Rumah Gadang dengan sangat baik, menunjukkan pemahaman mendalam tentang prinsip-prinsip tradisional & Mengintegrasikan kearifan arsitektur tradisional dengan baik dan relevan & Menyebutkan kearifan tradisional namun integrasinya masih superfisial & Tidak mampu mengaitkan dengan kearifan tradisional atau tidak relevan \\
\hline
\textbf{Kelogisan \& Struktur Argumen} & Argumen sangat logis, terstruktur dengan baik, didukung bukti kuat dari kedua perspektif & Argumen logis dan terstruktur, didukung bukti yang memadai & Argumen cukup logis namun struktur dan bukti pendukung masih lemah & Argumen tidak logis, tidak terstruktur, dan tidak didukung bukti yang memadai \\
\hline
\end{longtable}

\section{DAFTAR PUSTAKA ETNOSAINS}

\begin{enumerate}
\item Couto, N. (2008). \textit{Budaya Visual Rumah Gadang}. Padang: Andalas University Press.
\item Hakimi, I. (2015). \textit{Arsitektur Tradisional Minangkabau: Filosofi dan Teknologi}. Jakarta: Rajawali Pers.
\item Navis, A.A. (1984). \textit{Alam Terkembang Jadi Guru: Adat dan Kebudayaan Minangkabau}. Jakarta: Grafiti Pers.
\item Syafwandi. (2014). \textit{Rumah Gadang: Arsitektur Tradisional Minangkabau}. Padang: Minangkabau Press.
\end{enumerate}

\end{document}