\documentclass[12pt,a4paper]{article}
\usepackage[margin=1.5cm]{geometry}
\usepackage[utf8]{inputenc}
\usepackage{amsmath}
\usepackage{amsfonts}
\usepackage{amssymb}
\usepackage{graphicx}
\usepackage{xcolor}
\usepackage{tcolorbox}
\usepackage{enumitem}
\usepackage{multicol}
\usepackage{fancyhdr}
\usepackage{titlesec}
\usepackage{array}
\usepackage{longtable}
\usepackage{booktabs}

% Setup header and footer
\pagestyle{fancy}
\fancyhf{}
\fancyhead[L]{\textbf{RPP IPA Kelas VIII - Model KESAN}}
\fancyhead[R]{\textbf{Semester Genap 2025/2026}}
\fancyfoot[C]{\thepage}

% Define colors
\definecolor{primarycolor}{RGB}{0,0,0}
\definecolor{secondarycolor}{RGB}{255,255,255}

% Custom tcolorbox styles
\tcbset{
    mystyle/.style={
        colback=white,
        colframe=black,
        boxrule=1pt,
        arc=2pt,
        left=3pt,
        right=3pt,
        top=3pt,
        bottom=3pt
    }
}

\begin{document}

\begin{center}
\textbf{\Large MODUL AJAR: Kearifan Sistem Irigasi Sawah \& Hukum Tekanan Hidrostatis}
\end{center}

\vspace{0.5cm}

\begin{tcolorbox}[mystyle]
\textbf{Nama Penyusun:} Irfan Ananda \\
\textbf{Institusi:} SMP \\
\textbf{Mata Pelajaran:} Ilmu Pengetahuan Alam (IPA) \\
\textbf{Tahun Ajaran:} 2025/2026 \\
\textbf{Semester:} Genap \\
\textbf{Jenjang Sekolah:} SMP \\
\textbf{Kelas/Fase:} VIII / D \\
\textbf{Alokasi waktu:} 3 x 40 Menit (1 Pertemuan)
\end{tcolorbox}

\section{DIMENSI PROFIL PELAJAR PANCASILA}
\textit{(Clue untuk Guru: Sebutkan dimensi ini secara eksplisit saat apersepsi agar siswa sadar tujuan non-akademis yang sedang mereka kembangkan).}

\begin{itemize}
\item \textbf{Berkebinekaan Global:} Mengenal dan menghargai budaya, khususnya menganalisis kearifan lokal sistem irigasi tradisional Minangkabau dalam menerapkan prinsip-prinsip tekanan hidrostatis dan aliran fluida, lalu menghubungkannya dengan konsep fisika universal.
\item \textbf{Bernalar Kritis:} Ditempa secara intensif saat menganalisis informasi dari sumber sains dan budaya (Tahap S), menyintesiskan kedua perspektif (Tahap A), dan menyusun argumen berbasis bukti (Tahap N).
\item \textbf{Gotong Royong:} Kemampuan untuk bekerja sama secara kolaboratif dalam kelompok untuk merumuskan masalah (Tahap E), melakukan investigasi (Tahap S), dan membangun pemahaman bersama (Tahap A).
\item \textbf{Kreatif:} Menghasilkan argumen atau solusi orisinal yang terintegrasi pada tahap akhir pembelajaran (Tahap N).
\end{itemize}

\section{Sarana dan Prasarana}

\begin{itemize}
\item \textbf{Media:} LKPD 12 - Jurnal Investigasi Dual-Lensa, video singkat sistem irigasi sawah tradisional Minangkabau (durasi 2-3 menit), gambar/diagram sistem pengairan tradisional, artikel/infografis prinsip tekanan hidrostatis, model/simulasi aliran air.
\item \textbf{Alat:} Papan tulis/whiteboard, spidol, Proyektor \& Speaker, Kertas Plano atau Karton (1 per kelompok), sticky notes warna-warni, botol plastik, selang transparan, penggaris.
\item \textbf{Sumber Belajar:} Buku ajar IPA kelas VIII, Tautan video animasi tekanan hidrostatis (misal: bit.ly/animasi-hidrostatis), tautan artikel/video sistem irigasi tradisional (misal: bit.ly/irigasi-minang).
\end{itemize}

\section{Target Peserta didik}

\begin{itemize}
\item Peserta didik reguler kelas VIII (Fase D).
\end{itemize}

\section{Model Pembelajaran}

\begin{itemize}
\item Model Pembelajaran KESAN (Konektivitas Etnosains-Sains).
\end{itemize}

\section{Pemahaman Bermakna}
\textit{(Clue untuk Guru: Bacakan atau sampaikan narasi ini dengan intonasi yang menarik di akhir pembelajaran untuk mengikat semua pengalaman belajar siswa menjadi satu kesatuan yang bermakna).}

\begin{tcolorbox}[mystyle]
"Ananda Semua, hari ini kita telah mengungkap keajaiban di balik sistem irigasi sawah tradisional Minangkabau! Kita menemukan bahwa para petani nenek moyang kita telah menerapkan hukum-hukum tekanan hidrostatis dengan sangat cerdas. Mereka memahami bagaimana air mengalir dari tempat tinggi ke rendah, bagaimana tekanan air dapat dimanfaatkan untuk mengairi sawah bertingkat, dan bagaimana sistem saluran dapat didistribusikan secara merata. Semua ini dilakukan tanpa kalkulator atau rumus fisika, namun berdasarkan pengamatan alam dan pemahaman intuitif tentang prinsip-prinsip fisika. Dengan ini, kita sadar bahwa sains tidak hanya ada di laboratorium, tapi juga terwujud dalam kearifan pengelolaan sumber daya alam oleh masyarakat tradisional."
\end{tcolorbox}

\section{PERTEMUAN 2: Tekanan Hidrostatis dalam Sistem Irigasi Tradisional}

\subsection{Capaian Pembelajaran (Fase D)}
Menganalisis ragam gerak, gaya, dan tekanan dalam konteks kehidupan sehari-hari dan fenomena alam, khususnya tekanan hidrostatis dan aplikasinya dalam sistem pengairan.

\subsection{Tujuan Pembelajaran Berbasis Sintaks KESAN}
\textit{(Clue untuk Guru: Setiap TP ini adalah checkpoint yang harus Anda pastikan tercapai di setiap tahap)}

Melalui Model Pembelajaran KESAN, peserta didik mampu:

\begin{enumerate}
\item \textbf{(Tahap K)} Mengidentifikasi dan mengapresiasi kearifan sistem irigasi tradisional Minangkabau dalam menerapkan prinsip-prinsip tekanan hidrostatis dan aliran fluida.
\item \textbf{(Tahap E)} Merumuskan pertanyaan investigatif tentang hubungan antara sistem pengairan tradisional dan konsep tekanan hidrostatis dalam fisika.
\item \textbf{(Tahap S)} Menganalisis secara sistematis prinsip tekanan hidrostatis dari perspektif sains modern dan kearifan pengelolaan air tradisional Minangkabau.
\item \textbf{(Tahap A)} Menyintesiskan pemahaman tentang konsep tekanan hidrostatis dengan mengintegrasikan pengetahuan fisika dan kearifan pengelolaan air tradisional.
\item \textbf{(Tahap N)} Menerapkan pemahaman terintegrasi tentang tekanan hidrostatis untuk menganalisis dan merancang sistem pengairan yang efisien dalam konteks modern.
\end{enumerate}

\subsection{Pertanyaan Pemantik Berbasis Model KESAN}
\textit{(Clue untuk Guru: Ajukan secara berurutan dengan jeda reflektif. Biarkan ketegangan kognitif terbangun sebelum masuk ke tahap investigasi)}

\begin{itemize}
\item "Bagaimana mungkin petani tradisional Minangkabau bisa mengairi sawah bertingkat di lereng bukit tanpa menggunakan pompa listrik atau teknologi modern?"
\item "Mengapa air bisa mengalir dari satu sawah ke sawah lainnya dengan tekanan yang tepat? Apa rahasia di balik desain saluran irigasi tradisional yang begitu efisien?"
\end{itemize}

\section{LANGKAH-LANGKAH KEGIATAN PEMBELAJARAN BERBASIS SINTAKS KESAN}

\subsection{Kegiatan Pembuka (15 Menit)}

\begin{enumerate}
\item \textbf{Orientasi} (3 menit): Salam, doa, presensi, dan pengkondisian kelas
\item \textbf{Asesmen Diagnostik} (7 menit): K-W-L Chart untuk mengidentifikasi prior knowledge tentang tekanan hidrostatis dan sistem irigasi
   \begin{itemize}
   \item \textbf{Instruksi Guru}: "Anak-anak, mari kita isi kolom K (Know) dengan apa yang kalian ketahui tentang aliran air dan sistem irigasi. Lalu isi kolom W (Want to know) dengan apa yang ingin kalian ketahui tentang tekanan air dan sistem pengairan tradisional."
   \end{itemize}
\item \textbf{Apersepsi \& Motivasi} (5 menit): Pengenalan dimensi PPP dan misi pembelajaran
\end{enumerate}

\subsection{Kegiatan Inti (90 Menit) - IMPLEMENTASI SINTAKS MODEL KESAN}

\subsubsection{Tahap 1: (K) Kaitkan Konteks Kultural (15 menit)}

\textbf{Aktivitas Guru:}
\begin{itemize}
\item Menampilkan video singkat tentang sistem irigasi sawah tradisional Minangkabau
\item Mengajukan pertanyaan pemantik tentang kecanggihan sistem pengairan tradisional
\item Memfasilitasi sharing pengalaman siswa tentang sistem irigasi atau pengairan yang pernah mereka lihat
\item \textit{(Clue: Dorong siswa untuk berbagi pengalaman tentang sawah, kolam, atau sistem pengairan di sekitar mereka. Ciptakan rasa kagum terhadap kearifan pengelolaan air)}
\end{itemize}

\textbf{Aktivitas Siswa:}
\begin{itemize}
\item Mengamati video dan gambar sistem irigasi tradisional yang disajikan
\item Berbagi pengalaman pribadi tentang sistem pengairan yang pernah dilihat
\item Menuliskan pertanyaan awal di sticky notes untuk "Papan Penasaran"
\end{itemize}

\subsubsection{Tahap 2: (E) Eksplorasi Enigma (15 menit)}

\textbf{Aktivitas Guru:}
\begin{itemize}
\item Membentuk kelompok heterogen (3-4 siswa)
\item \textbf{Instruksi Guru}: "Sekarang, dalam kelompok kalian, diskusikan dan rumuskan 'Misi Penyelidikan Tim' kalian. Apa yang ingin kalian selidiki tentang hubungan antara sistem irigasi tradisional dan konsep tekanan hidrostatis? Tuliskan minimal 3 pertanyaan kunci yang akan menjadi fokus investigasi kelompok kalian."
\item \textit{(Clue: Pastikan pertanyaan yang dirumuskan bersifat investigatable dan menghubungkan fenomena tradisional dengan konsep fisika)}
\end{itemize}

\textbf{Aktivitas Siswa:}
\begin{itemize}
\item Berdiskusi dalam kelompok untuk merumuskan misi investigasi
\item Menuliskan minimal 3 pertanyaan kunci di kertas plano
\item Mempresentasikan rumusan masalah secara singkat
\end{itemize}

\subsubsection{Tahap 3: (S) Selidiki secara Sintetis (25 menit)}

\textbf{Aktivitas Guru:}
\begin{itemize}
\item Membagikan "LKPD 12 - Jurnal Investigasi Dual-Lensa"
\item \textbf{Instruksi Guru}: "Sekarang saatnya investigasi mendalam! Gunakan LKPD ini untuk mencatat temuan kalian dari dua perspektif. Di kolom 'Lensa Sains', catat semua informasi tentang tekanan hidrostatis, hukum Pascal, dan prinsip aliran fluida dari sumber-sumber ilmiah. Di kolom 'Lensa Etnosains', catat semua informasi tentang kearifan sistem irigasi tradisional Minangkabau dari sumber-sumber budaya."
\item \textit{(Clue: Sediakan demonstrasi sederhana dengan botol dan selang untuk menunjukkan prinsip tekanan hidrostatis. Pastikan siswa memahami konsep sebelum menganalisis aplikasinya)}
\end{itemize}

\textbf{Aktivitas Siswa:}
\begin{itemize}
\item Melakukan investigasi dengan pembagian tugas dalam tim
\item Mencatat temuan dari perspektif sains dan etnosains secara terpisah pada LKPD
\item Mengumpulkan data tentang prinsip tekanan hidrostatis dan kearifan pengelolaan air tradisional
\end{itemize}

\subsubsection{Tahap 4: (A) Asimilasi Analitis (20 menit)}

\textbf{Aktivitas Guru:}
\begin{itemize}
\item \textbf{Pertanyaan Pancingan Kunci}: "Bagaimana hukum tekanan hidrostatis yang kalian pelajari dapat menjelaskan mengapa sistem irigasi tradisional begitu efektif? Apa kesamaan prinsip antara sistem pengairan modern dan tradisional dalam memanfaatkan tekanan air?"
\item \textit{(Clue: Dorong siswa untuk mengidentifikasi prinsip-prinsip fisika yang sama dalam kedua pendekatan. Bantu mereka melihat bahwa teknologi tradisional juga berbasis sains)}
\end{itemize}

\textbf{Aktivitas Siswa:}
\begin{itemize}
\item Berdiskusi intensif untuk menghubungkan temuan dari kedua lensa
\item Mengidentifikasi kesamaan prinsip fisika dalam sistem tradisional dan modern
\item Menuliskan kesimpulan terpadu di kertas plano
\end{itemize}

\subsubsection{Tahap 5: (N) Nyatakan Pemahaman (15 menit)}

\textbf{Aktivitas Guru:}
\begin{itemize}
\item Memberikan studi kasus: "Sebuah desa di daerah perbukitan ingin membangun sistem irigasi untuk mengairi sawah bertingkat. Mereka memiliki sumber air di puncak bukit dan sawah-sawah di berbagai ketinggian. Berdasarkan pemahaman kalian tentang tekanan hidrostatis dan kearifan sistem irigasi tradisional, rancanglah sistem pengairan yang efisien dengan menggabungkan prinsip fisika dan kearifan lokal."
\item \textbf{Instruksi Guru}: "Tuliskan jawaban individual kalian dalam 3-5 kalimat. Gunakan bukti dari kedua lensa yang telah kalian investigasi untuk mendukung rancangan kalian."
\end{itemize}

\textbf{Aktivitas Siswa:}
\begin{itemize}
\item Menyusun rancangan sistem irigasi berdasarkan studi kasus
\item Mengintegrasikan pemahaman dari perspektif sains dan etnosains
\item Memberikan justifikasi yang didukung bukti dari investigasi
\end{itemize}

\subsection{Kegiatan Penutup (15 Menit)}

\begin{enumerate}
\item \textbf{Presentasi \& Penguatan} (7 menit): Beberapa siswa mempresentasikan rancangan mereka, guru memberikan penguatan konsep
\item \textbf{Refleksi} (5 menit): Melengkapi kolom L (Learned) pada K-W-L Chart, refleksi pembelajaran
\item \textbf{Tindak Lanjut \& Penutup} (3 menit): Preview pembelajaran berikutnya tentang getaran dan gelombang dengan narasi berkelanjutan
\end{enumerate}

\section{ASESMEN BERBASIS MODEL KESAN}

\subsection{Asesmen Diagnostik (Tahap K)}
\begin{itemize}
\item \textbf{Instrumen}: K-W-L Chart
\item \textbf{Tujuan}: Mengidentifikasi pengetahuan awal siswa tentang tekanan hidrostatis dan sistem irigasi
\end{itemize}

\subsection{Asesmen Formatif (Tahap E-S-A)}
\begin{itemize}
\item \textbf{Tahap E}: Penilaian kualitas "Misi Penyelidikan" kelompok
\item \textbf{Tahap S}: Kelengkapan dan kualitas "Jurnal Investigasi Dual-Lensa"
\item \textbf{Tahap A}: Observasi diskusi dan kemampuan sintesis kelompok
\end{itemize}

\subsection{Asesmen Sumatif (Tahap N)}
\begin{itemize}
\item \textbf{Instrumen}: Rancangan sistem irigasi dengan rubrik 4 level
\item \textbf{Aspek}: Ketepatan konsep tekanan hidrostatis, keterkaitan etnosains, kreativitas rancangan
\end{itemize}

\section{RUBRIK PENILAIAN STUDI KASUS}

\begin{longtable}{|p{3cm}|p{3cm}|p{3cm}|p{3cm}|p{3cm}|}
\hline
\textbf{Kriteria Penilaian} & \textbf{Skor 4 (Sangat Baik)} & \textbf{Skor 3 (Baik)} & \textbf{Skor 2 (Cukup)} & \textbf{Skor 1 (Kurang)} \\
\hline
\textbf{Ketepatan Konsep Tekanan Hidrostatis} & Menjelaskan dan menerapkan konsep tekanan hidrostatis dengan sangat tepat dan lengkap dalam rancangan sistem irigasi & Menjelaskan konsep tekanan hidrostatis dengan tepat dan dapat menerapkannya dengan baik & Menjelaskan konsep dengan cukup tepat namun penerapan masih terbatas & Penjelasan konsep kurang tepat dan penerapan tidak sesuai \\
\hline
\textbf{Keterkaitan dengan Etnosains} & Mengintegrasikan kearifan sistem irigasi tradisional dengan sangat baik, menunjukkan pemahaman mendalam tentang prinsip-prinsip pengelolaan air tradisional & Mengintegrasikan kearifan tradisional dengan baik dan relevan dalam rancangan & Menyebutkan kearifan tradisional namun integrasinya masih superfisial & Tidak mampu mengaitkan dengan kearifan tradisional atau tidak relevan \\
\hline
\textbf{Kreativitas \& Kelogisan Rancangan} & Rancangan sangat kreatif, logis, dan praktis, menggabungkan prinsip fisika dan kearifan tradisional secara inovatif & Rancangan kreatif dan logis, menggabungkan kedua perspektif dengan baik & Rancangan cukup logis namun kreativitas dan integrasi masih terbatas & Rancangan tidak logis, tidak kreatif, dan tidak menunjukkan integrasi yang memadai \\
\hline
\end{longtable}

\section{DAFTAR PUSTAKA ETNOSAINS}

\begin{enumerate}
\item Dt. Rajo Penghulu. (1994). \textit{Sistem Irigasi Tradisional Minangkabau}. Padang: Pusat Dokumentasi Kebudayaan Minang.
\item Kato, T. (2005). \textit{Adat Minangkabau dan Merantau dalam Perspektif Sejarah}. Jakarta: Balai Pustaka.
\item Naim, M. (1984). \textit{Pengelolaan Sumber Daya Air dalam Masyarakat Minangkabau}. Padang: Andalas University Press.
\item Syafwandi. (2016). \textit{Kearifan Lokal dalam Pengelolaan Lingkungan Hidup Masyarakat Minangkabau}. Padang: Minangkabau Press.
\end{enumerate}

\end{document}