\documentclass[12pt,a4paper]{article}
\usepackage[margin=1.8cm]{geometry} % A4 narrow margins
\usepackage[utf8]{inputenc}
\usepackage[indonesian]{babel}
\usepackage{tcolorbox}
\tcbuselibrary{breakable,skins}
\usepackage{graphicx}
\usepackage{longtable}
\usepackage{booktabs}
\usepackage{array}
\usepackage{fancyhdr}
\usepackage{tikz}
\usetikzlibrary{patterns,decorations.pathmorphing,positioning,shapes.geometric}
\usepackage{hyperref}
\usepackage{subfiles} % allow modular compilation of sub-documents
\hypersetup{
    colorlinks=false,
    linkcolor=black,
    filecolor=black,
    urlcolor=black,
    citecolor=black,
    pdftitle={Kumpulan RPP IPA Kelas VIII Model KESAN},
    pdfauthor={Irfan Ananda Ismail},
    pdfsubject={Rencana Pelaksanaan Pembelajaran},
    pdfkeywords={RPP, IPA, KESAN, Etnosains, Minangkabau},
}

% Black and white color scheme
\definecolor{darkgray}{RGB}{64,64,64}
\definecolor{lightgray}{RGB}{240,240,240}
\definecolor{mediumgray}{RGB}{128,128,128}

% Enhanced header and footer
\pagestyle{fancy}
\fancyhf{}
\renewcommand{\headrulewidth}{1pt}
\renewcommand{\footrulewidth}{1pt}
\fancyhead[L]{\textbf{Irfan Ananda Ismail, S.Pd, M.Pd, Gr.}}
\fancyhead[R]{\textbf{RPP IPA Kelas VIII Model KESAN}}
\fancyfoot[C]{\textbf{\thepage}}
\fancyfoot[L]{\small @RPP IPA MODEL KESAN KELAS VIII}
\fancyfoot[R]{\small 2025}

% Custom title page design
\newcommand{\customtitlepage}{
\thispagestyle{empty}
\begin{center}
\vspace*{3cm}

% Main title section
\begin{tcolorbox}[enhanced,
    colback=lightgray!30,
    colframe=black,
    boxrule=2pt,
    arc=5mm,
    width=16cm,
    before skip=0pt,
    after skip=0pt]
\centering
{\fontsize{18}{22}\selectfont\textbf{KUMPULAN RENCANA PELAKSANAAN PEMBELAJARAN}}\\
\vspace{0.3cm}
{\fontsize{16}{20}\selectfont\textbf{MATA PELAJARAN ILMU PENGETAHUAN ALAM}}\\
\vspace{0.2cm}
{\fontsize{14}{18}\selectfont\textbf{KELAS VIII SEMESTER GANJIL \& GENAP}}
\end{tcolorbox}

\vspace{1cm}

% Model KESAN highlight
\begin{tcolorbox}[enhanced,
    colback=darkgray,
    coltext=white,
    colframe=black,
    boxrule=2pt,
    arc=3mm,
    width=14cm]
\centering
{\fontsize{16}{20}\selectfont\textbf{BERBASIS MODEL KESAN}}\\
\vspace{0.2cm}
{\fontsize{12}{15}\selectfont\textit{Konektivitas Etnosains-SAiNs}}
\end{tcolorbox}

\vspace{0.8cm}

% Etnosains integration
\begin{tcolorbox}[enhanced,
    colback=lightgray!50,
    colframe=black,
    boxrule=1.5pt,
    arc=3mm,
    width=12cm]
\centering
{\fontsize{14}{18}\selectfont\textbf{TERINTEGRASI ETNOSAINS MINANGKABAU}}
\end{tcolorbox}

\vspace{1.5cm}

% Author info
\begin{tcolorbox}[enhanced,
    colback=white,
    colframe=darkgray,
    boxrule=1.5pt,
    arc=2mm,
    width=10cm]
\centering
{\fontsize{14}{16}\selectfont\textbf{Peneliti}}\\
\vspace{0.3cm}
{\fontsize{12}{15}\selectfont\textbf{Irfan Ananda Ismail, S.Pd, M.Pd, Gr.}}\\
\vspace{0.2cm}
{\fontsize{11}{14}\selectfont SMP Negeri/Swasta}\\
{\fontsize{11}{14}\selectfont Sumatera Barat}
\end{tcolorbox}

\vspace{0.8cm}

% Year
\begin{tcolorbox}[enhanced,
    colback=darkgray,
    coltext=white,
    colframe=black,
    boxrule=1.5pt,
    arc=2mm,
    width=6cm]
\centering
{\fontsize{14}{16}\selectfont\textbf{2025}}
\end{tcolorbox}
\end{center}
}

\begin{document}

% ===============================
% HALAMAN DEPAN
% ===============================
\customtitlepage

\newpage

% ===============================
% IDENTITAS RPP
% ===============================
\thispagestyle{empty}
\begin{center}
\vspace*{2cm}
\begin{tcolorbox}[enhanced,
    colback=lightgray!30,
    colframe=black,
    boxrule=2pt,
    arc=5mm,
    width=15cm,
    title={\centering\textbf{\large IDENTITAS DOKUMEN RPP}},
    fonttitle=\bfseries,
    coltitle=white,
    colbacktitle=darkgray]
\centering
\renewcommand{\arraystretch}{1.4}
\begin{tabular}{p{6cm}p{7cm}}
\textbf{Peneliti/Penyusun} & : Irfan Ananda Ismail, S.Pd, M.Pd, Gr. \\
\textbf{Jenjang Pendidikan} & : Sekolah Menengah Pertama (SMP) \\
\textbf{Mata Pelajaran} & : Ilmu Pengetahuan Alam (IPA) \\
\textbf{Kelas/Fase} & : VIII / D \\
\textbf{Kurikulum} & : Kurikulum Merdeka 2025 \\
\textbf{Model Pembelajaran} & : KESAN (Konektivitas Etnosains-Sains) \\
\textbf{Integrasi Budaya} & : Etnosains Minangkabau \\
\textbf{Tahun Penyusunan} & : 2025 \\
\textbf{Total RPP Semester 1} & : 20 Pertemuan \\
\textbf{Total RPP Semester 2} & : 12 Pertemuan \\
\textbf{Total Keseluruhan} & : 32 Pertemuan \\
\end{tabular}
\end{tcolorbox}
\end{center}

\newpage

% ===============================
% DAFTAR LKPD SEMESTER 1 & 2
% ===============================
\section*{\centering DAFTAR RPP SEMESTER GANJIL DAN GENAP}

\begin{tcolorbox}[colback=white, colframe=black, breakable, title=\textbf{Daftar LKPD IPA Kelas VIII Berbasis Model KESAN}]

\subsection*{SEMESTER GANJIL}
\begin{longtable}{|c|p{12cm}|p{2cm}|}
\hline
\textbf{Bab} & \textbf{Materi} & \textbf{Pertemuan Ke} \\
\hline
\endfirsthead

\multicolumn{3}{c}%
{{\bfseries \tablename\ \thetable{} -- Lanjutan dari halaman sebelumnya}} \\
\hline
\textbf{Bab} & \textbf{Materi} & \textbf{Pertemuan Ke} \\
\hline
\endhead

\hline \multicolumn{3}{|r|}{{Lanjutan pada halaman berikutnya}} \\ \hline
\endfoot

\hline
\endlastfoot

1 & Gerak Makhluk Hidup \& Tari Piring Minangkabau & 1 \\
\hline
1 & Gerak Makhluk Hidup \& Silek Minangkabau & 2 \\
\hline
2 & Gerak Benda \& Kincir Air Tradisional & 3 \\
\hline
2 & Gerak Benda \& Energi Kinetik Tradisional & 4 \\
\hline
3 & Pesawat Sederhana \& Katrol Tradisional & 5 \\
\hline
3 & Pesawat Sederhana \& Tuas Lesung & 6 \\
\hline
4 & Struktur Tumbuhan \& Tanaman Obat Minang & 7 \\
\hline
4 & Fotosintesis \& Sistem Sawah Tradisional & 8 \\
\hline
5 & Klasifikasi \& Tumbuhan Obat Tradisional & 9 \\
\hline
5 & Organisasi Kehidupan \& Ekosistem Minang & 10 \\
\hline
6 & Sistem Pencernaan \& Kuliner Minangkabau & 11 \\
\hline
6 & Pencernaan \& Pengobatan Herbal Tradisional & 12 \\
\hline
7 & Zat Aditif \& Pengawetan Tradisional & 13 \\
\hline
7 & Zat Adiktif \& Kearifan Pencegahan & 14 \\
\hline
8 & Peredaran Darah \& Aktivitas Tradisional & 15 \\
\hline
8 & Peredaran Darah \& Terapi Tradisional & 16 \\
\hline
9 & Sistem Pernapasan \& Teknik Tradisional & 17 \\
\hline
9 & Pernapasan \& Pengobatan Herbal & 18 \\
\hline
10 & Sistem Ekskresi \& Kesehatan Ginjal & 19 \\
\hline
10 & Ekskresi \& Pengobatan Tradisional & 20 \\
\hline

\end{longtable}

\subsection*{SEMESTER GENAP}
\begin{longtable}{|c|p{12cm}|p{2cm}|}
\hline
\textbf{Bab} & \textbf{Materi} & \textbf{Pertemuan Ke} \\
\hline

11 & Tekanan \& Arsitektur Rumah Gadang & 21 \\
\hline
11 & Tekanan \& Sistem Irigasi Tradisional & 22 \\
\hline
11 & Tekanan \& Perahu Tradisional & 23 \\
\hline
12 & Getaran \& Alat Musik Tradisional & 24 \\
\hline
12 & Gelombang Bunyi \& Randai Minangkabau & 25 \\
\hline
12 & Gelombang Air \& Permainan Tradisional & 26 \\
\hline
13 & Cahaya \& Cermin Rumah Gadang & 27 \\
\hline
13 & Pembiasan \& Alat Optik Tradisional & 28 \\
\hline
13 & Dispersi Cahaya \& Pelangi Danau Maninjau & 29 \\
\hline
14 & Gaya \& Bajak Tradisional & 30 \\
\hline
14 & Usaha \& Kincir Air Tradisional & 31 \\
\hline
14 & Daya \& Pengolahan Pangan Tradisional & 32 \\
\hline

\end{longtable}

\end{tcolorbox}

\newpage

% ===============================
% DAFTAR ETNOSAINS YANG DIANGKAT
% ===============================
\section*{\centering DAFTAR ETNOSAINS YANG DIANGKAT}

\begin{tcolorbox}[colback=white, colframe=black, breakable, title=\textbf{Integrasi Etnosains Minangkabau dalam RPP IPA Kelas VIII}]

\subsection*{SEMESTER GANJIL}
\begin{longtable}{|p{1cm}|p{5.5cm}|p{2cm}|p{5.5cm}|}
\hline
\textbf{Bab} & \textbf{Materi} & \textbf{Pertemuan Ke} & \textbf{Etnosains yang Diangkat} \\
\hline
\endfirsthead

\multicolumn{4}{c}%
{{\bfseries \tablename\ \thetable{} -- Lanjutan dari halaman sebelumnya}} \\
\hline
\textbf{Bab} & \textbf{Materi} & \textbf{Pertemuan Ke} & \textbf{Etnosains yang Diangkat} \\
\hline
\endhead

\hline \multicolumn{4}{|r|}{{Lanjutan pada halaman berikutnya}} \\ \hline
\endfoot

\hline
\endlastfoot

1 & Gerak Makhluk Hidup \& Tari Piring & 1 & Analisis gerak dalam seni tari Piring tradisional Minangkabau \\
\hline
1 & Gerak Makhluk Hidup \& Silek Minang & 2 & Biomekanika dalam seni bela diri Silek tradisional \\
\hline
2 & Gerak Benda \& Kincir Air & 3 & Teknologi kincir air untuk penggilingan padi tradisional \\
\hline
2 & Gerak Benda \& Energi Kinetik & 4 & Pemanfaatan energi kinetik dalam teknologi tradisional \\
\hline
3 & Pesawat Sederhana \& Katrol & 5 & Sistem katrol dalam konstruksi Rumah Gadang \\
\hline
3 & Pesawat Sederhana \& Tuas Lesung & 6 & Prinsip tuas dalam lesung penumbuk padi tradisional \\
\hline
4 & Struktur Tumbuhan \& Tanaman Obat & 7 & Kearifan pengobatan tradisional dengan tanaman lokal \\
\hline
4 & Fotosintesis \& Sistem Sawah & 8 & Teknik pertanian padi dalam sistem sawah tradisional \\
\hline
5 & Klasifikasi \& Tumbuhan Obat & 9 & Sistem klasifikasi tradisional tumbuhan obat Minangkabau \\
\hline
5 & Organisasi Kehidupan \& Ekosistem & 10 & Kearifan pengelolaan ekosistem dalam budaya Minangkabau \\
\hline
6 & Sistem Pencernaan \& Kuliner Minang & 11 & Analisis nutrisi dalam masakan tradisional Minangkabau \\
\hline
6 & Pencernaan \& Pengobatan Herbal & 12 & Ramuan tradisional untuk kesehatan pencernaan \\
\hline
7 & Zat Aditif \& Pengawetan & 13 & Teknik pengawetan makanan tradisional Minangkabau \\
\hline
7 & Zat Adiktif \& Kearifan Pencegahan & 14 & Kearifan lokal dalam pencegahan penyalahgunaan zat \\
\hline
8 & Peredaran Darah \& Aktivitas & 15 & Analisis aktivitas fisik dalam kegiatan tradisional \\
\hline
8 & Peredaran Darah \& Terapi & 16 & Teknik pijat dan terapi tradisional untuk sirkulasi darah \\
\hline
9 & Sistem Pernapasan \& Teknik & 17 & Teknik pernapasan dalam seni budaya Minangkabau \\
\hline
9 & Pernapasan \& Pengobatan Herbal & 18 & Ramuan herbal tradisional untuk kesehatan pernapasan \\
\hline
10 & Sistem Ekskresi \& Kesehatan Ginjal & 19 & Kearifan tradisional dalam menjaga kesehatan ginjal \\
\hline
10 & Ekskresi \& Pengobatan Tradisional & 20 & Terapi tradisional untuk sistem ekskresi tubuh \\
\hline

\end{longtable}

\subsection*{SEMESTER GENAP}
\begin{longtable}{|c|p{4.5cm}|c|p{5.5cm}|}
\hline
\textbf{Bab} & \textbf{Materi} & \textbf{Pertemuan Ke} & \textbf{Etnosains yang Diangkat} \\
\hline

11 & Tekanan \& Arsitektur Rumah Gadang & 21 & Prinsip tekanan dalam konstruksi Rumah Gadang \\
\hline
11 & Tekanan \& Sistem Irigasi & 22 & Teknologi irigasi sawah dalam budaya Minangkabau \\
\hline
11 & Tekanan \& Perahu Tradisional & 23 & Prinsip tekanan air dalam perahu tradisional Minangkabau \\
\hline
12 & Getaran \& Alat Musik & 24 & Analisis getaran pada alat musik tradisional Minangkabau \\
\hline
12 & Gelombang Bunyi \& Randai & 25 & Akustik dalam pertunjukan Randai tradisional \\
\hline
12 & Gelombang Air \& Permainan & 26 & Fenomena gelombang air dalam permainan tradisional \\
\hline
13 & Cahaya \& Cermin Rumah Gadang & 27 & Pemanfaatan cermin dalam arsitektur Rumah Gadang \\
\hline
13 & Pembiasan \& Alat Optik & 28 & Teknologi kaca dan alat optik dalam budaya Minangkabau \\
\hline
13 & Dispersi \& Pelangi Danau Maninjau & 29 & Fenomena dispersi cahaya di alam Minangkabau \\
\hline
14 & Gaya \& Bajak Tradisional & 30 & Prinsip gaya dalam alat pertanian tradisional \\
\hline
14 & Usaha \& Kincir Air & 31 & Konsep usaha dan energi dalam teknologi air tradisional \\
\hline
14 & Daya \& Pengolahan Pangan & 32 & Efisiensi daya dalam teknologi pengolahan pangan \\
\hline

\end{longtable}

\end{tcolorbox}

\vfill
\begin{center}
\begin{tcolorbox}[enhanced,
    colback=lightgray!30,
    colframe=black,
    boxrule=2pt,
    arc=5mm,
    width=13cm,
    title={\centering\textbf{\large PENUTUP}},
    fonttitle=\bfseries,
    coltitle=white,
    colbacktitle=darkgray]
\centering
{\fontsize{11}{14}\selectfont Dokumen ini merupakan kumpulan lengkap RPP IPA Kelas VIII berbasis Model KESAN yang mengintegrasikan etnosains Minangkabau. Semoga bermanfaat untuk meningkatkan kualitas pembelajaran sains yang kontekstual, bermakna, dan berkarakter.}

\vspace{0.3cm}

{\fontsize{10}{12}\selectfont\textit{"Alam terkembang jadi guru, adat bersendi syarak, syarak bersendi Kitabullah"}}\\
{\fontsize{9}{11}\selectfont\textit{- Filosofi Minangkabau -}}
\end{tcolorbox}
\end{center}

% ===============================
% KUMPULAN RPP (AUTOMATIS)
% ===============================
\clearpage
\section*{\centering KOMPILASI LENGKAP RPP SEMESTER 1 DAN 2}
\addcontentsline{toc}{section}{Kompilasi Lengkap RPP}
\begingroup
\setlength{\parskip}{0pt}
\setlength{\parindent}{0pt}

% ====== Semester 1 ======
% Title page macro for each RPP
% Usage: \rppsectionpage{Judul Besar}{NomorPertemuan}
\newcommand{\rppsectionpage}[2]{%
    \clearpage
    \thispagestyle{empty}%
    \vspace*{6cm}
    \begin{center}
        % Background box
        \tikz[remember picture,overlay] \node[fill=lightgray!20,rounded corners=8pt,minimum width=18cm,minimum height=8cm] at (current page.center) {};
        
        % Title text with line breaks handled automatically
        \parbox{16cm}{\centering
            {\fontsize{28}{34}\selectfont\textbf{#1}}
            
            \vspace{2cm}
            
            {\fontsize{20}{24}\selectfont\textbf{(Pertemuan #2)}}
        }
    \end{center}
    \clearpage
    \setcounter{page}{1}% Reset page numbering to 1 for each RPP
}

% ========= SEMESTER 1 =========
\rppsectionpage{Gerak Makhluk Hidup Tari Piring}{1}
\subfile{Semester_1/[1.1]_Gerak_Makhluk_Hidup_Tari_Piring_Pert1}
\rppsectionpage{Gerak Makhluk Hidup Silek Minang}{2}
\subfile{Semester_1/[1.2]_Gerak_Makhluk_Hidup_Silek_Minang_Pert2}
\rppsectionpage{Gerak Benda Kincir Air}{3}
\subfile{Semester_1/[2.1]_Gerak_Benda_Kincir_Air_Pert1}
\rppsectionpage{Gerak Benda Energi Kinetik}{4}
\subfile{Semester_1/[2.2]_Gerak_Benda_Energi_Kinetik_Pert2}
\rppsectionpage{Pesawat Sederhana Katrol Tradisional}{5}
\subfile{Semester_1/[3.1]_Usaha_Pesawat_Sederhana_Katrol_Tradisional_Pert1}
\rppsectionpage{Pesawat Sederhana Tuas Lesung}{6}
\subfile{Semester_1/[3.2]_Usaha_Pesawat_Sederhana_Tuas_Lesung_Pert2}
\rppsectionpage{Struktur Tumbuhan Tanaman Obat}{7}
\subfile{Semester_1/[4.1]_Struktur_Fungsi_Tumbuhan_Tanaman_Obat_Pert1}
\rppsectionpage{Struktur Tumbuhan Fotosintesis Sawah}{8}
\subfile{Semester_1/[4.2]_Struktur_Fungsi_Tumbuhan_Fotosintesis_Sawah_Pert2}
\rppsectionpage{Klasifikasi Tumbuhan Obat}{9}
\subfile{Semester_1/[5.1]_Klasifikasi_Makhluk_Hidup_Tumbuhan_Obat_Pert1}
\rppsectionpage{Klasifikasi Organisasi Kehidupan}{10}
\subfile{Semester_1/[5.2]_Klasifikasi_Makhluk_Hidup_Organisasi_Kehidupan_Pert2}
\rppsectionpage{Sistem Pencernaan Kuliner Minang}{11}
\subfile{Semester_1/[6.1]_Sistem_Pencernaan_Kuliner_Minang_Pert1}
\rppsectionpage{Sistem Pencernaan Pengobatan Tradisional}{12}
\subfile{Semester_1/[6.2]_Sistem_Pencernaan_Pengobatan_Tradisional_Pert2}
\rppsectionpage{Zat Aditif Pengawetan Tradisional}{13}
\subfile{Semester_1/[7.1]_Zat_Aditif_Adiktif_Pengawetan_Tradisional_Pert1}
\rppsectionpage{Zat Adiktif Kearifan Pencegahan}{14}
\subfile{Semester_1/[7.2]_Zat_Aditif_Adiktif_Kearifan_Pencegahan_Pert2}
\rppsectionpage{Sistem Peredaran Darah Aktivitas Tradisional}{15}
\subfile{Semester_1/[8.1]_Sistem_Peredaran_Darah_Aktivitas_Tradisional_Pert1}
\rppsectionpage{Sistem Peredaran Darah Pengobatan Tradisional}{16}
\subfile{Semester_1/[8.2]_Sistem_Peredaran_Darah_Pengobatan_Tradisional_Pert2}
\rppsectionpage{Sistem Pernapasan Teknik Tradisional}{17}
\subfile{Semester_1/[9.1]_Sistem_Pernapasan_Teknik_Tradisional_Pert1}
\rppsectionpage{Sistem Pernapasan Pengobatan Herbal}{18}
\subfile{Semester_1/[9.2]_Sistem_Pernapasan_Pengobatan_Herbal_Pert2}
\rppsectionpage{Sistem Ekskresi Kesehatan Ginjal}{19}
\subfile{Semester_1/[10.1]_Sistem_Ekskresi_Kesehatan_Ginjal_Pert1}
\rppsectionpage{Sistem Ekskresi Pengobatan Tradisional}{20}
\subfile{Semester_1/[10.2]_Sistem_Ekskresi_Pengobatan_Tradisional_Pert2}

% ====== Semester 2 ======
% ========= SEMESTER 2 =========
\rppsectionpage{Tekanan Arsitektur Rumah Gadang}{21}
\subfile{Semester_2/[11.1]_Tekanan_Arsitektur_Rumah_Gadang_Pert1}
\rppsectionpage{Tekanan Sistem Irigasi Tradisional}{22}
\subfile{Semester_2/[11.2]_Tekanan_Sistem_Irigasi_Tradisional_Pert2}
\rppsectionpage{Tekanan Perahu Tradisional}{23}
\subfile{Semester_2/[11.3]_Tekanan_Perahu_Tradisional_Pert3}
\rppsectionpage{Getaran Alat Musik Tradisional}{24}
\subfile{Semester_2/[12.1]_Getaran_Alat_Musik_Tradisional_Pert1}
\rppsectionpage{Gelombang Bunyi Randai Minangkabau}{25}
\subfile{Semester_2/[12.2]_Gelombang_Bunyi_Randai_Minangkabau_Pert2}
\rppsectionpage{Gelombang Air Permainan Tradisional}{26}
\subfile{Semester_2/[12.3]_Gelombang_Air_Permainan_Tradisional_Pert3}
\rppsectionpage{Cahaya Cermin Rumah Gadang}{27}
\subfile{Semester_2/[13.1]_Cahaya_Cermin_Rumah_Gadang_Pert1}
\rppsectionpage{Pembiasan Kaca Tradisional}{28}
\subfile{Semester_2/[13.2]_Pembiasan_Kaca_Tradisional_Pert2}
\rppsectionpage{Dispersi Pelangi Danau Maninjau}{29}
\subfile{Semester_2/[13.3]_Dispersi_Pelangi_Danau_Maninjau_Pert3}
\rppsectionpage{Gaya Bajak Tradisional}{30}
\subfile{Semester_2/[14.1]_Gaya_Bajak_Tradisional_Pert1}
\rppsectionpage{Usaha Kincir Air Tradisional}{31}
\subfile{Semester_2/[14.2]_Usaha_Kincir_Air_Tradisional_Pert2}
\rppsectionpage{Daya Pengolahan Pangan Tradisional}{32}
\subfile{Semester_2/[14.3]_Daya_Pengolahan_Pangan_Tradisional_Pert3}
\endgroup

\end{document}